\documentclass{article}

\usepackage{ctex}
\usepackage{tikz}
\usetikzlibrary{cd}
\usetikzlibrary{decorations.pathreplacing}

\usepackage{amsthm}
\usepackage{amsmath}
\usepackage{amssymb}

\usepackage{enumitem}

\usepackage[textwidth=18cm]{geometry} % 设置页宽=18

\usepackage{blindtext}
\usepackage{bm}
\parindent=0pt
\setlength{\parindent}{2em} 
\usepackage{indentfirst}



\usepackage{xcolor}
\usepackage{titlesec}
\titleformat{\section}[block]{\color{blue}\Large\bfseries\filcenter}{}{1em}{}
\titleformat{\subsection}[hang]{\color{red}\Large\bfseries}{}{0em}{}
%\setcounter{secnumdepth}{1} %section 序号

\newtheorem{theorem}{Theorem}[section]
\newtheorem{lemma}[theorem]{Lemma}
\newtheorem{corollary}[theorem]{Corollary}
\newtheorem{proposition}[theorem]{Proposition}
\newtheorem{example}[theorem]{Example}
\newtheorem{definition}[theorem]{Definition}
\newtheorem{remark}[theorem]{Remark}
\newtheorem{exercise}{Exercise}[section]
\newtheorem{annotation}[theorem]{Annotation}

\newcommand*{\xfunc}[4]{{#2}\colon{#3}{#1}{#4}}
\newcommand*{\func}[3]{\xfunc{\to}{#1}{#2}{#3}}

\newcommand\Set[2]{\{\,#1\mid#2\,\}} %集合
\newcommand\SET[2]{\Set{#1}{\text{#2}}} %

\newcommand{\norm}[1]{\left\lVert#1\right\rVert} % 范数
\newcommand{\vect}[1]{\mathbf{#1}} % vector

\begin{document}
\title{考研高数}
\author{枫聆}
\maketitle
\tableofcontents

\newpage
\section{经典证明}
\begin{theorem}
\rm {\color{red} (连续函数在闭区间上有界)} 若real-valued函数$f$在闭区间$[a,b]$上连续,那么它在其上有界(上下界).
\end{theorem}

\begin{proof}
{\color{blue} (方法1: 构造$f(x)$非空子区间$[a,x]$,求其上确界)} 假设$B$是使得$f(x)$在形如闭区间$[a,x]$上有界的$x \in [a,b]$集合,显然$a \in B$,所以$B$非空。若$e \in B$且$e > a$,那么$a$和$e$之间的点都是在$B$里面的,所以实际上$B$是一个闭区间. 我们再考虑$B$的上确界,根据$x$的取法,有$x \leq b$,如果我们能证明它的上确界在$b$出取得,那么整个命题就得证. 现在假设$\sup(B) < b$, 由于$B$是一个闭区间,所以$\sup(B) \in B$. 由于$f$是连续的,那么足够靠近$\sup(B)$的地方,即$s -\sup(B) < \delta$且$s > \sup(B)$,有$|f(s) - f(\sup(B))| < \varepsilon$,那么$f(x),\;x \in [\sup(B),s]$也是有界,这是和$\sup{B}$是$B$的上确界矛盾的.

{\color{blue} (方法2: 构造一个严格递增的数列,其子列收敛造矛盾)}.
\end{proof}

\begin{theorem}
\rm {\color{red} (确界原理)} 任一有上界的非空实数集必有上确界,同理任一有下界的非空实数集必有下确界.
\end{theorem}

\begin{proof}
{\color{blue} (构造一个实数划分,用戴德金分割定理说明界数就是确界)}假设非空实数集$S$有上界$M$,取$S$所有上界为集合$B$. 因为$M \in B$所以$B$非空,取$A = \mathbb{R} \setminus B$,要证明$A$是非空是trivial的,取$x = x_0 - 1,\; x_0 \in S$,那么$x \in A$. 显然地$A$里面所有的元素都小于$B$里面的元素(若是大于$B$里面某个元素,那么它就是$S$的一个上界了,这是矛盾的),这样我们就可以得到一个实数上的划分,根据{\color{red} 戴德金实数分割定理},存在一个$\beta$,它要么是$A$里面最大值或者要么$B$里面的最小值. 假设它是$A$里面的最大值,根据$A$的定义,对于任意$a \in A$都存在一个$x_0 \in S$使得$a < x_0$,将其作用到$\beta$上,我们得到某个$x_0' \in S$使得$\beta < x_0'$. 我们考虑$\frac{x_0'+\beta}{2}$,有
$$
\beta < \frac{x_0' + \beta}{2} < x_0'
$$
所以$\frac{x_0' + \beta}{2} \in A$,这和$\beta$是$A$里面最大值是矛盾的,所以$\beta \in B$,即这个$\beta$就是$S$的上确界.
\end{proof}

\begin{theorem}
\rm {\color{red} (极值定理)} 若real-valued函数$f$在闭区间$[a,b]$连续,那们存在$c,d \in [a,b]$使得
$$
f(c) \leq f(x) \leq f(d),\; x \in [a,b].
$$
\end{theorem}

\begin{proof}
{\color{blue} (构造一个特殊连续函数说明原函数可以取到确界)}$f$在闭区间$[a,b]$上连续({\color{red} 连续闭有界}),那么马上可以得到$f$在$[a,b]$上有界. 取集合$Y = \Set{f(x)}{ x \in [a,b]}$,即$Y$有界,根据{\color{red}确界原理}$Y$有确界,那么我们下面证明思路,就是看$f(x)$是不是能取到这个确界. 取其上确界为$m$,假设不存在$d \in [a,b]$使得$f(d) = m$,那么我们考虑函数$g(x)=\frac{1}{m - f(x)}$,由于$m > f(x),\; x \in [a,b]$,所以$g(x)$在$[a,b]$上是连续的,又因为$f$在$[a,b]$是上有界的,那么$g$在其上也是有界的. 由于$m$是上确界,所以对任意的正实数$\varepsilon$,都有$m - f(x) \leq \varepsilon$,那么$g(x) \geq \frac{1}{\varepsilon}$,这说明$g(x)$是发散的,造成了矛盾. 所以$f$是可以取到上确界的.
\end{proof}

\begin{theorem}
\rm {\color{red} (罗尔定理)}如果real-valued函数$f$在闭区间$[a,b]$上连续,且在开区间$(a,b)$内可导,若有$f(a) = f(b)$,那么存在至少一个$c \in (a,b)$使得
$$
f'(c) = 0.
$$
\end{theorem}

\begin{proof}
{\color{blue} (确界处导数存在的充分必要条件)}$f$在$[a,b]$上连续,那么根据{\color{red} 极值定理}其在$[a,b]$是可以取到极值的,分两种情况讨论: (1 如果其最大值和最小值同时在$a,b$取得,那么$f$就是常函数,对任意的$x \in [a,b]$都有$f'(x) = 0$. (2 不失一般性,我们假设$f$在一点$c \in (a,b)$处$f(c)$为最大值(若是最小值,考虑$-f$即可),我们来考虑$c$的一个邻域$(c-\varepsilon, c+\varepsilon)$两边,其中$c-\varepsilon$和$c+\varepsilon$均在$[a,b]$里面. 对任意的$h \in (c-\varepsilon, c)$都有
$$
f'(c^-) = \lim\limits_{h \rightarrow c^-}\frac{f(c)- f(h)}{c-h} \leq 0. 
$$
同理对任意的$t \in (c,c+\varepsilon)$都有
$$
f'(c^+) = \lim\limits_{t \rightarrow c^+}\frac{f(t)-f(c)}{t-c} \geq 0.
$$
由于$f$在$c$点可导,那么$f'(c) = f'(c^-) = f'(c^+) = 0$.
\end{proof}


\begin{theorem}
\rm {\color{red} (中值定理)} 若real-valued函数$f$在闭区间$[a,b]$($a < b$)上连续,且在$(a,b)$上可导,那么存在一个实数$c \in (a,b)$使得
$$
f'(c) = \frac{f(b) - f(a)}{b-a}
$$ 
\end{theorem}

\begin{proof}
{\color{blue} (中值定理是罗尔定理的推广)} 构造函数$g(x) = f(x) - rx$,通过选择合适的$r$,使得$g(a) = g(b)$,即
$$
\begin{array}{ll}
g(a) = g(b) &\Leftrightarrow f(a) - ra = f(b) - rb \\
			&\Leftrightarrow r = \frac{f(b) - f(a)}{b-a}.
\end{array} 
$$
那么根据{\color{red} 罗尔定理},我们知道存在一点$c \in (a,b)$,使得$g'(c) = 0$,
$$
\begin{array}{ll}
g'(x) = f'(x) - r \\
g'(c) = f'(c) - r = 0 \\
f'(c) = r = \frac{f(b) - f(a)}{b-a}.
\end{array}  
$$
证毕.
\end{proof}


\begin{theorem}
\rm {\color{red} (柯西中值定理)} 若两个real-valued函数$f$和$g$都在闭区间$[a,b]$($a < b$)上连续,且都在$(a,b)$上可导. 那么存在一点$c \in (a,b)$,使得
$$
(f(b) - f(a))g'(c) = (g(b) - g(a))f'(c). 
$$
特别地,若$g(b) \neq g(a)$且$g'(c) \neq 0$,等价于
$$
\frac{f'(c)}{g'(c)} = \frac{f(b)-f(a)}{g(b) - g(a)}.
$$
\end{theorem}

\begin{proof}
{\color{blue} (柯西中值定理是中值定理的扩展)} 构造函数$h(x) = f(x) - rg(x)$,选择合适$r$使得$h(a) = h(b)$,若$g(b) \neq g(a)$即$r = \frac{f(b)-f(a)}{g(b) - g(a)}$,那么根据{\color{red}罗尔定理}可以得到$h'(c) = 0$, 即
$$
0 = g'(c) - rf'(c) = f'(c) -  \frac{f(b)-f(a)}{g(b) - g(a)}g'(x).
$$
若$g(a) = g(b)$,同样根据{\color{red}罗尔定理}有$g'(c) = 0$,这个条件显然是使得前面第一个等式成立的.
\end{proof}


\begin{theorem}
\rm {\color{red} (夹逼准则)} 若函数$f,g,h$均在以点$a$为聚点的区间$I$上定义着,且对任意的$x \in I$,其中$x \neq a$都有
$$
g(x) \leq f(x) \leq h(x),
$$
并且同时满足
$$
\lim\limits_{x \rightarrow a}g(x) = \lim\limits_{x \rightarrow a}h(x) = L.
$$
那么$\lim\limits_{x \rightarrow a}f(x) = L$.
\end{theorem}

\begin{proof}
{\color{blue} (经典两边夹)} 取任意的正实数$\varepsilon > 0$,根据极限地定义对$g(x)$和$h(x)$我们可以分别找到$|x| < \delta_1$和$|x| < \delta_2$,使得
$$
L-\varepsilon < g(x) < L + \varepsilon~\text{和}~L-\varepsilon < h(x) < L + \varepsilon
$$
成立,我们取$\delta = \min(\delta_1, \delta_2)$,那么当$|x| < \delta$时有
$$
L-\varepsilon < g(x) \leq f(x) \leq h(x) < L+\varepsilon.
$$
由于$\varepsilon$的任意性,所以有$\lim\limits_{x \rightarrow a} f(x) = L$.
\end{proof}

\newpage
\section{数列极限}

\subsection{数列极限的定义}

\begin{definition}
\rm 若对于每一整数$\varepsilon$,不论它怎样小,恒有序号$N$,使在$n > N$时,一切$x_n$满足不等式\[|x_n-a| < \varepsilon\],则称常数$a$为数列$(x_n)$当$n$趋向于无穷时的{\color{red}极限},记为$\lim\limits_{n \rightarrow \infty}=a$. 也可以说这个序列收敛于$a$.
\end{definition}

\subsection{数列极限的几何意义}

\subsection{数列左右极限}

\subsection{数列极限的基本性质}
%局部性
\begin{proposition}
\rm 若$\lim x_n = a$,又$a > p$,那么存在某个正整数$N$,当$n > N$时,一切$x_n$都满足$x_n > p$.
\end{proposition}

%包号性
\begin{proposition}
\rm 若$\lim x_n = a$,有$a > 0$,那么存在某个正整数$N$,当$n > N$时,一切$x_n$都满足$x_n > 0$.
\end{proposition}

%保值性?
\begin{proposition}
\rm 若$\lim x_n = a$,且$a \neq 0$,则必有充分远的$x_n$值,其绝对值大于某个正数$r$:
$$
|x_n| > r > 0 \; (n > N).
$$
{\color{blue} 性质1的推论}.
\end{proposition}

%有界性
\begin{proposition}
\rm 若$\lim x_n = a$,则$(x_n)$必有界.
\end{proposition}

\begin{proof}
我们知道任意给定一个$\varepsilon > 0$,可以找到正整数$N$,使得当$n > N$时的一切$x_n$都满足$ a-\varepsilon < x_n < a + \varepsilon$,因此当$n > N$是存在一个界数使得$|x_n| < M$. 接着再考虑$n \leq N$时,注意这时候只有有限个$x_n$,我们把$M$和它们放在一起,取它们里面绝对值最大的数$M'$,即有$|x_n| \leq M'$.
\end{proof} 

%极限唯一
\begin{proposition}
\rm 若同时有$\lim x_n = a,\; \lim x_n = b$,则$a = b$.  
\end{proposition}

\subsection{数列无穷小和无穷大}

\subsection{数列极限运算}

\begin{proposition}
\rm 若$x_n = y_n$,则$\lim x_n = \lim y_n$.
\end{proposition}

\begin{proposition}
\rm 若恒有$x_n \leq y_n$,且各自趋于有限极限,则$\lim x_n \leq \lim y_n$.
\end{proposition}

\begin{proposition}
\rm {\color{red} 夹闭准则} {\color{blue} 见经典证明}.
\end{proposition}

\begin{proposition}
\rm 任意有限个无穷小的和亦是无穷小.
\end{proposition}

\begin{proposition}
\rm 有界数列$(x_n)$与无穷小${\alpha_n}$的乘积仍是无穷小.
\end{proposition}

\begin{proposition}
\rm 若$\lim x_n = a,\; \lim y_n =b$,则$\lim(x_n\pm y_n) = a \pm b$. 考虑两个极限的尾巴
$$
\lim (x_n \pm y_n) = a+b+ \alpha + \beta.
$$
\end{proposition}

\begin{proposition}
\rm 若$\lim x_n = a,\; \lim y_n =b$,则$\lim(x_ny_n) = ab$. 考虑两个极限的尾巴
$$
\lim x_ny_n = ab + a\beta + b\alpha + \alpha\beta.
$$
\end{proposition}


\begin{proposition}
\rm 若$\lim x_n = a,\; \lim y_n =b$,且$b\neq 0$,则$\lim \frac{x_n}{y_n} = \frac{a}{b}$. 考虑两个极限的尾巴
$$
\lim \frac{x_n}{y_n} = \frac{a+\alpha}{b+\beta}.
$$
\end{proposition}

\subsection{不定式}

\begin{annotation}
$$
\frac{x_n}{y_n} \sim \frac{0}{0}
$$
\end{annotation}

\begin{annotation}
$$
\frac{x_n}{y_n} \sim \frac{\infty}{\infty}
$$
\end{annotation}

\begin{annotation}
$$
x_n y_n \sim  0 \cdot \infty
$$
\end{annotation}

\begin{annotation}
$$
x_n - y_n \sim \infty - \infty
$$
\end{annotation}

\subsection{单调数列的极限}

\begin{theorem}
\rm 给定单调增加的数列$(x_n)$,若它有上界
$$
x_n \leq M,\; n = 1,2,\cdots
$$
则它必有一有限极限,此极限为上确界. 同理单调减少的数理$(x_n)$,若它有下界
$$
x_n \geq M, \; n = 1,2,\cdots
$$
则它必有一有限极限,此极限为下确界.
\end{theorem}

\subsection{收敛原理}

\begin{theorem}
\rm {\color{red} (柯西收敛原理)}数列$(x_n)$有有限极限的必要且充分条件是: 对于任意的数$\varepsilon > 0$,总存在着整数$N$,使得当$n > N$和$n' > N$时有下面不等式成立
$$
|x_n - x_{n'}| < \varepsilon.
$$
\end{theorem}

\begin{proof}
{\color{red} (必要性)} 若$\lim x_n = a$,即对任意的$\frac{\varepsilon}{2}$,能找到一个整数$N$,使得$n, n' > N$时有
$$
|x_n - a| < \frac{\varepsilon}{2},\; |x_{n'} - a| < \frac{\varepsilon}{2}
$$
成立. 那么
$$
|x_n - x_{n'}| = |(x_n -a ) + (a - x_{n'})| \leq |x_n-a| + |x_{n'}-a| < \varepsilon
$$

{\color{red} (充分性)} 若满足上述定理中的条件,{\color{blue}我们要证明$\lim x_n = a$,我们得想办法把这个$a$表示出来,这里手法将会用戴德金实数划分的结论来把这个$a$弄出来}.

在全体实数域下构造一个划分. 对于任何实数$\alpha$,若$x_n$从某项其满足不等式\[x_n > \alpha,\]则取这种实数$\alpha$归入下组$A$,其余的(即不落在$A$里面的)一起实数归入上组$A'$.

首先我们来说明这样确实产生了一个实数上的划分. 由前提条件,对于任意数$\varepsilon>0$及其对应的$N$. 若$n > N$及$n' > N$,则下面不等式成立\[x_{n'} - \varepsilon < x_n < x_{n'}+\varepsilon.\]现在我们可以看到每一个数$x_{n'} - \varepsilon$都是小于$x_n$的,所以它归入下组$A$. 另一方面$x_{n'}+\varepsilon$都大于$x_n$,所以$x_{n'}+\varepsilon$放不进去$A$,那它只能归入$A'$了,所以$A$和$A'$都是非空的. 我们的划分方式对于每一个数$\alpha$和确定序列$x_n$,要么它属于$A$或者属于$A'$. 同时$A$中实数都小于$A'$的实数. 如果$\alpha > \alpha', \alpha \in A , \alpha' \in A'$,则$x_n$从某一项其也都大于$a'$,这样就产生矛盾了. 因此的确产生了一个实数上的划分.

根据戴德金基本定理,有实数$a$存在它是$A$和$A'$的界数,即\[\alpha \leq a \leq \alpha', \; \alpha \in A,\alpha' \in A'.\]我们注意到当$n > N$时,$x_{n'} - \varepsilon$是一个$\alpha$,而$x_{n'} +\varepsilon$是一个$\alpha'$. 所以我们有\[x_{n'} - \varepsilon \leq a \leq x_{n'} + \varepsilon.\]即$|x_{n'}-a| \leq \varepsilon$,所以$\lim x_n = a$.

{\color{blue} 若是不用实数划分的手法,也可以构造$a_n = \inf\limits_{k \geq n} x_k$和$b_n=\sup\limits_{k \geq n}$,证明$\lim a_n = \lim b_n = c$,再用一下夹逼准则$a_n \leq x_n \leq b_n$.}
\end{proof}

\subsection{上下极限}
%https://en.wikipedia.org/wiki/Limit_inferior_and_limit_superior
\begin{definition}
\rm 序列$(x_n)$的部分极限的最大值和最小值,称为$x_n$的上极限和下极限,各记为
$$
\overline{\lim} x_n~\text{及}~\underline{\lim} x_n.
$$
\end{definition}

\newpage
\section{函数极限}

\subsection{函数极限的定义}

\subsection{函数左右极限}

\subsection{两个重要极限}

\begin{proposition}
$$
\lim\limits_{x \rightarrow 0} \frac{\sin x}{x}.
$$
\end{proposition}

\begin{proof}
几何证明的手法,或者直接上洛必达.
\end{proof}

\begin{proposition}
$$
\lim\limits_{x \rightarrow +\infty} (1+\frac{1}{x})^x.
$$
\end{proposition}


\begin{proof}
单调有界.
\end{proof}

\subsection{函数极限的基本性质}

%局部性
%保号性
%有界性
\begin{proposition}
\rm 若$\lim\limits_{x \rightarrow a} = A$,则对于充分接近$a$的$x$的函数值是有界的
$$
|f(x)| \leq M ,\; |x-a| < \delta.
$$
{\color{blue} 注意这里和$\lim x_n = b$,而导致整个$(x_n)$有界是区别的,因为这里当$x-a < -\delta$或者$x-a > \delta$可能是有无限多个函数值的,它们是否有界是无法确定的}.
\end{proposition}

\subsection{函数极限运算}



\subsection{洛必达法则}

\begin{definition}
\rm 若real-value函数$f$和$g$都在去心邻域$\tilde{U}(c,\delta)$可导,有
$$
\lim\limits_{x \rightarrow c} f(x) = \lim\limits_{x \rightarrow c} g(x) = 0 ~\text{或者}~ \lim\limits_{x \rightarrow c} g(x) = \infty,
$$
且对任意$x \in \tilde{U}$都有$g'(x) \neq 0$,同时有$\lim\limits_{x \rightarrow c}\frac{f'(x)}{g'(x)}$存在,那么
$$
\lim\limits_{x \rightarrow c} \frac{f(x)}{g(x)} = \lim\limits_{x \rightarrow c} \frac{f'(x)}{g'(x)}.
$$
\end{definition}

\begin{proof}
首先来看一个比较特殊的情况,除满足上述条件之外,若还满足$f(c) = g(c) = 0$,并且$g'(c) \neq 0$,那么
$$
\lim\limits_{x \rightarrow c}  \frac{f(x)}{g(x)} = \lim\limits_{x \rightarrow c} \frac{f(x) - f(c)}{g(x) - g(c)} = \lim\limits_{x \rightarrow c} \frac{ \frac{f(x) - f(c)}{x-c}}{\frac{g(x) - g(c)}{x-c}} = \frac{f'(c)}{g'(c)} = \lim\limits_{x \rightarrow c} \frac{f'(x)}{g'(x)}.
$$
下面来严格证明分两种情况来证明,由于$\tilde{U}$在$c$这里间断,后面需要频繁使用到柯西中值定理,所以自然地在$\title{U}$的两端来分析,取开区间$\mathcal{I}$以$c$点为端点,且$\mathcal{I} \subset \tilde{U}$. {\color{blue}注意到条件满足对任意的$x \in \mathcal{I}$有$g'(x) \neq 0$,并且$g$在$\mathcal{I}$上是连续的,那么是可以在$\mathcal{I}$里面找到一个足够小的区间使得$g(x)\neq 0$,那这个小区间代替$\mathcal{I}$}. 

我们定义$m(x) = \inf\frac{f'(\alpha)}{g'(\alpha)}$和$M(x) = \sup\frac{f'(\alpha)}{g'(\alpha)}$其中$x \in \mathcal{I}$,$\alpha$取遍$x$和$c$之间的数({\color{blue} 为什么确保可以取到确界? 任意$\alpha$处$f$和$g$导数均有意义}). 在确定$x$之后,我们再取定$x$和$c$之间一点$y$,结合柯西中值定理可以保证在它们之间找到一个$c$使得
$$
 m(x) \leq \frac{f(x) - f(y)}{g(x) - g(y)} = \frac{f'(\alpha)}{g'(\alpha)} \leq M(x).
$$
注意为什么这里可以保证$g(x) - g(y) \neq 0$? 假设存在$g(x) = g(y)$, 那么根据罗尔定理,就存在一点$p$使得$g'(p) = 0$,这是和前提条件$\forall x \in \bar{U},\;g'(x) \neq 0$矛盾的.

{\color{blue}情况一}:$\lim\limits_{x \rightarrow c} f(x) = \lim\limits_{x \rightarrow c} g(x) = 0.$

对任意的$x \in \mathcal{I}$, 取$y$位于$x$和$c$之间,为了得到$\frac{f(x)}{g(x)}$,我们让
$$
m(x) \leq \frac{f(x) - f(y)}{g(x) - g(y)} = \frac{\frac{f(x)}{g(x)} - \frac{f(y)}{g(x)}}{1 - \frac{g(y)}{g(x)}} \leq M(x).
$$
当$y \rightarrow c$时,$\frac{f(y)}{g(x)}$和$\frac{g(y)}{g(x)}$都趋向于$0$,所以
$$
m(x) \leq \frac{f(x)}{g(x)} \leq M(x).
$$

{\color{blue}情况二}:$\lim\limits_{x \rightarrow c} g(x) = \infty.$

对任意的$x \in \mathcal{I}$, 取$y$位于$x$和$c$之间. 如果我们还是用上面的分式,直接尝试把$\frac{f(x)}{g(x)}$构造出来,尝试分式对$y \rightarrow c$取极限的时候,显然是无法处理的. {\color{blue}同时你注意到在当前条件下是对$\lim\limits_{x \rightarrow c} f(x)$是没有特别说明的,言下之意它不会对我们的证明产生影响}. 现在我们考虑把前面分式上下都除以$g(y)$,同时上下取负,即
$$
m(x) \leq \frac{f(y) - f(x)}{g(y) - g(x)} = \frac{\frac{f(y)}{g(y)} - \frac{f(x)}{g(y)}}{1 - \frac{g(x)}{g(y)}} \leq M(x).
$$
那么当$y \rightarrow c$时,$\frac{f(x)}{g(y)}$和$\frac{g(x)}{g(y)}$都是趋于$0$,那么此刻关键是我们如何需要考虑$\lim\limits_{y \rightarrow c}\frac{f(y)}{g(y)}$? 让$S_x = \SET{y}{y~\text{位于}~x~\text{和}~c~\text{之间}}$, 我们取遍$y \in S_x$, 我们可以得到得到一个有界数列$\{\frac{f(y)}{g(y)}\}$({\color{red} 为什么有界? $f$和$g$在$[x,c]$上连续}),我们考虑其上下极限
$$
m(x) \leq \lim\limits_{y \rightarrow c }\inf\frac{f(y)}{g(y)} \leq \lim\limits_{y \rightarrow c }\sup\frac{f(y)}{g(y)} \leq M(x).
$$

当对$m(x)$和$M(x)$也取极限$x \rightarrow c$时,有
$$
\lim\limits_{x \rightarrow c} m(x) = \lim\limits_{x \rightarrow c} M(x) = \lim\limits_{x \rightarrow c} \frac{f'(x)}{g'(x)}.
$$
对{\color{blue}情况一}使用{\color{red} 夹逼准则},可以很快得到$\lim\limits_{x \rightarrow c} \frac{f(x)}{g(x)} = \lim\limits_{x \rightarrow c} \frac{f'(x)}{g'(x)}$. 对{\color{blue} 情况二 }也同样使用{\color{red} 夹逼准则},可以得到
$$
\lim\limits_{y \rightarrow c }\inf\frac{f(y)}{g(y)} = \lim\limits_{y \rightarrow c }\sup\frac{f(y)}{g(y)} = \lim\limits_{x \rightarrow c} \frac{f'(x)}{g'(x)},
$$
上下极限相等可以马上得到$\lim\limits_{x \rightarrow c} \frac{f(x)}{g(x)} = \lim\limits_{x \rightarrow c} \frac{f'(x)}{g'(x)}$. 最终证毕.
\end{proof}

\end{document}