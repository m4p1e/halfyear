\documentclass{article}

\usepackage{ctex}
\usepackage{tikz}
\usetikzlibrary{cd}
\usetikzlibrary{decorations.pathreplacing}

\usepackage{amsthm}
\usepackage{amsmath}
\usepackage{amssymb}

\usepackage{unicode-math}

\usepackage{enumitem}

\usepackage[textwidth=18cm]{geometry} % 设置页宽=18

\usepackage{blindtext}
\usepackage{bm}
\parindent=0pt
\setlength{\parindent}{2em} 
\usepackage{indentfirst}



\usepackage{xcolor}
\usepackage{titlesec}
\titleformat{\section}[block]{\color{blue}\Large\bfseries\filcenter}{}{1em}{}
\titleformat{\subsection}[hang]{\color{red}\Large\bfseries}{}{0em}{}
%\setcounter{secnumdepth}{1} %section 序号

\newtheorem{theorem}{Theorem}[section]
\newtheorem{lemma}[theorem]{Lemma}
\newtheorem{corollary}[theorem]{Corollary}
\newtheorem{proposition}[theorem]{Proposition}
\newtheorem{example}[theorem]{Example}
\newtheorem{definition}[theorem]{Definition}
\newtheorem{remark}[theorem]{Remark}
\newtheorem{exercise}{Exercise}[section]

\newcommand*{\xfunc}[4]{{#2}\colon{#3}{#1}{#4}}
\newcommand*{\func}[3]{\xfunc{\to}{#1}{#2}{#3}}

\newcommand\Set[2]{\{\,#1\mid#2\,\}} %集合
\newcommand\SET[2]{\Set{#1}{\text{#2}}} %

\newcommand{\norm}[1]{\left\lVert#1\right\rVert} % 范数
\newcommand{\vect}[1]{\mathbf{#1}} % vector

\begin{document}
\title{考研高数}
\author{枫聆}
\maketitle
\tableofcontents

\newpage
\section{经典证明}
\begin{definition}
\rm {\color{red} 连续函数在闭区间上有界} 若real-valued函数$f$在闭区间$[a,b]$上连续,那么它在其上有界.
\end{definition}

\begin{proof}
{(\color{blue} $f(x)$非空子区间$[a,x]$,求其上确界)} 假设$B$是使得$f(x)$在形如闭区间$[a,x]$上有界的$x \in [a,b]$集合,显然$a \in B$,所以$B$非空。若$e \in B$且$e > a$,那么$a$和$e$之间的点都是在$B$里面的,所以实际上$B$是一个闭区间. 我们再考虑$B$的上确界,根据$x$的取法,有$x \leq b$,如果我们能证明它的上确界在$b$出取得,那么整个命题就得证. 现在假设$\sup(B) < b$, 由于$B$是一个闭区间,所以$\sup(B) \in B$. 由于$f$是连续的,那么足够靠近$\sup(B)$的地方,即$s -\sup(B) < \delta$且$s > \sup(B)$,有$|f(s) - f(\sup(B))| < \varepsilon$,那么$[\sup(B),s]$也是有界,这是和$\sup{B}$是$B$的上确界矛盾的.
\end{proof}

\begin{definition}
\rm 若real-valued函数$f$在闭区间$[a,b]$连续,那们存在$c,d \in [a,b]$使得
$$
f(c) \leq f(x) \leq f(d),\; x \in [a,b].
$$
\end{definition}

\begin{proof}

\end{proof}

\begin{definition}
\rm 如果real-valued函数$f$在闭区间$[a,b]$上连续,且在开区间$(a,b)$内可导,若有$f(a) = f(b)$,那么存在至少一个$c \in (a,b)$使得
$$
f'(c) = 0.
$$
\end{definition}

\begin{proof}

\end{proof}

\newpage
\section{函数极限}



\end{document}