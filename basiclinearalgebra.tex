\documentclass{article}

\usepackage{ctex}
\usepackage{tikz}
\usetikzlibrary{cd}
%\usetikzlibrary{paths.ortho} % path折线
%\usetikzlibrary{decorations.pathreplacing}
\usetikzlibrary{calc}
\usetikzlibrary{graphs, graphs.standard, quotes}% quotes library is for the [""] edges
\usetikzlibrary{positioning} %right of 描述位置的

\usepackage{amsthm}
\usepackage{amsmath}
\usepackage{amssymb}

\usepackage{unicode-math}

\usepackage{enumitem}

\usepackage[textwidth=18cm]{geometry} % 设置页宽=18

\usepackage{blindtext}
\usepackage{bm}
\parindent=0pt
\setlength{\parindent}{2em} 
\usepackage{indentfirst}



\usepackage{xcolor}
\usepackage{titlesec}
\titleformat{\section}[block]{\color{blue}\Large\bfseries\filcenter}{}{1em}{}
\titleformat{\subsection}[hang]{\color{red}\Large\bfseries}{}{0em}{}
%\setcounter{secnumdepth}{1} %section 序号

\newtheorem{theorem}{Theorem}[section]
\newtheorem{lemma}[theorem]{Lemma}
\newtheorem{corollary}[theorem]{Corollary}
\newtheorem{proposition}[theorem]{Proposition}
\newtheorem{example}[theorem]{Example}
\newtheorem{definition}[theorem]{Definition}
\newtheorem{remark}[theorem]{Remark}
\newtheorem{exercise}{Exercise}[section]

\newcommand*{\xfunc}[4]{{#2}\colon{#3}{#1}{#4}}
\newcommand*{\func}[3]{\xfunc{\to}{#1}{#2}{#3}}

\newcommand\Set[2]{\{\,#1\mid#2\,\}} %集合
\newcommand\SET[2]{\Set{#1}{\text{#2}}} %

\newcommand{\norm}[1]{\left\lVert#1\right\rVert} % 范数
\newcommand{\vect}[1]{\mathbf{#1}} % vector

\begin{document}
\title{考研高数}
\author{枫聆}
\maketitle
\tableofcontents

\newpage 
\section{高等代数的研究对象}

linear algebra在研究多元一次方程组的解的过程中逐渐形成的一门学科. 如有这样一个方程组
$$
\begin{array}{cl}
x_1 + 3x_2 + x_3 &= 2,\\
3x_1 + 4x_2 + 2x_3 &= 9, \\
-x_1 - 5x_2 + 4x_3 &= 10,  \\
2x_2 + 7x_2 + x_3 &= 1
\end{array}
$$
我们求解方程组的解过程通常是把用某一行的$n$倍加上或者减去另一行,这种情况下我们其实是在对每一行方程的变元前面的系数做运算,只要我们排列好对应系数的位置就可以避免每一次都写变元,自然而然地,多元一次方程组对应的{\color{red}系数矩阵}和 {\color{red} 增广矩阵}就诞生了,例如上面方程组的增广矩阵如下
$$
\begin{bmatrix}
1 & 3 & 1 & 2 \\
3 & 4 & 2 & 9 \\
-1 & -5 & 4 & 10  \\
2 & 7 & 1 & 1
\end{bmatrix}
$$
这个矩阵也可以称为4阶矩阵($4 \times 4$). 去掉最后常数项列就是对应的系数矩阵. 我们通过研究增广矩阵来研究原方程的解,那么研究方程组的解可以从两个方向出发,把解直接解出来或者判别方程组解的情况.

%两种折线
\tikzset{
  -|-/.style={
    to path={
      (\tikztostart) -| ($(\tikztostart)!#1!(\tikztotarget)$) |- (\tikztotarget)
      \tikztonodes
    }
  },
  -|-/.default=0.5,
  |-|/.style={
    to path={
      (\tikztostart) |- ($(\tikztostart)!#1!(\tikztotarget)$) -| (\tikztotarget)
      \tikztonodes
    }
  },
  |-|/.default=0.5,
}

\begin{center}
\begin{tikzpicture}[v/.style={rectangle,draw=blue}]
\node (linear equation) [v] {n元线性方程组};
\node (matrix) [v, below right = of linear equation] {矩阵};
%\node (vector) [v, below = of linear equation] {n维向量};
\node (vector space) [v, below = of linear equation] {n维向量空间};
\node (linear space) [v, below = of vector space] {线性空间};
\node (linear map)	[v, below = of matrix] {线性映射};
\node (linear transform) [v, below = of linear map] {线性变换};
\node (double vector-value function) [v, left = of linear space] {双线性函数};
\node (linear metric space) [v, below = of double vector-value function] {具有度量的线性空间};
\node (metric via linear transformation) [v,below = of $(linear metric space)!0.5!(linear transform)$] {与度量相关的线性变换};
\coordinate (metric join linear transformation) at ([yshift=0.5cm]metric via linear transformation.north); 

\draw [->]	(linear equation) to (vector space);
\draw [->]	(linear equation.east) -| (matrix);
\draw [->]	(vector space) to (linear space);
\draw [->]	(linear space) to (linear map);
\draw [->]	(linear map) to (linear transform);
\draw [->]	(linear space) to (double vector-value function);
\draw [<->]	(matrix) to (linear map);
\draw [->]	(double vector-value function) to (linear metric space);
\draw [-] (linear transform) |- (metric join linear transformation);
\draw [-] (linear metric space) |- (metric join linear transformation);
\draw [->] (metric join linear transformation) -> (metric via linear transformation.north);
\end{tikzpicture}
\end{center}
上图是整个linear algebra研究过程的发展,值得关注是{\color{red}线性空间是向量空间的一般推广};{\color{red} 矩阵不光可以表示$n$维线性方程,同样也是可以用来表示线性映射的}; {\color{red}线性空间到自身的线性映射叫做线性变换}; {\color{red} 从线性空间到线性度量空间需要用到一个双线性函数来描述两个其空间两个元素的度量};
\end{document}