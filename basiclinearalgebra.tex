\documentclass{article}

\usepackage{ctex}
\usepackage{tikz}
\usetikzlibrary{cd}
%\usetikzlibrary{paths.ortho} % path折线
%\usetikzlibrary{decorations.pathreplacing}
\usetikzlibrary{calc}
\usetikzlibrary{graphs, graphs.standard, quotes}% quotes library is for the [""] edges
\usetikzlibrary{positioning} %right of 描述位置的

\usepackage{amsthm}
\usepackage{amsmath}
\usepackage{amssymb}

\usepackage{unicode-math}

\usepackage{enumitem}

\usepackage[textwidth=18cm]{geometry} % 设置页宽=18

\usepackage{blindtext}
\usepackage{bm}
\parindent=0pt
\setlength{\parindent}{2em} 
\usepackage{indentfirst}



\usepackage{xcolor}
\usepackage{titlesec}
\titleformat{\section}[block]{\color{blue}\Large\bfseries\filcenter}{}{1em}{}
\titleformat{\subsection}[hang]{\color{red}\Large\bfseries}{}{0em}{}
%\setcounter{secnumdepth}{1} %section 序号

\newtheorem{theorem}{Theorem}[section]
\newtheorem{lemma}[theorem]{Lemma}
\newtheorem{corollary}[theorem]{Corollary}
\newtheorem{proposition}[theorem]{Proposition}
\newtheorem{example}[theorem]{Example}
\newtheorem{definition}[theorem]{Definition}
\newtheorem{remark}[theorem]{Remark}
\newtheorem{exercise}{Exercise}[section]
\newtheorem{annotation}[theorem]{Annotation}

\newcommand*{\xfunc}[4]{{#2}\colon{#3}{#1}{#4}}
\newcommand*{\func}[3]{\xfunc{\to}{#1}{#2}{#3}}

\newcommand\Set[2]{\{\,#1\mid#2\,\}} %集合
\newcommand\SET[2]{\Set{#1}{\text{#2}}} %

\newcommand{\norm}[1]{\left\lVert#1\right\rVert} % 范数
\newcommand{\vect}[1]{\mathbf{#1}} % vector

\newcommand\bigzero{\makebox(0,0){\text{\huge0}}} %bigzero

\begin{document}
\title{考研线代}
\author{枫聆}
\maketitle
\tableofcontents

\newpage 
\section{高等代数的研究对象}

linear algebra在研究多元一次方程组的解的过程中逐渐形成的一门学科. 如有这样一个方程组
$$
\begin{array}{cl}
x_1 + 3x_2 + x_3 &= 2,\\
3x_1 + 4x_2 + 2x_3 &= 9, \\
-x_1 - 5x_2 + 4x_3 &= 10,  \\
2x_2 + 7x_2 + x_3 &= 1
\end{array}
$$
我们求解方程组的解过程通常是把用某一行的$n$倍加上或者减去另一行,这种情况下我们其实是在对每一行方程的变元前面的系数做运算,只要我们排列好对应系数的位置就可以避免每一次都写变元,自然而然地,多元一次方程组对应的{\color{red}系数矩阵}和 {\color{red} 增广矩阵}就诞生了,例如上面方程组的增广矩阵如下
$$
\begin{bmatrix}
1 & 3 & 1 & 2 \\
3 & 4 & 2 & 9 \\
-1 & -5 & 4 & 10  \\
2 & 7 & 1 & 1
\end{bmatrix}
$$
这个矩阵也可以称为4阶矩阵($4 \times 4$). 去掉最后常数项列就是对应的系数矩阵. 我们通过研究增广矩阵来研究原方程的解,那么研究方程组的解可以从两个方向出发,把解直接解出来或者判别方程组解的情况.

%两种折线
\tikzset{
  -|-/.style={
    to path={
      (\tikztostart) -| ($(\tikztostart)!#1!(\tikztotarget)$) |- (\tikztotarget)
      \tikztonodes
    }
  },
  -|-/.default=0.5,
  |-|/.style={
    to path={
      (\tikztostart) |- ($(\tikztostart)!#1!(\tikztotarget)$) -| (\tikztotarget)
      \tikztonodes
    }
  },
  |-|/.default=0.5,
}

\begin{center}
\begin{tikzpicture}[v/.style={rectangle,draw=blue}]
\node (linear equation) [v] {n元线性方程组};
\node (matrix) [v, below right = of linear equation] {矩阵};
%\node (vector) [v, below = of linear equation] {n维向量};
\node (vector space) [v, below = of linear equation] {n维向量空间};
\node (linear space) [v, below = of vector space] {线性空间};
\node (linear map)	[v, below = of matrix] {线性映射};
\node (linear transform) [v, below = of linear map] {线性变换};
\node (double vector-value function) [v, left = of linear space] {双线性函数};
\node (linear metric space) [v, below = of double vector-value function] {具有度量的线性空间};
\node (metric via linear transformation) [v,below = of $(linear metric space)!0.5!(linear transform)$] {与度量相关的线性变换};
\coordinate (metric join linear transformation) at ([yshift=0.5cm]metric via linear transformation.north); 

\draw [->]	(linear equation) to (vector space);
\draw [->]	(linear equation.east) -| (matrix);
\draw [->]	(vector space) to (linear space);
\draw [->]	(linear space) to (linear map);
\draw [->]	(linear map) to (linear transform);
\draw [->]	(linear space) to (double vector-value function);
\draw [<->]	(matrix) to (linear map);
\draw [->]	(double vector-value function) to (linear metric space);
\draw [-] (linear transform) |- (metric join linear transformation);
\draw [-] (linear metric space) |- (metric join linear transformation);
\draw [->] (metric join linear transformation) -> (metric via linear transformation.north);
\end{tikzpicture}
\end{center}
上图是整个linear algebra研究过程的发展,值得关注是{\color{red}线性空间是向量空间的一般推广};{\color{red} 矩阵不光可以表示$n$维线性方程,同样也是可以用来表示线性映射}; {\color{red}线性空间到自身的线性映射叫做线性变换}; {\color{red} 从线性空间到线性度量空间需要用到一个双线性函数来描述两个其空间两个元素的度量};


\begin{definition}
\rm {\color{red} 阶梯型矩阵} 若给定矩阵满足下面条件
\begin{enumerate}
	\item $0$行全在下方;
	\item {\color{red}主元}(非零行第一个非零元素)的列指标随着行指标的增加而严格增大.
\end{enumerate}
则称其为行阶梯形矩阵. i.e.
$$
\begin{pmatrix}
{\color{red}1} & 3 & 1 & 2 \\
0 & {\color{red}1} & -1 & -3 \\
0 & 0 & {\color{red}3} & 6 \\
0 & 0 & 0 & 0 \\
\end{pmatrix}
$$
同理还有列阶梯形矩阵,把上面这个矩阵顺时针旋转90°.
$$
\begin{pmatrix}
2 & -3 & 6 & 0 \\
1 & -1 & 3 & 0 \\
3 & 3 & 0 & 0  \\
1 & 0 & 0 & 0 \\
\end{pmatrix}
$$
\end{definition}

\begin{definition}
\rm {\color{red} (简化阶梯型矩阵)} 若给定行阶梯形满足下面条件
\begin{enumerate}
	\item 主元都是$1$.
	\item 主元所在列的其余元素都是$0$.
\end{enumerate}
则称其为简化行阶梯形矩阵. i.e.
$$
\begin{pmatrix}
{\color{red}1} & 0 & 0 & 3 \\
0 & {\color{red}1} & 0 & -1 \\
0 & 0 & {\color{red}1} & 2 \\
0 & 0 & 0 & 0 \\
\end{pmatrix}
$$
\end{definition}

\begin{definition}
\rm 下面几种矩阵变换操作称之为矩阵的{\color{red}初等行变换}
\begin{enumerate}
	\item 把一行的倍数加到另一行上;
	\item 两行互换;
	\item 一行乘以一个非零数.
\end{enumerate}
\end{definition}

\begin{theorem}
\rm {\color{red} (初等行变换的性质)}矩阵的初等行变换得到的方程组与原来的方程组同解.
\end{theorem}

\begin{proposition}
\rm {\color{red} (线性方程组的判别方法)} $n$元线性方程组解的情况只有三种可能,其对应的行阶梯行矩阵的特征如下
\begin{enumerate}
	\item 若非零行个数等于未知数个数,则有唯一解;
	\item 若非零行个数小于未知数个数,则有无穷多个解;
	\item 若存在非零行对应等式$0=d$,其中$d$不等于零,则无解.
\end{enumerate}
有一个很形象的理解方式类比{\color{blue}到平面上两条直线,它们可能存在三种关系: 平行,相交(相交于一点),重合}.
\end{proposition}

\begin{proof}
其中无解对应的情况是显然的,那么我们在确保不会出现$0=d$的情况下有解,且为什么只会出现唯一解或者无穷多个解,而不会出现有两个解的这种情况呢? 若给定$n$元线性方程组的增广矩阵,经过初等行变换化成行阶梯形矩阵$J$. 设$J$有$r$个非零行,显然$J$有$n+1$列,其中$n+1$列表示等式右边的常数项.
$$
J=
\begin{pmatrix}
\cdots & \cdots &c_1 \\
\vdots & \vdots & \vdots \\
\cdots & b_{rt} &\cdots \\
\end{pmatrix}
$$
设$r$行的主元为$b_{rt}$,即它在第$t$列,那么必有$t \leq n$. 根据行阶梯形的定义,列指标是随着行指标的增加而严格增大的,因此也有$t \geq r$. 所以结合前面两个关系,可以得到$r \leq n$,即非零行的个数是不会超过$n$的. 此刻先把$J$通过初等行变换变成简化行阶梯形矩阵$J_1$,那么$J_1$也有$r$非零行,即有$r$个主元,下面分两种情况分别讨论: {\color{red}当$r=n$时},$J_1$就有$n$个主元,那么按照行阶梯形主元排列顺序,
$$
J_1 =
\begin{pmatrix}
1 & \cdots & \cdots &\cdots & c_1 \\
0 & 1 & \cdots & \cdots & c_2 \\
0 & 0 & 1 & \cdots & c_3 \\
\vdots & \vdots & \vdots & \vdots & \vdots \\
\cdots & \cdots & \cdots & 1 & c_n 
\end{pmatrix}
$$
即$n$个主元,要分别放到$n$列,那么就是每列都有一个主元,即原方程组有唯一解. {\color{red}当$r < n$时},此时简化行梯形矩阵为
$$
J_1 =
\begin{pmatrix}
1 & \cdots & \cdots &\cdots & \cdots & c_1 \\
\cdots & 1 & \cdots & \cdots & \cdots & c_2 \\
\cdots & \cdots & 1 & \cdots & \cdots & c_3 \\
\vdots & \vdots & \vdots & \vdots & \vdots & \vdots \\
\cdots & \cdots & \cdots & 1 & \cdots & c_r 
\end{pmatrix}
$$
那么现在我们可以把上面这$r$个主元对应的未知数表示出来
$$
\begin{array}{ll}
x_1 &= a_{1,1}x_{i_1} + a_{1,2}x_{i_2} + \cdots + a_{1,n-r} x_{n-r} \\
x_{j_2} &= a_{j_2,1}x_{i_1} + a_{j_2,2}x_{i_2} + \cdots + a_{j-2,n-r} x_{n-r} \\
&\vdots \\
x_{j_r} &= a_{j_r,1}x_{i_1} + a_{j_r,2}x_{i_2} + \cdots + a_{j-r,n-r} x_{n-r}
\end{array}
$$
其中$x_1,x_{j_2},\cdots x_{j_r}$表示对应列的$r$个主元,其余的自由变量有$n-r$个用$x_{i_1},x_{i_2},\cdots,x_{n-r}$表示. 那么只要任意确定一组自由变量的取值,也就确定了一组解,即这种情况下只有无穷多个解,上面这个形式也称为{\color{red}一般解}.
\end{proof}

\begin{definition}
\rm $n$元齐次线性方程组如下
$$
\begin{array}{l}
a_{11}x_1 + a_{12}x_2 + \cdots a_{1n}x_n = 0 \\
a_{21}x_1 + a_{22}x_2 + \cdots a_{2n}x_n = 0 \\
\vdots \\
a_{m1}x_1 + a_{m2}x_2 + \cdots a_{mn}x_n = 0 \\
\end{array}
$$
显然$(0,0,\cdots,0)$是它的一个解,称为{\color{red}零解}; 其余的解(如果有)称为{\color{red}非零解}.
\end{definition}

\begin{lemma}
\rm {\color{red} (齐次线性方程组有非零解的充要条件)} $n$元齐次线性方程组有零解当且仅当{\color{red}系数矩阵}经过初等行变换得到的阶梯形矩阵的非零行个数$r<n$.
\end{lemma}

\begin{lemma}
\rm {\color{red} (齐次线性方程族有非零解的充分条件)} $n$元齐次线性方程组的方程个数$m < n$,则它有非零解.
\end{lemma}


\newpage
\section{行列式}
\subsection{基本定义}
%Determinant
\begin{annotation}
\rm {\color{red} 行列式的意义} 在用矩阵解$n$元线性方程组的过程中,我们需要将对应的增广矩阵转换为阶梯型矩阵,然后向上回溯求解. 如果对于一个$n$元线性方程组,我们只想关注它的解结构或者更明确一点,是否有解就够了,那我们是否可以在不把具体解解出来的情况下,回答这个问题呢?也就是说我们是否在可以在不进行初等行变换化阶梯形矩阵的情况下,来回答这个问题?

从二元线性方程组作为例子出发来进行讨论,给定一个二元线性方程组
$$
\begin{array}{ll}
a_{11}x_1 + a_{12}x_2 = c_1 \\
a_{21}x_1 + a_{22}x_2 = c_2
\end{array}
$$
其中$a_{11},a_{21}$不全为$0$,不妨设$a_{11}$不等于$0$. 其对应的增广矩阵通过初等行变换化作阶梯形矩阵如下
$$
\begin{pmatrix}
a_{11} & a_{12} & c_1 \\
0 & a_{22}-a_{12}\frac{a_{21}}{a_{11}} & c_2 -c_1\frac{a_{21}}{a_{11}} \\
\end{pmatrix}
=
\begin{pmatrix}
a_{11} & a_{12} & c_1 \\ 
0 & \frac{a_{11}a_{22}-a_{12}a_{21}}{a_{11}} & \frac{c_2a_{11}-c_1a_{21}}{a_{11}} \\
\end{pmatrix}
$$
若$a_{11}a_{22}-a_{12}a_{21} \neq 0$,则原方程组有唯一解; 反之,若$a_{11}a_{22}-a_{12}a_{21} = 0$,则原方程组有无穷多个解或者无解. 可以很快发现$a_{11}a_{22}-a_{12}a_{21}$就是$\begin{pmatrix}
a_{11} & a_{12} \\
a_{21} & a_{21} 
\end{pmatrix}$的行列式,称其为{\color{red} 二阶行列式}. 基于这个直觉就将其推广到$n$元方程组对应的系数矩阵行列式来判别解的情况.
\end{annotation}

\begin{definition}
\rm {\color{red} (逆序数的概率)} 若给定一个排列$j_1j_2\cdots j_n$,取其中任意两个不同的数$j_p$和$j_q$,使得$1 \leq p< q <n$. 若$j_p > j_q$,就称$j_p$和$j_q$构成一个{\color{red}逆序}. 一个排列中所有逆序的总和称为这个排列的{\color{red}逆序数}. 特别地,如果一个排列的逆序数是偶数,则称这个排列为{\color{red}偶排列},否则称为{\color{red}奇排列}.
\end{definition}

\begin{definition}
\rm {\color{red} ($n$阶行列式的概念)} $n$阶行列式的{\color{red}完全展开式}如下
$$
\begin{vmatrix}
a_{11} & a_{12} & \cdots & a_{1n} \\
a_{21} & a_{22} & \cdots & a_{2n} \\
\vdots & \vdots & 		 & \vdots \\
a_{n1} & a_{n2} & \cdots & a_{nn} \\
\end{vmatrix}
=\sum\limits_{j_1,j_2,\cdots,j_n} (-1)^{\tau(j_1j_2\cdots j_n)} a_{1j_1}a_{2j_2}\cdots a_{nj_n}.
$$
其中$j_1,j_2,\cdots,j_n$表示一组列指标的排列,即$a_{1j_1}a_{2j_2}\cdots a_{nj_n}$表示取自不同行不同列的$n$个元素的乘积. 用$\tau(j_1j_2\cdots j_n)$表示$j_1j_2\cdots j_n$的逆序数.  
\end{definition}

\begin{definition}
\rm 记$|\mathbf{A}| = \begin{vmatrix}
a_{11} & a_{12} & \cdots & a_{1n} \\
a_{21} & a_{22} & \cdots & a_{2n} \\
\vdots & \vdots & 		 & \vdots \\
a_{n1} & a_{n2} & \cdots & a_{nn} \\
\end{vmatrix}$,$|\mathbf{A}^T| = \begin{vmatrix}
a_{11} & a_{21} & \cdots & a_{n1} \\
a_{12} & a_{22} & \cdots & a_{n2} \\
\vdots & \vdots & 		 & \vdots \\
a_{1n} & a_{n2} & \cdots & a_{nn} \\
\end{vmatrix}$称为$|A|$的{\color{red}转置行列式}.
\end{definition}


\begin{definition}
\rm 在$n$阶行列式中划去$a_{ij}$所在的第$i$行和第$j$列,留下来的$n-1$阶行列式叫做$a_{ij}$的{\color{red}余子式}; 称$(-1)^{i+j}M_{ij}$为{\color{red}代数余子式},记为$A_{ij}$.
\end{definition}


\newpage
\subsection{基本性质}

\begin{lemma}
\rm 一个排列中任意两个元素对换,其逆序数的奇偶性改变.
\end{lemma}

\begin{proof}
设给定一个排列为$j_1j_2\cdots j_n$,先考虑两个相邻的数交换,即
$$
j_1j_2\cdots j_{i}j_{i+1}\cdots j_n \longrightarrow j_1j_2\cdots j_{i+1}j_{i}\cdots j_n
$$
其中$1\leq i <n$. 很显然相邻的两个数交换,并不会影响它们两边的数的逆序个数,分两种情况,若$j_i > j_k$,那么变换之后的排列的逆序数减1; 若$j_i < j_k$,那么变换之后的排列数的逆序数加1即,因此做一次相邻交换其奇偶性改变. 再来考虑交换任意两个数$j_i$和$j_k$,我们可以把这个交换拆解为多个相邻数的交换,即
$$
j_1j_2\cdots j_{i}\cdots j_{k}\cdots j_n \longrightarrow j_1j_2\cdots j_{i-1}\cdots j_{i}j_{k}\cdots j_n \longrightarrow j_1j_2\cdots j_{i-1}j_k\cdots j_{i}\cdots j_n
$$
那么第一个箭头做了$k-i-1$次两两交换,第二个箭头做了$k-i$次两两交换,那么一共做了$2(k-i)-1$次,因此这个排列的逆序数的奇偶性一定会改变.
\end{proof}

\begin{proposition}
\rm 行列式的基本性质如下
\begin{enumerate}
	\item 行列式和它的转置行列式相等,即$|A^T| = |A|$;
	\item 两行互换位置,行列式的值变号;
	\item 某行若有公因子$k$,则可以把$k$提出行列式记号之外;
	\item 如果行列式某行(或列)是两个元素之和,则可把行列式拆成两个行列式之和;
	$$
	\begin{vmatrix}
a_{11} & a_{12} & \cdots & a_{1n} \\
\vdots & \vdots & 		& \\
a_{i1}+b_{i1} & a_{i2}+b_{i1} & \cdots & a_{in}+b_{in} \\
\vdots & \vdots & 		 & \vdots \\
a_{n1} & a_{n2} & \cdots & a_{nn} \\
\end{vmatrix}
	 = \begin{vmatrix}
a_{11} & a_{12} & \cdots & a_{1n} \\
\vdots & \vdots & 		& \\
a_{i1} & a_{i2} & \cdots & a_{in} \\
\vdots & \vdots & 		 & \vdots \\
a_{n1} & a_{n2} & \cdots & a_{nn} \\
\end{vmatrix} +
	 \begin{vmatrix}
a_{11} & a_{12} & \cdots & a_{1n} \\
\vdots & \vdots & 		& \\
b_{i1} & b_{i1} & \cdots & b_{in} \\
\vdots & \vdots & 		 & \vdots \\
a_{n1} & a_{n2} & \cdots & a_{nn} \\
\end{vmatrix}
	$$
	\item 把某行(或者列)的$k$加到另一行(或者列),行列式的值不变.
\end{enumerate}
\end{proposition}

\begin{proof}
给定$n$阶行列式为$\begin{vmatrix}
a_{11} & a_{12} & \cdots & a_{1n} \\
a_{21} & a_{22} & \cdots & a_{2n} \\
\vdots & \vdots & 		 & \vdots \\
a_{n1} & a_{n2} & \cdots & a_{nn} \\
\end{vmatrix}$.

{\color{red}(1)} 设$|\mathbf{A}^T|=\sum\limits_{p_1,p_2,\cdots,p_n} (-1)^{\tau(p_1p_2\cdots p_n)} b_{1p_1}b_{2p_2}\cdots b_{np_n} = \sum\limits_{p_1,p_2,\cdots,p_n} (-1)^{\tau(p_1p_2\cdots p_n)} a_{p_11}a_{p_22}\cdots a_{p_nn}$,其中$b_{ij} = a_{ij}$. 要证明$|\mathbf{A}| = |\mathbf{A}^T|$,{\color{blue}我们需要证明对于$|A^T|$代数和中的任意一项都有$|\mathbf{A}|$代数和的一项与其对应,反之亦然}. 那么任取$|\mathbf{A}^T|$中一项$\sum\limits (-1)^t a_{p_11}a_{p_22}\cdots a_{p_nn}$   ,我们考虑任意交换其中的两个元素$a_{p_ii}$和$a_{p_jj}$,即
$$
\sum\limits_{p_1,p_2,\cdots,p_n} (-1)^t a_{p_11}a_{p_22}\cdots a_{p_ii} \cdots a_{p_jj} \cdots a_{p_nn} = \sum\limits_{p_1,p_2,\cdots,p_n} (-1)^t a_{p_11}a_{p_22}\cdots a_{p_jj} \cdots a_{p_ii} \cdots a_{p_nn}
$$
这里观察两个排列的逆序数的改变,一个是列指标排列$1,2,\cdots,j,\cdots,i,\cdots,n$,原本$1,2,\cdots,i,j,\cdots,n$是自然序排列,它的逆序数为零(偶数排列),那么根据前面的lemma,现在它的逆序数设为$r$,$r$此刻应该是一个奇数. 另一个是行指标排列$p_1,p_2,\cdots,p_j,\cdots,p_i,\cdots,p_n$,原本为$t$,此刻设为$t_1$,那么还是根据前面的lemma,现在有$(-1)^t_1 = (-1)(-1)^t$. 我们尝试用$r$和$t_1$替换掉$t$,即
$$
(-1)^t = (-1) (-1)^{t_1} = (-1)^r (-1)^{t_1} = (-1)^{r+t}.
$$ 
{\color{blue}上面这个等式的告诉我们无论经过多少次的$a_{p_ii}$和$a_{p_jj}$的交换,$t$都可以用行指标排列和列指标排列的逆序数的和来代替}. 那么当通过有限次元素交换把$a_{p_11}a_{p_22}\cdots a_{p_nn}$变到$a_{1q_1}a_{2q_2}\cdots a_{nq_n}$之后,此时行指标排列是一个自然序,则$t = \tau(q_1,q_2,\cdots,q_n)$,这样就得到了$|A|$代数和中的某项,并且如果$a_{p_jj}$对应元素$a_{ij}$,那么当前$a_{iq_i}$也应该是对应着相同元素$a_{ij}$,即$p_1,p_2,\cdots,p_n$唯一确定了一组$q_1,q_2,\cdots,q_n$. 反之任取$|\mathbf{A}|$代数和中的一项$\sum\limits_{p_1,p_2,\cdots,p_n}(-1)^{\tau(p_1p_2\cdots p_n)}a_{1p_1}a_{2p_2}\cdots a_{np_n}$,也可以在$|A^T|$代数和中找到一项$\sum\limits_{q_1,q_2,\cdots,q_n}(-1)^{\tau(q_1q_2\cdots q_n)}a_{q_11}a_{q_22}\cdots a_{q_nn}$与之对应. {\color{blue}这个性质更本质的意义是行列式中的行和列具有同等的地位}.


{\color{red}(2)} 如果我们交换第$i$和$k$行得到的矩阵我们记为$A_2$,如下
$$
\begin{vmatrix}
a_{11} & a_{12} & \cdots & a_{1n} \\
\vdots & \vdots & 		 & \vdots \\
a_{i1} & a_{i2} & \cdots & a_{in} \\
\vdots & \vdots & 		 & \vdots \\
a_{k1} & a_{k2} & \cdots & a_{kn} \\
\vdots & \vdots & 		 & \vdots \\
a_{n1} & a_{n2} & \cdots & a_{nn} \\
\end{vmatrix} \longrightarrow
\begin{vmatrix}
a_{11} & a_{12} & \cdots & a_{1n} \\
\vdots & \vdots & 		 & \vdots \\
a_{k1} & a_{k2} & \cdots & a_{kn} \\
\vdots & \vdots & 		 & \vdots \\
a_{i1} & a_{i2} & \cdots & a_{in} \\
\vdots & \vdots & 		 & \vdots \\
a_{n1} & a_{n2} & \cdots & a_{nn} \\
\end{vmatrix} 
$$
那么
$$
\begin{array}{ll}
\det(\mathbf{A}_2) &=  \sum\limits_{j_1,j_2,\cdots,j_n} (-1)^{\tau(j_1j_2\cdots j_n)} b_{1j_1}b_{1j_1}b_{2j_1}\cdots b_{nj_n}\\
&= \sum\limits_{j_1,j_2,\cdots,j_n} (-1)^{\tau(j_1j_2\cdots {\color{red}j_i} \cdots {\color{red}j_k}\cdots j_n)} a_{1j_1}a_{2j_2}\cdots a_{kj_{k}}\cdots a_{ij_{i}} \cdots a_{nj_n} \\
&= \sum\limits_{j_1,j_2,\cdots,j_n} (-1)^{\tau(j_1j_2\cdots {\color{red}j_k} \cdots {\color{red}j_i}\cdots j_n)+1} a_{1j_1}a_{2j_2}\cdots a_{ij_{i}}\cdots a_{kj_{k}} \cdots a_{nj_n}\; ({\color{blue}\text{性质1}})\\
&= (-1)\sum\limits_{j_1,j_2,\cdots,j_n} (-1)^{\tau(j_1j_2\cdots {\color{red}j_k} \cdots {\color{red}j_i}\cdots j_n)} a_{1j_1}a_{2j_2}\cdots a_{ij_{i}}\cdots a_{kj_{k}} \cdots a_{nj_n}
\end{array}
$$
这里有一个推论是{\color{red}若行列式有两行(或者列)完全相同,那么此行列式等于零}. 这是因为$|\mathbf{A}|=-|\mathbf{A}|$,所以$|\mathbf{A}|=0$.

{\color{red}(3)} 考虑下面行列式
$$
|\mathbf{B}|=
\begin{vmatrix}
a_{11} & a_{12} & \cdots & a_{1n} \\
\vdots & \vdots & 		& \\
ka_{i1} & ka_{i2} & \cdots & ka_{in} \\
\vdots & \vdots & 		 & \vdots \\
a_{n1} & a_{n2} & \cdots & a_{nn} \\
\end{vmatrix}
$$
按照完全展开式
$$
\begin{array}{ll}
|\mathbf{B}| &= \sum\limits_{j_1j_2\cdots j_n}(-1)^{\tau(j_1j_2\cdots j_n)} a_{1j_1}\cdots ka_{ij_i} \cdots a_{nj_n}\\
&=  k\sum\limits_{j_1j_2\cdots j_n}(-1)^{\tau(j_1j_2\cdots j_n)} a_{1j_1}\cdots a_{ij_i} \cdots a_{nj_n}
\end{array}
$$

{\color{red}(4)} 按照完全展开式
$$
\begin{array}{ll}
|\mathbf{A}| &= \sum\limits_{j_1j_2\cdots j_n}(-1)^{\tau(j_1j_2\cdots j_n)} a_{1j_1}\cdots (a_{ij_i}+b_{ij_i}) \cdots a_{nj_n}\\
&=  \sum\limits_{j_1j_2\cdots j_n}(-1)^{\tau(j_1j_2\cdots j_n)} a_{1j_1}\cdots a_{ij_i} \cdots a_{nj_n} + \sum\limits_{j_1j_2\cdots j_n}(-1)^{\tau(j_1j_2\cdots j_n)} a_{1j_1}\cdots b_{ij_i} \cdots a_{nj_n}
\end{array}
$$

{\color{red}(5)} 若考虑把第$i$行的$k$倍加第到$q$行,那么当且的完全展开式如下
$$
\begin{array}{ll}
|\mathbf{A}_1| &= \sum\limits_{j_1j_2\cdots j_n}(-1)^{\tau(j_1j_2\cdots j_n)} a_{1j_1}\cdots (a_{qj_q}+ka_{ij_i}) \cdots a_{nj_n}\\
&= \sum\limits_{j_1j_2\cdots j_n}(-1)^{\tau(j_1j_2\cdots j_n)} a_{1j_1}\cdots (a_{qj_q}) \cdots a_{nj_n} + k\sum\limits_{j_1j_2\cdots j_n}(-1)^{\tau(j_1j_2\cdots j_n)} a_{1j_1}\cdots a_{ij_i} \cdots (a_{ij_i}) \cdots a_{nj_n}\\
&= \sum\limits_{j_1j_2\cdots j_n}(-1)^{\tau(j_1j_2\cdots j_n)} a_{1j_1}\cdots (a_{qj_q}) \cdots a_{nj_n}
\end{array}
$$
\end{proof}

\begin{lemma}
\rm 给定如下行列式
$$
\det(D)=
\begin{vmatrix}
a_{11} & 0 & \cdots & 0 \\
a_{21} & a_{22} & \cdots & a_{2n}\\
\vdots & \vdots &	&\vdots \\
a_{n1} & a_{n2} & \cdots & a_{nn}\\
\end{vmatrix},
$$
则$\det(D) = a_{11}A_{11}$.
\end{lemma}

\begin{proof}
这其实是很trivial的,其完全展开式如下
$$
\begin{array}{ll}
\det(D) &= \sum\limits_{j_1,j_2,\cdots,j_n} (-1)^{\tau(j_1j_2\cdots j_n)}a_{1j_1}a_{2j_2}\cdots a_{j_nn} \\
&= a_{11} \sum\limits_{j_2,\cdots,j_n} (-1)^{\tau(j_2\cdots j_n)}a_{2j_2}\cdots a_{j_nn} \\
&= a_{11}(-1)^{1+1}M_{11}\\
&= a_{11}A_{11}.
\end{array}
$$
\end{proof}

\begin{lemma}
\rm 若给定$n$行列式$D$存在元素$a_{ij}$所在行元素除了它自己以外全为零,如下
$$
\det(D)=
\begin{vmatrix}
a_{11} & \cdots & a_{1j} & \cdots & a_{1n}\\
\vdots & & \vdots & & \vdots \\
0	& \cdots & a_{ij} & \cdots & 0 \\
\vdots & &\vdots & &\vdots \\
a_{n1} & \cdots & a_{nj} & \cdots & a_{nn} \\ 
\end{vmatrix}.
$$ 
则$\det(D)=a_{ij}A_{ij}$
\end{lemma}

% ^_^
\begin{proof}
考虑将第$i$行先向上移到第一行,再将第$j$列向左移动到第一列,还原到Lemma 2.8的情况下我们设为$D_1$. {\color{blue}注意这里是相邻移动,而不是直接将第一行和第$i$行的对换,第一列和第$j$列的替换,这样做的目的就是在将$a_{ij}$移动到$(1,1)$了之后,对应的$M_{11}$是和原来的$M_{ij}$相同的}. 所以现在我们只需要考虑在移动过程中对$D$的影响,向上移动了$i-1$,向左移动了$j-1$次,那么一共移动了$i+j-2$次,即
$$
D = (-1)^{i+j-2}D_1  = (-1)^{i+j-2} a_{ij}M_{ij} = a_{ij}A_{ij}.  
$$
\end{proof}

\begin{theorem}
\rm {\color{red} (按行(列)展开定理)} $n$阶行列式等于它的任一行(列)的各元素与其对应的代数余子式乘积的代数和,即
$$
\begin{array}{ll}
D = a_{i1}A_{i1} + a_{i2}A_{i2} + \cdots + a_{in}A_{in} \; i=1,2,\cdots,n\\ \\
D = a_{1j}A_{1j} + a_{2j}A_{2j} + \cdots + a_{nj}A_{nj} \; j=1,2,\cdots,n. 
\end{array}
$$
\end{theorem}

\begin{proof}
将$D$按第$i$行拆成$n$个行列式.
$$
\begin{array}{ll}
D &= \begin{vmatrix}
a_{11} & a_{12} & \cdots & a_{1n} \\
\vdots & \vdots & &\vdots \\
a_{i1}+0+\cdots+0 & 0+a_{i2}+\cdots+0 & \cdots & 0+0+\cdots+a_{in}\\
a_{n1} & a_{n2} &  \cdots & a_{nn} 
\end{vmatrix} \\
&= \begin{vmatrix}
a_{11} & a_{12} & \cdots & a_{1n} \\
\vdots & \vdots & &\vdots \\
a_{i1} & 0 & \cdots & 0\\
a_{n1} & a_{n2} &  \cdots & a_{nn} 
\end{vmatrix}
+ \begin{vmatrix}
a_{11} & a_{12} & \cdots & a_{1n} \\
\vdots & \vdots & &\vdots \\
0 & a_{i2} & \cdots & 0\\
a_{n1} & a_{n2} &  \cdots & a_{nn} 
\end{vmatrix}
+ \cdots 
+ \begin{vmatrix}
a_{11} & a_{12} & \cdots & a_{1n} \\
\vdots & \vdots & &\vdots \\
0 & 0 & \cdots & a_{in}\\
a_{n1} & a_{n2} &  \cdots & a_{nn} 
\end{vmatrix} \\
&= a_{i1}A_{i1} + a_{i2}A_{i2} + \cdots + a_{in}A_{in}.
\end{array}
$$
\end{proof}

\begin{theorem}
\rm $n$阶行列式的任一行(列)元素与另一行(列)元素的代数余子式乘积的代数和等于零,即
$$
a_{i1}A_{j1} + a_{i2}A_{j2} + \cdots + a_{in}A_{jn} = 0,\;(i\neq j).
$$
\end{theorem}


\begin{proof}
我们把给定行列式$D$的第$j$行特别地标注出来
$$
D=
\begin{vmatrix}
a_{11} & \cdots & a_{1n} \\
\vdots && \vdots \\
b_1 & \cdots & b_n \\
\vdots && \vdots \\
a_{n1} & \cdot & a_{nn} \\
\end{vmatrix}
$$
其中$b_1,\cdots,b_n$表示第$j$行,那么此时
$$
D = b_1A_{j1} + \cdots +b_nA_{jn}.
$$
我们再第$j$行用第$i$行换掉得到$D_1$,那么此时第$j$行各元素的代数余子式是没有发生变化的,且现在$D_1$是有两行相同的元素,所以
$$
D_1 = a_{i1}A_{j1} + \cdots + a_{in}A_{jn} = 0.
$$
同理也可以对第$j$列做上述操作.
\end{proof}

\begin{proposition}
\rm {\color{red} (几种特殊的行列式)}
\begin{enumerate}
	\item {\color{red}上(下)三角形行列式}
	$$
	\begin{vmatrix}
	a_{11} & & \bigzero\\
	a_{21} & a_{22} \\
	\vdots & \vdots & \ddots \\
	a_{n1} & a_{n2} & \cdots & a_{nn}
	\end{vmatrix} =
	\begin{vmatrix}
	a_{11} & a_{12} & \cdots & a_{1n}\\
	&	a_{22} & \cdots & a_{2n} \\
	&\bigzero& \ddots & \vdots \\
	&&& a_{nn} \\
	\end{vmatrix} = a_{11}a_{22}\cdots a_{nn}.
	$$
	\item {\color{red}对角行列式}
	$$
	\begin{vmatrix}
	a_{11} &&& \\
	&a_{22} &\bigzero& \\
	&\bigzero&\ddots& \\
	&&&a_{nn}
	\end{vmatrix} = a_{11}a_{22}\cdots a_{nn}.
	$$
\end{enumerate}
\end{proposition}

\begin{proof}
{\color{red}(1)} 完全展开式需要取值不同行不同列元素乘积的代数和,那么在上三角行列式中,只考虑非零项,因此第一行只能取$a_{11}$,接着第二行也就只能取$a_{22}$,如此去下去,最后只有$a_{11}a_{22}\cdots a_{nn}$这个非零项.

{\color{red}(2)} (1)的一个特殊情况.
\end{proof}

\end{document}