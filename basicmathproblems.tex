\documentclass{article}

\usepackage{ctex}
\usepackage{tikz}
\usetikzlibrary{calc}
\usetikzlibrary{cd}
\usetikzlibrary{decorations.pathreplacing}

\usepackage{amsthm}
\usepackage{amsmath}
\usepackage{amssymb}

\usepackage{pgfplots}
\pgfplotsset{compat=newest}

\usepackage{hyperref} %url
\hypersetup{
    colorlinks=true,
    linkcolor=blue,
    filecolor=magenta,      
    urlcolor=cyan,
    pdftitle={Overleaf Example},
    pdfpagemode=FullScreen,
    }

\usepackage{enumitem}

\usepackage[textwidth=18cm]{geometry} % 设置页宽=18

\usepackage{blindtext}
\usepackage{bm}
\parindent=0pt
\setlength{\parindent}{2em} 
\usepackage{indentfirst}


\usepackage{xcolor}
\usepackage{titlesec}
\titleformat{\section}[block]{\color{blue}\Large\bfseries\filcenter}{}{1em}{}
\titleformat{\subsection}[hang]{\color{red}\Large\bfseries}{}{0em}{}
%\setcounter{secnumdepth}{1} %section 序号

\newtheorem{theorem}{Theorem}[section]
\newtheorem{lemma}[theorem]{Lemma}
\newtheorem{corollary}[theorem]{Corollary}
\newtheorem{proposition}[theorem]{Proposition}
\newtheorem{example}[theorem]{Example}
\newtheorem{definition}[theorem]{Definition}
\newtheorem{remark}[theorem]{Remark}
\newtheorem{exercise}{Exercise}[section]
\newtheorem{annotation}[theorem]{Annotation}

\newcommand*{\xfunc}[4]{{#2}\colon{#3}{#1}{#4}}
\newcommand*{\func}[3]{\xfunc{\to}{#1}{#2}{#3}}

\newcommand\Set[2]{\{\,#1\mid#2\,\}} %集合
\newcommand\SET[2]{\Set{#1}{\text{#2}}} %

\newcommand{\norm}[1]{\left\lVert#1\right\rVert} % 范数
\newcommand{\vect}[1]{\mathbf{#1}} % vector
\newcommand{\hints}{{\color{blue} \text{hints}}}

\DeclareMathOperator{\arcsec}{arcsec}%反三角函数
\DeclareMathOperator{\arccot}{arccot}
\DeclareMathOperator{\arccsc}{arccsc}

\begin{document}
\title{考研高数习题集}
\author{枫聆}
\maketitle
\tableofcontents


\newpage
\section{极限相关}

\subsection{$1^\infty$类型极限}

\begin{example}
若$\lim \alpha(x)=0,\lim \beta(x)=\infty$,且$\lim \alpha(x)\beta(x) = A$,其中$A$是一个常数,则
$$
\lim \left[ 1 + \alpha(x) \right]^{\beta(x)} = e^A.
$$
\hints\ 带指数形式的表达式,第一想法是把指数拿下来
$$
\lim \left[ 1 + \alpha(x) \right]^{\beta(x)} = \lim e^{\beta(x)\ln(1+\alpha(x))} = \lim e^{\beta(x)\alpha(x)} = e^A.
$$
\end{example}

\begin{example}
求极限
$$
\lim\limits_{x \rightarrow \infty} \left[ \frac{x^2}{(x-a)(x+b)}\right]^x.
$$
\hints
$$
\left[ \frac{x^2}{(x-a)(x+b)}\right]^x = \left(\frac{x}{x-a}\right)^x \cdot \left(\frac{x}{x+b} \right)^x = \left(1+\frac{a}{x-a}\right)^x \cdot \left(1-\frac{b}{x+b}\right)^x = e^{a-b}.
$$
\end{example}

\begin{example}
求极限
$$
\lim\limits_{n \rightarrow \infty} \left( \frac{\sqrt[n]{a} + \sqrt[n]{b} + \sqrt[n]{c}}{3} \right)^n.
$$
\hints\ 往$(1+\alpha(x))^{\beta(x)}$上凑
$$
\left( \frac{\sqrt[n]{a} + \sqrt[n]{b} + \sqrt[n]{c}}{3} \right)^n = \left( 1 + \frac{\sqrt[n]{a} + \sqrt[n]{b} + \sqrt[n]{c}-3}{3}\right)^n
$$
考虑$\alpha(x)\beta(x)$
$$
\frac{(\sqrt[n]{a}-1) + (\sqrt[n]{b}-1) + (\sqrt[n]{c}-1)}{3}\cdot n = \frac{1}{3}\left( \frac{\sqrt[n]{a}-1}{\frac{1}{n}} + \frac{\sqrt[n]{b}-1}{\frac{1}{n}} + \frac{\sqrt[n]{c}-1}{\frac{1}{n}}\right)
$$
\end{example}

\subsection{$1^0$类型极限}

\begin{example}
\rm 若$\lim \alpha(x) = 0, \lim \beta(x)\alpha(x) = 0$,则
$$
(1+\alpha(x))^{\beta(x)} - 1 \sim \alpha(x)\beta(x).
$$
\hints\ 取对数
$$
e^{\beta(x)\ln(1+\alpha(x))}-1 \sim e^{\beta(x)\alpha(x)} -1 \sim \beta(x)\alpha(x). 
$$ 

\end{example}


\newpage
\subsection{夹逼准则应用}

\begin{example}
求极限
$$
\lim\limits_{n \rightarrow \infty}\left( \frac{n}{n^2+1} + \frac{n}{n^2+2} + \cdots + \frac{n}{n^2+n} \right).
$$
\hints
$$
\frac{n^2}{n^2+n} \leq s \leq  \frac{n^2}{n^2+1}.
$$
\end{example}

\begin{example}
求极限
$$
\lim\limits_{n \rightarrow 0^+} x\left[ \frac{1}{x} \right].
$$
\hints
$$
x-1 \leq \left[ x \right] \leq x
$$
\end{example}

\begin{example}\label{motonone-sequence-1}
求极限
$$
\lim\limits_{n \rightarrow \infty} \frac{2^n}{n!}.
$$
\hints
$$
\left(\frac{2}{1}\right)\times\frac{2}{2}\times\frac{2}{3}\times \cdots \times \frac{2}{n}.
$$
\end{example}


\newpage
\subsection{级数相关的极限}

\begin{example} \label{ex:jishu1}
当$\lim\limits_{n \rightarrow \infty} a_n = A$,则
$$
\lim\limits_{n \rightarrow \infty} \frac{a_1+a_2+\cdots+a_n}{n} = A.
$$
\hints\ 直接考察
$$
\left|\frac{a_1+a_2+\cdots+a_n}{n} - A \right| = \left| \frac{(a_1-A)+(a_2-A)+\cdots+(a_n-A)}{n} \right|
$$
{\color{blue}用极限的定义等式右边分成两部分},即对任意的$\varepsilon > 0$,可以找到一个$n_1$,使得$n > n_1$时有$|x_n - A| < \varepsilon$,那么
$$
\begin{array}{ll}
\left|\frac{(a_1-A) + (a_2-A) + \cdots + (a_{n_1}-A)}{n} + \frac{(a_{n_1 + 1}-A)+ (a_{n_1 + 2}-A) + \cdots + (a_{n}-A)}{n} \right| \\ \leq \frac{|a_1-A| + |a_2-A| + \cdots + |a_{n_1}-A|}{n} + \frac{|a_{n_1 + 2}-A|+ |a_{n_1 + 1}-A| + \cdots + |a_{n}-A|}{n}
\end{array}
$$
上述不等式右边第一项,形如$\frac{C}{n}$,因为先对任意$n > n_1$都有上述不等式成立,那么只需要让$n$取的大一点,就能使得$\frac{C}{n} < \varepsilon$({\color{red}阿基米德公理}). 右边第二项显然小于$\frac{n-n_1}{n}\varepsilon$,于是综上
$$
\left|\frac{a_1+a_2+\cdots+a_n}{n} - A \right| < \varepsilon+\frac{n-n_1}{n}\varepsilon < 2\varepsilon.
$$
{\color{blue}如果题目中没有直接给出极限的具体值,我们可以用O.Stolz定理先猜出来,然后用初等方法来验证,再根据极限的唯一性,就得到了答案}. 把$a_n$换成形式,例如
$$
\lim\limits_{n \rightarrow \infty} \frac{1+\sqrt[2]{2}+\cdots+\sqrt[n]{n}}{n} = \lim\limits_{n \rightarrow \infty} \sqrt[n]{n} = 1.
$$ 
\end{example}


\begin{example}
求极限
$$
x_n  = \frac{1^k + 2^k + \cdots + n^k}{n^{k+1}}.
$$
\hints\ 用O.Stolz定理考虑
$$
\lim\limits_{n \rightarrow \infty} \frac{n^k}{n^{k+1} - (n-1)^{k+1}}
$$
分母二项式展开合并极有$\lim \frac{n^k}{(k+1)n^k + \cdots} = \frac{1}{k+1}$. 这道题初等方法似乎不能很好的把握,用和式的方法写出来其实就是黎曼积分的定义
$$
\lim\limits_{n \rightarrow \infty} \frac{1}{n} \sum\limits_{k=1}^k \frac{k}{n} = \int_0^1 x^k = \frac{1}{k+1}. 
$$
{\color{blue} 级数相关的问题往往可以尝试考虑用定积分的思路来解决}. 下面是$1^k + 2^k +\cdots + n^k$的转换思路
$$
{\color{red}
\sum_{i=1}^n i^k=n^{k+1}\frac1n\sum_{i=1}^n \left(\frac in\right)^k\sim_\infty n^{k+1}\int_0^1 x^kdx=\frac{n^{k+1}}{k+1}
}
$$
\end{example}

\newpage
\begin{example}\label{ex:jishu2}
当$\lim\limits_{n \rightarrow \infty} a_n = a, a_n > 0$,则
$$
\lim\limits_{n \rightarrow \infty } \ln \sqrt[n]{a_1a_2\cdots a_n} = \ln a.
$$
\hints\
$$
\ln \sqrt[n]{a_1a_2\cdots a_n} = \frac{\ln a_1 + \ln a_2 + \cdots + \ln a_n}{n} = \ln a.
$$
因为$\ln x$的连续性,所以$\lim \ln a_n = \ln a$,再根据\ref{ex:jishu1}. 
\end{example}

\begin{example}\label{ex:jishu3}
当$\lim\limits_{n \rightarrow \infty} a_n = a, a_n > 0$,则
$$
\lim\limits_{n \rightarrow \infty} \sqrt[n]{a_1a_2\cdots a_n} = a.
$$
\hints\ 取对数再根据\ref{ex:jishu2}
$$
\sqrt[n]{a_1a_2\cdots a_n} = e^{\ln \sqrt[n]{a_1a_2\cdots a_n}} = e^{ln a} = a.
$$
\end{example}

\begin{example}
求极限
$$
\lim\limits_{n \rightarrow \infty}\frac{\sqrt[n]{n!}}{n}.
$$
\hints\ 由 \ref{ex:jishu3} 可知$a_n$和$b_n = \sqrt[n]{a_1a_2\cdots a_n}$的极限是相同的(假设$a_n$的极限存在). 那么有一个推论,对于数列
$$
a_1, \frac{a_2}{a_1}, \frac{a_3}{a_2},\cdots,\frac{a_{n+1}}{a_n},\cdots
$$
则$\lim \sqrt[n]{a_n} = \lim \frac{a_{n+1}}{a_{n}}$,只要等式右边的极限存在就行. 在这里我们只要设$a_n = \frac{n!}{n^n}$即可,那么
$$
\lim \frac{(n+1)!}{(n+1)^{n+1}} \cdot \frac{n^n}{n!} = \lim \frac{n^n}{(n+1)^n} = \frac{1}{(1+\frac{1}{n})^n} = \frac{1}{e}.
$$
\end{example}

\newpage
\subsection{去除根式的尴尬}

%https://math.stackexchange.com/questions/3491441/limit-lim-limits-x-to-infty-sqrtnxa-1xa-2-xa-n-x?noredirect=1&lq=1
\begin{example}
求极限
$$
\lim\limits_{x \rightarrow +\infty} \left[ \sqrt[k]{(x+a_1)(x+a_2)\cdots(x+a_k)} - x\right].
$$
\hints\ 
$$
(x+a_1)(x+a_2)\cdots(x+a_k) = x^k\left(1+\frac{a_1+a_2+\cdots+a_k}{x}+\mathcal{O}\left(\frac{1}{x^2}\right)\right)
$$
那么
$$
x\left(1+\frac{a_1+a_2+\cdots+a_k}{x}+\mathcal{O}\left(\frac{1}{x^2}\right)\right)^{\frac{1}{n}} = x\left(1+ \frac{a_1+a_2+\cdots+a_n}{nx} + \mathcal{O}\left(\frac{1}{x^2}\right)\right) = x + \frac{a_1+a_2+\cdots+a_n}{nx} + \mathcal{O}\left(\frac{1}{x}\right),
$$
这里第一个等号右边对$(1+x)^p$在$x=0$处用了一下{\color{blue}泰勒展开}得到$(1+qx+\mathcal{O}(x^2))$,这个$\mathcal{O}$表示最高次的多项式. 

还有一种{\color{blue}升次}的方法,即下面的恒等式
$$
{\color{red}
y-z = \frac{y^k-z^k}{y^{k-1} + y^{k-2}z + \cdots + z^{k-1}}.
}
$$
这里我们使得$y = \sqrt[k]{(x+a_1)(x+a_2)\cdots(x+a_k)}$及$z = x$,那么原式就变成了
$$
\begin{array}{lll}
= &\frac{(x+a_1)(x+a_2)\cdots(x+a_k) - x^k}{\left[\sqrt[k]{(x+a_1)(x+a_2)\cdots(x+a_k)}\right]^{k-1}+\left[\sqrt[k]{(x+a_1)(x+a_2)\cdots(x+a_k)}\right]^{k-2}x + \cdots + x^{k-1} } \\
= & \frac{a_1+a_2+\cdots + a_k + \mathcal{O}(\frac{1}{x})}{\left[\sqrt[k]{(1+\frac{a_1}{x})(1+\frac{a_2}{x})\cdots(1+\frac{a_k}{x})}\right]^{k-1}+\left[\sqrt[k]{(1+\frac{a_1}{x})(1+\frac{a_2}{x})\cdots(1+\frac{a_k}{x})}\right]^{k-2}x + \cdots + 1} & {\color{blue}\text{上下除以$x^{k-1}$}}
\end{array}
$$
分母中$\sqrt[k]{(1+\frac{a_1}{x})(1+\frac{a_2}{x})\cdots(1+\frac{a_k}{x})}$是趋于$1$的,再用一下函数$x^{\frac{m}{n}}$的连续性,取其函数值也是等于$1$,所以分母就有$k \cdot 1$.
\end{example}

%https://math.stackexchange.com/questions/28348/proof-of-lim-n-to-infty-sqrtnn-1
\begin{example}
求极限
$$
\lim\limits_{n \rightarrow \infty} \sqrt[n]{n} = 1.
$$
\hints\ 取对数应用$e^x$的连续性
$$
\lim e^\frac{\ln n}{n} = e^{\lim \frac{\ln n}{n}} = 1. 
$$
也可以使用一下\ref{prop: bonuli}的伯努利不等式来证明,这里设$\sqrt[n]{n} = 1+h$,那么
$$
\begin{array}{ll}
&n = (1+h)^n = 1+nh + \frac{n(n-1)}{2}h^2 + \cdots \\
\Rightarrow & n \geq \frac{n(n-1)}{2}h^2  \\
\Rightarrow & h^2 \leq \frac{2}{n-1}.
\end{array}
$$
当$n \rightarrow \infty$时,$h \rightarrow 0$,即$\sqrt[n]{n}-1 \rightarrow 0$,所以$\lim \sqrt[n]{n} = 1$.
\end{example}

\newpage
\subsection{换元取极限}

\begin{example}
求极限
$$
\lim\limits_{x \rightarrow 0} \frac{\sqrt[m]{x+1}-1}{x},\; m \in \mathbb{N}.
$$
\hints\ 设$y = \sqrt[m]{x+1}-1$,显然$y$在$x = 0$处连续,所以当$x \rightarrow 0$时有$y \rightarrow 0$,那么此时的极限就变成了
$$
\lim\limits_{y \rightarrow 0} \frac{y}{(y+1)^m-1} = \frac{1}{m}.
$$
这样上下都变成我们熟悉的多项式,分母二项式展开.
\end{example}

\begin{example}
求极限
$$
\lim\limits_{x \rightarrow 0} \frac{(x+1)^{\frac{n}{m}}-1}{x}.
$$
\hints\ 还是使得$y = (x+1)^{\frac{1}{m}}-1$,那么就变成了
$$
\lim\limits_{y \rightarrow 0} \frac{(1+y)^n-1}{(1+y)^m-1} = \lim\limits_{y \rightarrow 0} \frac{(1+y)^n-1}{y}\frac{y}{(1+y)m-1} = \frac{n}{m}. 
$$
\end{example}

\subsection{递归求极限}

\begin{example}
\rm \ref{motonone-sequence-1} 单调数列求极限

\hints\ 考虑递归式
$$
x_{n+1} = x_n \cdot \frac{2}{n+1},
$$
等式两边同时取极限则有
$$
a = a \cdot 0 \Rightarrow a = 0. 
$$
\end{example}

\subsection{等价无穷小的替换}


\subsection{中值定理}

\begin{example}
\rm 求极限
$$
\lim\limits_{x \to +\infty} \frac{1}{2}x^2[\ln\arctan(x+1) - \ln\arctan x]. 
$$
\hints\ 对连续函数$\ln\arctan x$应用中值定理
$$
\lim\limits_{x \to +\infty} \frac{1}{2}x^2 \frac{1}{[1+(\theta+x)^2]\arctan (\theta+x)},
$$
其中$0 < \theta < 1$. 那么即有
$$
\lim\limits_{x \to +\infty} \frac{1}{2} \frac{x^2}{1+(\theta+x)^2} \frac{1}{\arctan (\theta+x)} = \frac{1}{\pi}.
$$
\end{example}

\subsection{含积分的极限}

%https://math.stackexchange.com/questions/4228847/how-to-calculate-lim-x-to-0-frac-int-0x-sqrtx-tetdt-sqrtx3
\begin{example}
\rm 求极限
$$
\lim _{x\to 0^+} \frac{\int _0^x\sqrt{x-t}e^tdt}{\sqrt{x^3}}
$$
\hints\ 这样的含参数积分最好的办法就是洛必达,但是这里首先需要换元一下,令$u = x-t$,则
$$
\int _0^x\sqrt{x-t}e^tdt = \int_{0}^x \sqrt{u}e^{x-u}du = e^x \sqrt{u}e^{-u}du.
$$
再用洛必达
$$
\lim _{x\to 0^+} = \frac{e^x \sqrt{u}e^{-u}du}{x^{\frac{3}{2}}}  = \lim _{x\to 0^+} \frac{( \int_{0}^x \sqrt{u}e^{-u}du)'}{(x^{\frac{3}{2}})'} = \frac{x^{\frac{1}{2}}e^{-x}}{\frac{3}{2}x^{\frac{1}{2}}} = \frac{2}{3}.
$$
\end{example}

\section{导数}

\subsection{导数定义相关的}

\begin{example}
\rm 已知$f'(x_0) = -1$,求
$$
\lim\limits_{x \to 0} \frac{x}{f(x_0-2x)-f(x_0-x)}.
$$
\hints 直觉上就是想办法凑导数的定义出来
$$
\begin{array}{ll}
\lim\limits_{x \to 0} \frac{f(x_0-2x)-f(x_0)}{-2x} = -1 \\
\lim\limits_{x \to 0} \frac{f(x_0-x)-f(x_0)}{-x} = -1\\
\end{array}
$$
求出需要$\lim\limits_{x \to 0} \frac{f(x_0-2x)-f(x_0)}{x}$和$\lim\limits_{x \to 0} \frac{f(x_0-x)-f(x_0)}{x}$,两项相减再取倒. 
\end{example}

\section{不定积分}

\subsection{多项式分式}

\begin{example}
\rm 求
$$
\int \frac{x^4-x^2}{1+x^2}dx.
$$
\hints\ 还是得部分分式
$$
\frac{x^4-x^2}{1+x^2} = \frac{(x^4-1) -(x^2+1) + 2 }{1+x^2}
= x^2 + \frac{2}{1+x^2}-2.
$$
因此原函数为
$$
\frac{x^3}{3} + 2\arctan x - 2x + C,
$$ 
\end{example}

\begin{example}
\rm 求
$$
\int \frac{x+5}{x^2-6x+13}dx.
$$
\hints 观察分子多项式次数小于分母的,且只小一次,所以我们考虑这样部分分式
$$
\frac{1}{2}\int \frac{2x-6}{x^2-6x+13}dx + 8\int \frac{1}{x^2-6x+13}dx = \frac{1}{2}\int \frac{1}{x^2-6x+13}d(x^2-6x+13) + 8\int \frac{1}{4+(x-3)^2}dx,
$$
因此原函数为
$$
\frac{1}{2}\ln(x^2-6x+13) +  4\arctan \frac{x-3}{2} +C.
$$
\end{example}

\begin{example}
\rm 求
$$
\int \frac{x}{x^4+2x^2+5}dx 
$$
\hints\ 观察分子多项式次数小于分母,且小两次,所以我们考虑这样部分分式
$$
\int \frac{x}{4+(x^2+1)^2}dx = \frac{1}{2}\int \frac{1}{4+(x^2+1)^2}d(x^2+1) = \frac{1}{4}\arctan \frac{x^2+1}{2} + C
$$
\end{example}

\subsection{分母带根号}

\begin{example}
\rm 求
$$
\int \frac{dx}{\sqrt{x(4-x)}}.
$$
\hints 根号下凑平方
$$
\int \frac{1}{\sqrt{4-(x-2)^2}}d(x-2) = \arcsin \frac{x-2}{2} +C
$$
\end{example}

\begin{example}
\rm 求
$$
\int \frac{2-x}{\sqrt{3+2x-x^2}}dx.
$$
\hints\ 先分式把分子变成常数$1$
$$
\int \frac{2-x}{\sqrt{3+2x-x^2}}dx = \int \frac{1-x}{\sqrt{3+2x-x^2}} dx + \int \frac{1}{\sqrt{3+2x-x^2}} dx = \frac{1}{2}\int \frac{1}{\sqrt{3+2x-x^2}} d(3+2x-x^2) +  \int \frac{1}{\sqrt{4-(x-1)^2}} dx,
$$
因此原函数为
$$
\sqrt{3+2x-x^2} + \arcsin \frac{x-1}{2} + C
$$
\end{example}

\begin{example}
\rm 求
$$
\int \frac{x^2}{\sqrt{a^2 - x^2}}dx 
$$
\hints\ 考虑第二类换元,令$x=a\sin t$,则
$$
\int \frac{a^2\sin^2 t}{a\cos t} \cdot a\cos t dt = \frac{a^2}{2}\int 1-\cos 2t dt = \frac{a^2 t}{2} - \frac{a^2}{4}\sin 2t.
$$
把$t$变成$x$也有一点技巧,第二项可以变成$\frac{1}{2}(a\sin t)(a \cos t)$,其中$a\sin t = x, a\cos t = \sqrt{a^2 - x^2}$,这样会方便一点
$$
\frac{a^2\arcsin \frac{x}{a}}{2} - \frac{x}{2}\sqrt{a^2 - x^2}+C
$$
\end{example}

\subsection{换元法}

\begin{example}
\rm 求
$$
\int \sqrt{1+e^x} dx
$$
\hints 考虑第二类换元,令$x = \ln (t^2-1)$,则
$$
\int t \cdot \frac{2t}{t^2-1} dt = 2 \int 1 + \frac{1}{t^2-1}dt =2t + \ln |\frac{t-1}{t+1}| + C
$$
带入$t=\sqrt{e^x +1}$,即得
$$
2\sqrt{e^x +1} + \ln \frac{\sqrt{e^x +1}-1}{\sqrt{e^x +1}+1} +C
$$
\end{example}

\subsection{高次}


\subsection{分部积分}

\subsection{三角有理式}

\begin{example}
\rm 求
$$
\int \frac{dx}{\cos x(1+\sin x)}.
$$
\hints\ 这里有一个非常巧妙的第二类换元,令$x=\arcsin u$,则
$$
\int \frac{1}{\sqrt{1-u^2}(1+u)} \frac{1}{\sqrt{1-u^2}}du = \int \frac{1}{(1+u)(1-u^2)}du .
$$
再把有理式拆开,这过程使用待定系数的方法
$$
\int \frac{1}{(1+u)(1-u^2)}du = \frac{1}{2}\int \frac{1}{1-u^2} + \frac{1}{(1+u)^2}du  = -\frac{1}{4} \ln \left| \frac{1-u}{1+u} \right| -\frac{1}{2}\frac{1}{(1+u)}. 
$$
最后即有
$$
-\frac{1}{4} \ln \left|\frac{1-\sin x}{1+\sin x}\right| - \frac{1}{2}\frac{1}{1+\sin x}+C. 
$$
\end{example}

\begin{example}
\rm 求
$$
\int \frac{dx}{\sin x(\sin x + \cos x)}.
$$
\hints\ 考虑第二类换元,令$x = \arccot u$,则有
$$
- \int \frac{1}{\frac{1}{\sqrt{1+u^2}}(\frac{1}{\sqrt{1+u^2}} + \frac{u}{\sqrt{1+u^2}})} \frac{1}{1+u^2} du = - \int \frac{1}{1+u}du = - \ln |u| + C = -\ln|1+\cot x| + C .
$$

\end{example}

\subsection{被积函数含不常见函数形式}

\begin{example}
\rm 求
$$
\int \frac{\arcsin e^x}{e^x}dx. 
$$
\hints\ 必须得想办法吧$\arcsin e^x$提出来,因为我们没有已知原函数导数为反三角的,这里自然地就要使用部分积分了
$$
-\int \arcsin e^x d(e^{-x}) = -\frac{\arcsin e^x}{e^x} + \int e^{-x} \frac{e^x}{\sqrt{1-e^{2x}}}dx = \int \frac{1}{\sqrt{1-e^{2x}}}dx . 
$$
这里令$t=\sqrt{1-e^{2x}}$,那么$x = \frac{\ln(1-t^2)}{2},dx = \frac{-t}{1-t^2}dt$,于是
$$
\int \frac{1}{t} \frac{-t}{1-t^2}dt = \int\frac{1}{t^2-1}dt =  \frac{1}{2}\ln \left| \frac{t-1}{t+1} \right| + C = \frac{1}{2}\ln  \frac{\sqrt{1-e^{2x}}-1}{\sqrt{1-e^{2x}}+1}  + C.  
$$
因此
$$
\int \frac{\arcsin e^x}{e^x}dx = -\frac{\arcsin e^x}{e^x} + \frac{1}{2}\ln  \frac{\sqrt{1-e^{2x}}-1}{\sqrt{1-e^{2x}}+1}  + C  
$$
\end{example}

\begin{example}
\rm 求
$$
\int \ln\left( 1+\sqrt{\frac{1+x}{x}} \right)dx, x>0
$$
\hints 首选分部积分,但是为了为了能部分积分,我们必须先第一类换元,令$t = \sqrt{\frac{1+x}{x}}$,那么$x =\frac{1}{t^2-1}$,于是
$$
\int \ln(1+t)d\left(\frac{1}{t^2-1}\right) = \frac{\ln(1+t)}{{t^2-1}}- \int \frac{1}{(1+t)^2(t-1)},
$$
其中
$$
\int \frac{1}{(1+t)^2(t-1)} = \frac{1}{2}\int \frac{(t+1)-(t-1)}{(1+t)^2(t-1)} = \frac{1}{2} \int \frac{1}{t^2-1} - \frac{1}{(1+t)^2} = \frac{1}{4}\ln\left|\frac{t-1}{t+1} \right| + \frac{1}{2(1+t)} + C. 
$$
因此
$$
\int \ln\left( 1+\sqrt{\frac{1+x}{x}} \right)dx = \frac{\ln(1+t)}{{t^2-1}} + \frac{1}{4}\ln\left|\frac{t-1}{t+1} \right| + \frac{1}{2(1+t)} + C. 
$$
\end{example}

\newpage
\section{定积分}



\subsection{参数积分求导}

\begin{example}
\rm 设$f(x)$连续,求
$$
\frac{d}{dx}\int_0^x tf(x^2-t^2)dt.
$$
\hints\ 对于这种第二类的参数积分,对于有比较简洁的结果的,首先应该换元试试,令$u = x^2 - t^2$,那么即有
$$
-\frac{1}{2}\int_{x^2}^{0} f(u)du = \frac{1}{2}\int_{0}^{x^2} f(u)du 
$$
因此
$$
\frac{1}{2}\frac{d}{dx}\int_{0}^{x^2} f(u)du = xf(x^2). 
$$
\end{example}

\subsection{奇怪的定积分}

\begin{example}
\rm 设$f(x)=\int_0^{\pi} \frac{\sin t}{\pi-t}dt$,求$\int_0^{\pi}f(x)dx$.

\hints\ 可以用分部积分
$$
\int_0^{\pi}f(x)dx =  xf(x)\big\vert_0^{\pi} - \int_{0}^{\pi} xf'(x)dx =  \pi\int_{0}^{\pi} \frac{\sin x}{\pi-x}dx - \int_{0}^{\pi} \frac{\sin x}{\pi-x}dx = \int_{0}^{\pi} \sin x dx = 2. 
$$
\end{example}

\newpage
\section{反常积分}

\subsection{含有$e^x$的被积函数}

\begin{example}
\rm 讨论下述积分的收敛性
$$
\int_a^{+\infty} x^\mu e^{-ax}dx ~ (\mu,a > 0).
$$
\hints 比较审敛法,取任意的$\lambda > 1$,即$\frac{1}{x^\lambda}$是收敛的,于是
$$
\lim\limits_{x \to +\infty} \frac{x^\mu e^{-ax}}{\frac{1}{x^\lambda}} = \frac{x^{u+\lambda}}{e^{ax}} = 0,
$$
因此原无穷积分也是收敛的. 
\end{example}

\begin{example}
\rm 讨论下述积分的收敛性
$$
\int_0^{+\infty} \frac{xdx}{\sqrt{e^{2x}-1}}.
$$
\hints 这里需要注意两个上下积分限都需要考察,我们可以将上述积分划分为
$$
\int_0^{+\infty} \frac{xdx}{\sqrt{e^{2x}-1}} = \int_0^{A} \frac{xdx}{\sqrt{e^{2x}-1}} + \int_A^{+\infty} \frac{xdx}{\sqrt{e^{2x}-1}},
$$
其中$A \in (0,+\infty)$. 当$x \to 0$时,取$0 < \lambda < 1$,于是
$$
\lim\limits_{x \to 0} \frac{\frac{x}{\sqrt{e^{2x}-1}}}{\frac{1}{x}^\lambda} = \frac{x^{1+\lambda}}{\sqrt{e^{2x}-1}} = 0,
$$
即积分$\int_0^{A} \frac{xdx}{\sqrt{e^{2x}-1}}$是收敛的. 当$x \to \infty$时,取$\lambda > 1$,于是
$$
\lim\limits_{x \to \infty} \frac{\frac{x}{\sqrt{e^{2x}-1}}}{\frac{1}{x}^\lambda} =  \frac{1}{\sqrt{e^{2x}\cdot x^{-(2\lambda+2)-x^{-(2\lambda+2)}}}} = 0,
$$
\end{example}

\subsection{待定参数}

\begin{example}
\rm 反常积分
$$
\int_0^{+\infty} \frac{1}{x^a(1+x)^b}dx
$$
收敛,求$a,b$.

\hints\ 这道题还是用柯西审敛法,注意要同时考虑积分上下限. 当$x \to +\infty$,那么就要和$\frac{1}{x^{\lambda}}(\lambda >1)$比较, 于是有
$$
\lim\limits_{x \to \infty} \frac{\frac{1}{x^a(1+x)^b}}{\frac{1}{x^{\lambda}}} = \frac{x^{\lambda-(a+b)}}{(\frac{1}{x}+1)^b},
$$
其中分母是趋于$0$,为保证分子不趋于无穷,则需要$\lambda < (a+b)$,即$a+b > 1$. 当$x \to 0$时,那么就要和$\frac{1}{x^{\lambda}}(\lambda < 1)$比较, 于是有
$$
\lim\limits_{x \to \infty} \frac{\frac{1}{x^a(1+x)^b}}{\frac{1}{x^{\lambda}}} = \frac{1}{x^{a-\lambda}(1+x)^b},
$$
其中$(1+x)^b \to 0$,则$a < \lambda$,即$a < 1$.
\end{example}

\newpage
\section{tricks}

\subsection{一些有趣的不等式}

\begin{proposition}\label{prop: bonuli}
$$
a^{\frac{1}{n}}-1 < \frac{a-1}{n},\; a > 1.
$$
\hints\ {\color{red}伯努利不等式}.
$$
(1+x)^n \leq 1 + nx, \; n \geq 0, x \leq -1.
$$
使得$(1+x) = a^{\frac{1}{n}}$,即可得到上式.
\end{proposition}

\end{document}