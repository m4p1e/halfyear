\documentclass{article}

\usepackage{ctex}
\usepackage{tikz}
\usetikzlibrary{calc}
\usetikzlibrary{cd}
\usetikzlibrary{decorations.pathreplacing}

\usepackage{amsthm}
\usepackage{amsmath}
\usepackage{amssymb}

\usepackage{pgfplots}
\pgfplotsset{compat=newest}

\usepackage{hyperref} %url
\hypersetup{
    colorlinks=true,
    linkcolor=blue,
    filecolor=magenta,      
    urlcolor=cyan,
    pdftitle={Overleaf Example},
    pdfpagemode=FullScreen,
    }

\usepackage{enumitem}

\usepackage[textwidth=18cm]{geometry} % 设置页宽=18

\usepackage{blindtext}
\usepackage{bm}
\parindent=0pt
\setlength{\parindent}{2em} 
\usepackage{indentfirst}


\usepackage{xcolor}
\usepackage{titlesec}
\titleformat{\section}[block]{\color{blue}\Large\bfseries\filcenter}{}{1em}{}
\titleformat{\subsection}[hang]{\color{red}\Large\bfseries}{}{0em}{}
%\setcounter{secnumdepth}{1} %section 序号

\newtheorem{theorem}{Theorem}[section]
\newtheorem{lemma}[theorem]{Lemma}
\newtheorem{corollary}[theorem]{Corollary}
\newtheorem{proposition}[theorem]{Proposition}
\newtheorem{example}[theorem]{Example}
\newtheorem{definition}[theorem]{Definition}
\newtheorem{remark}[theorem]{Remark}
\newtheorem{exercise}{Exercise}[section]
\newtheorem{annotation}[theorem]{Annotation}

\newcommand*{\xfunc}[4]{{#2}\colon{#3}{#1}{#4}}
\newcommand*{\func}[3]{\xfunc{\to}{#1}{#2}{#3}}

\newcommand\Set[2]{\{\,#1\mid#2\,\}} %集合
\newcommand\SET[2]{\Set{#1}{\text{#2}}} %

\newcommand{\norm}[1]{\left\lVert#1\right\rVert} % 范数
\newcommand{\vect}[1]{\mathbf{#1}} % vector
\newcommand{\hints}{{\color{blue} \text{hints}}}

\begin{document}
\title{考研高数习题集}
\author{枫聆}
\maketitle
\tableofcontents


\newpage
\section{极限相关}

\subsection{$1^\infty$类型极限}

\begin{example}
若$\lim \alpha(x)=0,\lim \beta(x)=\infty$,且$\lim \alpha(x)\beta(x) = A$,其中$A$是一个常数,则
$$
\lim \left[ 1 + \alpha(x) \right]^{\beta(x)} = e^A.
$$
\hints\ 带指数形式的表达式,第一想法是把指数拿下来
$$
\lim \left[ 1 + \alpha(x) \right]^{\beta(x)} = \lim e^{\beta(x)\ln(1+\alpha(x))} = \lim e^{\beta(x)\alpha(x)} = e^A.
$$
\end{example}

\begin{example}
求极限
$$
\lim\limits_{x \rightarrow \infty} \left[ \frac{x^2}{(x-a)(x+b)}\right]^x.
$$
\hints
$$
\left[ \frac{x^2}{(x-a)(x+b)}\right]^x = \left(\frac{x}{x-a}\right)^x \cdot \left(\frac{x}{x+b} \right)^x = \left(1+\frac{a}{x-a}\right)^x \cdot \left(1-\frac{b}{x+b}\right)^x = e^{a-b}.
$$
\end{example}

\begin{example}
求极限
$$
\lim\limits_{n \rightarrow \infty} \left( \frac{\sqrt[n]{a} + \sqrt[n]{b} + \sqrt[n]{c}}{3} \right)^n.
$$
\hints\ 往$(1+\alpha(x))^{\beta(x)}$上凑
$$
\left( \frac{\sqrt[n]{a} + \sqrt[n]{b} + \sqrt[n]{c}}{3} \right)^n = \left( 1 + \frac{\sqrt[n]{a} + \sqrt[n]{b} + \sqrt[n]{c}-3}{3}\right)^n
$$
考虑$\alpha(x)\beta(x)$
$$
\frac{(\sqrt[n]{a}-1) + (\sqrt[n]{b}-1) + (\sqrt[n]{c}-1)}{3}\cdot n = \frac{1}{3}\left( \frac{\sqrt[n]{a}-1}{\frac{1}{n}} + \frac{\sqrt[n]{b}-1}{\frac{1}{n}} + \frac{\sqrt[n]{c}-1}{\frac{1}{n}}\right)
$$
\end{example}

\subsection{$1^0$类型极限}

\begin{example}
\rm 若$\lim \alpha(x) = 0, \lim \beta(x)\alpha(x) = 0$,则
$$
(1+\alpha(x))^{\beta(x)} - 1 \sim \alpha(x)\beta(x).
$$
\hints\ 取对数
$$
e^{\beta(x)\ln(1+\alpha(x))}-1 \sim e^{\beta(x)\alpha(x)} -1 \sim \beta(x)\alpha(x). 
$$ 

\end{example}


\newpage
\subsection{夹逼准则应用}

\begin{example}
求极限
$$
\lim\limits_{n \rightarrow \infty}\left( \frac{n}{n^2+1} + \frac{n}{n^2+2} + \cdots + \frac{n}{n^2+n} \right).
$$
\hints
$$
\frac{n^2}{n^2+n} \leq s \leq  \frac{n^2}{n^2+1}.
$$
\end{example}

\begin{example}
求极限
$$
\lim\limits_{n \rightarrow 0^+} x\left[ \frac{1}{x} \right].
$$
\hints
$$
x-1 \leq \left[ x \right] \leq x
$$
\end{example}

\begin{example}\label{motonone-sequence-1}
求极限
$$
\lim\limits_{n \rightarrow \infty} \frac{2^n}{n!}.
$$
\hints
$$
\left(\frac{2}{1}\right)\times\frac{2}{2}\times\frac{2}{3}\times \cdots \times \frac{2}{n}.
$$
\end{example}


\newpage
\subsection{级数相关的极限}

\begin{example} \label{ex:jishu1}
当$\lim\limits_{n \rightarrow \infty} a_n = A$,则
$$
\lim\limits_{n \rightarrow \infty} \frac{a_1+a_2+\cdots+a_n}{n} = A.
$$
\hints\ 直接考察
$$
\left|\frac{a_1+a_2+\cdots+a_n}{n} - A \right| = \left| \frac{(a_1-A)+(a_2-A)+\cdots+(a_n-A)}{n} \right|
$$
{\color{blue}用极限的定义等式右边分成两部分},即对任意的$\varepsilon > 0$,可以找到一个$n_1$,使得$n > n_1$时有$|x_n - A| < \varepsilon$,那么
$$
\begin{array}{ll}
\left|\frac{(a_1-A) + (a_2-A) + \cdots + (a_{n_1}-A)}{n} + \frac{(a_{n_1 + 1}-A)+ (a_{n_1 + 2}-A) + \cdots + (a_{n}-A)}{n} \right| \\ \leq \frac{|a_1-A| + |a_2-A| + \cdots + |a_{n_1}-A|}{n} + \frac{|a_{n_1 + 2}-A|+ |a_{n_1 + 1}-A| + \cdots + |a_{n}-A|}{n}
\end{array}
$$
上述不等式右边第一项,形如$\frac{C}{n}$,因为先对任意$n > n_1$都有上述不等式成立,那么只需要让$n$取的大一点,就能使得$\frac{C}{n} < \varepsilon$({\color{red}阿基米德公理}). 右边第二项显然小于$\frac{n-n_1}{n}\varepsilon$,于是综上
$$
\left|\frac{a_1+a_2+\cdots+a_n}{n} - A \right| < \varepsilon+\frac{n-n_1}{n}\varepsilon < 2\varepsilon.
$$
{\color{blue}如果题目中没有直接给出极限的具体值,我们可以用O.Stolz定理先猜出来,然后用初等方法来验证,再根据极限的唯一性,就得到了答案}. 把$a_n$换成形式,例如
$$
\lim\limits_{n \rightarrow \infty} \frac{1+\sqrt[2]{2}+\cdots+\sqrt[n]{n}}{n} = \lim\limits_{n \rightarrow \infty} \sqrt[n]{n} = 1.
$$ 
\end{example}


\begin{example}
求极限
$$
x_n  = \frac{1^k + 2^k + \cdots + n^k}{n^{k+1}}.
$$
\hints\ 用O.Stolz定理考虑
$$
\lim\limits_{n \rightarrow \infty} \frac{n^k}{n^{k+1} - (n-1)^{k+1}}
$$
分母二项式展开合并极有$\lim \frac{n^k}{(k+1)n^k + \cdots} = \frac{1}{k+1}$. 这道题初等方法似乎不能很好的把握,用和式的方法写出来其实就是黎曼积分的定义
$$
\lim\limits_{n \rightarrow \infty} \frac{1}{n} \sum\limits_{k=1}^k \frac{k}{n} = \int_0^1 x^k = \frac{1}{k+1}. 
$$
{\color{blue} 级数相关的问题往往可以尝试考虑用定积分的思路来解决}. 下面是$1^k + 2^k +\cdots + n^k$的转换思路
$$
{\color{red}
\sum_{i=1}^n i^k=n^{k+1}\frac1n\sum_{i=1}^n \left(\frac in\right)^k\sim_\infty n^{k+1}\int_0^1 x^kdx=\frac{n^{k+1}}{k+1}
}
$$
\end{example}

\newpage
\begin{example}\label{ex:jishu2}
当$\lim\limits_{n \rightarrow \infty} a_n = a, a_n > 0$,则
$$
\lim\limits_{n \rightarrow \infty } \ln \sqrt[n]{a_1a_2\cdots a_n} = \ln a.
$$
\hints\
$$
\ln \sqrt[n]{a_1a_2\cdots a_n} = \frac{\ln a_1 + \ln a_2 + \cdots + \ln a_n}{n} = \ln a.
$$
因为$\ln x$的连续性,所以$\lim \ln a_n = \ln a$,再根据\ref{ex:jishu1}. 
\end{example}

\begin{example}\label{ex:jishu3}
当$\lim\limits_{n \rightarrow \infty} a_n = a, a_n > 0$,则
$$
\lim\limits_{n \rightarrow \infty} \sqrt[n]{a_1a_2\cdots a_n} = a.
$$
\hints\ 取对数再根据\ref{ex:jishu2}
$$
\sqrt[n]{a_1a_2\cdots a_n} = e^{\ln \sqrt[n]{a_1a_2\cdots a_n}} = e^{ln a} = a.
$$
\end{example}

\begin{example}
求极限
$$
\lim\limits_{n \rightarrow \infty}\frac{\sqrt[n]{n!}}{n}.
$$
\hints\ 由 \ref{ex:jishu3} 可知$a_n$和$b_n = \sqrt[n]{a_1a_2\cdots a_n}$的极限是相同的(假设$a_n$的极限存在). 那么有一个推论,对于数列
$$
a_1, \frac{a_2}{a_1}, \frac{a_3}{a_2},\cdots,\frac{a_{n+1}}{a_n},\cdots
$$
则$\lim \sqrt[n]{a_n} = \lim \frac{a_{n+1}}{a_{n}}$,只要等式右边的极限存在就行. 在这里我们只要设$a_n = \frac{n!}{n^n}$即可,那么
$$
\lim \frac{(n+1)!}{(n+1)^{n+1}} \cdot \frac{n^n}{n!} = \lim \frac{n^n}{(n+1)^n} = \frac{1}{(1+\frac{1}{n})^n} = \frac{1}{e}.
$$
\end{example}

\newpage
\subsection{去除根式的尴尬}

%https://math.stackexchange.com/questions/3491441/limit-lim-limits-x-to-infty-sqrtnxa-1xa-2-xa-n-x?noredirect=1&lq=1
\begin{example}
求极限
$$
\lim\limits_{x \rightarrow +\infty} \left[ \sqrt[k]{(x+a_1)(x+a_2)\cdots(x+a_k)} - x\right].
$$
\hints\ 
$$
(x+a_1)(x+a_2)\cdots(x+a_k) = x^k\left(1+\frac{a_1+a_2+\cdots+a_k}{x}+\mathcal{O}\left(\frac{1}{x^2}\right)\right)
$$
那么
$$
x\left(1+\frac{a_1+a_2+\cdots+a_k}{x}+\mathcal{O}\left(\frac{1}{x^2}\right)\right)^{\frac{1}{n}} = x\left(1+ \frac{a_1+a_2+\cdots+a_n}{nx} + \mathcal{O}\left(\frac{1}{x^2}\right)\right) = x + \frac{a_1+a_2+\cdots+a_n}{nx} + \mathcal{O}\left(\frac{1}{x}\right),
$$
这里第一个等号右边对$(1+x)^p$在$x=0$处用了一下{\color{blue}泰勒展开}得到$(1+qx+\mathcal{O}(x^2))$,这个$\mathcal{O}$表示最高次的多项式. 

还有一种{\color{blue}升次}的方法,即下面的恒等式
$$
{\color{red}
y-z = \frac{y^k-z^k}{y^{k-1} + y^{k-2}z + \cdots + z^{k-1}}.
}
$$
这里我们使得$y = \sqrt[k]{(x+a_1)(x+a_2)\cdots(x+a_k)}$及$z = x$,那么原式就变成了
$$
\begin{array}{lll}
= &\frac{(x+a_1)(x+a_2)\cdots(x+a_k) - x^k}{\left[\sqrt[k]{(x+a_1)(x+a_2)\cdots(x+a_k)}\right]^{k-1}+\left[\sqrt[k]{(x+a_1)(x+a_2)\cdots(x+a_k)}\right]^{k-2}x + \cdots + x^{k-1} } \\
= & \frac{a_1+a_2+\cdots + a_k + \mathcal{O}(\frac{1}{x})}{\left[\sqrt[k]{(1+\frac{a_1}{x})(1+\frac{a_2}{x})\cdots(1+\frac{a_k}{x})}\right]^{k-1}+\left[\sqrt[k]{(1+\frac{a_1}{x})(1+\frac{a_2}{x})\cdots(1+\frac{a_k}{x})}\right]^{k-2}x + \cdots + 1} & {\color{blue}\text{上下除以$x^{k-1}$}}
\end{array}
$$
分母中$\sqrt[k]{(1+\frac{a_1}{x})(1+\frac{a_2}{x})\cdots(1+\frac{a_k}{x})}$是趋于$1$的,再用一下函数$x^{\frac{m}{n}}$的连续性,取其函数值也是等于$1$,所以分母就有$k \cdot 1$.
\end{example}

%https://math.stackexchange.com/questions/28348/proof-of-lim-n-to-infty-sqrtnn-1
\begin{example}
求极限
$$
\lim\limits_{n \rightarrow \infty} \sqrt[n]{n} = 1.
$$
\hints\ 取对数应用$e^x$的连续性
$$
\lim e^\frac{\ln n}{n} = e^{\lim \frac{\ln n}{n}} = 1. 
$$
也可以使用一下\ref{prop: bonuli}的伯努利不等式来证明,这里设$\sqrt[n]{n} = 1+h$,那么
$$
\begin{array}{ll}
&n = (1+h)^n = 1+nh + \frac{n(n-1)}{2}h^2 + \cdots \\
\Rightarrow & n \geq \frac{n(n-1)}{2}h^2  \\
\Rightarrow & h^2 \leq \frac{2}{n-1}.
\end{array}
$$
当$n \rightarrow \infty$时,$h \rightarrow 0$,即$\sqrt[n]{n}-1 \rightarrow 0$,所以$\lim \sqrt[n]{n} = 1$.
\end{example}

\newpage
\subsection{换元取极限}

\begin{example}
求极限
$$
\lim\limits_{x \rightarrow 0} \frac{\sqrt[m]{x+1}-1}{x},\; m \in \mathbb{N}.
$$
\hints\ 设$y = \sqrt[m]{x+1}-1$,显然$y$在$x = 0$处连续,所以当$x \rightarrow 0$时有$y \rightarrow 0$,那么此时的极限就变成了
$$
\lim\limits_{y \rightarrow 0} \frac{y}{(y+1)^m-1} = \frac{1}{m}.
$$
这样上下都变成我们熟悉的多项式,分母二项式展开.
\end{example}

\begin{example}
求极限
$$
\lim\limits_{x \rightarrow 0} \frac{(x+1)^{\frac{n}{m}}-1}{x}.
$$
\hints\ 还是使得$y = (x+1)^{\frac{1}{m}}-1$,那么就变成了
$$
\lim\limits_{y \rightarrow 0} \frac{(1+y)^n-1}{(1+y)^m-1} = \lim\limits_{y \rightarrow 0} \frac{(1+y)^n-1}{y}\frac{y}{(1+y)m-1} = \frac{n}{m}. 
$$
\end{example}

\subsection{递归求极限}

\begin{example}
\rm \ref{motonone-sequence-1} 单调数列求极限

\hints\ 考虑递归式
$$
x_{n+1} = x_n \cdot \frac{2}{n+1},
$$
等式两边同时取极限则有
$$
a = a \cdot 0 \Rightarrow a = 0. 
$$
\end{example}

\subsection{等价无穷小的替换}

\newpage
\section{tricks}

\subsection{一些有趣的不等式}

\begin{proposition}\label{prop: bonuli}
$$
a^{\frac{1}{n}}-1 < \frac{a-1}{n},\; a > 1.
$$
\hints\ {\color{red}伯努利不等式}.
$$
(1+x)^n \leq 1 + nx, \; n \geq 0, x \leq -1.
$$
使得$(1+x) = a^{\frac{1}{n}}$,即可得到上式.
\end{proposition}

\end{document}