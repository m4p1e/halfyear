\documentclass{article}

\usepackage{ctex}
\usepackage{tikz}
\usetikzlibrary{cd}
%\usetikzlibrary{paths.ortho} % path折线
%\usetikzlibrary{decorations.pathreplacing}
\usetikzlibrary{calc}
\usetikzlibrary{graphs, graphs.standard, quotes}% quotes library is for the [""] edges
\usetikzlibrary{positioning} %right of 描述位置的

\usepackage{amsthm}
\usepackage{amsmath}
\usepackage{amssymb}
\usepackage{mathrsfs} %花写

%\usepackage{unicode-math}

\usepackage{enumitem}

\usepackage[textwidth=18cm]{geometry} % 设置页宽=18

\usepackage{blindtext}
\usepackage{bm}
\parindent=0pt
\setlength{\parindent}{2em} 
\usepackage{indentfirst}
\usepackage{hyperref} %url
\hypersetup{
    colorlinks=true,
    linkcolor=blue,
    filecolor=magenta,      
    urlcolor=cyan,
    pdftitle={Overleaf Example},
    pdfpagemode=FullScreen,
    }


\usepackage{xcolor}
\usepackage{titlesec}
\titleformat{\section}[block]{\color{blue}\Large\bfseries\filcenter}{}{1em}{}
\titleformat{\subsection}[hang]{\color{red}\Large\bfseries}{}{0em}{}
%\setcounter{secnumdepth}{1} %section 序号

\newtheorem{theorem}{Theorem}[section]
\newtheorem{lemma}[theorem]{Lemma}
\newtheorem{corollary}[theorem]{Corollary}
\newtheorem{proposition}[theorem]{Proposition}
\newtheorem{example}[theorem]{Example}
\newtheorem{definition}[theorem]{Definition}
\newtheorem{remark}[theorem]{Remark}
\newtheorem{exercise}{Exercise}[section]
\newtheorem{annotation}[theorem]{Annotation}

\newcommand*{\xfunc}[4]{{#2}\colon{#3}{#1}{#4}}
\newcommand*{\func}[3]{\xfunc{\to}{#1}{#2}{#3}}

\newcommand\Set[2]{\{\,#1\mid#2\,\}} %集合
\newcommand\SET[2]{\Set{#1}{\text{#2}}} %

\newcommand{\redt}[1]{\textcolor{red}{#1}}
\newcommand{\bluet}[1]{\textcolor{red}{#1}}

\begin{document}
\title{考研概率论}
\author{枫聆}
\maketitle

\tableofcontents

\newpage
\section{随机事件和概率}

\subsection{样本空间与事件}

\begin{definition}
\rm 对随机现象进行观察或者实验被成为{\color{red} 随机试验}当且仅当满足以下条件
\begin{enumerate}
	\item 可以在相同的条件下{\color{red}重复实验};
	\item 所得的可能结果不止一个,且所有可能结果都能{\color{red}事前已知};
	\item 每次具体实验之前{\color{red}无法预知}出现的结果.
\end{enumerate}
\end{definition}

\begin{definition}
\rm {\color{red}随机试验}的每一可能的结果为被称为{\color{red}样本点},所有{\color{red}样本点}构成的集合被称为{\color{red} 样本空间}.
\end{definition}

\begin{definition}
\rm {\color{red} 样本空间}的任一子集被称为{\color{red} 随机事件}. 其中每个单点集被称为{\color{red}基本事件}. 事件$\Omega$被称为{\color{red}必然事件}当且仅当每次试验必有$\Omega$中某一样本点发生. 特别地,把空集$\emptyset$称为{\color{red}不可能事件}.
\end{definition}

\begin{definition}
\rm 若事件$A$的发生{\color{red}必然导致}事件$B$发生,则称事件$B$包含事件$A$,记为$B \supset A$. 若$A \supset B$和$A \subset B$同时成立,则称事件$A$和事件$B$相等,记为$A=B$.
\end{definition}

\begin{definition}
\rm 给定事件$A$和$B$, 它们的交记为$A \cap B$或者$AB$, 表示其所有的公共样本点构成的事件. 这样事件的发生,将导致{\color{red}事件$A$和$B$同时发生}.它们的并记为$A \cup B$,表示它们所有样本点放在一起构成的事件,这样的事件发生将导致{\color{red}至少事件$A$和$B$其中一个发生}.
\end{definition}

\begin{definition}
\rm 给定事件$A$和$B$,若它们的交$AB = \emptyset$,则称事件$A$和$B${\color{red}互斥}或者{\color{red}互不相容}. 若它们的并$A\cup B = \Omega$,且$AB=\emptyset$,则称事件$A$和$B$为{\color{red}对立事件}或者{\color{red}互逆事件},记为$\bar{A} = B$或者$\bar{B} = A$.
\end{definition}

\begin{definition}
\rm 特别地,当$A_1,A_2,\cdots,A_n$两两无不相容时,并称为和,记做
$$
A_1 + A_2 + \cdots + A_n,
$$
或者$\sum\limits_{i=1}^n A_i$. 
\end{definition}

\begin{definition}
\rm 给定事件$A$和$B$,它们的差记为$A-B$,表示事件$A$有而$B$没有的样本点,通俗地来讲表示{\color{red}事件$A$发生而事件$B$不发生}的样本点组成的新事件.
\end{definition}

\begin{proposition}
\rm {\color{red}事件相关的运算法则}
\begin{enumerate}
	\item 交换律 $A \cup B = B \cup A;\;A \cap B = B \cap A$.
	\item 结合律 $(A \cup B) \cup C = A \cup (B \cup C);\; (A \cap B) \cap C = A \cap (B \cap C).$
	\item 分配律 $A \cap (B \cup C) = (A \cap B) \cup (A \cap C);\; A \cup (B \cap C) = (A \cup B) \cap (A \cup C).$
	\item 对偶律 $\overline{A \cup B} = \bar{A} \cap \bar{B};\; \overline{A \cap B} = \bar{A} \cup \bar{B};\; \overline{\bigcup\limits_{i=1}^n A_i} = \bigcap\limits_{i=1}^n \bar{A_i};\; \overline{\bigcap\limits_{i=1}^n A_i} = \bigcup\limits_{i=1}^n \bar{A_i}$ 
\end{enumerate}
\end{proposition}

\subsection{古典概型和几何概率}

\begin{definition}
\rm 当试验的样本空间由$n$个{\color{red}有限}样本点构成,且{\color{red}每个样本点的发生具有相同的可能性},即若事件$A$由$n_A$个样本点组成,则事件$A$对应的概率为
$$
P(A) = \frac{n_A}{n}.
$$
称这样的有限等可能试验中事件$A$的概率$P(A)$为{\color{red}古典型概率}.
\end{definition}


\begin{definition}
\rm 当试验的样本空间是某区域(该区域可以是一维,二维或者三维等等),以$L(\Omega)$表示其几何度量,$L(\Omega)$有限,{\color{red}且试验结果出现在$\Omega$中任何区域的可能性只与该区域几何度量成正比},事件$A$的样本点所表示的区域为$\Omega_A$,则事件$A$的概率为
$$
P(A) = \frac{L(\Omega_A)}{L(\Omega)}.
$$
称这种样本点个数无限但是其几何度量上的等可能试验中事件$A$的概率$P(A)$为{\color{red}几何型概率}.
\end{definition}

\begin{definition}
\rm 把一随机试验独立重复做若干次,即同一事件在各次试验中出现的概率相同. 这个过程称为{\color{red}独立重复试验}.
\end{definition}

\begin{definition}
\rm 如果每次试验只有两个结果$A$和$\bar{A}$,则称这种试验为{\color{red}伯努利试验},将伯努利试验独立重复进行$n$次,称为{\color{red}$n$重伯努利试验}. 设在每次试验中,概率$P(A)=p~(0<p<1)$,则在$n$重伯努利试验中事件$A$发生$k$次的概率为
$$
C_n^kp^k(1-p)^{n-k},\; k=1,2,\cdots,n,
$$
其又称为{\color{red}二项概率公式}.
\end{definition}

\begin{definition}
\rm 将每次伯努利试验结果推广至$A_1,A_2,\cdots,A_r$,而$P(A_i) = p_i, i = 1,2,\cdots,r$,且
$$
p_1 + p_2 + \cdots + p_r = 1, \,p_i \geq 0.
$$
当$r=2$时,即为伯努利试验. 将这种推广的伯努利试验独立重复进行$n$次, 设在$n$次试验中$A_1$出现$k_1$次,$A_2$出现$k_2$,$\cdots$, $A_r$出现$k_r$次的概率为
$$
\frac{n!}{k_1!k_2!\cdots k_r!}p_1^{k_1}p_2^{k_2}\cdots p_r^{k_r},
$$
其中$k_i \geq 0$,且$k_1 + k_2 + \cdots + k_r = n$. 该式被称为{\color{red}多项分布},即它是$(p_1+p_2 + \cdots + p_r)^n$的展开式的一般项. 
\end{definition}

\begin{definition}
\rm {\color{red} 加法原理} 若完成一件事,有$n$类方式,第一类方式有$m_1$种解决方法,第二类方式有$m_2$种解决方式,如此定义下去即第$i$类方法有$m_i$种解决方法. 那么完成这件事就一共有$m_1 + m_2 + \cdots + m_n$种不同的方法. 
\end{definition}

\begin{definition}
\rm {\color{red} 乘法原理} 若完成一件事分成$n$个步骤,其中第$i$步有$m_i$种不同的方法,必须依次完成每一步之后才能进行下一步,那么完成这件事就一共有$m_1m_2\cdots m_n$.
\end{definition}


\begin{definition}
\rm {\color{red} 排列数公式} 从$n$个元素中取$m$个元素出来进行排列,则不同排列的总数为
$$
P^m_n = \frac{n!}{(n-m)!} = n\times(n-1)\times\cdots(n-k+1).
$$
特别地,若$m=n$时,$P^n_n = n!$其被称为{\color{red}全排列};若有放回的取,则不同的排列总数为$n^n$;
\end{definition}

\begin{definition}
\rm {\color{red} 组合数公式} 从$n$个元素中取$m$个元素组成一组,即不管其顺序,则不同的组合总数为
$$
C^m_n = \binom nm  = \frac{n!}{m!(n-m)!} = \frac{P^m_n}{m!}. 
$$
\end{definition}

\begin{proposition}
\rm 如果把$n$个不同的元素分成$k$组($1\leq k \leq n$),使得第$i$组有$n_i$个元素,那么$\sum\limits_{i=1}^k n_i = n$,组内不考虑元素的排列,那么不同的分法总数有
$$
\frac{n!}{n_1!n_2!\cdots n_k!}.
$$
\end{proposition}

\begin{proof}
实际上就是
$$
\begin{array}{ll}
\binom {n}{n_1} \binom{n-n_1}{n_2} \cdots \binom{n-n_1-\cdots-n_{k-1}}{n_k} &= \frac{n!}{n_1!(n-n_1)!} \frac{(n-n_1)!}{n_2!((n-n_1)-n_2)!}\cdots \frac{(n-n_1-\cdots-n_{k-1})!}{n_k!((n-n_1-\cdots-n_{k-1})-n_{k})!} \\
&= \frac{n!}{n_1!n_2!\cdots n_k!}.
\end{array}
$$
\end{proof}

\begin{proposition}
\rm {\color{red} 常用的组合数公式}
$$
\begin{array}{rl}
C^k_n =& C^{n-k}_n \\ \\
C^k_{n+1} =& C^k_n + C^{k-1}_n \\ \\
\sum\limits_{i=0}^n C^i_n =& 2^n \\ \\
C^k_{n+m} =& \sum\limits_{i=0}^k C^i_n C^{k-i}_m \\ \\
\end{array}
$$
\end{proposition}

\begin{proof}
(1) 比较trivial.

(2) 用自然语言来解释,就是说我在$n+1$个元素里面取$k$等价于我先考虑在$n$个元素里面取$k$,然后现在又来了一个新的元素$e$,考虑多出的取法显然要包括这个$e$,那么现在我只要再去原来$n$个元素里面取$k-1$就够了. 这就是第二个等式的含义. 代数证明就略过了...

(3) 直接考虑$(1+1)^n$的展开式就够了.

(4) 实际上也比较trivial,即考虑$k$分别在$m$个元素和$n$元素里面取.
\end{proof}

\newpage
\subsection{概率空间}

\begin{annotation}
\rm 对事件和概率的长期研究,发觉事件的运算与集合运算完全相似,概率与测度有相同的性质,这个事实随着当时在实变函数论中关于勒贝格测度和积分的研究以及一般抽象测度和积分理论的发展而日益明确起来. 

另外,19世纪末以来,数学的各个分支广泛流行着一股公理化潮流,这个流派主张把最基本的假定公理化,其他结论则由它们经过演绎导出.

在这种背景下,1933年,前苏联数学家科尔莫戈罗夫提出了概率论公理化结构,这个结构综合了前人成果,明确定义了基本概率,使概率论成为严谨的数学分支.
\end{annotation}

\begin{definition}
\rm 若把事件的全体记为$\mathscr{F}$,它是由$\Omega$的一些子集构成的子集族. 当$\mathscr{F}$满足以下条件时
\begin{enumerate}
	\item $\Omega \in \mathscr{F}$;
	\item 若$A \in \mathscr{F}$,则$\bar{A} \in \mathscr{F}$;
	\item 若$A_n \in \mathscr{F}, \, n=1,2,\cdots$,则$\bigcup\limits_{n=1}^{\infty} A_n \in \mathscr{F}$.
\end{enumerate}
我们称$\mathscr{F}$是空间$\Omega$上的一个{\color{red}$\sigma$代数}. 
\end{definition}

\begin{definition}
\rm 给定$\Omega$上某个非空集合$A$,通过对$A$里面的元素进行可数并,可数交和取补操作构造的包含$A$本身最小$\sigma$代数称为$A${\color{red}生成}的$\sigma$代数.
\end{definition}

\begin{annotation}
\rm 包含$A$的$\sigma$代数是一定存在, i.e. $\Omega$. 将包含$A$的所有$\sigma$代数取交得到的就是包含$A$的最小$\sigma$代数. 
\end{annotation}

\begin{definition}
\rm 由某个拓扑空间上的一些开集生成的$\sigma$代数成为{\color{red}Borel set}. 特别地,由$\mathcal{R}$上全体的左开右闭(左闭右开)区间构成的集合族所产生的$\sigma$代数成为一维Borel set,记为$\mathscr{F}_1$. 

若$x,y$表示任意实数,由于
$$
\begin{array}{ll}
\{x\} =\bigcap\limits_{n=1}^{\infty}\left[x,x+\frac{1}{n}\right) \\
(x,y) = [x,y) - \{x\} \\
\left[x,y\right] = [x,y) + \{y\} \\
(x,y] = [x,y) + \{y\} - \{x\}
\end{array}
$$
因此$\mathscr{F}_1$中包含一切开区间,闭区间,单个实数,可列个实数,以及由它们经过可列次并,交运算而得出的集合. 
\end{definition}

\begin{definition}
\rm 定义在事件$\mathscr{F}$上的一个实值函数$P$称为{\color{red}概率函数}或者简称为概率,如果它满足如下三个条件(Kolmogorov axioms)
\begin{enumerate}
	\item 对于任意的$A \in \mathscr{F}$,有$P(A) \geq 0$;
	\item $P(\Omega) = 1$;
	\item 对于一个两两不相交的事件可数序列$A_1,A_2,\cdots$,有$P(\bigcup\limits_{i=1}^{\infty} A_i) = \bigcup\limits_{i=1}^{\infty}P(A_i)$成立.({\color{red}可数可加性或者可列可加性})
\end{enumerate}
\end{definition}


\begin{proposition}
\rm {\color{red}概率相关性质}
\begin{enumerate}
	\item $P(\emptyset) = 0$;
	\item 概率具有有限可加性;
	\item $P(\bar{A}) = 1 - P(A)$;
	\item 若$A \subset B$,则$P(B-A) = P(B) - P(A)$;
	\item 若$A\subset B$,则$P(A) \leq P(B)$; (单调性)
	\item $0 \leq P(A) \leq 1$;
\end{enumerate}
\end{proposition}

\begin{proof}
(1) 因为$\Omega = \Omega + \emptyset + \cdots$,所以
$$
P(\Omega) = P(\Omega) + P(\emptyset) + \cdots.
$$
因此$P(\Omega) = 0$.

(2) 对有限互斥事件$A_i,i=1,2,\cdots,n$,因为
$$
A_1 + A_2 + \cdots + A_n = A_1 + A_2 + \cdots + A_n + \emptyset +\emptyset + \cdots,
$$
由可数可加性和性质1有
$$
P(A_1 + A_2 + \cdots + A_n) = P(A_1) + P(A_2) + \cdots + P(A_n). 
$$

(3)
因$A \cup \bar{A} = \Omega$,且$A\bar{A} = \emptyset$,于是由性质3
$$
1 = P(\Omega) = P(A \cup \bar{A}) = P(A) + P(\bar{A}).
$$

(4) 若$A \subset B$,则$A \cup (B-A) = B$,于是
$$
P(B) = P(A \cup (B-A)) = P(A) + P(B-A).
$$
故$P(B-A) = P(B) - P(A)$. 

(5) 性质4的推论,由$P(B-A) \geq 0$,即得$P(B) \geq P(A)$. 

(6) 由性质5可知,对任意的事件$A$,有$A \subset \Omega$,即$P(A) \leq P(\Omega) = 1$. 
\end{proof}

\begin{proposition}
\rm 概率加法公式
$$
P(A \cup B) = P(A) + P(B) - P(AB) 
$$
\end{proposition}

\begin{proof}
因$A \cup B = A \cup (B - AB)$,且$A \cap (B-AB) = \emptyset$,于是由性质2有
$$
P(A \cup B) = P(A \cup (B-AB)) = P(A) + P(B-AB).
$$
因$AB \subset B$,那么用一下性质4,即可得到
$$
P(A \cup B) = P(A) + P(B-AB) = P(A) + P(B) - P(AB).
$$
\end{proof}

\begin{corollary}
\rm 布尔不等式
$$
P(A \cup B) \leq P(A) \cup P(B).
$$
\end{corollary}

\begin{corollary}
\rm Bonferroni不等式
$$
P(AB) \geq P(A) + P(B) -1.
$$
\end{corollary}

\begin{proposition}
\rm 一般加法公式
$$
P(A_1 \cup A_2 \cup \cdots \cup A_n) = \sum\limits_{1,\cdots,n} P(A_i) - \sum\limits_{i < j \atop i,j = 1,\cdots,n}P(A_iA_j) +  \sum\limits_{i < j < k \atop i,j,k = 1,\cdots,n}P(A_iA_jA_k) - \cdots + (-1)^{n-1}P(A_1A_2\cdots A_n). 
$$
\end{proposition}

\begin{proof}
可用数学归纳法证明. 
\end{proof}

\begin{proposition}
\rm 概率减法公式 
$$
P(A - B) = P(A) - P(AB).
$$
\end{proposition}

\begin{proof}
因$A = A(B \cup \bar{B}) = AB \cup A\bar{B} = AB \cup (A-B)$,且$AB \cap (A-B) = \emptyset$,于是
$$
P(A) = P(AB \cup (A-B)) = P(AB) \cup P(A-B). 
$$
移项可得原式. 
\end{proof}

\begin{definition}
\rm 在科尔莫戈洛夫的概率论公理化结构中,称三元组总体$(\Omega,\mathscr{F}, P)$为{\color{red}概率空间},其中$\Omega$是样本空间,$\mathscr{F}$是事件域,$P$是概率. 
\end{definition}



\newpage
\subsection{条件概率和事件独立性}

\begin{definition}
\rm 设$(\Omega,\mathscr{F},P)$是一个概率空间,给定事件$A \in \mathscr{F}$和$B \in \mathscr{F}$,且$P(A) > 0$, 称
$$
P(B|A) = \frac{P(AB)}{P(A)}.
$$
为在事件$A$发生的条件下事件$B$发生的{\color{red}条件概率}(conditional probability).
\end{definition}

\begin{proposition}
\rm 乘法公式(今后出现条件概率$P(B|A)$时,都假定$P(A) > 0$)
$$
P(AB) = P(A)P(B|A);
$$
\end{proposition}

\begin{corollary}
\rm 当$P(A) > 0,P(B) > 0$时,有
$$
P(AB) = P(A|B)P(B) = P(B|A)P(A).
$$
\end{corollary}

\begin{proposition}
\rm 条件概率$P(B|A)$具有概率的三个基本性质
\begin{enumerate}
	\item $P(B|A) \geq 0$ (非负性);
	\item $P(\Omega|A) = 1$ (规范性);
	\item $P(\sum\limits_{i=1}^{\infty} B_i | A) = \sum\limits_{i=1}^{\infty}P(B_i | A)$ (可列可加性); 
\end{enumerate}

利用这三个基本性质也可以导出其他的一些性质
\begin{enumerate}
	\item $P(\emptyset | B) = 0$;
	\item $P(B|A) = 1 - P(\bar{B} | A)$;
	\item $P(B_1 \cup B_2 | A) = P(B_1 | A) + P(B_2 | A) - P(B_1B_2 |A)$. 
\end{enumerate}
\end{proposition}

\begin{proposition}
\rm 乘法公式可以推广到$n$个事件之交
$$
P(A_1A_2\cdots A_n) = P(A_1)P(A_2|A_1)P(A_3|A_1A_2)]\cdots P(A_n|A_1A_2\cdots A_{n-1}).
$$
这里要求$P(A_1A_2\cdots A_{n-1}) > 0$. 
\end{proposition}

\begin{proposition}
\rm {\color{red}全概率公式} 设$B_1,B_2,\cdots,B_n$满足$\bigcup\limits_{i=1}^n B_i = \Omega$,$B_iB_j = \emptyset$且$P(B_k) > 0,\; k=1,2,\cdots,n$,则对任意事件有
$$
P(A) = \sum\limits_{i=1}^n P(B_i)P(A|B_i).
$$
称满足$\bigcup\limits_{i=1}^n B_i = \Omega$,$B_iB_j = \emptyset$的$B_1,B_2,\cdots,B_n$为一个{\color{red}完备事件组}.
\end{proposition}

\begin{proof}
由概率的有限可列可加性有
$$
P(A) = \bigcup\limits_{i=1}^n P(AB_i),
$$
再利用乘法公式即得
$$
P(A) = \bigcup\limits_{i=1}^n P(B_i)P(A|B_i).
$$
\end{proof}

\begin{annotation}
\rm {\color{red}(全概率公式的motivation)} 概率论的重要研究课题之一是希望从已知的简单事件的概率推算出未知的复杂的事件的概率. 为了达到这个目的,经常把一个复杂的事件分解成若干不相容的简单事件之和,再用过分别计算这些简单事件的概率,最后利用概率的可加性得到最终的结果.

\begin{center}
\includegraphics[width=5cm, height=4cm]{images/total_probability.jpg}
\end{center}
\end{annotation}

\begin{definition}
\rm {\color{red}贝叶斯公式} 设$B_1,B_2,\cdots,B_n$满足$\bigcup\limits_{i=1}^n B_i = \Omega$,$B_iB_j = \emptyset$且$P(A)>0, P(B_k) > 0,\; k=1,2,\cdots,n$,则
	$$
		P(B_j | A) = \frac{P(B_j)P(A|B_j)}{\sum\limits_{i=1}^n P(B_i)P(A|B_i)},\; j = 1,2,\cdots,n.
	$$
前提条件可以换成“若事件$A$能且只能与两两互不相容的事件$B_1,B_2,\cdots,B_n$之一同时发生”,这个条件实际上更有实际意义. 其中$P(B_i)$称为先验概率,条件概率$P(B_i | A)$称为后验概率. 
\end{definition}

\begin{proof}
由于$P(A) > 0,P(B_k) > 0$,那么此时有
$$
P(B_kA) = P(A)P(B_k|A) = P(B_k)P(A|B_k). 
$$
故
$$
P(B_j|A) = \frac{P(B_j)P(A|B_j)}{P(A)}.
$$
在利用全概率公式即有
$$
P(B_j|A) = \frac{P(B_j)P(A|B_j)}{\sum\limits_{i =1}^n P(B_i)P(A|B_i)}.
$$
\end{proof}

\begin{annotation}
\rm 贝叶斯公式中先验概率$P(B_i)$反应了各种“原因”发生的可能的性大小,一般是以往的经验总结,在当前试验前就已经知道了. 若当前试验产生了事件$A$,这个信息将有助于探讨事件发生的原因. 后验概率$P(B_i|A)$它反映了试验之后对各种原因发生的可能性大小的新知识.
\end{annotation}

\begin{example}
\rm {\color{red}(贝叶斯决策)} 为了判定一个字母是"C"还是"O",通常采取抽取它的一个特征$X$. 然后再根据这个特征作出判决,这时贝叶斯决策常用的方法之一. 

以$A_1,A_2$分别表示被检验的字母为$C$或者$O$这一事件,它们的先验概率$P(A_1)$及$P(A_2)$应预先给定,此外要通过试验确定$P(X|A_1)$或者$P(X|A_2)$,由贝叶斯公式得
$$
P(A_i|X) = \frac{P(A_i)P(X|A_i)}{\sum\limits_{i=1}P(A_i)P(X|A_i)} 
$$
其中$i=1,2$. 若$P(A_1 | X) > P(A_2 | X)$,则做出决策,具有特征$X$的字母是$C$.
\end{example}

\begin{definition}
\rm 若事件$A,B$满足等式
$$
P(AB) = P(A)P(B),
$$
则称$A$与$B${\color{red}相互独立}. 
\end{definition}

\begin{annotation}
\rm {\color{blue} 从条件概率看两个事件的独立性,也就是其中一个发生的概率是不会影响另一个发生的概率}
\end{annotation}

\begin{corollary}
\rm 若事件$A$与$B$独立,则下列各对立事件也相互独立
$$
\{\overline{A}, B\},\, \{A,\overline{B}\}, \{\overline{A},\overline{B}\}
$$
\end{corollary}

\begin{proof}
由于
$$
\begin{array}{ll}
P(\overline{A}B) &= P(B-AB)=P(B)-P(AB)\\
&= P(B)-P(A)P(B) = P(B)[1-P(A)]\\
&= P(\overline{A})P(B).  
\end{array}
$$
同理可得$\{A,\overline{B}\}$. 特别因为$\overline{A}$和$B$独立,马上可以得到
$$
P(\overline{A}) = P(\overline{A}B \cup \bar{A}\bar{B}) = P(\overline{A})P(B) + P(\bar{A}\bar{B}). 
$$
移项即可得到$P(\bar{A}\bar{B}) = P(\overline{A})P(\overline{B})$. 
\end{proof}

\begin{definition}
\rm 推广至$n$个事件$A_1,\cdots,A_n$相互独立,需要$\binom n2 + \binom n3 + \cdots + \binom nn = 2^n - n -1$等式成立,即设任意的$1<k \leq n$,对任意$1 \leq i_1 \leq \cdots \leq i_k \leq n$满足等式
$$
P(A_{i_1}\cdots A_{i_k})=P(A_{i_1})\cdots P(A_{i_k}).
$$
\end{definition}

\begin{annotation}
\rm {\color{blue}任意事件两两独立并不能推出它们相互独立}.
\end{annotation}

\begin{proposition}
\rm 若$A,B$是两个相互独立的事件,则
$$
P(A-B) = P(A)P(\overline{B}).
$$
\end{proposition}

\begin{proof}
\rm 
$$
P(A-B) = P(A)-P(AB) = P(A)(1-P(B)) = P(A)P(\overline{B}).
$$
\end{proof}

\begin{proposition}
\rm 若$A_1,A_2,\cdots,A_n$是$n$个相互独立的事件,则
$$
P(A_1 \cup A_2 \cup \cdots \cup A_n) = 1-P(\overline{A}_1)P(\overline{A}_2)\cdots P(\overline{A}_n). 
$$
\end{proposition}

\begin{proof}
由于$\overline{A_1\cup A_2 \cup \cdots \cup A_n} = \overline{A}_1\overline{A}_2\cdots\overline{A}_n$,因此
$$
\begin{array}{ll}
P(A_1 \cup A_2 \cup \cdots \cup A_n) &= 1-P(\overline{A}_1\overline{A}_2\cdots\overline{A}_n)\\
&= 1 - P(\overline{A}_1)P(\overline{A}_2)\cdots P(\overline{A}_n).
\end{array}
$$
\end{proof}



\newpage
\subsection{更深的思考和技巧}

\begin{annotation}
\rm {\color{red} 零概率事件$P(A) = 0$与不可能事件$A=\emptyset,P(A) = 0$不是一个概念}.
\end{annotation}


%https://zhuanlan.zhihu.com/p/53736521 独立和互斥
\begin{annotation}
\rm {\color{red} 如何理解事件之间独立性} 事件$A$和$B$独立,则有下面等式成立
$$
P(AB) = P(A)P(B). 
$$
你可能总想深究其更本质的含义,可能你会联想到当事件$A$和$B$独立时,那么$A$和$B$之间有什么关系呢? 实际上并{\color{blue}不能确定$A$和$B$有什么特别地确切的关系},所以你把上述等式理解为在描述一个代数结构就好了,从这个代数结构可以引发一些有趣的性质i.e. 例如化简某些运算{\color{blue}条件概率},所以就特别地把这个代数结构提出来了. 但是当$P(A)>0, P(B) > 0$时,有这样一种关系: {\color{red} 互斥不独立,独立不互斥},证明也是很显然的.
\end{annotation}

\begin{annotation}
\rm {\color{red} 当$P(A) = 0$时,如何计算$P(AB)$?}条件概率的定义中特别指明了$P(A) > 0$,那么$P(A)$等于$0$时候,条件概率$P(B|A)$是未定义的. 那么此时如果我们要计算$P(AB)$应该怎么办呢? 肯定不能使用乘法公式了. 我们可以利用$P$的{\color{blue}单调性},因为$AB \subseteq A$,所以$P(AB) \leq P(A)$,所以$P(AB)=0$.
\end{annotation}

\begin{annotation}
\rm {\color{red} 取部分对立事件不影响独立性} 若$P(AB) = P(A)P(B)$,有下面等式成立
$$
\begin{array}{l}
P(\bar{A}\bar{B}) = P(\bar{A})P(\bar{B})\\
P(\bar{A}B) = P(\bar{A})P(B)\\
P(A\bar{B}) = P(A)P(\bar{B})
\end{array}
$$
我们来证明第一个等式
$$
\begin{array}{rl}
P(AB) &= P(A)P(B) \\
	  &= (1-P(\bar{A}))(1-P(\bar{B}))\\
	  &= 1-P(\bar{A})-P(\bar{B}) + P(\bar{A})P(\bar{B})\\
	  \\
P(\bar{A}) + P(\bar{B}) - P(\bar{A} \cup \bar{B}) &= P(\bar{A})P(\bar{B}) \\
P(\bar{A}\bar{B}) &= P(\bar{A})P(\bar{B})
\end{array}
$$
其余的等式应用类似的手法来证明. 由此在$A$和$B$相互独立的情况下,延伸出来一些有用的等式
$$
\begin{array}{l}
P(A \cup B) = 1 - P(\bar{A}\bar{B}) = 1-P(\bar{A})P(\bar{B})\\
P(A-B) = P(A\bar{B}) = P(A)P(\bar{B}) 
\end{array}
$$
这些等式也可以推广为$n$个事件相互独立.
\end{annotation}


\section{随机变量及其概率分布}

\subsection{随机变量及其分部函数}

\begin{definition}
\rm 在样本空间$\Omega$上的实值函数$X=X(\omega),\, \omega \in \Omega$,称$X(\omega)$为{\color{red}随机变量},简记为$X$. $X$表示样本点到$\mathbb{R}$上一一映射.  
\end{definition}

\begin{annotation}
\rm 随机变量的概念引入是为了通过函数的image(实数)来描述preimage,即某一样本空间中的样本点, i.e. $P(X < 1)$. 随机变量是样本点的函数,因此在试验前我们只能知道它可能取那些值,而不确定它将取何值,这就是随机的含义. 
\begin{center}
\includegraphics[width=5cm, height=4cm]{images/random_variable.jpg}
\end{center}
\end{annotation}

\begin{definition}
\rm 如果一个随机变量的可能取值是有限多个或者可数无穷多个,则称它为{\color{red}离散型随机变量}.
\end{definition}

\begin{annotation}
\rm 离散型随机变量可以理解为$\mathbb{R}$上的点.
\end{annotation}



\begin{definition}
\rm 设离散型随机变量$X$的可能取值是$x_k(k=1,2,\cdots)$,$X$取各可能值的概率为
$$
P\{X=x_k\} = p_k, k=1,2,\cdots,
$$
称上式为离散型随机变量$X$的概率分布或者分部律,其中$P$是一个概率函数.
\end{definition}

\begin{proposition}
\rm 分部律充要条件
\begin{enumerate}
	\item $p_k \geq 0,k=1,2,\cdots$;
	\item $\sum\limits_{k=1}^\infty p_k = 1$.
\end{enumerate}
\end{proposition}

\begin{proof}
{\color{red}(2)} $1 = P[\bigcup\limits_{k=1}^{\infty}\{X= x_k\}] = \sum\limits_{k=1}^{\infty} P\{X=x_k\}$,这里说明了$\{X=x_i\} \cap \{X = x_j\} = \emptyset, i \neq j$.
\end{proof}

\begin{definition}
\rm 设$X$是一个随机变量,对于任意实数$x$,函数
$$
F(x) = P\{X \leq x\},\,-\infty < x < +\infty,
$$
称为随机变量$X$的分布函数(累积分布函数或者cumulative distribution function). 
\end{definition}

\begin{proposition}
\rm {\color{red} 分布函数的本质意义} 给定随机变量$X$的分布函数$F(x)$,则对任意的$x_1 < x_2$,有
$$
P\{x_1 < X \leq x_2\} = F(x_2) - F(x_1).
$$
\end{proposition}

\begin{proof}
$$
P\{x_1 < X \leq x_2\} = P(X \leq x_2) - P(X \leq x_1) = F(x_2)-F(x_1).
$$
\end{proof}

\begin{annotation}
\rm {\color{blue}从上式的证明中我们知道只要对一切实数$x$给出了概率$P\{X \leq x\}$,就能算出$X$落入某个区间$(a,b]$的概率,特别地再利用概率的性质还可以算出$X$属于$\mathbb{R}$上某些相当复杂的点集}. 为了计算概率,必要要求随机变量具有可测性,而分布函数的引进则把对于随机变量的概率计算转换为了对分布函数的数值运算. 
\end{annotation}

\begin{proposition}
\rm 分布函数性质如下
\begin{enumerate}
	\item $F(x)$是单调函数,即当$x_1 < x_2$时,$F(x_1) \leq F(x_2)$ ({\color{red}分部函数充要条件之一});
	\item $0 \leq F(x) \leq 1$; $\lim\limits_{x \rightarrow -\infty}F(x) =F(-\infty) = 0$; $\lim\limits_{x \rightarrow +\infty}F(x) =F(+\infty)= 1$ ({\color{red}分部函数充要条件之一});
	\item $F(x)$是右连续的,即$F(x+0) = F(x)$ ({\color{red}分部函数充要条件之一}); 
	\item 对任意的$x$,有$P\{X = x\} = F(x) - F(x-0)$;  
\end{enumerate}
\end{proposition}

\begin{proof}
\rm  
{\color{red}(1)} 若$x_2 > x_1$,则
$$
F(x_2)-F(x_1) = P\{x_1 < X \leq x_2\} \geq 0.
$$ 

{\color{red}(2)} 因
$$
\begin{array}{ll}
P\{-\infty < X < +\infty\} &= \sum\limits_{n=-\infty}^{+\infty}P\{n \leq X \leq n+1\} \\ 
&= \sum\limits_{n=-\infty}^{+\infty}\left[F(n+1)-F(n)\right]\\
&= \lim\limits_{n \rightarrow +\infty}F(n) - \lim\limits_{m \to -\infty} F(m)= 1 
\end{array}.
$$
上面只是说明了有理数趋于无穷的极限,需要推广任意实数上. 因为$F(x)$的单调性($ F(\lfloor x \rfloor) \leq F(x) \leq F(\lceil x \rceil)$),所以$\lim\limits_{x \rightarrow -\infty}F(x) = \lim\limits_{m \to -\infty} F(m)$,$\lim\limits_{x \rightarrow +\infty}F(x)= \lim\limits_{n \rightarrow +\infty}F(n)$存在. 因为$0\leq F(x) \leq 1$,故
$$
\lim\limits_{x \rightarrow -\infty}F(x) =0 , \lim\limits_{x \rightarrow +\infty}F(x)=1.
$$


{\color{red}(3)} 从(1)(2)可知$F(x)$是单调有界的,若存在间断点,那么只能是第一类间断点,即$F(x)$的任意一点$x_0$处的右极限$F(x_0+0)$是存在的,实际也可以从单调有界来看出从$x_0$右边趋于$x_0$时候$F(x)$极限存在. 现在来证明$F(x+0) = F(x)$. 这里可取一个单调减的数列$x_1 > x_2 > \cdots > x_n > \cdots> x_0$, 即证$\lim\limits_{n \rightarrow \infty} F(x_n) = F(x_0)$. 而
$$
\begin{array}{ll}
F(x_1) - F(x_0) =  P\{x_0 < X \leq x_1\} =  P\left[\bigcup\limits_{i=1}^\infty\{x_{i+1} < X \leq x_{i}\}\right] = \sum\limits_{i=1}^\infty P\{x_{i+1} < X \leq x_i\} \\
= \sum\limits_{i=1}^\infty \left[F(x_i) - F(x_{i+1})\right] =  F(x_1) - \lim\limits_{i \to \infty} F(x_{i+1}),
\end{array}
$$
因此有$F(x_0) = \lim\limits_{i \to \infty} F(x_{i+1}) = F(x+0)$. 注意第二个等号使用了一个重要的极限
$$
\{x_0 < X \leq x_1\} = \lim\limits_{n \rightarrow \infty}\bigcup\limits_{i=1}^n\{x_{i+1} < X \leq x_{i}\}
$$

{\color{red}(4)} 这里因为无法保证是右连续的,所以减去$x$除的右极限的跃度就是这$x$这一点的概率.
\end{proof}

\begin{proposition}
\rm 特殊点和区间的概率分布
$$
\begin{array}{ll}
P\{X = x\} = F(x)-F(x-0) \\
P\{X < x\} = F(x-0) \\
P\{X > x\} = 1-F(x) \\
P\{X \geq x\} = 1-F(x-0) 
\end{array} 
$$
\end{proposition}

\begin{annotation}
\rm {\color{red}上述更贴切的说明了分别函数是一种分析性质良好的函数,给定了分布函数就能计算处各种事件的概率}. 
\end{annotation}

\begin{proposition}
\rm 给定离散型随机变量的分布律$P\{X=x_i\}=p_i,i = 1,2,\cdots$,它的分布函数$F(x)$为
$$
F(x) = P\{X < x\} = \sum\limits_{x_k < x}p(x_k)
$$
\end{proposition}


\begin{definition}
\rm 如果对随机变量$X$的分布函数$F(x)$,存在一个非负可积函数$f(x)$,使得对任意的实数$x$,都有
$$
F(x) = \int_{-\infty}^x f(t)dt, - \infty < x < + \infty,
$$
那么称$X$为{\color{red}连续型随机变量},函数$f(x)$称为$X$的{\color{red}概率密度}. 
\end{definition}

\begin{annotation}
\rm 连续随机变量可以理解为$\mathbb{R}$上的区间. 自然地,事件运算对应了区间集合运算. 
\end{annotation}

\begin{proposition}
\rm 连续型随机变量的分布函数$F(x)$是在$(-\infty,+\infty)$上连续的.
\end{proposition}

\begin{proposition}\label{probability-density-func: prop1}
\rm 概率密度函数$f(x)$的性质如下
\begin{enumerate}
	\item $f(x) \geq 0$ ({\color{red}$f(x)$是概率密度函数的充要条件之一});
	\item 对于任意实数$x$,有$P\{X=x\} = F(x)-F(x-0) = 0$.
	\item $F(+\infty) = \int_{-\infty}^{+\infty} f(t)dt = 1$ ({\color{red}$f(x)$是概率密度函数的充要条件之一});
	\item 对任意实数$x_1 < x_2$,有$P\{x_1 < X \leq x_2 \} = F(x_2)-F(x_1) = \int_{x_1}^{x_2} f(t)dt$;
	\item 在$f(x)$的连续点处有$F'(x) = f(x)$. (证明见高数积分上限函数一节) 
\end{enumerate}
\end{proposition}

\begin{proof}
证明见高数积分上限函数一节
\end{proof}

\begin{annotation}
\rm {\color{blue}上述性质表明,一个事件的概率为0,这件事并不一定是不可能事件; 同样地,一个事件概率等于1,这件事也不一定是必然事件}. 
\end{annotation}

\begin{proposition}
\rm 若$X$是连续型随机变量,则
$$
P\{x_1 < X \leq x_2\} = P\{x_1 \leq  < x_2\} = P\{x_1 < X < x_2\} = P\{x_1 \leq  X \leq x_2\}.
$$
\end{proposition}

\begin{proof}
由proposition \ref{probability-density-func: prop1}中(2)易得. 
\end{proof}


\subsection{常用分布}

\begin{definition}
\rm 若随机变量$X$只取常数$c$,即$P\{X=c\} = 1$,那么这时的分布函数为
$$
F(x) = \left\{\begin{array}{ll}
1, &x \geq c \\
0, &x < c
\end{array} \right. .
$$
则称$X$服从\redt{退化分布},又称\redt{单点分布}. 
\end{definition}

\begin{definition}
\rm 如果随机变量$X$的分布律为
$$
\begin{array}{c|cc}
X & 0 & 1\\
\hline
P & 1-p & p
\end{array}
$$
其中$0 < p < 1$,则称$X$服从参数为$p$的\redt{$0 - 1$分布}或者\redt{两点分布}.
\end{definition}

\begin{definition}
\rm 如果随机变量$X$的分布律为
$$
P\{X = k\} = C_n^kp^kq^{n-k}, k = 0,1,2,\cdots,n,
$$
其中$0 < p < 1, q= 1- p$,则称$X$服从参数为$n,p$的\redt{二项分布},记做$X \sim B(n,p)$.
\end{definition}

\begin{annotation}
\rm 在$n$重伯努利试验中,若每次试验成功率为$p(0 < p < 1)$,则在$n$次独立重复试验中成功的总次数$X$服从二项分布.
\end{annotation}

\begin{proposition}
\rm \redt{二项分布分析性质} 若$X \sim B(n,p)$,当$n$固定时,$P\{X=k\}$先随$k$增加而增大,达到某一极值后又逐渐下降,其在$k=\lfloor(n+1)p\rfloor$取得极值. 
\end{proposition}

\begin{proof}
\rm 我们来考察
$$
\frac{P\{X=k\}}{P\{X=k-1\}} = \frac{C_n^k p}{C_n^{k-1}q} = \frac{(n+1-k)p}{kq} = \frac{(n+1)p-(1-q)k}{kq} = 1+ \frac{(n+1)p - k}{kq}
$$
因此

当$k < (n+1)p$时,$P\{X=k\} > P\{X = k-1\}$;

当$k = (n+1)p$时,$P\{X=k\} = P\{X = k-1\}$;

当$k > (n+1)p$时,$P\{X=k\} < P\{X = k-1\}$;

因为$(n+1)p$不一定是整数,所以存在整数$m$,使得$(n+1)p-1 < m \leq (n+1)p$,即当$k = m$时$P\{X=k\}$达到最大值.  
\end{proof}

\begin{definition}
\rm 如果随机变量$X$的分布律为
$$
P\{X=k\} = pq^{k-1}, k =1,2,\cdots,
$$
其中$0 < p < 1, q= 1- p$,则称$X$服从参数为$p$的\redt{几何分布},或称$X$具有几何分布.
\end{definition}

\begin{annotation}
\rm 在独立地重复做一系列伯努利试验中,若每次试验成功率为$p(0 < p < 1)$,则在第$k$次试验时才首次试验成功的概率服从几何分布.
\end{annotation}

\begin{proposition}
\rm \redt{几何分布具有无记忆性},即若假定前$m$次没有成功,设随机变量$X'$为为了达到首次成功还需要做的试验数,则$X'$依然服从几何分布.
\end{proposition}

\begin{proof}
$$
\begin{array}{ll}
P\{X' = k\} = P\{X=m+k| X > m\} = \frac{P\{X = m+k\}}{P\{X > m\}} = \frac{pq^{m+k-1}}{q^m} = pq^{k-1}. 
\end{array}
$$
\end{proof}

\begin{definition}
\rm 若随机变量$X$的分布律为
$$
P\{X=k\} = C_{k-1}^{r-1}p^rq^{k-r},k=r,r+1,\cdots,
$$
其中$r \geq 1$,则称$X$服从\redt{巴斯卡分布}. 特别地当$r=1$时,就是几何分布. 
\end{definition}

\begin{annotation}
\rm 在独立地重复做一系列伯努利试验中,若每次试验成功率为$p(0 < p < 1)$,那么第$r$次成功时的试验次数的概率服从巴斯卡分布. 若考虑第$r$次成功时的失败试验次数的概率,即
$$
P\{X = k\} = \binom{r+k-1}{r-1}p^{r}q^{k}.
$$
上述概率分布被称为\redt{负二项分布},其中的组合数项可以写作
$$
(-1)^{k}{\frac {(-r)(-r-1)(-r-2)\dotsm (-r-k+1)}{k!}}=(-1)^{k}{\binom {-r}{k}},
$$
那么原式就为
$$
P\{X = k\} = \binom {-r}{k} p^r(-q)^k. 
$$
这就是"负"字的来源. 
\end{annotation}

\begin{definition}
\rm 如果随机变量$X$的分布律为
$$
P\{X=k\} = \frac{C_M^kC_{N-M}^{n-k}}{C_{N}^n}, k=l_1,\cdots,l_2,
$$
其中$l_1 = \max(0,n-N+M), l_2 = \min(M,n)$,则称随机变量$X$服从参数$n,N,M$的超几何分布.
\end{definition}

\begin{annotation}
\rm 如果$N$件产品中含有$M$件次品,从中任意一次取出$n$件,令$X=$抽取的$n$件产品中的次品件数,则$X$服从参数$n,N,M$的\redt{超几何分布}. 
\end{annotation}

\begin{definition}
\rm 如果随机变量$X$的分布律为
$$
P\{X=k\} = \frac{\lambda^k}{k!}e^{-\lambda}, k = 0,1,2,\cdots,
$$
其中$\lambda > 0$为常数,则称随机变量$X$服从参数为$\lambda$的泊松分布,记为$X \sim \P(\lambda)$. 
\end{definition}

\begin{annotation}
\rm 
$$
\sum\limits_{k=0}^{\infty} P\{X =k\} = \sum\limits_{k=0}^{\infty}\frac{\lambda^k}{k!}e^{-\lambda} = e^{-\lambda}\sum\limits_{k=0}^{\infty}\frac{\lambda^k}{k!} = e^{-\lambda}e^{\lambda} = 1,
$$
其中$e^x = \sum\limits_{k=0}^{\infty}\frac{\lambda^k}{k!}$.
\end{annotation}

\begin{theorem}
\rm {\color{red} 泊松定理} 设$\lambda > 0$是一个常数,$n$是任意整数,设$np_n = \lambda$,则对于任一固定的非负整数$k$,有
$$
\lim\limits_{n \rightarrow \infty}C_n^kp_n^k(1-p_n)^{n-k} = \frac{\lambda^ke^{-\lambda}}{k!}.
$$
\end{theorem}

\begin{annotation}
\rm 应用泊松定理来做二项式近似计算要求$n$较大,$p$较少,并且$np$大小适中,则
$$
C_n^kp^k(1-p)^{n-k} \approx \frac{\lambda^ke^{-\lambda}}{k!},
$$
其中$\lambda = np$.
\end{annotation}

\begin{proposition}
\rm 若我们关注的随机变量$X$是表示某个事件$A$发生的次数$k$且$X~P(\lambda)$,假定它具有有下面三个性质
\begin{enumerate}
	\item {\color{red}平稳性} 在$[t_0,t_0+t)$中$A$发生个数只与时间间隔长度有关而与时间起点$t_0$无关. 若以$P_k(t)$表示在长度为$t$的时间区间中$A$发生的$k$次的概率,那么对任意的$t$有
	$$
	\sum\limits_{k=0}^{\infty} P_k(t) = 1
	$$
	成立. 过程的平稳性表示了它的概率规律不随时间的推移而改变.
	\item {\color{red}独立增量性(无后效性)} 在$[t_0,t_0+t)$中$A$发生的$k$次这一事件与时刻$t_0$以前发生的事件独立. 独立增量性表明在互不相交的时间区间内过程进行的相互独立性.
	\item {\color{red}普通性} 在充分小的时间间隔内,$A$最多发生一次. 即,若记
	$$
	\psi(t) = \sum\limits_{k=2}^{\infty} P_k(t) = 1-P_0(t)-P_1(t),
	$$
	那么有
	$$
	\lim\limits_{t \rightarrow 0} \frac{\psi(t)}{t} = 0.
	$$
	换句话说,就是在同一时间瞬时$A$不可能发生$2$次以上. 由上述条件求出来的$P_k(t)$固定$t$就是泊松分布. 
\end{enumerate}
\end{proposition}

\begin{definition}
\rm 如果连续型随机变量$X$的概率密度为
$$
f(x) = \left\{\begin{array}{ll}
\frac{1}{b-a} & a \leq x \leq b\\
0 & \text{other}
\end{array}\right.,
$$
则称$X$在区间$[a,b]$上服从\redt{均匀分布},记做$X \sim U[a,b]$. 其的分布函数为
$$
F(x) = \left\{\begin{array}{ll}
0 & x < a \\
\frac{x-a}{b-a} & a \leq x < b\\
1 & x \geq b
\end{array}\right..
$$
\end{definition}

\begin{proposition}
\rm 设$X \sim U[a,b]$,则对$a \leq c < d \leq b$,有
$$
P\{c\leq x \leq d\} = \frac{d-c}{b-a}.
$$
\end{proposition}

\begin{definition}
\rm 如果连续型随机变量$X$的概率密度为
$$
f(x) = \left\{ \begin{array}{ll}
\lambda e^{-\lambda x} & x > 0 \\
0 & x \leq 0
\end{array}\right.,
$$
其中$\lambda > 0$是个常数,则称$X$服从参数为$\lambda$的\redt{指数分布},记做$X \sim E(\lambda)$. 其分布函数为
$$
F(x) = \left\{\begin{array}{ll}
1 - e^{-\lambda x} & x > 0 \\
0 & x \leq 0
\end{array}\right..
$$
\end{definition}

\begin{proposition}
\rm 设$X \sim E(\lambda)$,则有
\begin{enumerate}
	\item 当$t > 0$时,$P\{X > t\} = \int_t^{+\infty} \lambda e^{-\lambda t}dt$;
	\item 当$t > 0, s >0$时,$P\{X > t+s |x > s\} = \frac{P\{t+s\}}{P\{s\}} = \frac{e^{-\lambda(t+s)}}{e^{-\lambda}(s)} = e^{-\lambda t} = P\{X > t\}$. 此性质称为指数分布具有"无记忆性". 
\end{enumerate}
\end{proposition}

\begin{definition}
\rm 如果连续性随机变量的概率密度为
$$
f(x) = \frac{1}{\sqrt{2\pi}\sigma}e^{-\frac{(x-\mu)^2}{2\sigma^2}},
$$
其中$\mu,\sigma$为常数且$\sigma > 0$,则称$X$服从参数为$\mu,\sigma$的\redt{正态分布},记做$X~N(\mu,\sigma^2)$. 当$\mu=0,\sigma^2 =1$时,即$X~N(0,1)$,称$X$服从\redt{标准正态分布},此时用$\varphi(x)$表示$X$的概率密度,即
$$
\varphi(x) = \frac{1}{\sqrt{2\pi}}e^{-\frac{x^2}{2}}.
$$
当$X \sim N(\mu,\sigma^2)$,其分布函数为
$$
F(x) = \frac{1}{\sqrt{2\pi}\sigma}\int_{-\infty}^{x}e^{-\frac{(t-\mu)^2}{2\sigma^2}}dt. 
$$
当$X~N(0,1)$,其分布函数为
$$
\Phi(x) = \frac{1}{\sqrt{2\pi}}\int_{-\infty}^x e^{-\frac{t^2}{2}}dt.
$$
\end{definition}

\begin{annotation}
\rm \redt{正态分布的图像性质} 正态分布概率密度图像是关于$x = \mu$对称的. 当$\sigma$越小时,分布越集中在$x=a$附近; $\sigma$越大时,分布就越平坦. 若$X$服从$N(\mu,\sigma^2)$,在一次试验中$X$几乎都落在$(\mu-3\sigma,\mu + 3\sigma)$.
\end{annotation}

\begin{proposition}
\rm \redt{正态分布到标准正态分布的转换} 设$X \sim N(\mu,\sigma^2)$,则$Z = \frac{X-\mu}{\sigma} \sim N(0,1)$. 
\end{proposition}

\begin{proof}
\rm $Z=\frac{X-\mu}{\sigma}$的分布函数为
$$
P\{Z \leq x\} = P\{\frac{X-\mu}{\sigma} \leq x\} = P\{X \leq \mu + \sigma x\} =  \frac{1}{\sqrt{2\pi}\sigma}\int_{-\infty}^{\mu+\sigma x}e^{-\frac{(t-\mu)^2}{2\sigma^2}}dt.
$$
令$u = \frac{t-\mu}{\sigma}$,那么$t = \sigma u + \mu$,于是
$$
P\{Z \leq x\} = \frac{1}{\sqrt{2\pi}}\int_{-\infty}^{x} e^{\frac{u^2}{2}}du = \Phi(x). 
$$
由此$Z \sim N(0,1)$.
\end{proof}

\begin{corollary}
\rm 若$X \sim N(\mu,\sigma^2)$,则其分布函数$F(x)$可写成
$$
F(x) = P\{X \leq x \} = P\{ \frac{X-\mu}{\sigma} \leq \frac{x-\mu}{\sigma} \} = \Phi(\frac{x-\mu}{\sigma}). 
$$
\end{corollary}

\begin{proposition}
\rm 设$X~N(\mu,\sigma^2)$,其分布函数为$F(x)$,则
\begin{enumerate}
	\item $P\{a < x \leq b\} = \Phi(\frac{b-\mu}{\sigma}) - \Phi(\frac{a-\mu}{\sigma})$;
	\item 概率函数$f(x)$关于$x=\mu$对称,$\varphi(x)$是偶函数;
	\item $\Phi(-x) = 1-\Phi(x)$,$\Phi(0) = \frac{1}{2}$;
	\item 当$X \sim N(0,1)$且$a > 0$时,$P\{|X| \leq a\} = 1-2\Phi(-a) = 2\Phi(a) - 1$. 
\end{enumerate}
\end{proposition}

\begin{proof}
{\color{red}(4)}
$$
I(a)=\int_0^{\infty}e^{-ax^2}dx =\frac12 \sqrt{\frac{\pi}{a}}
$$
\end{proof}

\subsection{随机变量的函数的分布}

\begin{theorem}
\rm 设随机变量$X$具有概率密度$f_X(x)$对任意实数都有定义,又设函数$g(x)$处处可导且恒有$g'(x) > 0$(或恒有$g'(x) < 0$),则$Y = g(X)$是连续型随机变量,其概率密度为
$$
f_Y(y) = 
$$
\end{theorem}



\end{document}