\documentclass{article}

\usepackage{ctex}
\usepackage{tikz}
\usetikzlibrary{cd}
%\usetikzlibrary{paths.ortho} % path折线
%\usetikzlibrary{decorations.pathreplacing}
\usetikzlibrary{calc}
\usetikzlibrary{graphs, graphs.standard, quotes}% quotes library is for the [""] edges
\usetikzlibrary{positioning} %right of 描述位置的

\usepackage{amsthm}
\usepackage{amsmath}
\usepackage{amssymb}

%\usepackage{unicode-math}

\usepackage{enumitem}

\usepackage[textwidth=18cm]{geometry} % 设置页宽=18

\usepackage{blindtext}
\usepackage{bm}
\parindent=0pt
\setlength{\parindent}{2em} 
\usepackage{indentfirst}



\usepackage{xcolor}
\usepackage{titlesec}
\titleformat{\section}[block]{\color{blue}\Large\bfseries\filcenter}{}{1em}{}
\titleformat{\subsection}[hang]{\color{red}\Large\bfseries}{}{0em}{}
%\setcounter{secnumdepth}{1} %section 序号

\newtheorem{theorem}{Theorem}[section]
\newtheorem{lemma}[theorem]{Lemma}
\newtheorem{corollary}[theorem]{Corollary}
\newtheorem{proposition}[theorem]{Proposition}
\newtheorem{example}[theorem]{Example}
\newtheorem{definition}[theorem]{Definition}
\newtheorem{remark}[theorem]{Remark}
\newtheorem{exercise}{Exercise}[section]

\newcommand*{\xfunc}[4]{{#2}\colon{#3}{#1}{#4}}
\newcommand*{\func}[3]{\xfunc{\to}{#1}{#2}{#3}}

\newcommand\Set[2]{\{\,#1\mid#2\,\}} %集合
\newcommand\SET[2]{\Set{#1}{\text{#2}}} %

\begin{document}
\title{考研概率论}
\author{枫聆}
\maketitle

\tableofcontents

\newpage
\section{随机事件和概率}

\subsection{基本定义}

\begin{definition}
\rm 对随机现象进行观察或者实验被成为{\color{red} 随机试验}当且仅当满足以下条件
\begin{enumerate}
	\item 可以在相同的条件下{\color{red}重复实验};
	\item 所得的可能结果不止一个,且所有可能结果都能{\color{red}事前已知};
	\item 每次具体实验之前{\color{red}无法预知}出现的结果.
\end{enumerate}
\end{definition}

\begin{definition}
\rm {\color{red}随机试验}的每一可能的结果为被称为{\color{red}样本点},所有{\color{red}样本点}构成的集合被称为{\color{red} 样本空间}.
\end{definition}

\begin{definition}
\rm {\color{red} 样本空间}的任一子集被称为{\color{red} 随机事件}. 其中每个单点集被称为{\color{red}基本事件}. 事件$\Omega$被称为{\color{red}必然事件}当且仅当每次试验必有$\Omega$中某一样本点发生. 特别地,把空集$\emptyset$称为{\color{red}不可能事件}.
\end{definition}

\begin{definition}
\rm 若事件$A$的发生{\color{red}必然导致}事件$B$发生,则称事件$B$包含事件$A$,记为$B \supset A$. 若$A \supset B$和$A \subset B$同时成立,则称事件$A$和事件$B$相等,记为$A=B$.
\end{definition}

\begin{definition}
\rm 给定事件$A$和$B$, 它们的交记为$A \cap B$或者$AB$, 表示其所有的公共样本点构成的事件. 这样事件的发生,将导致{\color{red}事件$A$和$B$同时发生}.它们的并记为$A \cup B$,表示它们所有样本点放在一起构成的事件,这样的事件发生将导致{\color{red}至少事件$A$和$B$其中一个发生}.
\end{definition}

\begin{definition}
\rm 给定事件$A$和$B$,若它们的交$AB = \emptyset$,则称事件$A$和$B${\color{red}互斥}或者{\color{red}互不相容}. 若它们的并$A\cup B = \Omega$,且$AB=\emptyset$,则称事件$A$和$B$为{\color{red}对立事件}或者{\color{red}互逆事件},记为$\bar{A} = B$或者$\bar{B} = A$.
\end{definition}

\begin{definition}
\rm 给定事件$A$和$B$,它们的差记为$A-B$,表示事件$A$有而$B$没有的样本点,通俗地来讲表示事件$A$发生而$B$不发生的样本点组成的新事件.
\end{definition}

\begin{definition}
\rm 设试验$E$的样本空间为$\Omega$,real-valued函数$\func{P}{\mathcal{A}}{\mathbb{R}}$被为一个概率函数,其中$\mathcal{A}$被称为{\color{red}输入空间}或者{\color{red}事件空间},即样本空间的{\color{red}幂集}. 当其满足如下条件(Kolmogorov axioms)时
\begin{enumerate}
	\item 对于任意的$A \in \mathcal{A}$,有$P(A) \geq 0$;
	\item $P(\Omega) = 1$;
	\item 对于一个两两不相交的事件可数序列$A_1,A_2,\cdots$,有$P(\bigcup\limits_{i=1}^{\infty} A_i) = \bigcup\limits_{i=1}^{\infty}P(A_i)$成立.
\end{enumerate}
则成$P$是试验$E$的一个{\color{red}概率分布}. {\color{blue} 这个real-value $P$其实看做一个代数形式,其需要满足3个公理.}
\end{definition}

\begin{definition}
\rm 给定事件$A$和$B$,且$P(A) > 0$, 称
$$
P(B|A) = \frac{P(AB)}{P(A)}.
$$
为在事件$A$发生的条件下事件$B$发生的{\color{red}条件概率}.
\end{definition}

\begin{definition}
\rm 若事件$A,B$满足等式
$$
P(AB) = P(A)P(B),
$$
则称$A$与$B${\color{red}相互独立}. {\color{blue} 从条件概率看两个事件的独立性,也就是其中一个发生的概率是不会影响另一个发生的概率}.
\end{definition}

\newpage
\subsection{基本性质和运算法则}

\begin{proposition}
\rm {\color{red}事件相关的运算法则}
\begin{enumerate}
	\item 交换律 $A \cup B = B \cup A;\;A \cap B = B \cap A$.
	\item 结合律 $(A \cup B) \cup C = A \cup (B \cup C);\; (A \cap B) \cap C = A \cap (B \cap C).$
	\item 分配律 $A \cap (B \cup C) = (A \cap B) \cup (A \cap C);\; A \cup (B \cap C) = (A \cup B) \cap (A \cup C).$
	\item 对偶律 $\overline{A \cup B} = \bar{A} \cap \bar{B};\; \overline{A \cap B} = \bar{A} \cup \bar{B};\; \overline{\bigcup\limits_{i=1}^n A_i} = \bigcap\limits_{i=1}^n \bar{A_i};\; \overline{\bigcap\limits_{i=1}^n A_i} = \bigcup\limits_{i=1}^n \bar{A_i}$ 
\end{enumerate}
\end{proposition}


\begin{proposition}
\rm {\color{red}概率分布相关性质}
\begin{enumerate}
	\item $P(\emptyset) = 0$;
	\item $P(\bar{A}) = 1 - P(A)$;
	\item $A\subset B$,则$P(A) \leq P(B)$;
	\item $0 \leq P(A) \leq 1$;
\end{enumerate}
\end{proposition}

\begin{proof}
(2) 因为$P(\Omega) = P(A\bar{A}) =P(A) + P(\bar{A})$.

(1) 可以马上通过(2)直接得到.

(3) $P$满足{\color{blue}单调性},可以根据Kolmogorov axioms(3)很自然地可以得到.

(4) $P(\emptyset) \leq P(A) \leq P(\Omega).$
\end{proof}

\begin{proposition}
\rm 五大概率公式
\begin{enumerate}
	\item {\color{red}加法公式} $P(A \cup B) = P(A) + P(B) - P(AB)$.
	\item {\color{red}减法公式} $P(A - B) = P(A) - P(AB)$.
	\item {\color{red}乘法公式} 当$P(A) > 0$时,$P(AB) = P(A)P(B|A)$;
	
	当$P(A_1A_2\cdots A_{n-1}) > 0$时,$P(A_1A_2\cdots A_n) = P(A_1)P(A_2|A_1)\cdots P(A_n|A_1A_2\cdots A_{n-1}).$
	\item {\color{red}全概率公式} 设$B_1,B_2,\cdots,B_n$满足$\bigcup\limits_{i=1}^n B_i = \Omega$,$B_iB_j = \emptyset$且$P(B_k) > 0,\; k=1,2,\cdots,n$,则对任意事件有
	$$
		P(A) = \sum\limits_{i=1}^n P(B_i)P(A|B_i).
	$$
	其中称满足$\bigcup\limits_{i=1}^n B_i = \Omega$,$B_iB_j = \emptyset$的$B_1,B_2,\cdots,B_n$为一个{\color{red}完备事件组}. 通常把$P(B_1), P(B_2),\cdots,P(B_n)$叫做{\color{red}先验概率}. {\color{blue}全概率公式的意义在于可以将复杂的事件$A$划分为简单互斥事件$AB_1,AB_2,\cdots,AB_n$,再结乘法公式计算出$A$的概率}.
	\item {\color{red}贝叶斯公式} 设$B_1,B_2,\cdots,B_n$满足$\bigcup\limits_{i=1}^n B_i = \Omega$,$B_iB_j = \emptyset$且$P(A)>0, P(B_k) > 0,\; k=1,2,\cdots,n$,则
	$$
		P(B_j | A) = \frac{P(B_j)P(A|B_j)}{\sum\limits_{i=1}^n P(B_i)P(A|B_i)},\; j = 1,2,\cdots,n.
	$$
\end{enumerate}
\end{proposition}

\begin{proof}
(2) $P(A) = P(AB \cup A\bar{B}) = P(AB)+P(A-B).$

(1) $P(A \cup B) = P((A-AB) \cup (B-AB) \cup AB)$, 再根据(2)有
$$
\begin{array}{ll}
P((A-AB) \cup (B-AB) \cup AB) &= P(A)-P(AB) + P(B) - P(AB) + P(AB) \\
&= P(A) + P(B) -P(AB).
\end{array}
$$
{\color{blue} 实际上这里还是要分类讨论一下当$A=B$的时候.}

(3) 条件概率的另一种写法.

(4) $P(A) = P(\bigcup\limits_{i=1}^n AB_i) = \bigcup\limits_{i=1}^n P(AB_i)$,再用(3)替换一下即可.

(5) $P(B_j | A) =  \frac{P(B_jA)}{P(A)}$,用(3)和(4)分别替换分子和分母即可.

\end{proof}
\end{document}