\documentclass{article}

\usepackage{ctex}
\usepackage{tikz}
\usetikzlibrary{cd}
%\usetikzlibrary{paths.ortho} % path折线
%\usetikzlibrary{decorations.pathreplacing}
\usetikzlibrary{calc}
\usetikzlibrary{graphs, graphs.standard, quotes}% quotes library is for the [""] edges
\usetikzlibrary{positioning} %right of 描述位置的

\usepackage{amsthm}
\usepackage{amsmath}
\usepackage{amssymb}

%\usepackage{unicode-math}

\usepackage{enumitem}

\usepackage[textwidth=18cm]{geometry} % 设置页宽=18

\usepackage{blindtext}
\usepackage{bm}
\parindent=0pt
\setlength{\parindent}{2em} 
\usepackage{indentfirst}
\usepackage{hyperref} %url
\hypersetup{
    colorlinks=true,
    linkcolor=blue,
    filecolor=magenta,      
    urlcolor=cyan,
    pdftitle={Overleaf Example},
    pdfpagemode=FullScreen,
    }


\usepackage{xcolor}
\usepackage{titlesec}
\titleformat{\section}[block]{\color{blue}\Large\bfseries\filcenter}{}{1em}{}
\titleformat{\subsection}[hang]{\color{red}\Large\bfseries}{}{0em}{}
%\setcounter{secnumdepth}{1} %section 序号

\newtheorem{theorem}{Theorem}[section]
\newtheorem{lemma}[theorem]{Lemma}
\newtheorem{corollary}[theorem]{Corollary}
\newtheorem{proposition}[theorem]{Proposition}
\newtheorem{example}[theorem]{Example}
\newtheorem{definition}[theorem]{Definition}
\newtheorem{remark}[theorem]{Remark}
\newtheorem{exercise}{Exercise}[section]
\newtheorem{annotation}[theorem]{Annotation}

\newcommand*{\xfunc}[4]{{#2}\colon{#3}{#1}{#4}}
\newcommand*{\func}[3]{\xfunc{\to}{#1}{#2}{#3}}

\newcommand\Set[2]{\{\,#1\mid#2\,\}} %集合
\newcommand\SET[2]{\Set{#1}{\text{#2}}} %

\begin{document}
\title{考研概率论}
\author{枫聆}
\maketitle

\tableofcontents

\newpage
\section{随机事件和概率}

\subsection{基本定义}

\begin{definition}
\rm 对随机现象进行观察或者实验被成为{\color{red} 随机试验}当且仅当满足以下条件
\begin{enumerate}
	\item 可以在相同的条件下{\color{red}重复实验};
	\item 所得的可能结果不止一个,且所有可能结果都能{\color{red}事前已知};
	\item 每次具体实验之前{\color{red}无法预知}出现的结果.
\end{enumerate}
\end{definition}

\begin{definition}
\rm {\color{red}随机试验}的每一可能的结果为被称为{\color{red}样本点},所有{\color{red}样本点}构成的集合被称为{\color{red} 样本空间}.
\end{definition}

\begin{definition}
\rm {\color{red} 样本空间}的任一子集被称为{\color{red} 随机事件}. 其中每个单点集被称为{\color{red}基本事件}. 事件$\Omega$被称为{\color{red}必然事件}当且仅当每次试验必有$\Omega$中某一样本点发生. 特别地,把空集$\emptyset$称为{\color{red}不可能事件}.
\end{definition}

\begin{definition}
\rm 若事件$A$的发生{\color{red}必然导致}事件$B$发生,则称事件$B$包含事件$A$,记为$B \supset A$. 若$A \supset B$和$A \subset B$同时成立,则称事件$A$和事件$B$相等,记为$A=B$.
\end{definition}

\begin{definition}
\rm 给定事件$A$和$B$, 它们的交记为$A \cap B$或者$AB$, 表示其所有的公共样本点构成的事件. 这样事件的发生,将导致{\color{red}事件$A$和$B$同时发生}.它们的并记为$A \cup B$,表示它们所有样本点放在一起构成的事件,这样的事件发生将导致{\color{red}至少事件$A$和$B$其中一个发生}.
\end{definition}

\begin{definition}
\rm 给定事件$A$和$B$,若它们的交$AB = \emptyset$,则称事件$A$和$B${\color{red}互斥}或者{\color{red}互不相容}. 若它们的并$A\cup B = \Omega$,且$AB=\emptyset$,则称事件$A$和$B$为{\color{red}对立事件}或者{\color{red}互逆事件},记为$\bar{A} = B$或者$\bar{B} = A$.
\end{definition}

\begin{definition}
\rm 给定事件$A$和$B$,它们的差记为$A-B$,表示事件$A$有而$B$没有的样本点,通俗地来讲表示{\color{red}事件$A$发生而事件$B$不发生}的样本点组成的新事件.
\end{definition}

\begin{definition}
\rm 设试验$E$的样本空间为$\Omega$,real-valued函数$\func{P}{\mathcal{A}}{\mathbb{R}}$被为一个概率函数,其中$\mathcal{A}$被称为{\color{red}输入空间}或者{\color{red}事件空间},即样本空间的{\color{red}幂集}. 当其满足如下条件(Kolmogorov axioms)时
\begin{enumerate}
	\item 对于任意的$A \in \mathcal{A}$,有$P(A) \geq 0$;
	\item $P(\Omega) = 1$;
	\item 对于一个两两不相交的事件可数序列$A_1,A_2,\cdots$,有$P(\bigcup\limits_{i=1}^{\infty} A_i) = \bigcup\limits_{i=1}^{\infty}P(A_i)$成立.
\end{enumerate}

{\color{blue} 这个real-value $P$其实看做一个代数形式,其需要满足3个公理.}
\end{definition}

\begin{definition}
\rm 给定事件$A$和$B$,且$P(A) > 0$, 称
$$
P(B|A) = \frac{P(AB)}{P(A)}.
$$
为在事件$A$发生的条件下事件$B$发生的{\color{red}条件概率}.
\end{definition}

\begin{definition}
\rm 若事件$A,B$满足等式
$$
P(AB) = P(A)P(B),
$$
则称$A$与$B${\color{red}相互独立}. {\color{blue} 从条件概率看两个事件的独立性,也就是其中一个发生的概率是不会影响另一个发生的概率}. 推广至$n$个事件$A_1,\cdots,A_n$相互独立,需要$\binom n2 + \binom n3 + \cdots + \binom nn = 2^n - n -1$等式成立,即设任意的$1<k \leq n$,对任意$1 \leq i_1 \leq \cdots \leq i_k \leq n$满足等式
$$
P(A_{i_1}\cdots A_{i_k})=P(A_{i_1})\cdots P(A_{i_k}).
$$
\end{definition}

\begin{definition}
\rm 当试验的样本空间由$n$个{\color{red}有限}样本点构成,且{\color{red}每个样本点的发生具有相同的可能性},即若事件$A$由$n_A$个样本点组成,则事件$A$对应的概率为
$$
P(A) = \frac{n_A}{n}.
$$
称这样的有限等可能试验中事件$A$的概率$P(A)$为{\color{red}古典型概率}.
\end{definition}


\begin{definition}
\rm 当试验的样本空间是某区域(该区域可以是一维,二维或者三维等等),以$L(\Omega)$表示其几何度量,$L(\Omega)$有限,{\color{red}且试验结果出现在$\Omega$中任何区域的可能性只与该区域几何度量成正比},事件$A$的样本点所表示的区域为$\Omega_A$,则事件$A$的概率为
$$
P(A) = \frac{L(\Omega_A)}{L(\Omega)}.
$$
称这种样本点个数无限但是其几何度量上的等可能试验中事件$A$的概率$P(A)$为{\color{red}几何型概率}.
\end{definition}

\begin{definition}
\rm 把一随机试验独立重复做若干次,即同一事件在各次试验中出现的概率相同. 这个过程称为{\color{red}独立重复试验}.
\end{definition}

\begin{definition}
\rm 如果每次试验只有两个结果$A$和$\bar{A}$,则称这种试验为{\color{red}伯努利试验},将伯努利试验独立重复进行$n$次,称为{\color{red}$n$重伯努利试验}. 设在每次试验中,概率$P(A)=p~(0<p<1)$,则在$n$重伯努利试验中事件$A$发生$k$次的概率为
$$
C_n^kp^k(1-p)^{n-k},\; k=1,2,\cdots,n,
$$
其又称为{\color{red}二项概率公式}.
\end{definition}

\newpage
\subsection{基本性质和运算法则}

\begin{proposition}
\rm {\color{red}事件相关的运算法则}
\begin{enumerate}
	\item 交换律 $A \cup B = B \cup A;\;A \cap B = B \cap A$.
	\item 结合律 $(A \cup B) \cup C = A \cup (B \cup C);\; (A \cap B) \cap C = A \cap (B \cap C).$
	\item 分配律 $A \cap (B \cup C) = (A \cap B) \cup (A \cap C);\; A \cup (B \cap C) = (A \cup B) \cap (A \cup C).$
	\item 对偶律 $\overline{A \cup B} = \bar{A} \cap \bar{B};\; \overline{A \cap B} = \bar{A} \cup \bar{B};\; \overline{\bigcup\limits_{i=1}^n A_i} = \bigcap\limits_{i=1}^n \bar{A_i};\; \overline{\bigcap\limits_{i=1}^n A_i} = \bigcup\limits_{i=1}^n \bar{A_i}$ 
\end{enumerate}
\end{proposition}


\begin{proposition}
\rm {\color{red}概率分布相关性质}
\begin{enumerate}
	\item $P(\emptyset) = 0$;
	\item $P(\bar{A}) = 1 - P(A)$;
	\item $A\subset B$,则$P(A) \leq P(B)$;
	\item $0 \leq P(A) \leq 1$;
\end{enumerate}
\end{proposition}

\begin{proof}
(2) 因为$P(\Omega) = P(A\bar{A}) =P(A) + P(\bar{A})$.

(1) 可以马上通过(2)直接得到.

(3) $P$满足{\color{blue}单调性},可以根据Kolmogorov axioms(3)很自然地可以得到.

(4) $P(\emptyset) \leq P(A) \leq P(\Omega).$
\end{proof}


%https://zhuanlan.zhihu.com/p/78297343 全概率公式和贝叶斯公式的相关lectures
\begin{proposition}
\rm 五大概率公式
\begin{enumerate}
	\item {\color{red}加法公式} 
	$$
	\begin{array}{c}
	P(A \cup B) = P(A) + P(B) - P(AB);\\
	P(A \cup B \cup C) = P(A) + P(B) +P(C) -P(AB) - P(AC) -P(BC) {\color{red}+}P(ABC).
	\end{array}
	$$
	\item {\color{red}减法公式} 
	$$
	P(A - B) = P(A) - P(AB).
	$$
	\item {\color{red}乘法公式} 当$P(A) > 0$时,
	$$
	P(AB) = P(A)P(B|A);
	$$
	当$P(A_1A_2\cdots A_{n-1}) > 0$时,
	$$
	P(A_1A_2\cdots A_n) = P(A_1)P(A_2|A_1)\cdots P(A_n|A_1A_2\cdots A_{n-1}).
	$$
	\item {\color{red}全概率公式} 设$B_1,B_2,\cdots,B_n$满足$\bigcup\limits_{i=1}^n B_i = \Omega$,$B_iB_j = \emptyset$且$P(B_k) > 0,\; k=1,2,\cdots,n$,则对任意事件有
	$$
		P(A) = \sum\limits_{i=1}^n P(B_i)P(A|B_i).
	$$
	其中称满足$\bigcup\limits_{i=1}^n B_i = \Omega$,$B_iB_j = \emptyset$的$B_1,B_2,\cdots,B_n$为一个{\color{red}完备事件组}. 通常把$P(B_1), P(B_2),\cdots,P(B_n)$叫做{\color{red}先验概率}. {\color{blue}全概率公式的意义在于可以将复杂的事件$A$划分为简单互斥事件$AB_1,AB_2,\cdots,AB_n$,再结乘法公式计算出$A$的概率}.
	\item {\color{red}贝叶斯公式} 设$B_1,B_2,\cdots,B_n$满足$\bigcup\limits_{i=1}^n B_i = \Omega$,$B_iB_j = \emptyset$且$P(A)>0, P(B_k) > 0,\; k=1,2,\cdots,n$,则
	$$
		P(B_j | A) = \frac{P(B_j)P(A|B_j)}{\sum\limits_{i=1}^n P(B_i)P(A|B_i)},\; j = 1,2,\cdots,n.
	$$
	贝叶斯公式的意义在于{\color{blue}在事件$A$已经发生的条件下,贝叶斯公式可以用来寻找导致$A$发生各种“原因”$B_i$的概率}. 其中$P(B_j|A)$被称为{\color{red}后验概率}.
\end{enumerate}
\end{proposition}

\begin{proof}
(2) $P(A) = P(AB \cup A\bar{B}) = P(AB)+P(A-B).$

(1) $P(A \cup B) = P((A-AB) \cup (B-AB) \cup AB)$, 再根据(2)有
$$
\begin{array}{ll}
P((A-AB) \cup (B-AB) \cup AB) &= P(A)-P(AB) + P(B) - P(AB) + P(AB) \\
&= P(A) + P(B) -P(AB).
\end{array}
$$
{\color{blue} 实际上这里还是要分类讨论一下当$A=B$的时候.}

(3) 条件概率的另一种写法.

(4) $P(A) = P(\bigcup\limits_{i=1}^n AB_i) = \bigcup\limits_{i=1}^n P(AB_i)$,再用(3)替换一下即可.

(5) $P(B_j | A) =  \frac{P(B_jA)}{P(A)}$,用(3)和(4)分别替换分子和分母即可.
\end{proof}


{\color{blue}下面都是一些高中学过的排列组合的性质}.

\begin{definition}
\rm {\color{red} 加法原理} 若完成一件事,有$n$类方式,第一类方式有$m_1$种解决方法,第二类方式有$m_2$种解决方式,如此定义下去即第$i$类方法有$m_i$种解决方法. 那么完成这件事就一共有$m_1 + m_2 + \cdots + m_n$种不同的方法. 
\end{definition}

\begin{definition}
\rm {\color{red} 乘法原理} 若完成一件事分成$n$个步骤,其中第$i$步有$m_i$种不同的方法,必须依次完成每一步之后才能进行下一步,那么完成这件事就一共有$m_1m_2\cdots m_n$.
\end{definition}

\begin{definition}
\rm {\color{red} 排列数公式} 从$n$个元素中取$m$个元素出来进行排列,则不同排列的总数为
$$
P^m_n = \frac{n!}{(n-m)!} = n\times(n-1)\times\cdots(n-k+1).
$$
特别地,若$m=n$时,$P^n_n = n!$其被称为{\color{red}全排列};若有放回的取,则不同的排列总数为$n^n$;
\end{definition}

\begin{definition}
\rm {\color{red} 组合数公式} 从$n$个元素中取$m$个元素组成一组,即不管其顺序,则不同的组合总数为
$$
C^m_n = \binom nm  = \frac{n!}{m!(n-m)!} = \frac{P^m_n}{m!}. 
$$
\end{definition}

\begin{proposition}
\rm 如果把$n$个不同的元素分成$k$组($1\leq k \leq n$),使得第$i$组有$n_i$个元素,那么$\sum\limits_{i=1}^k n_i = n$,组内不考虑元素的排列,那么不同的分法总数有
$$
\frac{n!}{n_1!n_2!\cdots n_k!}.
$$
\end{proposition}

\begin{proof}
实际上就是
$$
\begin{array}{ll}
\binom {n}{n_1} \binom{n-n_1}{n_2} \cdots \binom{n-n_1-\cdots-n_{k-1}}{n_k} &= \frac{n!}{n_1!(n-n_1)!} \frac{(n-n_1)!}{n_2!((n-n_1)-n_2)!}\cdots \frac{(n-n_1-\cdots-n_{k-1})!}{n_k!((n-n_1-\cdots-n_{k-1})-n_{k})!} \\
&= \frac{n!}{n_1!n_2\cdots n_k!}.
\end{array}
$$
\end{proof}

\begin{proposition}
\rm {\color{red} 常用的组合数公式}
$$
\begin{array}{rl}
C^k_n &= C^{n-k}_n \\ \\
C^k_{n+1} &= C^k_n + C^{k-1}_n \\ \\
\sum\limits_{i=0}^n C^i_n &= 2^n \\ \\
C^k_{n+m} &= \sum\limits_{i=0}^k C^i_n C^{k-i}_m \\ \\
\end{array}
$$
\end{proposition}

\begin{proof}
(1) 比较trivial.

(2) 用自然语言来解释,就是说我在$n+1$个元素里面取$k$等价于我先考虑在$n$个元素里面取$k$,然后现在又来了一个新的元素$e$,考虑多出的取法显然要包括这个$e$,那么现在我只要再去原来$n$个元素里面取$k-1$就够了. 这就是第二个等式的含义. 代数证明就略过了...

(3) 直接考虑$(1+1)^n$的展开式就够了.

(4) 实际上也比较trivial,即考虑$k$分别在$m$个元素和$n$元素里面取.
\end{proof}

\newpage
\subsection{更深的思考和技巧}

\begin{annotation}
\rm {\color{red} 零概率事件$P(A) = 0$与不可能事件$A=\emptyset,P(A) = 0$不是一个概念}.
\end{annotation}


%https://zhuanlan.zhihu.com/p/53736521 独立和互斥
\begin{annotation}
\rm {\color{red} 如何理解事件之间独立性} 事件$A$和$B$独立,则有下面等式成立
$$
P(AB) = P(A)P(B). 
$$
你可能总想深究其更本质的含义,可能你会联想到当事件$A$和$B$独立时,那么$A$和$B$之间有什么关系呢? 实际上并{\color{blue}不能确定$A$和$B$有什么特别地确切的关系},所以你把上述等式理解为在描述一个代数结构就好了,从这个代数结构可以引发一些有趣的性质i.e. 例如化简某些运算{\color{blue}条件概率},所以就特别地把这个代数结构提出来了. 但是当$P(A)>0, P(B) > 0$时,有这样一种关系: {\color{red} 互斥不独立,独立不互斥},证明也是很显然的.
\end{annotation}

\begin{annotation}
\rm {\color{red} 当$P(A) = 0$时,如何计算$P(AB)$?}条件概率的定义中特别指明了$P(A) > 0$,那么$P(A)$等于$0$时候,条件概率$P(B|A)$是未定义的. 那么此时如果我们要计算$P(AB)$应该怎么办呢? 肯定不能使用乘法公式了. 我们可以利用$P$的{\color{blue}单调性},因为$AB \subseteq A$,所以$P(AB) \leq P(A)$,所以$P(AB)=0$.
\end{annotation}

\begin{annotation}
\rm {\color{red} 取部分对立事件不影响独立性} 若$P(AB) = P(A)P(B)$,有下面等式成立
$$
\begin{array}{l}
P(\bar{A}\bar{B}) = P(\bar{A})P(\bar{B})\\
P(\bar{A}B) = P(\bar{A})P(B)\\
P(A\bar{B}) = P(A)P(\bar{B})
\end{array}
$$
我们来证明第一个等式
$$
\begin{array}{rl}
P(AB) &= P(A)P(B) \\
	  &= (1-P(\bar{A}))(1-P(\bar{B}))\\
	  &= 1-P(\bar{A})-P(\bar{B}) + P(\bar{A})P(\bar{B})\\
	  \\
P(\bar{A}) + P(\bar{B}) - P(\bar{A} \cup \bar{B}) &= P(\bar{A})P(\bar{B}) \\
P(\bar{A}\bar{B}) &= P(\bar{A})P(\bar{B})
\end{array}
$$
其余的等式应用类似的手法来证明. 由此在$A$和$B$相互独立的情况下,延伸出来一些有用的等式
$$
\begin{array}{l}
P(A \cup B) = 1 - P(\bar{A}\bar{B}) = 1-P(\bar{A})P(\bar{B})\\
P(A-B) = P(A\bar{B}) = P(A)P(\bar{B}) 
\end{array}
$$
这些等式也可以推广为$n$个事件相互独立.
\end{annotation}


\section{随机变量及其概率分布}

\subsection{随机变量及其分部函数}

\begin{definition}
\rm 在样本空间$\Omega$上的实值函数$X=X(\omega),\, \omega \in \Omega$,称$X(\omega)$为{\color{red}随机变量},简记为$X$.  
\end{definition}

\begin{annotation}
\rm 随机变量的概念引入是为了通过函数的image(实数)来描述preimage,即某一样本空间中的样本点, i.e. $P(X < 1)$. 
\end{annotation}

\begin{definition}
\rm 如果一个随机变量的可能取值是有限多个或者可数无穷多个,则称它为{\color{red}离散型随机变量}.
\end{definition}

\begin{definition}
\rm 设离散型随机变量$X$的可能取值是$x_k(k=1,2,\cdots)$,$X$取各可能值的概率为
$$
P\{X=x_k\} = p_k, k=1,2,\cdots,
$$
称上式为离散型随机变量$X$的概率分部或者分部律,其中$P$是一个概率函数.
\end{definition}

\begin{proposition}
\rm 分部律的性质如下
\begin{enumerate}
	\item $p_k \geq 0,k=1,2,\cdots$;
	\item $\sum\limits_{k=1}^\infty p_k = 1$.
\end{enumerate}
\end{proposition}

\begin{proof}
{\color{red}(2)} $1 = P[\bigcup\limits_{k=1}^{\infty}\{X= x_k\}] = \sum\limits_{k=1}^{\infty} P\{X=x_k\}$,这里说明了$\{X=x_i\} \cap \{X = x_j\} = \emptyset, i \neq j$.
\end{proof}

\begin{definition}
\rm 设$X$是一个随机变量,对于任意实数$x$,函数
$$
F(x) = P\{X \leq x\},\,-\infty < x < +\infty,
$$
称为随机变量$X$的分布函数(累积分布函数或者cumulative distribution function). 
\end{definition}

\begin{proposition}
\rm 分布函数性质如下
\begin{enumerate}
	\item $0 \leq F(x) \leq 1$; $\lim\limits_{x \rightarrow -\infty}F(x) =F(-\infty) = 0$; $\lim\limits_{x \rightarrow +\infty}F(x) =F(+\infty)= 1$;
	\item $F(x)$是单调函数,即当$x_1 < x_2$时,$F(x_1) \leq F(x_2)$;
	\item $F(x)$是右连续的,即$F(x+0) = F(x)$; 
	\item 对任意的$x_1 < x_2$,有$P\{x_1 < X \leq x_2\} = F(x_2) - F(x_1)$;
	\item 对任意的$x$,有$P\{X = x\} = F(x) - F(x-0)$;  
\end{enumerate}
\end{proposition}

\begin{proof}
\rm  {\color{red}(3)} 从(1)(2)可知$F(x)$是单调有界的,若存在间断点,那么只能是第一类间断点,即$F(x)$的任意一点$x_0$处的右极限$F(x_0+0)$是存在的. 现在来证明$F(x+0) = F(x)$. 这里需要用一下函数极限用数列的表示方法"若$f(x)$在$x_0$处有极限当且仅当任意极限$\lim x_n = x_0$的数列其对应函数值极限$\lim f(x_n)$存在且相等". 这里取一个单调减的数列$x_1 > x_2 > \cdots > x_n > \cdots> x_0$, 即证$\lim\limits_{n \rightarrow \infty} F(x_n) = F(x_0)$. 而
$$
\begin{array}{ll}
F(x_1) - F(x_0) =  P\{x_0 < X \leq x_1\} =  P\left[\bigcup\limits_{i=1}^\infty\{x_{i+1} < X \leq x_{i}\}\right] = \sum\limits_{i=1}^\infty P\{x_{i+1} < X \leq x_i\} \\
= \sum\limits_{i=1}^\infty F(x_i) - F(x_{i+1}) =  F(x_1) - \sum\limits_{i=1}^\infty F(x_{i+1}),
\end{array}
$$
因此有$F(x_0) = \sum\limits_{i=1}^\infty F(x_{i+1}) = F(x+0)$. 注意第二个等号使用了一个重要的极限
$$
\{x_0 < X \leq x_1\} = \lim\limits_{n \rightarrow \infty}\bigcup\limits_{i=1}^n\{x_{i+1} < X \leq x_{i}\}
$$

{\color{red}(5)} 这里因为无法保证是右连续的,所以减去$x$除的右极限的跃度就是这$x$这一点的概率.
\end{proof}

\begin{definition}
\rm 如果对随机变量$X$的分布函数$F(x)$,存在一个非负可积函数$f(x)$,使得对任意的实数$x$,都有
$$
F(x) = \int_{-\infty}^x f(t)dt, - \infty < x < + \infty,
$$
那么称$X$为{\color{red}连续型随机变量},函数$f(x)$称为$X$的{\color{red}概率密度}. 
\end{definition}

\begin{proposition}
\rm 连续型随机变量的分布函数$F(x)$是在$(-\infty,+\infty)$上连续的.
\end{proposition}

\begin{proposition}\label{probability-density-func: prop1}
\rm 概率密度函数$f(x)$的性质如下
\begin{enumerate}
	\item $f(x) \geq 0$ ({\color{red}$f(x)$是概率密度函数的充要条件之一});
	\item 对于任意实数$x$,有$P\{X=x\} = F(x)-F(x-0) = 0$.
	\item $F(+\infty) = \int_{-\infty}^{+\infty} f(t)dt = 1$ ({\color{red}$f(x)$是概率密度函数的充要条件之一});
	\item 对任意实数$x_1 < x_2$,有$P\{x_1 < X \leq x_2 \} = F(x_2)-F(x_1) = \int_{x_1}^{x_2} f(t)dt$;
	\item 在$f(x)$的连续点处有$F'(x) = f(x)$. (证明见高数积分上限函数一节) 
\end{enumerate}
\end{proposition}

\begin{proof}
证明见高数积分上限函数一节
\end{proof}

\begin{proposition}
\rm 若$X$是连续型随机变量,则
$$
P\{x_1 < X \leq x_2\} = P\{x_1 \leq  < x_2\} = P\{x_1 < X < x_2\} = P\{x_1 \leq  X \leq x_2\}.
$$
\end{proposition}

\begin{proof}
由proposition \ref{probability-density-func: prop1}中(2)易得. 
\end{proof}


\subsection{常用分布}

\begin{definition}
\rm 如果随机变量$X$的分布律为
$$
\begin{array}{c|cc}
X & 0 & 1\\
\hline
P & 1-p & p
\end{array}
$$
其中$0 < p < 1$,则称$X$服从参数为$p$的$0 - 1$分布或者两点分布.
\end{definition}

\begin{definition}
\rm 如果随机变量$X$的分布律为
$$
P\{X = k\} = C_n^kp^kq^{n-k}, k = 0,1,2,\cdots,n,
$$
其中$0 < p < 1, q= 1- p$,则称$X$服从参数为$n,p$的二项分布,记做$X \sim B(n,p)$.
\end{definition}

\begin{annotation}
\rm 在$n$重伯努利试验中,若每次试验成功率为$p(0 < p < 1)$,则在$n$次独立重复试验中成功的总次数$X$服从二项分布.
\end{annotation}

\begin{definition}
\rm 如果随机变量$X$的分布律为
$$
P\{X=k\} = pq^{k-1}, k =1,2,\cdots,
$$
其中$0 < p < 1, q= 1- p$,则称$X$服从参数为$p$的几何分布,或称$X$具有几何分布.
\end{definition}

\begin{annotation}
\rm 在独立地重复做一系列伯努利试验中,若每次试验成功率为$p(0 < p < 1)$,则在第$k$次试验时才首次试验成功的概率服从几何分布.
\end{annotation}

\begin{definition}
\rm 如果随机变量$X$的分布律为
$$
P\{X=k\} = \frac{C_M^kC_{N-M}^{n-k}}{C_{N}^n}, k=l_1,\cdots,l_2,
$$
其中$l_1 = \max(0,n-N+M), l_2 = \min(M,n)$,则称随机变量$X$服从参数$n,N,M$的超几何分布.
\end{definition}

\begin{annotation}
\rm 如果$N$件产品中含有$M$件次品,从中任意一次取出$n$件,令$X=$抽取的$n$件产品中的次品件数,则$X$服从参数$n,N,M$的超几何分布. 
\end{annotation}

\begin{definition}
\rm 如果随机变量$X$的分布律为
$$
P\{X=k\} = \frac{\lambda^k}{k!}e^{-\lambda}, k = 0,1,2,\cdots,
$$
其中$\lambda > 0$为常数,则称随机变量$X$服从参数为$\lambda$的泊松分布. 
\end{definition}

\end{document}