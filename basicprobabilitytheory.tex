\documentclass{article}

\usepackage{ctex}
\usepackage{tikz}
\usetikzlibrary{cd}
%\usetikzlibrary{paths.ortho} % path折线
%\usetikzlibrary{decorations.pathreplacing}
\usetikzlibrary{calc}
\usetikzlibrary{graphs, graphs.standard, quotes}% quotes library is for the [""] edges
\usetikzlibrary{positioning} %right of 描述位置的

\usepackage{amsthm}
\usepackage{amsmath}
\usepackage{amssymb}

%\usepackage{unicode-math}

\usepackage{enumitem}

\usepackage[textwidth=18cm]{geometry} % 设置页宽=18

\usepackage{blindtext}
\usepackage{bm}
\parindent=0pt
\setlength{\parindent}{2em} 
\usepackage{indentfirst}



\usepackage{xcolor}
\usepackage{titlesec}
\titleformat{\section}[block]{\color{blue}\Large\bfseries\filcenter}{}{1em}{}
\titleformat{\subsection}[hang]{\color{red}\Large\bfseries}{}{0em}{}
%\setcounter{secnumdepth}{1} %section 序号

\newtheorem{theorem}{Theorem}[section]
\newtheorem{lemma}[theorem]{Lemma}
\newtheorem{corollary}[theorem]{Corollary}
\newtheorem{proposition}[theorem]{Proposition}
\newtheorem{example}[theorem]{Example}
\newtheorem{definition}[theorem]{Definition}
\newtheorem{remark}[theorem]{Remark}
\newtheorem{exercise}{Exercise}[section]

\newcommand*{\xfunc}[4]{{#2}\colon{#3}{#1}{#4}}
\newcommand*{\func}[3]{\xfunc{\to}{#1}{#2}{#3}}

\newcommand\Set[2]{\{\,#1\mid#2\,\}} %集合
\newcommand\SET[2]{\Set{#1}{\text{#2}}} %

\begin{document}
\title{考研概率论}
\author{枫聆}
\maketitle

\tableofcontents

\newpage
\section{随机事件和概率}

\subsection{基本定义}

\begin{definition}
\rm 对随机现象进行观察或者实验被成为{\color{red} 随机试验}当且仅当满足以下条件
\begin{enumerate}
	\item 可以在相同的条件下{\color{red}重复实验};
	\item 所得的可能结果不止一个,且所有可能结果都能{\color{red}事前已知};
	\item 每次具体实验之前{\color{red}无法预知}出现的结果.
\end{enumerate}
\end{definition}

\begin{definition}
\rm {\color{red}随机试验}的每一可能的结果为被称为{\color{red}样本点},所有{\color{red}样本点}构成的集合被称为{\color{red} 样本空间}.
\end{definition}

\begin{definition}
\rm {\color{red} 样本空间}的任一子集被称为{\color{red} 随机事件}. 其中每个单点集被称为{\color{red}基本事件}. 事件$\Omega$被称为{\color{red}必然事件}当且仅当每次试验必有$\Omega$中某一样本点发生. 特别地,把空集$\emptyset$称为{\color{red}不可能事件}.
\end{definition}

\begin{definition}
\rm 
\end{definition}

\end{document}