\documentclass{article}

\usepackage{ctex}
\usepackage{tikz}
\usetikzlibrary{cd}
%\usetikzlibrary{paths.ortho} % path折线
%\usetikzlibrary{decorations.pathreplacing}
\usetikzlibrary{calc}
\usetikzlibrary{graphs, graphs.standard, quotes}% quotes library is for the [""] edges
\usetikzlibrary{positioning} %right of 描述位置的

\usepackage{amsthm}
\usepackage{amsmath}
\usepackage{amssymb}
\usepackage{mathrsfs} %花写

%\usepackage{unicode-math}

\usepackage{enumitem}

\usepackage[textwidth=18cm]{geometry} % 设置页宽=18

\usepackage{blindtext}
\usepackage{bm}
\parindent=0pt
\setlength{\parindent}{2em} 
\usepackage{indentfirst}
\usepackage{hyperref} %url
\hypersetup{
    colorlinks=true,
    linkcolor=blue,
    filecolor=magenta,      
    urlcolor=cyan,
    pdftitle={Overleaf Example},
    pdfpagemode=FullScreen,
    }


\usepackage{xcolor}
\usepackage{titlesec}
\titleformat{\section}[block]{\color{blue}\Large\bfseries\filcenter}{}{1em}{}
\titleformat{\subsection}[hang]{\color{red}\Large\bfseries}{}{0em}{}
%\setcounter{secnumdepth}{1} %section 序号

\counterwithin*{equation}{section} %equation重新编号
\counterwithin*{equation}{subsection}

\newtheorem{theorem}{Theorem}[section]
\newtheorem{lemma}[theorem]{Lemma}
\newtheorem{corollary}[theorem]{Corollary}
\newtheorem{proposition}[theorem]{Proposition}
\newtheorem{example}[theorem]{Example}
\newtheorem{definition}[theorem]{Definition}
\newtheorem{remark}[theorem]{Remark}
\newtheorem{exercise}{Exercise}[section]
\newtheorem{annotation}[theorem]{Annotation}

\newcommand*{\xfunc}[4]{{#2}\colon{#3}{#1}{#4}}
\newcommand*{\func}[3]{\xfunc{\to}{#1}{#2}{#3}}

\newcommand\Set[2]{\{\,#1\mid#2\,\}} %集合
\newcommand\SET[2]{\Set{#1}{\text{#2}}} %

\newcommand{\redt}[1]{\textcolor{red}{#1}}
\newcommand{\bluet}[1]{\textcolor{blue}{#1}}

\newcommand{\hints}{{\color{blue} \text{hints}}}

\begin{document}
\title{考研概率论}
\author{枫聆}
\maketitle

\tableofcontents

\newpage
\section{概率运算}

\subsection{翻译事件要准确}

\begin{example}
\rm 某种产品由自动生产线进行生成,一旦出现不合格品就立即对其进行调整,经过调整后生产出的产品为不合格的概率为$0.1$,求两次调整之间至少产生$3$件产品的概率. 

\hints\ 设$A_i = \{\text{一次调整之后生产的第$i$件为次品}\}, B=\{\text{两次调整之间至少产生$3$件产品}\}$,那么
$$
P(B) =  1- P(A_1) - P(A_2) = 1-0.1-0.1*0.9 = 0.81. 
$$
这里有个迷惑的地方,题目说的产品也包括不合格产品. 
\end{example}

\subsection{贝叶斯的应用}

\begin{example}
\rm 假设有两箱同种零件: 第一箱内装有$50$件,其中$10$件一等品; 第二箱内装有$30$件,其中$18$件一等品. 现从两箱中随意挑选一箱,然后从箱中随机取两个零件,试求在第一次取出的零件是一等品的条件下,第二次取出一等品的概率.

\hints\ 设事件$A$为选择第一个箱子,事件$B_1$为第一次取出一等品,事件$B_2$为第二次取出一等品. 这里要求的是一个条件概率$P(B_2|B_1)$,首先我们用贝叶斯公式分别计算$P(A|B_1)$和$P(\bar{A}|B_1)$,即
$$
P(A|B_1) = \frac{P(A)P(B_1|A)}{P(A)P(B_1|A) + P(\bar{A})P(B_1|\bar{A})} = \frac{\frac{10}{50}}{\frac{10}{50} + \frac{18}{30}} = \frac{1}{4},
$$ 
因此$P(\bar{A}|B_1) = \frac{3}{4}$. 于是
$$
P(B_2|B_1) = P(B_2|AB_1)P(A|B_1) + P(B_2|\bar{A}B_1)P(\bar{A}|B_1) = \frac{9}{49}\times\frac{1}{4} + \frac{17}{29} \times \frac{3}{4}
$$
\end{example}

\newpage
\section{常用分布}

\subsection{参数确定}

\begin{example}
\rm 设随机变量$X \sim U(a,b)$,已知$P\{-2< X < 0\} = \frac{1}{4}$和$P\{1<X<3\} = \frac{1}{2}$,求$a,b$. 

\hints\ 由已知条件,我们知道$[a,b]$和区间$[-2,0]$及$[1,3]$都是有重叠部分的,因此
$$
\left\{
\begin{array}{ll}
a < 0 \\
b > 1  
\end{array} \right.
$$ 
由此我们知道$[0,1]$完全躺在$[a,b]$里面的. (1 考虑由于$P\{1< X < 3\} =\frac{1}{2}$,那么
$$
P\{-2 \leq X \leq 1\} \leq \frac{1}{2}  \Rightarrow  P\{0 \leq X \leq 1\} \leq \frac{1}{4}.
$$
考虑$|[0,1]| = \frac{1}{2}|[1,2]|$,那么
$$
P\{0 \leq X \leq 1\} \geq \frac{1}{4}. 
$$
综上$P\{0 \leq X \leq 1 \} = \frac{1}{4}$. 因此$a = -1, b=3$. 
\end{example}


\section{随机变量函数}

\subsection{连续性判定}

\begin{example}
\rm 设随机变量$X$与$Y$相互独立,$X$服从参数为$\lambda$的指数分布,$Y$的分布律为$P\{Y=-1\}=\frac{1}{2},P\{Y=1\}=\frac{1}{2}$. 判定$Z=X+Y$的分布函数$F_Z(z)$的连续性.

\hints\ 求出$F_Z(z)$来判断是下下策! 这里要结合分布函数的性质来做就比较简单,如果$F_Z(z)$有间断点$a$,那么它是左间断的,即$P\{Z=a\} = F_Z(a) - F_Z(a-0) > 0$. 因此我们来求$P\{Z=a\}$,
$$
P\{Z=a\} = P\{X = a+1\}P\{Y = -1\} + P\{X = a-1\}P\{Y = 1\} = \frac{1}{2}\left[ P\{X = a+1\} + P\{X = a-1\} \right] = 0    
$$
最后一个等式成立条件是因为$X$是连续的. 
\end{example}

\newpage
\section{正态分布}


\subsection{线性运算}

\begin{example}
\rm 设$X_1,X_2$是两个独立的正态分布$(\mu,\sigma^2)$,证明: $X_1 - X_2$和$X_1 + X_2$也是独立的. 

\hints\ 
$$
\text{Cov}(X_1 - X_2, X_1 + X_2) = D(X_1) - D(X_2) = 0,
$$
且$X_1 - X_2 \sim (0, 2\sigma^2), X_1 + X_2 \sim (2u, 2\sigma^2)$. 
\end{example}

\newpage
\section{期望和方差}

\subsection{复杂随机变量函数}

\begin{example}
\rm 相互独立的随机变量$X_1$和$X_2$均服从正态分布$N(0,\frac{1}{2})$,求$D(|X_1 - X_2|)$. 

\hints\ 这里求期望不需要计算出$|X_1-X_2|$的概率分布,只需要确定$X_1 - X_2$概率分布即可,设$Z= X_1 - X_2$,那么显然有$Z \sim N(0,1)$. 首先求$E(|X_1-X_2|)$
$$
E(|X_1-X_2|) = \int_{-\infty}^{+\infty} |z|f_z(z)dz = 2\int_0^{+\infty} z\frac{1}{\sqrt{2\pi}}e^{-\frac{z^2}{2}} = \frac{\sqrt{2}}{\sqrt{\pi}}.
$$
再来求$D(|X_1 - X_2|)$
$$
D(|X_1 - X_2|) = D(|Z|) = E(Z^2) - E^2(|Z|) = 1-\frac{2}{\pi}.
$$
\end{example}

\begin{example}
\rm 设随机变量$X$的分布函数为$F(x) = 0.4\Phi(\frac{x-5}{2})+0.6\Phi(\frac{x+1}{3})$,其中$\Phi(x)$为标准正态分布的分布函数,求$E(X)$.

\hints\ 常规思路是先求出$f(x)$,再积分
$$
\int_{-\infty}^{+\infty} \frac{x}{5}f \left(\frac{x-5}{2}\right)dx + \int_{-\infty}^{+\infty} \frac{x}{5}f\left(\frac{x+1}{3}\right)dx = \int_{-\infty}^{+\infty}\left(\frac{4}{5}t+2\right)f(t)dt + \int_{-\infty}^{+\infty} \left(\frac{9}{5}t - \frac{3}{5}\right)f(t)dt = \frac{7}{5}. 
$$
也可以这样思考$\Phi(\frac{x-5}{2}) \sim N(5,4), \Phi(\frac{x+1}{3}) \sim N(-1,9)$,因此$E(X) = \frac{2}{5} \cdot 5 - 1 \cdot \frac{3}{5}= \frac{7}{5}$.
\end{example}


\begin{example}
\rm 设连续型随机变量$X_1$与$X_2$相互独立且方差均存在,$X_1$和$X_2$的概率密度分别为$f_1(x)$与$f_2(x)$,随机变量$Y_1$的概率密度为$f_{Y_1}(y) = \frac{1}{2}[f_1(y)+f_2(y)]$,随机变量$Y_2=\frac{1}{2}(X_1 + X_2)$. 比较$D(Y_1)$和$D(Y_2)$.

\hints\ 首先计算它们的期望
$$
\begin{aligned}
E(Y_1) &= \frac{1}{2}\int_{-\infty}^{+\infty}yf_1(y)dy + \frac{1}{2}\int_{-\infty}^{+\infty}yf_2(y) = \frac{E(X_1)+E(X_1)}{2} \\
E(Y_2) &= \frac{E(X_1)+E(X_1)}{2}
\end{aligned}
$$
而$D(Y) = E(Y^2) - E^2(Y)$,这里$E(Y_1) = E(Y_2)$,因此我们可以通过比较$E(Y_1^2)$和$E(Y_2^2)$来比较$D(Y_1)$和$D(Y_2)$. 
$$
\begin{aligned}
E(Y_1^2) &= \frac{E(X_1^2)+E(X_2^2)}{2} \\
E(Y_2^2) &= \frac{E(X_1^2) + 2E(X_1X_2) + E(X_2^2)}{4}
\end{aligned}
$$
那么
$$
E(Y_1^2) - E(Y_2^2) = \frac{E[(X_1-X_2)^2]}{4} > 0. 
$$
\end{example}

\subsection{随机变量乘积}

\begin{example}
\rm 设随机变量$X$服从标准正态分布$N(0,1)$,求$E[(X-2)^2e^{2X}]$.

\hints\ 这里要用求随机变量函数期望的公式. 
$$
\int_{-\infty}^{+\infty} (x-2)^2 e^{2x} \frac{1}{\sqrt{2\pi}}e^{-\frac{x^2}{2}} dx = e^2  \int_{-\infty}^{+\infty}(x-2)^2 \frac{1}{\sqrt{2\pi}} e^{-\frac{(x-2)^2}{2}} dx = e^2 
$$
\end{example}

\begin{example}
\rm 设随机变量$X,Y$不相关,且$E(X) = 2, E(Y) = 1, D(X) = 3$,求$E[X(X+Y-2)]$.

\hints\
$$
\text{Cov}(X, X+Y-2) = E[X(X+Y-2)]-E(X)E(X+Y-2).
$$
\end{example}

\subsection{拟合}

\begin{example}
\rm 设随机变量$X_1, X_2, \cdots, X_{2n}, \cdots$相互独立服从指数分布$E(\lambda)$,记$Z_i = X_{2i} - X_{2i-1}, i = 1,2,3\cdots$,则$\sum\limits_{i=1}^n Z_i$近似服从正态分布,求其参数. 

\hints\ 不要去构造尝试构造$\sum\limits_{i=1}^n Z_i$,题目已经告诉你是正态分布了,那么直接求其期望和方差即可. 
\end{example}


\subsection{线性相关}

\begin{example}
\rm 设随机变量$X \sim N(0,1), Y \sim N(1,4)$,且相关系数$\rho_{XY} = 1$,求$Y=aX+b$. 

\hints\ 这个题非常经典,$\rho_{XY} = 1$表示$X$和$Y$线性相关,来求它们之间的线性表达式. 

\redt{方法1.} 直接祭关键表达式
$$
E\{[t(X-E(X)) + (Y-E(Y))]^2\} = t^2D(X)+2t\text{Cov}(X,Y)+D(Y),
$$
使得上式左边等于$0$,求出$t$. 因此
$$
t(X-E(X)) + (Y-E(Y)) = 0,
$$
带入$t=2$,即可求得$Y=2X+1$. 

\redt{方法2.} 由$Y=aX+b$,那么$E(Y)=E(aX+b)$,求出$b = 1$. 再由
$$
\text{Cov}(X,Y) = \text{Cov}(X,aX+b) = a\text{Cov}(X,X) = 2,
$$
求出$a=2$.  
\end{example}

\section{参数估计}

\subsection{均匀分布的参数估计}

\begin{example}
\rm 设总体$X$的概率密度为
$$
f(x;\theta) = \left\{
\begin{array}{ll}
\frac{1}{1-\theta}, & \theta\leq x \leq 1, \\
0, & \text{其他}. 
\end{array} \right. 
$$

\hints\ 那么似然函数为
$$
\begin{aligned}
L(x_1,x_2,\cdots,x_n;\theta) &= 
\left\{
\begin{array}{ll}
\frac{1}{(1-\theta)^n}, & \theta\leq x_i \leq 1, i=1,2,\cdots,n\\
0, & \text{其他}. 
\end{array} \right. \\
&= \left\{
\begin{array}{ll}
\frac{1}{(1-\theta)^n}, & \theta\leq \min\{x_1,x_2,\cdots,x_n\} \\
0, & \text{其他}. 
\end{array} \right.
\end{aligned}
$$
因此当$\theta = \min\{x_1,x_2,\cdots,x_n\}$时,$L(x_1,x_2,\cdots,x_n;\theta)$取最大值. \redt{一定理解这个表示形式,因为拿到手你不太好表示这个似然函数的}. 
\end{example}

\section{三大分布}

\subsection{三大分布重要性质的应用}

\begin{example}
\rm 设$X_1,X_2,\cdots,X_n$来自正态总体$N(\mu,\sigma^2)$的简单随机样本,求$E\left\{\left(\sum\limits_{i=1}^n X_i\right) \left[\sum\limits_{j=1}^n(nX_j-\sum\limits_{k=1}^nX_k)^2 \right] \right\}$. 

\hints\ $\overline{X}$和$S^2$线性无关. 
\end{example}

\begin{example}
\rm 设$X_1,X_2,\cdots,X_n$和$Y_1,Y_2,\cdots,Y_n$分别来自正态总体$N(\mu,\sigma^2)$两个相互独立简单随机样本,设它们样本方差分别为$S_X^2$和$S_Y^2$,求统计量$T=(n-1)(S_X^2 + S_Y^2)$的方差$D(T)$.

\hints\ $\frac{(n-1)S^2}{\sigma^2} \sim \chi^2(n-1)$.
\end{example}

\begin{example}
\rm 设随机变量$X \sim F(n,n), p_1 = P\{X \geq 1\}, p_2 = P\{X \leq 1\}$,证明: $p_1 = p_2$.

\hints\ $X \sim F(n_1,n_2) \Rightarrow \frac{1}{X} \sim F(n_2,n_1)$. 
\end{example}

\section{假设检验}

\subsection{犯错的概率}

\begin{example}
\rm 设$X_1,X_2,\cdots,X_16$是来自总体$N(\mu,4)$的简单样本,考虑假设检验问题$H_0: \mu \leq 10, H_1: u > 10$. 若该检验问题的拒绝域为$W=\{\overline{X} > 11 \}$,其中$\overline{X} = \frac{1}{16}\sum\limits_{i=1}^{16}X_i$,则$\mu = 11.5$时,求该检验犯第二类错误的概率. 

\hints\ 给定$\mu = 11.5$,那么原假设$H_0$为假的,因此按照定义
$$
\beta = P\{\overline{X} \leq 11\} = P\{\frac{\overline{X}-11.5}{\frac{1}{2}} \leq -1\} = \Phi(-1) = 1-\Phi(1). 
$$
\end{example}

\end{document}