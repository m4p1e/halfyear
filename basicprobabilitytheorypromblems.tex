\documentclass{article}

\usepackage{ctex}
\usepackage{tikz}
\usetikzlibrary{cd}
%\usetikzlibrary{paths.ortho} % path折线
%\usetikzlibrary{decorations.pathreplacing}
\usetikzlibrary{calc}
\usetikzlibrary{graphs, graphs.standard, quotes}% quotes library is for the [""] edges
\usetikzlibrary{positioning} %right of 描述位置的

\usepackage{amsthm}
\usepackage{amsmath}
\usepackage{amssymb}
\usepackage{mathrsfs} %花写

%\usepackage{unicode-math}

\usepackage{enumitem}

\usepackage[textwidth=18cm]{geometry} % 设置页宽=18

\usepackage{blindtext}
\usepackage{bm}
\parindent=0pt
\setlength{\parindent}{2em} 
\usepackage{indentfirst}
\usepackage{hyperref} %url
\hypersetup{
    colorlinks=true,
    linkcolor=blue,
    filecolor=magenta,      
    urlcolor=cyan,
    pdftitle={Overleaf Example},
    pdfpagemode=FullScreen,
    }


\usepackage{xcolor}
\usepackage{titlesec}
\titleformat{\section}[block]{\color{blue}\Large\bfseries\filcenter}{}{1em}{}
\titleformat{\subsection}[hang]{\color{red}\Large\bfseries}{}{0em}{}
%\setcounter{secnumdepth}{1} %section 序号

\counterwithin*{equation}{section} %equation重新编号
\counterwithin*{equation}{subsection}

\newtheorem{theorem}{Theorem}[section]
\newtheorem{lemma}[theorem]{Lemma}
\newtheorem{corollary}[theorem]{Corollary}
\newtheorem{proposition}[theorem]{Proposition}
\newtheorem{example}[theorem]{Example}
\newtheorem{definition}[theorem]{Definition}
\newtheorem{remark}[theorem]{Remark}
\newtheorem{exercise}{Exercise}[section]
\newtheorem{annotation}[theorem]{Annotation}

\newcommand*{\xfunc}[4]{{#2}\colon{#3}{#1}{#4}}
\newcommand*{\func}[3]{\xfunc{\to}{#1}{#2}{#3}}

\newcommand\Set[2]{\{\,#1\mid#2\,\}} %集合
\newcommand\SET[2]{\Set{#1}{\text{#2}}} %

\newcommand{\redt}[1]{\textcolor{red}{#1}}
\newcommand{\bluet}[1]{\textcolor{blue}{#1}}

\newcommand{\hints}{{\color{blue} \text{hints}}}

\begin{document}
\title{考研概率论}
\author{枫聆}
\maketitle

\tableofcontents

\newpage
\section{概率运算}

\subsection{贝叶斯的应用}

\begin{example}
\rm 假设有两箱同种零件: 第一箱内装有$50$件,其中$10$件一等品; 第二箱内装有$30$件,其中$18$件一等品. 现从两箱中随意挑选一箱,然后从箱中随机取两个零件,试求在第一次取出的零件是一等品的条件下,第二次取出一等品的概率.

\hints\ 设事件$A$为选择第一个箱子,事件$B_1$为第一次取出一等品,事件$B_2$为第二次取出一等品. 这里要求的是一个条件概率$P(B_2|B_1)$,首先我们用贝叶斯公式分别计算$P(A|B_1)$和$P(\bar{A}|B_1)$,即
$$
P(A|B_1) = \frac{P(A)P(B_1|A)}{P(A)P(B_1|A) + P(\bar{A})P(B_1|\bar{A})} = \frac{\frac{10}{50}}{\frac{10}{50} + \frac{18}{30}} = \frac{1}{4},
$$ 
因此$P(\bar{A}|B_1) = \frac{3}{4}$. 于是
$$
P(B_2|B_1) = P(B_2|AB_1)P(A|B_1) + P(B_2|\bar{A}B_1)P(\bar{A}|B_1) = \frac{9}{49}\times\frac{1}{4} + \frac{17}{29} \times \frac{3}{4}
$$
\end{example}

\newpage
\section{期望和方差}

\subsection{复杂随机变量函数}

\begin{example}
\rm 相互独立的随机变量$X_1$和$X_2$均服从正态分布$N(0,\frac{1}{2})$,求$D(|X_1 - X_2|)$. 

\hints\ 这里求期望不需要计算出$|X_1-X_2|$的概率分布,只需要确定$X_1 - X_2$概率分布即可,设$Z= X_1 - X_2$,那么显然有$Z \sim N(0,1)$. 首先求$E(|X_1-X_2|)$
$$
E(|X_1-X_2|) = \int_{-\infty}^{+\infty} |z|f_z(z)dz = 2\int_0^{+\infty} z\frac{1}{\sqrt{2\pi}}e^{-\frac{z^2}{2}} = \frac{\sqrt{2}}{\sqrt{\pi}}.
$$
再来求$D(|X_1 - X_2|)$
$$
D(|X_1 - X_2|) = D(|Z|) = E(Z^2) - E^2(|Z|) = 1-\frac{2}{\pi}.
$$
\end{example}

\end{document}