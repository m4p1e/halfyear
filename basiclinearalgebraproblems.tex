\documentclass{article}

\usepackage{ctex}
\usepackage{tikz}
\usetikzlibrary{calc}
\usetikzlibrary{cd}
\usetikzlibrary{decorations.pathreplacing}

\usepackage{amsthm}
\usepackage{amsmath}
\usepackage{amssymb}
\usepackage{mathdots} %iddots

\usepackage{pgfplots}
\pgfplotsset{compat=newest}

\usepackage{hyperref} %url
\hypersetup{
    colorlinks=true,
    linkcolor=blue,
    filecolor=magenta,      
    urlcolor=cyan,
    pdftitle={Overleaf Example},
    pdfpagemode=FullScreen,
    }

\usepackage{enumitem}

\usepackage[textwidth=18cm]{geometry} % 设置页宽=18

\usepackage{blindtext}
\usepackage{bm}
\parindent=0pt
\setlength{\parindent}{2em} 
\usepackage{indentfirst}


\usepackage{xcolor}
\usepackage{titlesec}
\titleformat{\section}[block]{\color{blue}\Large\bfseries\filcenter}{}{1em}{}
\titleformat{\subsection}[hang]{\color{red}\Large\bfseries}{}{0em}{}
%\setcounter{secnumdepth}{1} %section 序号

\newtheorem{theorem}{Theorem}[section]
\newtheorem{lemma}[theorem]{Lemma}
\newtheorem{corollary}[theorem]{Corollary}
\newtheorem{proposition}[theorem]{Proposition}
\newtheorem{example}[theorem]{Example}
\newtheorem{definition}[theorem]{Definition}
\newtheorem{remark}[theorem]{Remark}
\newtheorem{exercise}{Exercise}[section]
\newtheorem{annotation}[theorem]{Annotation}

\newcommand*{\xfunc}[4]{{#2}\colon{#3}{#1}{#4}}
\newcommand*{\func}[3]{\xfunc{\to}{#1}{#2}{#3}}

\newcommand\Set[2]{\{\,#1\mid#2\,\}} %集合
\newcommand\SET[2]{\Set{#1}{\text{#2}}} %

\newcommand{\norm}[1]{\left\lVert#1\right\rVert} % 范数
\newcommand{\vect}[1]{\mathbf{#1}} % vector
\newcommand{\hints}{{\color{blue} \text{hints}}}
\newcommand{\mbf}[1]{\bm{#1}} 
\newcommand{\rank}[1]{\text{rank}\left(#1\right)} % rank
\newcommand\inp[2]{\langle #1, #2 \rangle} %inner product

\newcommand{\redt}[1]{\textcolor{red}{#1}}
\newcommand{\bluet}[1]{\textcolor{blue}{#1}}


\begin{document}
\title{考研高数习题集}
\author{枫聆}
\maketitle
\tableofcontents

\section{行列式}

\subsection{定义}

\begin{annotation}
\rm 这类题特征
\begin{enumerate}
	\item 按照行列式的完全展开式来计算某种特殊的矩阵
	\item 给定某个具体的行列式值的基础上,通过行列式的性质来计算行列式. 
\end{enumerate}
\end{annotation}

\begin{example}
\rm 证明: 如果在$n$阶行列式中,第$i_1,i_2,\cdots,i_k$行分别与第$j_1,j_2,\cdots,j_l$列交叉位置的元素都是$0$,并且$k+l > n$,那么这个行列式的值等于$0$.
\begin{proof}
按照行列式的完全展开式,每一项都必须要包含第$i_1,i_2,\cdots,i_k$行中位于不用列的元素,则有$k$个元素. 由已知的条件,第$i_1,i_2,\cdots,i_k$行只与$j_1,j_2,\cdots,j_l$之外的$n-l$元素可能不为零,但是$k > n-l$,说明每一项必取到0,因此行列式为$0$. 
\end{proof}
\end{example}

\begin{example}
\rm 证明
$$
\begin{vmatrix}
a_1 + c_1 &  b_1 + a_1 & c_1+b_1 \\
a_2 + c_2 &  b_2 + a_2 & c_2+b_2 \\
a_3 + c_3 &  b_3 + a_3 & c_3+b_3 \\
\end{vmatrix} = 2 \begin{vmatrix}
a_1 & b_1 & c_1 \\
a_2 & b_2 & c_2 \\
a_3 & b_3 & c_3 	
\end{vmatrix}
$$
\end{example}

\begin{example}
\rm \redt{行列式最大值} 求元素为$1$和$0$的三阶行列式可取的最大值.

\hints\ 从完全展开式我们应使得带正号的项尽可能的都是1,而带负号项尽可能是0
\begin{enumerate}
	\item 若3个正项都是1,那么此时行列式等于0;
	\item 若2个正项是1,此时任取一个2个项是正的行列式其行列式均等于2,且其他负项也都是0,例如
	$$
	\begin{vmatrix}
	0 & 1 & 1\\
	1 & 0 & 1\\
	1 & 1 & 0\\
	\end{vmatrix}
	$$
\end{enumerate}
综上最大值肯定就是$2$. 
\end{example}

\begin{example}
\rm 设$n \geq 2$,证明: 如果$n$阶矩阵$\mbf{A}$的元素为$1$或者$-1$,则$|A|$闭为偶数. 

\hints\ 这个证明按行展开可能更简单
$$
|\mbf{A}| = a_{11}A_{11} + a_{12}A_{12} +a_{13}A_{13},
$$
其中三个代数余子式都是二阶行列式的正值或者负值,那么我们来看一下元素为$1$或者$-1$的二阶行列式
$$
\begin{vmatrix}
b_{11} & b_{12} \\
b_{21} & b_{22} \\
\end{vmatrix}  = b_{11}b_{22} - b_{12}b{21}
$$
这里一共有4种可能的取值,由$b_{11}b_{22} = \pm 1, b_{12}b{21} = 、\pm 1$决定,经过计算该二阶行列式取值可能为$0,2$,因此$|A|$是由3个偶数相加得到的,那么$|A|$也一定是偶数. 
\end{example}

\begin{example}
\rm 求元素为$1$或者$-1$的三阶行列式的最大值. 

\hints\ 从完全展开式出发,三阶行列式有6项,其中每一项只可能为$-1$和$1$. 再有前面证明,我们知道这样的三阶行列式的值只能是偶数,那么最大的偶数就是6项全为1加起来为6,即
$$
\begin{array}{ll}
a_{11}a_{22}a_{33} = 1, a_{12}a_{23}a_{31} = 1, a_{13}a_{21}a_{32} = 1 \\
-a_{13}a_{22}a_{31} = 1, -a_{12}a_{21}a_{33} = 1, -a_{11}a_{23}a_{32} = 1 
\end{array}
$$
由此得出
$$
\begin{array}{ll}
a_{11}a_{22}a_{33}a_{12}a_{23}a_{31}a_{13}a_{21}a_{32} = 1 \\
a_{13}a_{22}a_{31}a_{12}a_{21}a_{33}a_{11}a_{23}a_{32} = -1 
\end{array}
$$
这是矛盾的. 因此我们再考虑行列式最大值为$4$的可能,可以找到
$$
\begin{vmatrix}
1 & -1 & -1\\
-1 & 1 & -1 \\
1 & 1 & 1 
\end{vmatrix} = 1 + 1 + 1 + 1 - 1 + 1 = 4
$$
那么这样的行列式的最大值为$4$. 
\end{example}

\begin{example}
\rm 设$n \leq 3$,证明: 元素为$1$或者$-1$的$n$阶行列式的绝对值不超过$(n-1)!(n-1)$.

\hints\ 借助前面的例子做归纳. 
\end{example}

\begin{example}
\rm 设$n \geq 2$,证明: 元素为$1$或$-1$的$n$阶行列式的值能被$2^{n-1}$整除. 

\hints\ 设$|\mbf{A}|$是这样的行列式,首先将第一列上的$(-1)$所在行都提一个$-1$出来
$$
|\mbf{A}| = (-1)^m \begin{vmatrix}
1 & b_{12} & \cdots & b_{1n} \\
1 & b_{22} & \cdots & b_{2n} \\
\vdots & \vdots & \cdots & \vdots \\
1 & b_{n2} & \cdots & b_{nn} \\
\end{vmatrix} = (-1)^m \begin{vmatrix} 
1 & b_{12} & \cdots & b_{1n} \\
0 & c_{22} & \cdots & c_{2n} \\
\vdots & \vdots & \cdots & \vdots \\
0 & c_{n2} & \cdots & c_{nn} \\
\end{vmatrix}
$$
其中$c_{ij}$值可能为$2,-2,0$,因此从第$2$列到第$n$列都是可以提一个因子$2$出来,最终就可以凑成$2^{n-1}$. 
\end{example}



\subsection{化行阶梯形}

\begin{annotation}
\rm 不是特殊矩阵的第一选择.
\end{annotation}

\subsection{按一行展开}

\begin{annotation}
\rm 若是可以将某一行或者某一列消去,只留下一个非零元素,按行和按列展开是不错的选择. 
\end{annotation}

\begin{example}
\rm 计算
$$
|\mbf{A}| = \begin{vmatrix}
1 & 2 & 3 & \cdots & n-1 & n \\
1 & -1 & 0 & \cdots & 0 & 0 \\
0 & 2 & -2 & \cdots & 0 & 0 \\
\vdots & \vdots & \vdots &  & \vdots & \vdots \\
0 & 0 & 0 & \cdots & n-1 & 1-n \\
\end{vmatrix}
$$
\hints 可以考虑把所有列都加到第一列,再按第一列展开
$$
|\mbf{A}| = \begin{vmatrix}
\frac{(1+n)n}{2} & 2 & 3 & \cdots & n-1 & n \\
0 & -1 & 0 & \cdots & 0 & 0 \\
0 & 2 & -2 & \cdots & 0 & 0 \\
\vdots & \vdots & \vdots &  & \vdots & \vdots \\
0 & 0 & 0 & \cdots & n-1 & n-1 \\
\end{vmatrix} = \frac{(1+n)n}{2} \begin{vmatrix}
 -1 & 0 & \cdots & 0 & 0 \\
2 & -2 & \cdots & 0 & 0 \\
 \vdots & \vdots &  & \vdots & \vdots \\
  0 & 0 & \cdots & n-1 & n-1 \\
\end{vmatrix} 
$$
同样上述矩阵也是所有列加到第一列,最终有$|\mbf{A}| = (-1)^{n-1}\frac{(n+1)!}{2}$.
\end{example}

\subsection{按多行展开}

\begin{annotation}
\rm 好像没有直接使用拉普拉斯定理的习惯,比较特殊的分块矩阵可以考虑. 
\end{annotation}

\subsection{特殊矩阵}



\begin{annotation}
\rm 常见的特殊矩阵\url{https://www.bilibili.com/read/cv266516}
\begin{enumerate}
	\item 范德蒙德行列式
	\item 爪型行列式	
\end{enumerate}
\end{annotation}

\subsection{数学归纳法}

\begin{annotation}
\rm 通常证明手法也是按行或者列展开.
\end{annotation}

\begin{example}
\rm 计算$n$阶行列式
$$
\mbf{D}_n = \begin{vmatrix}
x & 0 & 0  & \cdots & 0 & 0 & a_0 \\
-1 & x & 0  & \cdots & 0 & 0 & a_1 \\
0 & -1 & x  & \cdots & 0 & 0 & a_2 \\
\vdots & \vdots & \vdots  & \cdots & \vdots & \vdots & \vdots \\
0 & 0 & 0  & \cdots & -1 & x & a_{n-2} \\
0 & 0 & 0  & \cdots & 0 & -1 & x+a_{n-1} \\
\end{vmatrix}
$$
\begin{proof}
当$n=2$时,有
$$
\mbf{D}_2 = \begin{vmatrix}
x & a_0 \\
-1 & x+a_1 
\end{vmatrix} = x^2 + a_1x+ a_0
$$
假设对于上述形式的$n-1$阶行列式,有
$$
\begin{vmatrix}
x & 0 & 0  & \cdots & 0 & 0 & a_0 \\
-1 & x & 0  & \cdots & 0 & 0 & a_1 \\
0 & -1 & x  & \cdots & 0 & 0 & a_2 \\
\vdots & \vdots & \vdots  & \cdots & \vdots & \vdots & \vdots \\
0 & 0 & 0  & \cdots & 0 & -1 & x+a_{n-2} \\
\end{vmatrix} = x^{n-1} + a_{n-2}x^{n-2} + \cdots + a_1x + a_0
$$
那么$n$阶行列式,把它按第一行展开,有
$$
\begin{array}{ll}
\mbf{D}_n &= x\begin{vmatrix}
 x & 0  & \cdots & 0 & 0 & a_1 \\
 -1 & x  & \cdots & 0 & 0 & a_2 \\
 \vdots & \vdots  & \cdots & \vdots & \vdots & \vdots \\
 0 & 0  & \cdots & -1 & x & a_{n-2} \\
 0 & 0  & \cdots & 0 & -1 & x+a_{n-1} \\
\end{vmatrix}+ (-1)^{1+n}a_0 \begin{vmatrix}
-1 & x & 0  & \cdots & 0 & 0  \\
0 & -1 & x  & \cdots & 0 & 0 \\
\vdots & \vdots & \vdots  & \cdots & \vdots & \vdots  \\
0 & 0 & 0  & \cdots & -1 & x  \\
0 & 0 & 0  & \cdots & 0 & -1  \\
\end{vmatrix}  \\
&=x(x^{n-1} + a_{n-1}x^{n-2} + \cdots + a_2x + a_1) + (-1)^{1+n}a_0(-1)^{n-1} \\
&=  x^n + a_{n-1}x^{n-1} + a_{n-2}x^{n-2} + \cdots + a_1x + a_0
\end{array}
$$
\end{proof}
\end{example}

\subsection{递推式}


\begin{example}
\rm 计算$n$阶行列式
$$
\mbf{D}_n = \begin{vmatrix}
2 & -1 & 0 & 0 & \cdots & 0 & 0 & 0\\
-1 & 2 & -1 & 0 & \cdots & 0 & 0 & 0\\
0 & -1 & 2 & -1 & \cdots & 0 & 0 & 0\\
\vdots & \vdots & \vdots & \vdots & & \vdots & \vdots & \vdots\\
0 & 0 & 0 & 0 & \cdots & -1 & 2 & -1\\
0 & 0 & 0 & 0 & \cdots & 0 & -1 & 2\\
\end{vmatrix}
$$
\hints\ 显然$\mbf{D}_1 = 2$. 将$\mbf{D}_n$按第一列展开,则有
$$
\mbf{D}_n = 2\mbf{D}_{n-1}+
\begin{vmatrix}
 -1 & 0 & 0 & \cdots & 0 & 0 & 0\\
 -1 & 2 & -1 & \cdots & 0 & 0 & 0\\
 \vdots & \vdots & \vdots & & \vdots & \vdots & \vdots\\
 0 & 0 & 0 & \cdots & -1 & 2 & -1\\
 0 & 0 & 0 & \cdots & 0 & -1 & 2\\
\end{vmatrix} = 2\mbf{D}_{n-1} - \mbf{D}_{n-2}
$$
这也意味着$\mbf{D}_n - \mbf{D}_{n-1} = \mbf{D}_{n-1}-\mbf{D}_{n-2}$,可以马上推出$\mbf{D}_n - \mbf{D}_{n-1} = \mbf{D}_2-\mbf{D}_1 = 1$,即该行列式是一个等差数列$D_n = 2+(n-1) = n+1$. 
\end{example}


\begin{example}
\rm 计算$n$阶行列式
$$
\mbf{D}_n = \begin{vmatrix}
a+b & ab & 0 & 0 & \cdots & 0 & 0 & 0\\
1 & a+b & ab & 0 & \cdots & 0 & 0 & 0\\
0 & 1 & a+b & ab & \cdots & 0 & 0 & 0\\
\vdots & \vdots & \vdots & \vdots & & \vdots & \vdots & \vdots\\
0 & 0 & 0 & 0 & \cdots & 1 & a+b & ab\\
0 & 0 & 0 & 0 & \cdots & 0 & 1 & a+b\\
\end{vmatrix}
$$
其中$a \neq b$.

\hints\ 还是按第一列展开
$$
\mbf{D}_n = (a+b)\mbf{D}_{n-1}-\begin{vmatrix}
 ab & 0 & 0 & \cdots & 0 & 0 & 0\\
 1 & a+b & ab & \cdots & 0 & 0 & 0\\
 \vdots & \vdots & \vdots & & \vdots & \vdots & \vdots\\
 0 & 0 & 0 & \cdots & 1 & a+b & ab\\
 0 & 0 & 0 & \cdots & 0 & 1 & a+b\\ 
\end{vmatrix} = (a+b)\mbf{D}_{n-1} -ab\mbf{D}_{n-2}. 
$$
注意其上最后一个等式成立的条件是$a \neq 0$和$b \neq 0$,那么推出
$$
\begin{array}{ll}
\mbf{D}_n - a\mbf{D}_{n-1} = b(\mbf{D}_{n-1}-a\mbf{D}_{n-2}) \Rightarrow \mbf{D}_n -a\mbf{D}_{n-1} = (\mbf{D}_2 - a\mbf{D}_1)b^{n-2}  \\
\mbf{D}_n - b\mbf{D}_{n-1} = a(\mbf{D}_{n-1}-b\mbf{D}_{n-2}) \Rightarrow \mbf{D}_n -b\mbf{D}_{n-1} = (\mbf{D}_2 - b\mbf{D}_1)a^{n-2}
\end{array}
$$
而$\mbf{D}_1 = a+b$,$\mbf{D}_2 = a^2 + ab + b^2$. 因此
$$
\begin{array}{ll}
\mbf{D}_n -a\mbf{D}_{n-1}  = b^n \\
\mbf{D}_n -b\mbf{D}_{n-1}  = a^n
\end{array}
$$
所以$D_n  = \frac{b^{n+1}-a^{n+1}}{b-a}$. 

当$a = 0$时,$\mbf{D}_n = b^n$; 当$b=0$时,$\mbf{D}_n=a^n$. 
\end{example}

\subsection{Laplace展开}

\begin{example}
\rm 计算下述$2n$阶行列式
$$
\mbf{D}_{2n} = \begin{vmatrix}
a  & & & &  & b \\
   & \ddots & & & \iddots \\
  & & a & b &  &  \\
  & & b & a &  & \\
  &  \iddots & & &  \ddots \\
  b& & & &  &a\\
\end{vmatrix}
$$
\hints\ 尝试从第一行和最后一行展开,于是得到
$$
\begin{array}{ll}
\mbf{D}_{2n} &= \begin{vmatrix}
a & b \\
b & a 
\end{vmatrix}\cdot (-1)^{(1+2n)+(1+2n)}\cdot \mbf{D}_{2n-2} \\
&=(a^2-b^2)\mbf{D}_{2n-2}
\end{array} 
$$
而$\mbf{D}_2=a^2 - b^2$,因此$\mbf{D}_{2n} = (a^2-b^2)^{n}$ 
\end{example}



\newpage
\section{向量空间}

\subsection{向量运算下的线性性质判定}

\begin{example}
\rm 证明: 如果向量组$\mbf{\alpha}_1,\mbf{\alpha}_2,\mbf{\alpha}_3$线性无关,那么向量组$3\mbf{\alpha}_1-\mbf{\alpha}_2,5\mbf{\alpha}_2 + 2\mbf{\alpha}_3, 4\mbf{\alpha}_3-7\mbf{\alpha}_1$
\hints\ 直接用线性无关的定义,即
$$
k_1(3\mbf{\alpha}_1-\mbf{\alpha}_2) + k_2(5\mbf{\alpha}_2 + 2\mbf{\alpha}_3)+ k_3(4\mbf{\alpha}_3-7\mbf{\alpha}_1) = \mbf{0}
$$
此时需要证明$k_1 = k_2 = k_3$,展开上式
$$
(3k_1-7k_3)\mbf{\alpha}_1 + (5k_2 - k_1)\mbf{\alpha}_2 + (2k_2 +4k_3)\mbf{\alpha}_3 = 0. 
$$
根据已知条件,于是得到下述方程组
$$
\left \{
\begin{array}{ll}
3k_1-7k_3  = 0 \\
5k_2 - k_1 = 0 \\
2k_2 +4k_3 = 0
\end{array} \right.
$$
由此构造系数矩阵,最后系数矩阵的行列式不为零,即原方程只有零解,命题得证. 
\end{example}

\subsection{向量组的极大线性无关组}

\begin{annotation}
\rm 若给定$n$个列向量$\mbf{\alpha}_1,\cdots,\mbf{\alpha}_n$,操作步骤为
\begin{enumerate}
	\item 写出对应矩阵的形式$\mbf{A}$,做初等行变换化行阶梯型矩阵$\mbf{D}$.
	\item 观察行列式不为零最大子式所在的列,它们构成一个极大线性无关组. 
\end{enumerate}
\end{annotation}


\newpage
\section{秩}

\subsection{特殊矩阵的秩}

\begin{example}
\rm 设$n$阶矩阵$\mbf{A}$,满足
$$
|a_{ii}| >  \sum\limits_{j = 1 \atop j \neq i}^{n} |a_{ij}|, \, i = 1,2,\cdots,n.
$$
证明$\mbf{A}$的秩等于$n$. 这样的矩阵称为\redt{主对角占优矩阵}. 

\hints\  设$\mbf{A}$的列向量为$\mbf{\alpha}_1,\mbf{\alpha}_2,\cdots,\mbf{\alpha}_n$,那么只需证明$\mbf{A}$的列向量都是线性无关的. 假设$\mbf{A}$的列向量是线性相关的,则存在
$$
k_1\mbf{\alpha}_1 + k_2\mbf{\alpha}_2 + \cdots + k_n\mbf{\alpha}_n = \mbf{0},
$$
其中$k_1,k_2,\cdots,k_n$不全为$0$. 取
$$
|k_l| = \max\{|k_1|,|k_2|,\cdots,|k_n|\}. 
$$
然后我们列向量的第$l$个分量有
$$
k_1a_{l1} + k_1a_{l2} + \cdots + k_na_{ln} = 0,
$$
等式两边除以$k_l$,得到
$$
a_{ll} = - \frac{k_1}{k_l}a_{l1} - \frac{k_2}{k_l}a_{l2} - \cdots - - \frac{k_n}{k_l}a_{ln} =  -\sum\limits_{j = 1 \atop j \neq l}^{n} \frac{k_j}{k_l}|a_{lj}|.
$$
因此有
$$
|a_{ll}| \leq \sum\limits_{j = 1 \atop j \neq l}^{n} |a_{lj}|,
$$
与前提条件矛盾. 从而$\mbf{A}$等于$n$. 
\end{example}

\newpage
\section{方程组的解}

\subsection{方程组性质}

\begin{example}
\rm 证明: 给定$s$个$n$线性方程组成的线性方程组,如果该方程组的增广矩阵的第$i$个行向量$\mbf{\alpha}_i$可以由其余行向量线性表出,即
$$
\mbf{\alpha}_i = k_1\mbf{\alpha}_1 + k_{i-1}\mbf{\alpha}_{i-1} + k_{i+1}\mbf{\alpha}_{k+1} + \cdots + k_s\mbf{\alpha}_s 
$$
那么将第$i$个方程去掉之后得到的方程组与原方程组通解. 
\end{example}

\begin{example}
\rm 设一个$m \times n$矩阵$H$的列向量组为$\mbf{\alpha}_1,\cdots,\mbf{\alpha}_n$. 证明: $H$的任意$s$列都线性无关当且仅当,齐次线性方程组
$$
x_1 \mbf{\alpha}_1 + \cdots + x_n \mbf{\alpha}_n = \mbf{0}
$$
的任一非零解的非零分量的数目大于$s$. 

\hints\ \emph{必要性}\ 若$H$的任意$s$列都线性无关. 假设上述线性方程存在一个非零解为
$$
\eta = (0,\cdots,c_{i_1},\cdots,c_{i_j},\cdots)^{T}. 
$$
其中$c_{i_1},\cdots,c_{i_j}$均不为零,且$ l \leq s$. 则
$$
c_{i_1}\mbf{\alpha}_{i_1}+\cdots+c_{i_j}\mbf{\alpha}_{i_j} = \mbf{0},
$$
这意味着存在$l$列线性相关,与前提数矛盾,因此任一非零解的非零分量的数目大于$s$

\emph{充分性}\ 若$H$的任一非零向量的非零分量的数目大于$s$. 假设有$l\leq s$个列向量线性相关
$$
k_{i_1}\mbf{\alpha}_{i_1}+\cdots+k_{i_j}\mbf{\alpha}_{i_j} = \mbf{0},
$$
那么存在一个非零解,即
$$
\eta = (0,\cdots,k_{i_1},\cdots,k_{i_j},\cdots)^{T}. 
$$
就是将其他分量扩充为0即可,这样和前提是矛盾的. 因此$H$的任意$s$列都线性无关. 
\end{example}

\subsection{给定方程组解的情况}


\subsection{带参数的方程组解的情况}


\subsection{线性方程充要条件}

\begin{annotation}
\rm \redt{系数矩阵的秩和增广矩阵的秩相同}!
\end{annotation}

\begin{example}
\rm 证明: 线性方程组
$$
\left\{
\begin{array}{l}
a_{11}x_1 + a_{12}x_2 + \cdots + a_{1n}x_n = b_1, \\
a_{21}x_1 + a_{22}x_2 + \cdots + a_{2n}x_n = b_2, \\
\cdots \\
a_{m1}x_1 + a_{m2}x_2 + \cdots + a_{mn}x_n = b_m, \\
\end{array} \right.
$$
有解的当且仅当下述线性方程
$$
\left\{
\begin{array}{l}
a_{11}x_1 + a_{21}x_2 + \cdots + a_{s1}x_m = 0, \\
a_{12}x_1 + a_{22}x_2 + \cdots + a_{s2}x_m = 0, \\
\cdots \\
a_{1m}x_1 + a_{2m}x_2 + \cdots +a_{sm}x_m = 0, \\
b_{1}x_1 + b_{2}x_2 + \cdots+ b_{m}x_m = 1, \\
\end{array} \right.
$$

\hints\ 设第一个方程组的系数矩阵为$\mbf{A}$,增广矩阵为$\tilde{\mbf{A}}$. 那么第二个方程组的系数矩阵及增广矩阵分别为
$$
\begin{array}{ll}
\mbf{B} = \tilde{\mbf{A}}^T \\
\tilde{\mbf{B}}= \begin{pmatrix}
\mbf{A}^T & \mbf{0} \\\
\mbf{\beta} & 1
\end{pmatrix}
\end{array}
$$
其中$\mbf{\beta} = (b_1,\cdots,b_m)$. 这里存在一些等式
$$
\begin{array}{ll}
\rank{\mbf{B}} = \rank{\tilde{\mbf{A}}} \\
\rank{\tilde{\mbf{B}}} = \rank{\mbf{A}} + 1.
\end{array}
$$

当$\rank{\mbf{A}} = \rank{\tilde{\mbf{A}}}$时,有$\rank{\mbf{B}} < \rank{\tilde{\mbf{B}}}$. 

当$\rank{\mbf{B}} < \rank{\tilde{\mbf{B}}}$时,有$\rank{\tilde{\mbf{A}}} + 1 = \rank{\mbf{A}} + 1$. \bluet{这里能得到这个结果是因为增广矩阵的秩那么等于系数矩阵的秩,那么等于系数矩阵的秩加1}. 
\end{example}

\subsection{齐次线性方程组的性质}

\begin{example}
\rm 设$n$个的方程的$n$元齐次线性方程组的系数矩阵$\mbf{A}$的行列式等于$0$,且$\mbf{A}$的$(k,l)$元的代数余子式$A_{kl} \neq 0$. 证明$\mbf{\eta}=(A_{k1},A_{k2},\cdots,A_{kn})^T$是原齐次线性方程组的一个基础解系. 

\hints\ 证明的要求暗示了解空间是一维的. 这是因为由$A_{kl} \neq 0$,意味着$\mbf{A}$存在一个$n-1$阶的子式行列式不为0,而$|\mbf{A}| = 0$,因此$\rank{\mbf{A}} = n-1$. 

将该$\mbf{\eta}$带入原方程,当$i = k$时
$$
a_{i1}A_{k1} + a_{i1}A_{k1} +\cdots +a_{in}A_{kn} = |\mbf{A}| = 0; 
$$
当$i \neq k$时,显然有
$$
a_{i1}A_{k1} + a_{i1}A_{k1} +\cdots +a_{in}A_{kn} = 0.
$$
因此$\mbf{\eta}$的确是原方程组的一个解,其中第$k$个分量$A_{kl} \neq 0$,结合解空间是一维的,从而$\mbf{\eta}$是一个基础解系. 
\end{example}

\newpage
\section{矩阵相似}

\subsection{相似判定}

\begin{proposition}
\rm 常用判定矩阵相似的方法,遇题依次向下使用下述方法.
\begin{enumerate}
	\item 必要条件:相似必行列值相等;
	\item 必要条件:特征值相等;
	\item 充分条件: 对于都可对角化的矩阵,判定其特征值是否相同;
	\item 否命题的充分条件: 一个可对角化,一个不可对角化,则它们不相似;
	\item 对于都不可对角的矩阵,同一个特征值的特征子空间的维数相同;	
	\item 对于都不可对角的矩阵,则对应的特征向量满足: 若$\mbf{B}$对应$\lambda$的特征向量$\lambda$,则$\mbf{A}$对应$\lambda$的特征向量为$P\alpha$. 这里需要求出可逆矩阵$P$
\end{enumerate}
\end{proposition}

\subsection{对角化判定}

\begin{proposition}
\rm 常用判定对角化的方法,遇题依次向下使用下述方法
\begin{enumerate}
	\item 实对称矩阵一定相似于对角矩阵;	
	\item 有$n$个不同的特征值,那么一定相似于对角矩阵;
	\item $n$重特征值对应特征子空间是否为$n$维;
\end{enumerate}
\end{proposition}

\section{二次型}

\subsection{正定性的判定}



\end{document}