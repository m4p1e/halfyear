\documentclass{article}

\usepackage{ctex}
\usepackage{tikz}
\usetikzlibrary{calc}
\usetikzlibrary{cd}
\usetikzlibrary{decorations.pathreplacing}

\usepackage{amsthm}
\usepackage{amsmath}
\usepackage{amssymb}
\usepackage{mathdots} %iddots

\usepackage{pgfplots}
\pgfplotsset{compat=newest}

\usepackage{hyperref} %url
\hypersetup{
    colorlinks=true,
    linkcolor=blue,
    filecolor=magenta,      
    urlcolor=cyan,
    pdftitle={Overleaf Example},
    pdfpagemode=FullScreen,
    }

\usepackage{enumitem}

\usepackage[textwidth=18cm]{geometry} % 设置页宽=18

\usepackage{blindtext}
\usepackage{bm}
\parindent=0pt
\setlength{\parindent}{2em} 
\usepackage{indentfirst}


\usepackage{xcolor}
\usepackage{titlesec}
\titleformat{\section}[block]{\color{blue}\Large\bfseries\filcenter}{}{1em}{}
\titleformat{\subsection}[hang]{\color{red}\Large\bfseries}{}{0em}{}
%\setcounter{secnumdepth}{1} %section 序号

\newtheorem{theorem}{Theorem}[section]
\newtheorem{lemma}[theorem]{Lemma}
\newtheorem{corollary}[theorem]{Corollary}
\newtheorem{proposition}[theorem]{Proposition}
\newtheorem{example}[theorem]{Example}
\newtheorem{definition}[theorem]{Definition}
\newtheorem{remark}[theorem]{Remark}
\newtheorem{exercise}{Exercise}[section]
\newtheorem{annotation}[theorem]{Annotation}

\newcommand*{\xfunc}[4]{{#2}\colon{#3}{#1}{#4}}
\newcommand*{\func}[3]{\xfunc{\to}{#1}{#2}{#3}}

\newcommand\Set[2]{\{\,#1\mid#2\,\}} %集合
\newcommand\SET[2]{\Set{#1}{\text{#2}}} %

\newcommand{\norm}[1]{\left\lVert#1\right\rVert} % 范数
\newcommand{\vect}[1]{\mathbf{#1}} % vector
\newcommand{\hints}{{\color{blue} \text{hints}}}
\newcommand{\mbf}[1]{\bm{#1}} 
\newcommand{\rank}[1]{\text{rank}\left(#1\right)} % rank
\newcommand\inp[2]{\langle #1, #2 \rangle} %inner product

\begin{document}
\title{考研高数习题集}
\author{枫聆}
\maketitle
\tableofcontents

\section{行列式}

\subsection{定义}

\begin{annotation}
\rm 这类题特征
\begin{enumerate}
	\item 按照行列式的完全展开式来计算某种特殊的矩阵
	\item 给定某个具体的行列式值的基础上,通过行列式的性质来计算行列式. 
\end{enumerate}
\end{annotation}

\begin{example}
\rm 证明: 如果在$n$阶行列式中,第$i_1,i_2,\cdots,i_k$行分别与第$j_1,j_2,\cdots,j_l$列交叉位置的元素都是$0$,并且$k+l > n$,那么这个行列式的值等于$0$.
\begin{proof}
按照行列式的完全展开式,每一项都必须要包含第$i_1,i_2,\cdots,i_k$行中位于不用列的元素,则有$k$个元素. 由已知的条件,第$i_1,i_2,\cdots,i_k$行只与$j_1,j_2,\cdots,j_l$之外的$n-l$元素可能不为零,但是$k > n-l$,说明每一项必取到0,因此行列式为$0$. 
\end{proof}
\end{example}

\begin{example}
\rm 证明
$$
\begin{vmatrix}
a_1 + c_1 &  b_1 + a_1 & c_1+b_1 \\
a_2 + c_2 &  b_2 + a_2 & c_2+b_2 \\
a_3 + c_3 &  b_3 + a_3 & c_3+b_3 \\
\end{vmatrix} = 2 \begin{vmatrix}
a_1 & b_1 & c_1 \\
a_2 & b_2 & c_2 \\
a_3 & b_3 & c_3 	
\end{vmatrix}
$$
\end{example}

\begin{example}
\rm 求元素为$1$和$0$的三阶行列式可取的最大值.

\hints\ 从完全展开式我们应使得带正号的项尽可能的都是1,而带负号项尽可能是0
\begin{enumerate}
	\item 若3个正项都是1,那么此时行列式等于0;
	\item 若2个正项是1,此时任取一个2个项是正的行列式其行列式均等于2,且其他负项也都是0,例如
	$$
	\begin{vmatrix}
	0 & 1 & 1\\
	1 & 0 & 1\\
	1 & 1 & 0\\
	\end{vmatrix}
	$$
\end{enumerate}
综上最大值肯定就是$2$. 
\end{example}

\begin{example}
\rm 设$n \geq 2$,证明: 如果$n$阶矩阵$\mbf{A}$的元素为$1$或者$-1$,则$|A|$闭为偶数. 
\hints\ 这个证明按行展开可能更简单
$$
|\mbf{A}| = a_{11}A_{11} + a_{12}A_{12} +a_{13}A_{13},
$$
其中三个代数余子式都是二阶行列式的正值或者负值,那么我们来看一下元素为$1$或者$-1$的二阶行列式
$$
\begin{vmatrix}
b_{11} & b_{12} \\
b_{21} & b_{22} \\
\end{vmatrix}  = b_{11}b_{22} - b_{12}b{21}
$$
这里一共有4种可能的取值,由$b_{11}b_{22} = \pm 1, b_{12}b{21} = 、\pm 1$决定,经过计算该二阶行列式取值可能为$0,2$,因此$|A|$是由3个偶数相加得到的,那么$|A|$也一定是偶数. 
\end{example}

\begin{example}
\rm 求元素为$1$或者$-1$的三阶行列式的最大值. 
\hints\ 从完全展开式出发,三阶行列式有6项,其中每一项只可能为$-1$和$1$. 再有前面证明,我们知道这样的三阶行列式的值只能是偶数,那么最大的偶数就是6项全为1加起来为6,即
$$
\begin{array}{ll}
a_{11}a_{22}a_{33} = 1, a_{12}a_{23}a_{31} = 1, a_{13}a_{21}a_{32} = 1 \\
-a_{13}a_{22}a_{31} = 1, -a_{12}a_{21}a_{33} = 1, -a_{11}a_{23}a_{32} = 1 
\end{array}
$$
由此得出
$$
\begin{array}{ll}
a_{11}a_{22}a_{33}a_{12}a_{23}a_{31}a_{13}a_{21}a_{32} = 1 \\
a_{13}a_{22}a_{31}a_{12}a_{21}a_{33}a_{11}a_{23}a_{32} = -1 
\end{array}
$$
这是矛盾的. 因此我们再考虑行列式最大值为$4$的可能,可以找到
$$
\begin{vmatrix}
1 & -1 & -1\\
-1 & 1 & -1 \\
1 & 1 & 1 
\end{vmatrix} = 1 + 1 + 1 + 1 - 1 + 1 = 4
$$
那么这样的行列式的最大值为$4$. 
\end{example}

\subsection{化行阶梯形}

\begin{annotation}
\rm 不是特殊矩阵的第一选择.
\end{annotation}

\subsection{按一行展开}

\begin{annotation}
\rm 若是可以将某一行或者某一列消去,只留下一个非零元素,按行和按列展开是不错的选择. 
\end{annotation}

\begin{example}
\rm 计算
$$
|\mbf{A}| = \begin{vmatrix}
1 & 2 & 3 & \cdots & n-1 & n \\
1 & -1 & 0 & \cdots & 0 & 0 \\
0 & 2 & -2 & \cdots & 0 & 0 \\
\vdots & \vdots & \vdots &  & \vdots & \vdots \\
0 & 0 & 0 & \cdots & n-1 & 1-n \\
\end{vmatrix}
$$
\hints 可以考虑把所有列都加到第一列,再按第一列展开
$$
|\mbf{A}| = \begin{vmatrix}
\frac{(1+n)n}{2} & 2 & 3 & \cdots & n-1 & n \\
0 & -1 & 0 & \cdots & 0 & 0 \\
0 & 2 & -2 & \cdots & 0 & 0 \\
\vdots & \vdots & \vdots &  & \vdots & \vdots \\
0 & 0 & 0 & \cdots & n-1 & n-1 \\
\end{vmatrix} = \frac{(1+n)n}{2} \begin{vmatrix}
 -1 & 0 & \cdots & 0 & 0 \\
2 & -2 & \cdots & 0 & 0 \\
 \vdots & \vdots &  & \vdots & \vdots \\
  0 & 0 & \cdots & n-1 & n-1 \\
\end{vmatrix} 
$$
同样上述矩阵也是所有列加到第一列,最终有$|\mbf{A}| = (-1)^{n-1}\frac{(n+1)!}{2}$.
\end{example}

\subsection{按多行展开}

\begin{annotation}
\rm 好像没有直接使用拉普拉斯定理的习惯,比较特殊的分块矩阵可以考虑. 
\end{annotation}

\subsection{特殊矩阵}



\begin{annotation}
\rm 常见的特殊矩阵\url{https://www.bilibili.com/read/cv266516}
\begin{enumerate}
	\item 范德蒙德行列式
	\item 爪型行列式	
\end{enumerate}
\end{annotation}

\subsection{数学归纳法}

\begin{annotation}
\rm 通常证明手法也是按行或者列展开.
\end{annotation}

\begin{example}
\rm 计算$n$阶行列式
$$
\mbf{D}_n = \begin{vmatrix}
x & 0 & 0  & \cdots & 0 & 0 & a_0 \\
-1 & x & 0  & \cdots & 0 & 0 & a_1 \\
0 & -1 & x  & \cdots & 0 & 0 & a_2 \\
\vdots & \vdots & \vdots  & \cdots & \vdots & \vdots & \vdots \\
0 & 0 & 0  & \cdots & -1 & x & a_{n-2} \\
0 & 0 & 0  & \cdots & 0 & -1 & x+a_{n-1} \\
\end{vmatrix}
$$
\begin{proof}
当$n=2$时,有
$$
\mbf{D}_2 = \begin{vmatrix}
x & a_0 \\
-1 & x+a_1 
\end{vmatrix} = x^2 + a_1x+ a_0
$$
假设对于上述形式的$n-1$阶行列式,有
$$
\begin{vmatrix}
x & 0 & 0  & \cdots & 0 & 0 & a_0 \\
-1 & x & 0  & \cdots & 0 & 0 & a_1 \\
0 & -1 & x  & \cdots & 0 & 0 & a_2 \\
\vdots & \vdots & \vdots  & \cdots & \vdots & \vdots & \vdots \\
0 & 0 & 0  & \cdots & 0 & -1 & x+a_{n-2} \\
\end{vmatrix} = x^{n-1} + a_{n-2}x^{n-2} + \cdots + a_1x + a_0
$$
那么$n$阶行列式,把它按第一行展开,有
$$
\begin{array}{ll}
\mbf{D}_n &= x\begin{vmatrix}
 x & 0  & \cdots & 0 & 0 & a_1 \\
 -1 & x  & \cdots & 0 & 0 & a_2 \\
 \vdots & \vdots  & \cdots & \vdots & \vdots & \vdots \\
 0 & 0  & \cdots & -1 & x & a_{n-2} \\
 0 & 0  & \cdots & 0 & -1 & x+a_{n-1} \\
\end{vmatrix}+ (-1)^{1+n}a_0 \begin{vmatrix}
-1 & x & 0  & \cdots & 0 & 0  \\
0 & -1 & x  & \cdots & 0 & 0 \\
\vdots & \vdots & \vdots  & \cdots & \vdots & \vdots  \\
0 & 0 & 0  & \cdots & -1 & x  \\
0 & 0 & 0  & \cdots & 0 & -1  \\
\end{vmatrix}  \\
&=x(x^{n-1} + a_{n-1}x^{n-2} + \cdots + a_2x + a_1) + (-1)^{1+n}a_0(-1)^{n-1} \\
&=  x^n + a_{n-1}x^{n-1} + a_{n-2}x^{n-2} + \cdots + a_1x + a_0
\end{array}
$$
\end{proof}
\end{example}

\subsection{递推式}


\begin{example}
\rm 计算$n$阶行列式
$$
\mbf{D}_n = \begin{vmatrix}
2 & -1 & 0 & 0 & \cdots & 0 & 0 & 0\\
-1 & 2 & -1 & 0 & \cdots & 0 & 0 & 0\\
0 & -1 & 2 & -1 & \cdots & 0 & 0 & 0\\
\vdots & \vdots & \vdots & \vdots & & \vdots & \vdots & \vdots\\
0 & 0 & 0 & 0 & \cdots & -1 & 2 & -1\\
0 & 0 & 0 & 0 & \cdots & 0 & -1 & 2\\
\end{vmatrix}
$$
\hints\ 显然$\mbf{D}_1 = 2$. 将$\mbf{D}_n$按第一列展开,则有
$$
\mbf{D}_n = 2\mbf{D}_{n-1}+
\begin{vmatrix}
 -1 & 0 & 0 & \cdots & 0 & 0 & 0\\
 -1 & 2 & -1 & \cdots & 0 & 0 & 0\\
 \vdots & \vdots & \vdots & & \vdots & \vdots & \vdots\\
 0 & 0 & 0 & \cdots & -1 & 2 & -1\\
 0 & 0 & 0 & \cdots & 0 & -1 & 2\\
\end{vmatrix} = 2\mbf{D}_{n-1} - \mbf{D}_{n-2}
$$
这也意味着$\mbf{D}_n - \mbf{D}_{n-1} = \mbf{D}_{n-1}-\mbf{D}_{n-2}$,可以马上推出$\mbf{D}_n - \mbf{D}_{n-1} = \mbf{D}_2-\mbf{D}_1 = 1$,即该行列式是一个等差数列$D_n = 2+(n-1) = n+1$. 
\end{example}


\begin{example}
\rm 计算$n$阶行列式
$$
\mbf{D}_n = \begin{vmatrix}
a+b & ab & 0 & 0 & \cdots & 0 & 0 & 0\\
1 & a+b & ab & 0 & \cdots & 0 & 0 & 0\\
0 & 1 & a+b & ab & \cdots & 0 & 0 & 0\\
\vdots & \vdots & \vdots & \vdots & & \vdots & \vdots & \vdots\\
0 & 0 & 0 & 0 & \cdots & 1 & a+b & ab\\
0 & 0 & 0 & 0 & \cdots & 0 & 1 & a+b\\
\end{vmatrix}
$$
其中$a \neq b$.

\hints\ 还是按第一列展开
$$
\mbf{D}_n = (a+b)\mbf{D}_{n-1}-\begin{vmatrix}
 ab & 0 & 0 & \cdots & 0 & 0 & 0\\
 1 & a+b & ab & \cdots & 0 & 0 & 0\\
 \vdots & \vdots & \vdots & & \vdots & \vdots & \vdots\\
 0 & 0 & 0 & \cdots & 1 & a+b & ab\\
 0 & 0 & 0 & \cdots & 0 & 1 & a+b\\ 
\end{vmatrix} = (a+b)\mbf{D}_{n-1} -ab\mbf{D}_{n-2}. 
$$
注意其上最后一个等式成立的条件是$a \neq 0$和$b \neq 0$,那么推出
$$
\begin{array}{ll}
\mbf{D}_n - a\mbf{D}_{n-1} = b(\mbf{D}_{n-1}-a\mbf{D}_{n-2}) \Rightarrow \mbf{D}_n -a\mbf{D}_{n-1} = (\mbf{D}_2 - a\mbf{D}_1)b^{n-2}  \\
\mbf{D}_n - b\mbf{D}_{n-1} = a(\mbf{D}_{n-1}-b\mbf{D}_{n-2}) \Rightarrow \mbf{D}_n -b\mbf{D}_{n-1} = (\mbf{D}_2 - b\mbf{D}_1)a^{n-2}
\end{array}
$$
而$\mbf{D}_1 = a+b$,$\mbf{D}_2 = a^2 + ab + b^2$. 因此
$$
\begin{array}{ll}
\mbf{D}_n -a\mbf{D}_{n-1}  = b^n \\
\mbf{D}_n -b\mbf{D}_{n-1}  = a^n
\end{array}
$$
所以$D_n  = \frac{b^{n+1}-a^{n+1}}{b-a}$. 

当$a = 0$时,$\mbf{D}_n = b^n$; 当$b=0$时,$\mbf{D}_n=a^n$. 
\end{example}

\subsection{Laplace展开}

\begin{example}
\rm 计算下述$2n$阶行列式
$$
\mbf{D}_{2n} = \begin{vmatrix}
a  & & & &  & b \\
   & \ddots & & & \iddots \\
  & & a & b &  &  \\
  & & b & a &  & \\
  &  \iddots & & &  \ddots \\
  b& & & &  &a\\
\end{vmatrix}
$$
\hints\ 尝试从第一行和最后一行展开,于是得到
$$
\begin{array}{ll}
\mbf{D}_{2n} &= \begin{vmatrix}
a & b \\
b & a 
\end{vmatrix}\cdot (-1)^{(1+2n)+(1+2n)}\cdot \mbf{D}_{2n-2} \\
&=(a^2-b^2)\mbf{D}_{2n-2}
\end{array} 
$$
而$\mbf{D}_2=a^2 - b^2$,因此$\mbf{D}_{2n} = (a^2-b^2)^{n}$ 
\end{example}


\section{方程组的解}

\subsection{给定方程组解的情况}


\subsection{带参数的方程组解的情况}

\section{矩阵相似}

\subsection{相似判定}

\begin{proposition}
\rm 常用判定矩阵相似的方法,遇题依次向下使用下述方法.
\begin{enumerate}
	\item 必要条件:相似必行列值相等;
	\item 必要条件:特征值相等;
	\item 充分条件: 对于都可对角化的矩阵,判定其特征值是否相同;
	\item 否命题的充分条件: 一个可对角化,一个不可对角化,则它们不相似;
	\item 对于都不可对角的矩阵,同一个特征值的特征子空间的维数相同;	
	\item 对于都不可对角的矩阵,则对应的特征向量满足: 若$\mbf{B}$对应$\lambda$的特征向量$\lambda$,则$\mbf{A}$对应$\lambda$的特征向量为$P\alpha$. 这里需要求出可逆矩阵$P$
\end{enumerate}
\end{proposition}

\subsection{对角化判定}

\begin{proposition}
\rm 常用判定对角化的方法,遇题依次向下使用下述方法
\begin{enumerate}
	\item 实对称矩阵一定相似于对角矩阵;	
	\item 有$n$个不同的特征值,那么一定相似于对角矩阵;
	\item $n$重特征值对应特征子空间是否为$n$维;
\end{enumerate}
\end{proposition}

\section{二次型}

\subsection{正定性的判定}



\end{document}