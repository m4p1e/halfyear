\documentclass{article}

\usepackage{ctex}
\usepackage{tikz}
\usetikzlibrary{calc}
\usetikzlibrary{cd}
\usetikzlibrary{decorations.pathreplacing}

\usepackage{amsthm}
\usepackage{amsmath}
\usepackage{amssymb}

\usepackage{pgfplots}
\pgfplotsset{compat=newest}

\usepackage{hyperref} %url
\hypersetup{
    colorlinks=true,
    linkcolor=blue,
    filecolor=magenta,      
    urlcolor=cyan,
    pdftitle={Overleaf Example},
    pdfpagemode=FullScreen,
    }

\usepackage{enumitem}

\usepackage[textwidth=18cm]{geometry} % 设置页宽=18

\usepackage{blindtext}
\usepackage{bm}
\parindent=0pt
\setlength{\parindent}{2em} 
\usepackage{indentfirst}


\usepackage{xcolor}
\usepackage{titlesec}
\titleformat{\section}[block]{\color{blue}\Large\bfseries\filcenter}{}{1em}{}
\titleformat{\subsection}[hang]{\color{red}\Large\bfseries}{}{0em}{}
%\setcounter{secnumdepth}{1} %section 序号

\newtheorem{theorem}{Theorem}[section]
\newtheorem{lemma}[theorem]{Lemma}
\newtheorem{corollary}[theorem]{Corollary}
\newtheorem{proposition}[theorem]{Proposition}
\newtheorem{example}[theorem]{Example}
\newtheorem{definition}[theorem]{Definition}
\newtheorem{remark}[theorem]{Remark}
\newtheorem{exercise}{Exercise}[section]
\newtheorem{annotation}[theorem]{Annotation}

\newcommand*{\xfunc}[4]{{#2}\colon{#3}{#1}{#4}}
\newcommand*{\func}[3]{\xfunc{\to}{#1}{#2}{#3}}

\newcommand\Set[2]{\{\,#1\mid#2\,\}} %集合
\newcommand\SET[2]{\Set{#1}{\text{#2}}} %

\newcommand{\norm}[1]{\left\lVert#1\right\rVert} % 范数
\newcommand{\vect}[1]{\mathbf{#1}} % vector
\newcommand{\hints}{{\color{blue} \text{hints}}}
\newcommand{\mbf}[1]{\bm{#1}} 
\newcommand{\rank}[1]{\text{rank}\left(#1\right)} % rank
\newcommand\inp[2]{\langle #1, #2 \rangle} %inner product

\begin{document}
\title{考研高数习题集}
\author{枫聆}
\maketitle
\tableofcontents

\section{行列式}

\section{矩阵相似}

\subsection{相似判定}

\begin{proposition}
\rm 常用判定矩阵相似的方法,遇题依次向下使用下述方法.
\begin{enumerate}
	\item 必要条件:相似必行列值相等;
	\item 必要条件:特征值相等;
	\item 充分条件: 对于都可对角化的矩阵,判定其特征值是否相同;
	\item 否命题的充分条件: 一个可对角化,一个不可对角化,则它们不相似;
	\item 对于都不可对角的矩阵,同一个特征值的特征子空间的维数相同;	
	\item 对于都不可对角的矩阵,则对应的特征向量满足: 若$\mbf{B}$对应$\lambda$的特征向量$\lambda$,则$\mbf{A}$对应$\lambda$的特征向量为$P\alpha$. 这里需要求出可逆矩阵$P$
\end{enumerate}
\end{proposition}

\subsection{对角化判定}

\begin{proposition}
\rm 常用判定对角化的方法,遇题依次向下使用下述方法
\begin{enumerate}
	\item 实对称矩阵一定相似于对角矩阵;	
	\item 有$n$个不同的特征值,那么一定相似于对角矩阵;
	\item $n$重特征值对应特征子空间是否为$n$维;
\end{enumerate}
\end{proposition}

\section{二次型}

\subsection{正定性的判定}



\end{document}