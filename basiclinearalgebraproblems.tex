\documentclass{article}

\usepackage{ctex}
\usepackage{tikz}
\usetikzlibrary{calc}
\usetikzlibrary{cd}
\usetikzlibrary{decorations.pathreplacing}
\usetikzlibrary{patterns}
\usetikzlibrary{matrix}
\usetikzlibrary{decorations.pathreplacing,calligraphy}

\usepackage{amsthm}
\usepackage{amsmath}
\usepackage{amssymb}
\usepackage{mathdots} %iddots

\usepackage{pgfplots}
\pgfplotsset{compat=newest}

\usepackage{hyperref} %url
\hypersetup{
    colorlinks=true,
    linkcolor=blue,
    filecolor=magenta,      
    urlcolor=cyan,
    pdftitle={Overleaf Example},
    pdfpagemode=FullScreen,
    }

\usepackage{enumitem}
\usepackage{nicematrix} %矩阵的package

\usepackage[textwidth=18cm]{geometry} % 设置页宽=18

\usepackage{blindtext}
\usepackage{bm}
\parindent=0pt
\setlength{\parindent}{2em} 
\usepackage{indentfirst}


\usepackage{xcolor}
\usepackage{titlesec}
\titleformat{\section}[block]{\color{blue}\Large\bfseries\filcenter}{}{1em}{}
\titleformat{\subsection}[hang]{\color{red}\Large\bfseries}{}{0em}{}
%\setcounter{secnumdepth}{1} %section 序号

\newtheorem{theorem}{Theorem}[section]
\newtheorem{lemma}[theorem]{Lemma}
\newtheorem{corollary}[theorem]{Corollary}
\newtheorem{proposition}[theorem]{Proposition}
\newtheorem{example}[theorem]{Example}
\newtheorem{definition}[theorem]{Definition}
\newtheorem{remark}[theorem]{Remark}
\newtheorem{exercise}{Exercise}[section]
\newtheorem{annotation}[theorem]{Annotation}

\newcommand*{\xfunc}[4]{{#2}\colon{#3}{#1}{#4}}
\newcommand*{\func}[3]{\xfunc{\to}{#1}{#2}{#3}}

\newcommand\Set[2]{\{\,#1\mid#2\,\}} %集合
\newcommand\SET[2]{\Set{#1}{\text{#2}}} %

\newcommand{\norm}[1]{\left\lVert#1\right\rVert} % 范数
\newcommand{\vect}[1]{\mathbf{#1}} % vector
\newcommand{\hints}{{\color{blue} \text{hints}}}
\newcommand{\mbf}[1]{\bm{#1}} 
\newcommand{\rank}[1]{\text{rank}\left(#1\right)} % rank
\newcommand\inp[2]{\langle #1, #2 \rangle} %inner product

\newcommand{\redt}[1]{\textcolor{red}{#1}}
\newcommand{\bluet}[1]{\textcolor{blue}{#1}}


\begin{document}
\title{考研高数习题集}
\author{枫聆}
\maketitle
\tableofcontents

\newpage
\section{行列式}

\subsection{定义}

\begin{annotation}
\rm 这类题特征
\begin{enumerate}
	\item 按照行列式的完全展开式来计算某种特殊的矩阵
	\item 给定某个具体的行列式值的基础上,通过行列式的性质来计算行列式. 
\end{enumerate}
\end{annotation}

\begin{example}
\rm 证明: 如果在$n$阶行列式中,第$i_1,i_2,\cdots,i_k$行分别与第$j_1,j_2,\cdots,j_l$列交叉位置的元素都是$0$,并且$k+l > n$,那么这个行列式的值等于$0$.
\begin{proof}
按照行列式的完全展开式,每一项都必须要包含第$i_1,i_2,\cdots,i_k$行中位于不同列的元素,即$k$个元素. 由已知的条件,第$i_1,i_2,\cdots,i_k$行只与$j_1,j_2,\cdots,j_l$之外的$n-l$元素可能不为零,但是$k > n-l$,说明每一项必取到0,因此行列式为$0$. 
\end{proof}
\end{example}

\begin{example}
\rm 证明
$$
\begin{vmatrix}
a_1 + c_1 &  b_1 + a_1 & c_1+b_1 \\
a_2 + c_2 &  b_2 + a_2 & c_2+b_2 \\
a_3 + c_3 &  b_3 + a_3 & c_3+b_3 \\
\end{vmatrix} = 2 \begin{vmatrix}
a_1 & b_1 & c_1 \\
a_2 & b_2 & c_2 \\
a_3 & b_3 & c_3 	
\end{vmatrix}
$$
\end{example}

\begin{example}
\rm \redt{行列式最大值} 求元素为$1$和$0$的三阶行列式可取的最大值.

\hints\ 从完全展开式我们应使得带正号的项尽可能的都是1,而带负号项尽可能是0
\begin{enumerate}
	\item 若3个正项都是1,那么此时行列式等于0;
	\item 若2个正项是1,例如
	$$
	\begin{vmatrix}
	0 & 1 & 1\\
	1 & 0 & 1\\
	1 & 1 & 0\\
	\end{vmatrix}
	$$
\end{enumerate}
综上最大值肯定就是$2$. 
\end{example}

\begin{example}
\rm 设$n \geq 2$,证明: 如果$n$阶矩阵$\mbf{A}$的元素为$1$或者$-1$,则$|A|$必为偶数. 

\hints\ 这个证明按行展开可能更简单,当$n=3$时
$$
|\mbf{A}| = a_{11}A_{11} + a_{12}A_{12} +a_{13}A_{13},
$$
那么我们来看一下元素为$1$或者$-1$的二阶行列式
$$
\begin{vmatrix}
b_{11} & b_{12} \\
b_{21} & b_{22} \\
\end{vmatrix}  = b_{11}b_{22} - b_{12}b{21}
$$
这里一共有4种可能的取值,由$b_{11}b_{22} = \pm 1, b_{12}b_{21} = \pm 1$决定,经过计算该二阶行列式取值可能为$-2,0,2$,因此$|A|$是由3个偶数相加得到的,那么$|A|$也一定是偶数. 类似可以推广到$n$阶行列式. 
\end{example}

\begin{example}
\rm 求元素为$1$或者$-1$的三阶行列式的最大值. 

\hints\ 从完全展开式出发,三阶行列式有6项,其中每一项只可能为$-1$和$1$. 再有前面证明,我们知道这样的三阶行列式的值只能是偶数,那么最大的偶数就是6项全为1加起来为6,即
$$
\begin{array}{ll}
a_{11}a_{22}a_{33} = 1, a_{12}a_{23}a_{31} = 1, a_{13}a_{21}a_{32} = 1 \\
-a_{13}a_{22}a_{31} = 1, -a_{12}a_{21}a_{33} = 1, -a_{11}a_{23}a_{32} = 1 
\end{array}
$$
由此得出
$$
\begin{array}{ll}
a_{11}a_{22}a_{33}a_{12}a_{23}a_{31}a_{13}a_{21}a_{32} = 1 \\
a_{13}a_{22}a_{31}a_{12}a_{21}a_{33}a_{11}a_{23}a_{32} = -1 
\end{array}
$$
这是矛盾的. 因此我们再考虑行列式最大值为$4$的可能,可以找到
$$
\begin{vmatrix}
1 & -1 & -1\\
-1 & 1 & -1 \\
1 & 1 & 1 
\end{vmatrix} = 1 + 1 + 1 + 1 - 1 + 1 = 4
$$
那么这样的行列式的最大值为$4$. 
\end{example}

\begin{example}
\rm 设$n \geq 3$,证明: 元素为$1$或者$-1$的$n$阶行列式的绝对值不超过$(n-1)!(n-1)$.

\hints\ 借助前面的例子做归纳. 
\end{example}

\begin{example}
\rm 设$n \geq 2$,证明: 元素为$1$或$-1$的$n$阶行列式的值能被$2^{n-1}$整除. 

\hints\ 设$|\mbf{A}|$是这样的行列式,首先将第一列上的$(-1)$所在行都提一个$-1$出来
$$
|\mbf{A}| = (-1)^m \begin{vmatrix}
1 & b_{12} & \cdots & b_{1n} \\
1 & b_{22} & \cdots & b_{2n} \\
\vdots & \vdots & \cdots & \vdots \\
1 & b_{n2} & \cdots & b_{nn} \\
\end{vmatrix} = (-1)^m \begin{vmatrix} 
1 & b_{12} & \cdots & b_{1n} \\
0 & c_{22} & \cdots & c_{2n} \\
\vdots & \vdots & \cdots & \vdots \\
0 & c_{n2} & \cdots & c_{nn} \\
\end{vmatrix}
$$
其中$c_{ij}$值可能为$2,-2,0$,因此从第$2$列到第$n$列都是可以提一个因子$2$出来,最终就可以凑成$2^{n-1}$. 
\end{example}



\subsection{化行阶梯形}

\begin{annotation}
\rm 不是特殊矩阵的第一选择.
\end{annotation}

\subsection{按一行展开}

\begin{annotation}
\rm 若是可以将某一行或者某一列消去,只留下一个非零元素,按行和按列展开是不错的选择. 
\end{annotation}

\begin{example}
\rm 计算
$$
|\mbf{A}| = \begin{vmatrix}
1 & 2 & 3 & \cdots & n-1 & n \\
1 & -1 & 0 & \cdots & 0 & 0 \\
0 & 2 & -2 & \cdots & 0 & 0 \\
\vdots & \vdots & \vdots &  & \vdots & \vdots \\
0 & 0 & 0 & \cdots & n-1 & 1-n \\
\end{vmatrix}
$$
\hints 可以考虑把所有列都加到第一列,再按第一列展开
$$
|\mbf{A}| = \begin{vmatrix}
\frac{(1+n)n}{2} & 2 & 3 & \cdots & n-1 & n \\
0 & -1 & 0 & \cdots & 0 & 0 \\
0 & 2 & -2 & \cdots & 0 & 0 \\
\vdots & \vdots & \vdots &  & \vdots & \vdots \\
0 & 0 & 0 & \cdots & n-1 & n-1 \\
\end{vmatrix} = \frac{(1+n)n}{2} \begin{vmatrix}
 -1 & 0 & \cdots & 0 & 0 \\
2 & -2 & \cdots & 0 & 0 \\
 \vdots & \vdots &  & \vdots & \vdots \\
  0 & 0 & \cdots & n-1 & n-1 \\
\end{vmatrix} 
$$
同样上述矩阵也是所有列加到第一列,最终有$|\mbf{A}| = (-1)^{n-1}\frac{(n+1)!}{2}$.
\end{example}

\subsection{按多行展开}

\begin{annotation}
\rm 好像没有直接使用拉普拉斯定理的习惯,比较特殊的分块矩阵可以考虑. 
\end{annotation}

\subsection{特殊矩阵}



\begin{annotation}
\rm 常见的特殊矩阵\url{https://www.bilibili.com/read/cv266516}
\begin{enumerate}
	\item 范德蒙德行列式
	\item 爪型行列式	
\end{enumerate}
\end{annotation}

\subsection{数学归纳法}

\begin{annotation}
\rm 通常证明手法也是按行或者列展开.
\end{annotation}

\begin{example}
\rm 计算$n$阶行列式
$$
\mbf{D}_n = \begin{vmatrix}
x & 0 & 0  & \cdots & 0 & 0 & a_0 \\
-1 & x & 0  & \cdots & 0 & 0 & a_1 \\
0 & -1 & x  & \cdots & 0 & 0 & a_2 \\
\vdots & \vdots & \vdots  & \cdots & \vdots & \vdots & \vdots \\
0 & 0 & 0  & \cdots & -1 & x & a_{n-2} \\
0 & 0 & 0  & \cdots & 0 & -1 & x+a_{n-1} \\
\end{vmatrix}
$$
\begin{proof}
当$n=2$时,有
$$
\mbf{D}_2 = \begin{vmatrix}
x & a_0 \\
-1 & x+a_1 
\end{vmatrix} = x^2 + a_1x+ a_0
$$
假设对于上述形式的$n-1$阶行列式,有
$$
\begin{vmatrix}
x & 0 & 0  & \cdots & 0 & 0 & a_0 \\
-1 & x & 0  & \cdots & 0 & 0 & a_1 \\
0 & -1 & x  & \cdots & 0 & 0 & a_2 \\
\vdots & \vdots & \vdots  & \cdots & \vdots & \vdots & \vdots \\
0 & 0 & 0  & \cdots & 0 & -1 & x+a_{n-2} \\
\end{vmatrix} = x^{n-1} + a_{n-2}x^{n-2} + \cdots + a_1x + a_0
$$
那么$n$阶行列式,把它按第一行展开,有
$$
\begin{array}{ll}
\mbf{D}_n &= x\begin{vmatrix}
 x & 0  & \cdots & 0 & 0 & a_1 \\
 -1 & x  & \cdots & 0 & 0 & a_2 \\
 \vdots & \vdots  & \cdots & \vdots & \vdots & \vdots \\
 0 & 0  & \cdots & -1 & x & a_{n-2} \\
 0 & 0  & \cdots & 0 & -1 & x+a_{n-1} \\
\end{vmatrix}+ (-1)^{1+n}a_0 \begin{vmatrix}
-1 & x & 0  & \cdots & 0 & 0  \\
0 & -1 & x  & \cdots & 0 & 0 \\
\vdots & \vdots & \vdots  & \cdots & \vdots & \vdots  \\
0 & 0 & 0  & \cdots & -1 & x  \\
0 & 0 & 0  & \cdots & 0 & -1  \\
\end{vmatrix}  \\
&=x(x^{n-1} + a_{n-1}x^{n-2} + \cdots + a_2x + a_1) + (-1)^{1+n}a_0(-1)^{n-1} \\
&=  x^n + a_{n-1}x^{n-1} + a_{n-2}x^{n-2} + \cdots + a_1x + a_0
\end{array}
$$
\end{proof}
\end{example}

\subsection{递推式}


\begin{example}
\rm 计算$n$阶行列式
$$
\mbf{D}_n = \begin{vmatrix}
2 & -1 & 0 & 0 & \cdots & 0 & 0 & 0\\
-1 & 2 & -1 & 0 & \cdots & 0 & 0 & 0\\
0 & -1 & 2 & -1 & \cdots & 0 & 0 & 0\\
\vdots & \vdots & \vdots & \vdots & & \vdots & \vdots & \vdots\\
0 & 0 & 0 & 0 & \cdots & -1 & 2 & -1\\
0 & 0 & 0 & 0 & \cdots & 0 & -1 & 2\\
\end{vmatrix}
$$
\hints\ 显然$\mbf{D}_1 = 2$. 将$\mbf{D}_n$按第一列展开,则有
$$
\mbf{D}_n = 2\mbf{D}_{n-1}+
\begin{vmatrix}
 -1 & 0 & 0 & \cdots & 0 & 0 & 0\\
 -1 & 2 & -1 & \cdots & 0 & 0 & 0\\
 \vdots & \vdots & \vdots & & \vdots & \vdots & \vdots\\
 0 & 0 & 0 & \cdots & -1 & 2 & -1\\
 0 & 0 & 0 & \cdots & 0 & -1 & 2\\
\end{vmatrix} = 2\mbf{D}_{n-1} - \mbf{D}_{n-2}
$$
这也意味着$\mbf{D}_n - \mbf{D}_{n-1} = \mbf{D}_{n-1}-\mbf{D}_{n-2}$,可以马上推出$\mbf{D}_n - \mbf{D}_{n-1} = \mbf{D}_2-\mbf{D}_1 = 1$,即该行列式是一个等差数列$D_n = 2+(n-1) = n+1$. 
\end{example}


\begin{example}
\rm 计算$n$阶行列式
$$
\mbf{D}_n = \begin{vmatrix}
a+b & ab & 0 & 0 & \cdots & 0 & 0 & 0\\
1 & a+b & ab & 0 & \cdots & 0 & 0 & 0\\
0 & 1 & a+b & ab & \cdots & 0 & 0 & 0\\
\vdots & \vdots & \vdots & \vdots & & \vdots & \vdots & \vdots\\
0 & 0 & 0 & 0 & \cdots & 1 & a+b & ab\\
0 & 0 & 0 & 0 & \cdots & 0 & 1 & a+b\\
\end{vmatrix}
$$
其中$a \neq b$.

\hints\ 还是按第一列展开
$$
\mbf{D}_n = (a+b)\mbf{D}_{n-1}-\begin{vmatrix}
 ab & 0 & 0 & \cdots & 0 & 0 & 0\\
 1 & a+b & ab & \cdots & 0 & 0 & 0\\
 \vdots & \vdots & \vdots & & \vdots & \vdots & \vdots\\
 0 & 0 & 0 & \cdots & 1 & a+b & ab\\
 0 & 0 & 0 & \cdots & 0 & 1 & a+b\\ 
\end{vmatrix} = (a+b)\mbf{D}_{n-1} -ab\mbf{D}_{n-2}. 
$$
注意其上最后一个等式成立的条件是$a \neq 0$和$b \neq 0$,那么推出
$$
\begin{array}{ll}
\mbf{D}_n - a\mbf{D}_{n-1} = b(\mbf{D}_{n-1}-a\mbf{D}_{n-2}) \Rightarrow \mbf{D}_n -a\mbf{D}_{n-1} = (\mbf{D}_2 - a\mbf{D}_1)b^{n-2}  \\
\mbf{D}_n - b\mbf{D}_{n-1} = a(\mbf{D}_{n-1}-b\mbf{D}_{n-2}) \Rightarrow \mbf{D}_n -b\mbf{D}_{n-1} = (\mbf{D}_2 - b\mbf{D}_1)a^{n-2}
\end{array}
$$
而$\mbf{D}_1 = a+b$,$\mbf{D}_2 = a^2 + ab + b^2$. 因此
$$
\begin{array}{ll}
\mbf{D}_n -a\mbf{D}_{n-1}  = b^n \\
\mbf{D}_n -b\mbf{D}_{n-1}  = a^n
\end{array}
$$
所以$D_n  = \frac{b^{n+1}-a^{n+1}}{b-a}$. 

当$a = 0$时,$\mbf{D}_n = b^n$; 当$b=0$时,$\mbf{D}_n=a^n$. 
\end{example}

\subsection{Laplace展开}

\begin{example}
\rm 计算下述$2n$阶行列式
$$
\mbf{D}_{2n} = \begin{vmatrix}
a  & & & &  & b \\
   & \ddots & & & \iddots \\
  & & a & b &  &  \\
  & & b & a &  & \\
  &  \iddots & & &  \ddots \\
  b& & & &  &a\\
\end{vmatrix}
$$
\hints\ 尝试从第一行和最后一行展开,于是得到
$$
\begin{array}{ll}
\mbf{D}_{2n} &= \begin{vmatrix}
a & b \\
b & a 
\end{vmatrix}\cdot (-1)^{(1+2n)+(1+2n)}\cdot \mbf{D}_{2n-2} \\
&=(a^2-b^2)\mbf{D}_{2n-2}
\end{array} 
$$
而$\mbf{D}_2=a^2 - b^2$,因此$\mbf{D}_{2n} = (a^2-b^2)^{n}$ 
\end{example}


\subsection{用矩阵运算拆开计算}

\begin{example}
\rm 设
$$
s_k = x_1^k +x_2^k + \cdots + x_n^k, ~ k = 0,1,2,\cdots, 
$$
设$\mbf{A} = (a_{ij})_{n \times n}$,其中
$$
a_{ij} = s_{i+j-2}, \, i,j = 1,2,\cdots,n.
$$
证明: $|\mbf{A}| = \prod\limits_{1 \leq j < i \leq n}(x_i - x_j)^2$. 

\hints\ 这个左边是范德蒙行列式的通项的平方. 其中
$$
a_{ij} = x_1^{i-1}x_1^{j-1} + x_2^{i-1}x_2^{j-1} + \cdots + x_n^{i-1}x_n^{j-1} = \sum\limits_{r=1}^n x_r^{i-1}x_r^{j-1}. 
$$
因此
$$
|\mbf{A}| = \begin{vmatrix}
1 & 1  & \cdots & 1 & 1 \\
x_1 & x_2 & \cdots & x_{n-1} & x_n \\
\vdots & \vdots &  & \vdots & \vdots \\
x_1^{n-2} & x_2^{n-2} & \cdots & x_{n-1}^{n-2} & x_n^{n-2} \\
x_1^{n-1} & x_2^{n-1} & \cdots & x_{n-1}^{n-1} & x_n^{n-1} \\
\end{vmatrix}
\begin{vmatrix}
1 & x_1  & \cdots & x_1^{n-2} & x_1^{n-1} \\
1 & x_2 & \cdots & x_2^{n-2} & x_2^{n-1} \\
\vdots & \vdots &  & \vdots & \vdots \\
1 & x_{n-1} & \cdots & x_{n-1}^{n-2} & x_{n-1}^{n-1} \\
1 & x_{n} & \cdots & x_{n}^{n-1} & x_n^{n-1} \\
\end{vmatrix}  
$$
显然这是两个范德蒙行列式的乘积
\end{example}

\begin{example}
\rm 设
$$
u_j = \sum\limits_{i=1}^{n}c_i{a_i}^{j},\, 1 \leq j < 2n. 
$$
令
$$
\mbf{A} = \begin{pmatrix}
u_1 & u_2 & \cdots & u_n \\
u_2 & u_3 & \cdots & u_{n+1} \\
\vdots & \vdots &  & \vdots \\
u_n & u_{n+1} & \cdots & u_{2n-1}\\
\end{pmatrix}
$$
证明: 对任意$\mbf{\beta} \in \mathbb{R}^n$,线性方程组$\mbf{A}\mbf{X} = \mbf{\beta}$有唯一解的充分必要条件是$a_1,a_2,\cdots,a_n$两两不等,且$a_i$和$c_i$全不为$0$($i=1,2,\cdots,n$). 

\hints\ 就是我们要考察行列式$|\mbf{A}|$. 观察$\mbf{A}$也是可以拆开的
$$
\begin{array}{ll}
\mbf{A} &= \begin{pmatrix}
c_1 & c_2 & \cdots & c_{n-1} & c_n \\
c_1a_1 & c_2a_1 & \cdots & c_{n-1}a_{n-1} & c_na_n \\
\vdots & \vdots &  & \vdots & \vdots \\
c_1a_1^{n-2} & c_2a_2^{n-2} & \cdots & c_{n-1}a_{n-1}^{n-2} & c_na_n^{n-2} \\
c_1a_1^{n-1} & c_2a_2^{n-1} & \cdots & c_{n-1}a_{n-1}^{n-1} & c_na_n^{n-1}
\end{pmatrix} 
\begin{pmatrix}
a_1 & a_1^2 & \cdots & a_1^{n-1} & a_1^n \\
a_2 & a_2^2 & \cdots & a_2^{n-1} & a_2^n \\
\vdots & \vdots &  & \vdots & \vdots \\
a_{n-1} & a_{n-1}^2 & \cdots & a_{n-1}^{n-1} & a_{n-1}^{n} \\
a_{n} & a_n^2 & \cdots & a_n^{n-1} & a_n^{n}  
\end{pmatrix} \\
&= c_1\cdots c_n a_1\cdots a_n \begin{pmatrix}
1 & 1 & \cdots & 1 & 1 \\
a_1 & a_2 & \cdots & a_{n-1} & a_n \\
\vdots & \vdots &  & \vdots & \vdots \\
a_1^{n-2} & a_2^{n-2} & \cdots & a_{n-1}^{n-2} & a_n^{n-2} \\
a_1^{n-1} & a_2^{n-1} & \cdots & a_{n-1}^{n-1} & a_n^{n-1}
\end{pmatrix} 
\begin{pmatrix}
1 & a_1 & \cdots & a_1^{n-2} & a_1^{n-1} \\
1 & a_2 & \cdots & a_2^{n-2} & a_2^{n-1} \\
\vdots & \vdots &  & \vdots & \vdots \\
1 & a_{n-1} & \cdots & a_{n-1}^{n-2} & a_{n-1}^{n-1} \\
1 & a_n & \cdots & a_n^{n-2} & a_n^{n-1}  
\end{pmatrix} \\
&= c_1\cdots c_2 a_1 \cdots a_n \sum\limits_{1 \leq j < i \leq n}(a_i -a_j)^2  
\end{array}
$$
\end{example}

\begin{example}
\rm 计算$n$阶行列式
$$
|\mbf{A}| = 
\begin{vmatrix}
a_1 - b_1 & a_1 - b_2 & \cdots & a_1 - b_n \\
a_2 - b_1 & a_2 - b_2 & \cdots & a_2 - b_n \\
\vdots & \vdots && \vdots \\
a_n - b_1 & a_n - b_2 & \cdots & a_n - b_n
\end{vmatrix}
$$

\hints\ 矩阵$\mbf{A}$是可以拆开的
$$
\mbf{A} = \begin{pmatrix}
a_1 & -1 \\
a_2 & -1 \\
\vdots & \vdots \\
a_{n-1} & -1 \\
a_n  & -1 
\end{pmatrix} \begin{pmatrix}
1 & 1 & \cdots & 1 & 1\\ 
b_1 & b_2 & \cdots & b_{n-1} & b_n
\end{pmatrix}
$$
当$n > 2$时,$\rank{\mbf{A}} \leq 2$,即$|\mbf{A}| = 0$. 

当$n = 2$时,$|\mbf{A}| = (a_2 - a_1)(b_2 - b_1)$.

当$n = 1$时,$|\mbf{A}| = a_1 - b_1$. 
\end{example}

\subsection{矩阵行列式Lemma}

\begin{proposition}\label{sylvester determinant}
\rm \redt{Weinstein–Aronszajn identity} 设$\mbf{A}$是$m \times n$矩阵,$\mbf{B}$是$n \times m$矩阵,则有
$$
|\mbf{I}-\mbf{A}\mbf{B}| = |\mbf{I}-\mbf{B}\mbf{A}|
$$
\end{proposition}

\begin{proof}
因为有
$$
\det\begin{pmatrix} I_m & A \\ B & I_n \end{pmatrix} 
  = \det\begin{pmatrix} I_m & 0 \\ B & I_n - BA \end{pmatrix} = \det(I_n - BA).
$$
和
$$
\det\begin{pmatrix} I_m & A \\ B & I_n \end{pmatrix} 
  = \det\begin{pmatrix} I_m-AB & 0 \\ B & I_n  \end{pmatrix} = \det(I_m - AB).
$$
\end{proof}

\begin{corollary}
\rm 给定两个列向量$\mbf{u},\mbf{v}$,则有
$$
|\mbf{I} + \mbf{u}\mbf{v}^{T}| = (1 + \mbf{v}^{T}\mbf{u}). 
$$
\end{corollary}

\begin{proof}
直接用\ref{sylvester determinant}即可. 
\end{proof}

\begin{corollary}
\rm 设$\mbf{A}$可逆,给定两个列向量$\mbf{u},\mbf{v}$,则有
$$
|\mbf{A} + \mbf{u}\mbf{v}^{T}| = (1+\mbf{v}^{T}\mbf{A}^{-1}\mbf{u})|\mbf{A}|.
$$
\end{corollary}

\begin{proof}
\rm 
$$
|\mbf{A} + \mbf{u}\mbf{v}^{T}| = |\mbf{A}(\mbf{I} - \mbf{A}^{-1}\mbf{u}\mbf{v}^{T})| = (1+\mbf{v}^{T}\mbf{A}^{-1}\mbf{u})|\mbf{A}|.
$$
\end{proof}

\newpage
\section{向量空间}

\subsection{向量运算下的线性性质判定}

\begin{example}
\rm 证明: 如果向量组$\mbf{\alpha}_1,\mbf{\alpha}_2,\mbf{\alpha}_3$线性无关,那么向量组$3\mbf{\alpha}_1-\mbf{\alpha}_2,5\mbf{\alpha}_2 + 2\mbf{\alpha}_3, 4\mbf{\alpha}_3-7\mbf{\alpha}_1$

\hints\ 直接用线性无关的定义,即
$$
k_1(3\mbf{\alpha}_1-\mbf{\alpha}_2) + k_2(5\mbf{\alpha}_2 + 2\mbf{\alpha}_3)+ k_3(4\mbf{\alpha}_3-7\mbf{\alpha}_1) = \mbf{0}
$$
此时需要证明$k_1 = k_2 = k_3$,展开上式
$$
(3k_1-7k_3)\mbf{\alpha}_1 + (5k_2 - k_1)\mbf{\alpha}_2 + (2k_2 +4k_3)\mbf{\alpha}_3 = 0. 
$$
根据已知条件,于是得到下述方程组
$$
\left \{
\begin{array}{ll}
3k_1-7k_3  = 0 \\
5k_2 - k_1 = 0 \\
2k_2 +4k_3 = 0
\end{array} \right.
$$
由此构造系数矩阵,最后系数矩阵的行列式不为零,即原方程只有零解,命题得证. 
\end{example}

\subsection{向量组的极大线性无关组}

\begin{annotation}
\rm 若给定$n$个列向量$\mbf{\alpha}_1,\cdots,\mbf{\alpha}_n$,操作步骤为
\begin{enumerate}
	\item 写出对应矩阵的形式$\mbf{A}$,做初等行变换化行阶梯型矩阵$\mbf{D}$.
	\item 观察行列式不为零最大子式所在的列,它们构成一个极大线性无关组. 
\end{enumerate}
\end{annotation}


\newpage
\section{秩}

\subsection{特殊矩阵的秩}

\begin{example}
\rm 设$n$阶矩阵$\mbf{A}$,满足
$$
|a_{ii}| >  \sum\limits_{j = 1 \atop j \neq i}^{n} |a_{ij}|, \, i = 1,2,\cdots,n.
$$
证明$\mbf{A}$的秩等于$n$. 这样的矩阵称为\redt{主对角占优矩阵}. 

\hints\  设$\mbf{A}$的列向量为$\mbf{\alpha}_1,\mbf{\alpha}_2,\cdots,\mbf{\alpha}_n$,那么只需证明$\mbf{A}$的列向量都是线性无关的. 假设$\mbf{A}$的列向量是线性相关的,则存在
$$
k_1\mbf{\alpha}_1 + k_2\mbf{\alpha}_2 + \cdots + k_n\mbf{\alpha}_n = \mbf{0},
$$
其中$k_1,k_2,\cdots,k_n$不全为$0$. 取
$$
|k_l| = \max\{|k_1|,|k_2|,\cdots,|k_n|\}. 
$$
然后我们看第$l$个分量有
$$
k_1a_{l1} + k_1a_{l2} + \cdots + k_na_{ln} = 0,
$$
等式两边除以$k_l$,得到
$$
a_{ll} = - \frac{k_1}{k_l}a_{l1} - \frac{k_2}{k_l}a_{l2} - \cdots -  \frac{k_n}{k_l}a_{ln} =  -\sum\limits_{j = 1 \atop j \neq l}^{n} \frac{k_j}{k_l} a_{lj}.
$$
因此有
$$
|a_{ll}| \leq \sum\limits_{j = 1 \atop j \neq l}^{n} |a_{lj}|,
$$
与前提条件矛盾. 从而$\mbf{A}$等于$n$. 
\end{example}

\subsection{矩阵运算中的秩}

\begin{example}
\rm 证明: $\rank{\mbf{A}+\mbf{B}} \leq \rank{\mbf{A}} + \rank{\mbf{B}}$.

\hints\ 设$\mbf{A} = (\mbf{\alpha}_1,\mbf{\alpha}_2,\cdots,\mbf{\alpha}_n),\mbf{B}=(\mbf{\beta}_1,\mbf{\beta}_2,\cdots,\mbf{\beta}_n)$,那么$\mbf{A}+\mbf{B} = (\mbf{\alpha}_1+\mbf{\beta}_1,\mbf{\alpha}_2+\mbf{\beta}_2,\cdots,\mbf{\alpha}_n+\mbf{\beta}_n)$. 设$\mbf{A}$列向量的一个极大无关组$\mbf{\alpha}_{i_1},\mbf{\alpha}_{i_2},\cdots,\mbf{\alpha}_{i_n}$,$\mbf{B}$列向量的一个极大无关组$\mbf{\beta}_{j_1},\mbf{\beta}_{j_2},\cdots,\mbf{\beta}_{j_n}$. 显然$\mbf{\alpha}_1+\mbf{\beta}_1,\mbf{\alpha}_2+\mbf{\beta}_2,\cdots,\mbf{\alpha}_n+\mbf{\beta}_n$可以由$\mbf{\alpha}_{i_1},\mbf{\alpha}_{i_2},\cdots,\mbf{\alpha}_{i_n},\mbf{\beta}_{j_1},\mbf{\beta}_{j_2},\cdots,\mbf{\beta}_{j_n}$线性表出. 
\end{example}


\begin{example}
\rm \redt{矩阵的秩的相等可转化为对应的齐次线性方程组通解} 设$\mbf{A}$为$m \times n$矩阵,则
$$
\rank{\mbf{A}\mbf{A}^T} = \rank{\mbf{A}^T\mbf{A}} = \rank{\mbf{A}}. 
$$
\hints\ 在note上记录了一种化行阶梯形的证明方法,这里从判定$n$元齐次线性方程组$(\mbf{A}^T\mbf{A})\mbf{X}=\mbf{0}$与$\mbf{A}\mbf{X}= \mbf{0}$同解出发,同解意味着解空间维数相同,从而对应的系数矩阵的秩相同. 

若$\mbf{\eta}$是$\mbf{A}\mbf{X}= \mbf{0}$的一个解,那么显然$\mbf{\eta}$也是$(\mbf{A}^T\mbf{A})\mbf{X}=\mbf{0}$的解. 

若$\mbf{\delta}$是$(\mbf{A}^T\mbf{A})\mbf{X}=\mbf{0}$的解,于是有$(\mbf{A}^T\mbf{A})\mbf{\delta}=\mbf{0}$,将这个等式两边乘上$\mbf{\delta}^T$,得到
$$
\mbf{\delta}^T(\mbf{A}^T\mbf{A})\mbf{\delta}=\mbf{0} \Rightarrow (\mbf{\delta}^T\mbf{A}^T)(\mbf{A}\mbf{\delta})=\mbf{0} \Rightarrow (\mbf{A}\mbf{\delta})^T(\mbf{A}\mbf{\delta})=\mbf{0}
$$
那么设$(\mbf{A}\mbf{\delta})^T = (c_1,c_2,\cdots,c_n)$,那么则有
$$
c_1^2 + c_2^2 + \cdots + c_n^2 = 0,
$$
因此$c_1 = c_2 = \cdots = c_n = 0$,所以$\mbf{\delta}$也是$\mbf{A}\mbf{X}= \mbf{0}$的解. 

最后
$$
\rank{\mbf{A}\mbf{A}^T} = \rank{(\mbf{A}^T)^T\mbf{A}^T} = \rank{\mbf{A}^T} = \rank{\mbf{A}}.   
$$
\end{example}

\begin{example}
\rm 设$\mbf{A}$是$n$阶矩阵,证明对任意的正整数$k$,有
$$
\rank{\mbf{A}^{n+k}} = \rank{\mbf{A}^n}.
$$

\hints\ 若$\mbf{A}$可逆,命题是显然的; 若$\mbf{A}$不可逆,即$\rank{A} < n$,那么矩阵乘法秩的上界,我们有
$$
\rank{\mbf{A}} \geq \rank{\mbf{A}^2} \geq \cdots \geq \rank{\mbf{A}^n} \geq \rank{\mbf{A}^{n+1}}
$$
那么肯定存在一个$m \leq n$,使得上述等号成立,即
$$
\mbf{A}^m = \mbf{A}^{m+1},
$$
因此对任意的$k \geq 1$有
$$
\mbf{A}^n = \mbf{A}^{n+k},
$$
\end{example}

\subsection{秩为1的矩阵相关问题}

\begin{example}
\rm MSE上相关问题收集\href{https://math.stackexchange.com/questions/904926/determinant-of-a-rank-1-update-of-a-scalar-matrix-or-characteristic-polynomia}{determinant-of-a-rank-1-update-of-a-scalar-matrix-or-characteristic-polynomia}
\begin{enumerate}
	\item \href{https://math.stackexchange.com/q/153457/18880}{元素全是$1$的$n$阶矩阵的特征多项式通项}
	\item \href{https://math.stackexchange.com/q/55165/18880}{秩为$1$的$n$矩阵的特征值$uv^T$}
	\item \href{https://math.stackexchange.com/q/577937/18880}{计算$A+I$的行列式,其中$A$为秩为1的$n$阶矩阵}
	\item \href{https://math.stackexchange.com/q/84206/18880}{计算对角线元素都是$0$,其他元素都是$1$的$n$阶矩阵的行列式}
	\item \href{https://math.stackexchange.com/q/86644/18880}{计算对角线元素都是$a$,其他元素都是$b$的$n$阶矩阵的行列式}
	\item \href{https://math.stackexchange.com/questions/629892/determinant-of-a-special-n-times-n-matrix}{计算对角线元素都是$2$,其他元素都是$1$的$n$阶矩阵的行列式}
\end{enumerate}
\end{example}

\subsection{两个矩阵的秩相加}

\begin{lemma}
\rm \redt{构造新的分块矩阵} 给定矩阵$\mbf{A}=(a_{ij})_{m \times s},\mbf{B}=(b_{ij})_{s \times n}$,则
$$
\rank{\begin{pmatrix}
\mbf{A} & \mbf{0} \\
\mbf{0}	& \mbf{B} 
\end{pmatrix}} = \rank{\mbf{A}}+\rank{\mbf{B}}
$$
\end{lemma}

\begin{example}
\rm 若$n$阶矩阵满足$\mbf{A}^2 = \mbf{A}$,那么称$\mbf{A}$是幂等矩阵. 证明
$\mbf{A}$是幂等矩阵当且仅当
$$
\rank{\mbf{A}} + \rank{\mbf{I}-\mbf{A}} = n.
$$
\hints\ 若$\mbf{A}^2 = \mbf{A}$,那么$\mbf{A}^2 - \mbf{A} = \mbf{0}$. 由题意构造分块矩阵
$$
\begin{pmatrix}
\mbf{A} & \mbf{0} \\
\mbf{0} & \mbf{I}-\mbf{A}
\end{pmatrix}
$$
做对应的初等变换
$$
\begin{pmatrix}
\mbf{A} & \mbf{0} \\
\mbf{0} & \mbf{I}-\mbf{A}
\end{pmatrix} \longrightarrow 
\begin{pmatrix}
\mbf{A} & \mbf{0} \\
\mbf{A} & \mbf{I}-\mbf{A}
\end{pmatrix} \longrightarrow
\begin{pmatrix}
\mbf{A} & \mbf{A} \\
\mbf{A} & \mbf{I} 
\end{pmatrix} \longrightarrow
\begin{pmatrix}
\mbf{A}-\mbf{A}^2 & \mbf{0} \\
\mbf{A} & \mbf{I} 
\end{pmatrix} \longrightarrow
\begin{pmatrix}
\mbf{A}-\mbf{A}^2 & \mbf{0} \\
\mbf{0} & \mbf{I} 
\end{pmatrix}
$$
于是得到等式
$$
\rank{\mbf{A}} + \rank{\mbf{I}-\mbf{A}} = \rank{\mbf{A}-\mbf{A}^2} + n
$$ 
因此$\rank{\mbf{A}} + \rank{\mbf{I}-\mbf{A}} = n$. 反过来若$\rank{\mbf{A}} + \rank{\mbf{I}-\mbf{A}}$,还是构造上述分块矩阵,不难看出来易得$\mbf{A}^2 = \mbf{A}$. 


\end{example}

\begin{example}
\rm 
\end{example}

\newpage
\section{方程组的解}

\subsection{方程组性质}

\begin{example}
\rm 证明: 给定$s$个$n$元线性方程组成的线性方程组,如果该方程组的增广矩阵的第$i$个行向量$\mbf{\alpha}_i$可以由其余行向量线性表出,即
$$
\mbf{\alpha}_i = k_1\mbf{\alpha}_1 + k_{i-1}\mbf{\alpha}_{i-1} + k_{i+1}\mbf{\alpha}_{k+1} + \cdots + k_s\mbf{\alpha}_s 
$$
那么将第$i$个方程去掉之后得到的方程组与原方程组同解. 
\end{example}

\begin{example}
\rm 设一个$m \times n$矩阵$H$的列向量组为$\mbf{\alpha}_1,\cdots,\mbf{\alpha}_n$. 证明: $H$的任意$s$列都线性无关当且仅当,齐次线性方程组
$$
x_1 \mbf{\alpha}_1 + \cdots + x_n \mbf{\alpha}_n = \mbf{0}
$$
的任一非零解的非零分量的数目大于$s$. 

\hints\ \emph{必要性}\ 若$H$的任意$s$列都线性无关. 假设上述线性方程存在一个非零解为
$$
\eta = (0,\cdots,c_{i_1},\cdots,c_{i_l},\cdots)^{T}. 
$$
其中$c_{i_1},\cdots,c_{i_l}$均不为零,且$ l \leq s$. 则
$$
c_{i_1}\mbf{\alpha}_{i_1}+\cdots+c_{i_l}\mbf{\alpha}_{i_j} = \mbf{0},
$$
这意味着存在$l$列线性相关,与前提数矛盾,因此任一非零解的非零分量的数目大于$s$

\emph{充分性}\ 若$H$的任一非零向量的非零分量的数目大于$s$. 假设有$l\leq s$个列向量线性相关
$$
k_{i_1}\mbf{\alpha}_{i_1}+\cdots+k_{i_l}\mbf{\alpha}_{i_l} = \mbf{0},
$$
那么存在一个非零解,即
$$
\eta = (0,\cdots,k_{i_1},\cdots,k_{i_l},\cdots)^{T}. 
$$
就是将其他分量扩充为0即可,这样和前提是矛盾的. 因此$H$的任意$s$列都线性无关. 
\end{example}

\begin{example}
\rm 设$\mbf{A}$为$m \times n$矩阵,证明: 对于任意$\mbf{\beta} \in \mathbb{R}^m$,线性方程组$\mbf{A}^T\mbf{A}\mbf{X} = \mbf{A}^T\mbf{\beta}$有解. 

\hints\ 直接套线性方程有解的充要条件,证明增广矩阵$(\mbf{A}^T\mbf{A},\mbf{A}^T\mbf{\beta})$和系数矩阵$\mbf{A}^T\mbf{A}$的秩相同即可. 显然有
$$
\rank{(\mbf{A}^T\mbf{A},\mbf{A}^T\mbf{\beta})} \geq \rank{\mbf{A}^T\mbf{A}}. 
$$
利用分块矩阵乘法,也有
$$
\rank{(\mbf{A}^T\mbf{A},\mbf{A}^T\mbf{\beta})} = \rank{\mbf{A}^T(\mbf{A},\mbf{\beta})} \leq \rank{\mbf{A}^T\mbf{A}}.
$$
因此$\rank{(\mbf{A}^T\mbf{A},\mbf{A}^T\mbf{\beta})} = \rank{\mbf{A}^T\mbf{A}}$. 
\end{example}

\subsection{给定方程组解的情况}


\subsection{带参数的方程组解的情况}


\subsection{线性方程充要条件}

\begin{annotation}
\rm \redt{系数矩阵的秩和增广矩阵的秩相同}!
\end{annotation}

\begin{example}
\rm 证明: 线性方程组
$$
\left\{
\begin{array}{l}
a_{11}x_1 + a_{12}x_2 + \cdots + a_{1n}x_n = b_1, \\
a_{21}x_1 + a_{22}x_2 + \cdots + a_{2n}x_n = b_2, \\
\cdots \\
a_{m1}x_1 + a_{m2}x_2 + \cdots + a_{mn}x_n = b_m, \\
\end{array} \right.
$$
有解的当且仅当下述线性方程
$$
\left\{
\begin{array}{l}
a_{11}x_1 + a_{21}x_2 + \cdots + a_{m1}x_m = 0, \\
a_{12}x_1 + a_{22}x_2 + \cdots + a_{m2}x_m = 0, \\
\cdots \\
a_{1n}x_1 + a_{2n}x_2 + \cdots +a_{mn}x_m = 0, \\
b_{1}x_1 + b_{2}x_2 + \cdots+ b_{m}x_m = 1, \\
\end{array} \right.
$$
无解. 
\hints\ 设第一个方程组的系数矩阵为$\mbf{A}$,增广矩阵为$\tilde{\mbf{A}}$. 那么第二个方程组的系数矩阵及增广矩阵分别为
$$
\begin{array}{ll}
\mbf{B} = \tilde{\mbf{A}}^T \\
\tilde{\mbf{B}}= \begin{pmatrix}
\mbf{A}^T & \mbf{0} \\\
\mbf{\beta} & 1
\end{pmatrix}
\end{array}
$$
其中$\mbf{\beta} = (b_1,\cdots,b_m)$. 这里存在一些等式
$$
\begin{array}{ll}
\rank{\mbf{B}} = \rank{\tilde{\mbf{A}}} \\
\rank{\tilde{\mbf{B}}} = \rank{\mbf{A}} + 1.
\end{array}
$$

当方程组(1)有解时,即$\rank{\mbf{A}} = \rank{\tilde{\mbf{A}}}$时,则$\rank{\mbf{B}} < \rank{\tilde{\mbf{B}}}$,因此方程组(2)无解. 

当方程组(2)无解时,即$\rank{\mbf{B}} < \rank{\tilde{\mbf{B}}}$时,则$\rank{\tilde{\mbf{A}}} + 1 = \rank{\mbf{A}} + 1$(系数矩阵和增广矩阵的秩相差不超过$1$),因此方程组(1)有解.
\end{example}

\subsection{齐次线性方程组的性质}

\begin{example}
\rm 设$n$个的方程的$n$元齐次线性方程组的系数矩阵$\mbf{A}$的行列式等于$0$,且$\mbf{A}$的$(k,l)$元的代数余子式$A_{kl} \neq 0$. 证明$\mbf{\eta}=(A_{k1},A_{k2},\cdots,A_{kn})^T$是原齐次线性方程组的一个基础解系. 

\hints\ 证明的要求暗示了解空间是一维的. 这是因为由$A_{kl} \neq 0$,意味着$\mbf{A}$存在一个$n-1$阶的子式行列式不为0,而$|\mbf{A}| = 0$,因此$\rank{\mbf{A}} = n-1$. 

将该$\mbf{\eta}$带入原方程,当$i = k$时
$$
a_{i1}A_{k1} + a_{i1}A_{k1} +\cdots +a_{in}A_{kn} = |\mbf{A}| = 0; 
$$
当$i \neq k$时,显然有
$$
a_{i1}A_{k1} + a_{i1}A_{k1} +\cdots +a_{in}A_{kn} = 0.
$$
因此$\mbf{\eta}$的确是原方程组的一个解,其中第$k$个分量$A_{kl} \neq 0$,结合解空间是一维的,从而$\mbf{\eta}$是一个基础解系. 
\end{example}

\subsection{非齐次线性方程组的性质}

\begin{example}
\rm 给定$n$元非齐次线性方程组
$$
x_1\mbf{\alpha}_1 + x_2\mbf{\alpha}_2 + \cdots + x_n\mbf{\alpha}_n = \mbf{\beta},
$$
它的一个特解为$\mbf{\gamma}_0$和基础解系为$\mbf{\eta}_1,\mbf{\eta}_2,\cdots,\mbf{\eta}_t$. 令
$$
\mbf{\gamma}_1 = \mbf{\gamma}_0 + \mbf{\eta}_1, \mbf{\gamma}_2 = \mbf{\gamma}_0 + \mbf{\eta}_2,\cdots,\mbf{\gamma}_n = \mbf{\gamma}_0 + \mbf{\eta}_t.
$$
那么该非齐次线性方程的解集可以表示为
$$
U = \Set{k_0\mbf{\gamma}_0+k_1\mbf{\gamma}_1+\cdots+k_t\mbf{\gamma}_t}{k_0 + k_1 + \cdots + k_t = 1, k_i \in \mathbb{R},i =0,2,\cdots,t}
$$
\bluet{这个例子表明非齐次线性方程组也可以用有限个线性无关的解向量来表示解空间,但是系数上是有限制的}.
\end{example}

\newpage
\section{矩阵运算}

\subsection{矩阵乘法性质}

\begin{example}
\rm 设$\mbf{A}$和$\mbf{B}$均为$n$阶矩阵. 若$\mbf{A}^2 = \mbf{B}^2$,不一定有$\mbf{A} = \mbf{B}$或者$\mbf{A} = -\mbf{B}$.

\hints\ 首先$\mbf{A}$和$\mbf{B}$不一定交换,即
$$
\mbf{A}^2 - \mbf{B}^2 \neq (\mbf{A} + \mbf{B})(\mbf{A} - \mbf{B}).
$$
即使有$\mbf{A}$和$\mbf{B}$交换,即
$$
(\mbf{A} + \mbf{B})(\mbf{A} - \mbf{B}) = \mbf{0},
$$
也不一定有$\mbf{A} + \mbf{B} = \mbf{0}$或者$\mbf{A} - \mbf{B} = 0$. 例如$\mbf{A} = \mbf{I}_2$,$\mbf{B} = \begin{pmatrix}
1 & 0 \\
0 & -1
\end{pmatrix}.$
\end{example}

\subsection{幂运算}


\begin{example}
\rm 设$\mbf{A} = \begin{pmatrix}
2 & 3 \\
0 & 2
\end{pmatrix}$,求$\mbf{A}^m$. 

\hints\ 将$\mbf{A}$拆开
$$
\mbf{A} = \begin{pmatrix}
2 & 0 \\
0 & 2 
\end{pmatrix} + \begin{pmatrix}
0 & 3 \\
0 & 0
\end{pmatrix} = 2\mbf{I} + 3\mbf{B}.
$$
其中$\mbf{B}= \begin{pmatrix}
0 & 1 \\
0 & 0
\end{pmatrix}$. 我们的目标很明确了是尝试用二项式展开,因为这里
$$
\mbf{B}^2 = \mbf{0}. 
$$
还有一个关键是$\mbf{B}$和$\mbf{I}$是可交换的,因此展开式的每一项都可以写成
$$
2^i3^j \mbf{I}^i \mbf{B}^j,
$$ 
其中$i+j=m$,不然这里是不能用二项式展开的. 于是
$$
\mbf{A}^m = (2\mbf{I} + 3\mbf{B})^m = (2\mbf{I})^m + C_m^1 (2\mbf{I})^{m-1}3\mbf{B} = 2^m \mbf{I} + 3m 2^{m-1}\mbf{B} = \begin{pmatrix}
2^m & 3m2^{m-1} \\
0 & 2^m 
\end{pmatrix}
$$
\end{example}


\begin{example}
\rm 设$\mbf{A} = \begin{pmatrix}
a & c \\
0 & b
\end{pmatrix}$,求$\mbf{A}^m$. 


\hints\ 这里是不能用二项式展开的,因为不确定$\begin{pmatrix}
a & 0 \\
0 & b\\
\end{pmatrix}$和$\begin{pmatrix}
0 & c \\
0 & 0
\end{pmatrix}$是否交换. 可以先写出前几项,猜结果用数学归纳法来证. 
$$
\begin{array}{ll}
\mbf{A}^2 = \begin{pmatrix}
a & c \\
0 & b\\
\end{pmatrix}\begin{pmatrix}
a & c \\
0 & b\\
\end{pmatrix} = \begin{pmatrix}
a^2 & c(a+b) \\
0 & b^2 
\end{pmatrix} \\
 \mbf{A}^3 = \begin{pmatrix}
a^2 & c(a+b) \\
0 & b^2 
\end{pmatrix}\begin{pmatrix}
a & c \\
0 & b\\
\end{pmatrix} = \begin{pmatrix}
a^3 & c(a^2 + ab + b^2) \\
0 & b^3  
\end{pmatrix} \\
 \mbf{A}^4 = \begin{pmatrix}
a^3 & c(a^2 + ab + b^2) \\
0 & b^3  
\end{pmatrix}\begin{pmatrix}
a & c \\
0 & b\\
\end{pmatrix} = \begin{pmatrix}
a^4 & c(a^3 + a^2b + ab^2 + b^3) \\
0 & b^4  
\end{pmatrix}
\end{array} 
$$
因此我们猜测其通项为
$$
\mbf{A}^m = \begin{pmatrix}
a^m & c(a^{m-1} + a^{m-2}b + \cdots + ab^{m-2}+b^{m-1}) \\
0 & b^m
\end{pmatrix}
$$
当$m = 2$时,显然是满足. 假设$m = k-1$时满足,考虑$m = k$时. 根据幂运算,有
$$
\begin{array}{ll}
\mbf{A}^k = \mbf{A}^{k-1} \mbf{A} = \mbf{A}^m &= \begin{pmatrix}
a^{k-1} & c(a^{k-2} + a^{k-3}b + \cdots + ab^{k-3}+b^{k-2}) \\
0 & b^{k-1}
\end{pmatrix}\begin{pmatrix}
a & c \\
0 & b\\
\end{pmatrix} \\
&= \begin{pmatrix}
a^{k} & c(a^{k-1} + a^{k-2}b + \cdots + ab^{k-2}+b^{k-1}) \\
0 & b^{k}
\end{pmatrix}
\end{array}
$$
\end{example}

\begin{example}\label{power-of-matrix:m1}
\rm 求下述$n$阶矩阵的$m$阶幂
$$
\mbf{A} = \begin{pmatrix}
0 & 1 & 0 & \cdots & 0 \\
0 & 0 & 1 & \cdots & 0 \\
\vdots & \vdots & \vdots & \cdots & \vdots \\
0 & 0 & 0 & \cdots & 1 \\
0 & 0 & 0 & \cdots & 0 
\end{pmatrix}
$$

\hints\ 首先可以简单的计算前几项
$$
\mbf{A}^2 = \begin{pmatrix}
0 & 1 & 0 & \cdots & 0 \\
0 & 0 & 1 & \cdots & 0 \\
\vdots & \vdots & \vdots & \cdots & \vdots \\
0 & 0 & 0 & \cdots & 1 \\
0 & 0 & 0 & \cdots & 0 
\end{pmatrix} \begin{pmatrix}
0 & 1 & 0 & \cdots & 0 \\
0 & 0 & 1 & \cdots & 0 \\
\vdots & \vdots & \vdots & \cdots & \vdots \\
0 & 0 & 0 & \cdots & 1 \\
0 & 0 & 0 & \cdots & 0 
\end{pmatrix} = 
\begin{pmatrix}
0 & 0 & 1 & 0 & \cdots & 0 \\
0 & 0 & 0 & 1 & \cdots & 0 \\
\vdots & \vdots & \vdots & \vdots &  & \vdots \\
0 & 0 & 0 & 0 & \cdots & 1 \\
0 & 0 & 0 & 0 & \cdots & 0 \\
0 & 0 & 0 & 0 & \cdots & 0  
\end{pmatrix}
$$
可以看出来把每一行的$1$都往右移了一列,因此我们假设当$m < n$时,有

$$
\mbf{A}^m = \begin{pNiceMatrix}[name=matrix_1, ]
0 & 0 & \cdots & 0 & 1 & 0 & \cdots & 0 \\
0 & 0 & \cdots & 0 & 0 & 1 & \cdots & 0 \\
\vdots & \vdots &  & \vdots & \vdots & \vdots &  & \vdots \\
0 & 0 & \cdots & 0 & 0 & 0 & \cdots & 1 \\
0 & 0 & \cdots & 0 & 0 & 0 & \cdots & 0 \\
\vdots & \vdots &  & \vdots & \vdots & \vdots &  & \vdots \\
0 & 0 & \cdots & 0 & 0 & 0 & \cdots & 0  
\CodeAfter
\begin{tikzpicture}[decoration={calligraphic brace, amplitude=6pt,raise=5pt}] \draw[decorate,thick] (matrix_1-1-1.north west) -- (matrix_1-1-4.north east) node[midway,yshift=+1.5em]{m};
\draw[decorate, thick] (matrix_1-4-8.north east) -- (matrix_1-7-8.south east) node[midway,xshift=+1.5em]{m};
\end{tikzpicture}
\end{pNiceMatrix} 
$$
当$n=1$时,显然命题成立. 假设当$n = m-1$成立,那么

$$
\mbf{A}^m = \begin{pNiceMatrix}[name=matrix_2, ]
0 & 0 & \cdots & 0 & 1 & 0 & \cdots & 0 \\
0 & 0 & \cdots & 0 & 0 & 1 & \cdots & 0 \\
\vdots & \vdots &  & \vdots & \vdots & \vdots &  & \vdots \\
0 & 0 & \cdots & 0 & 0 & 0 & \cdots & 1 \\
0 & 0 & \cdots & 0 & 0 & 0 & \cdots & 0 \\
\vdots & \vdots &  & \vdots & \vdots & \vdots &  & \vdots \\
0 & 0 & \cdots & 0 & 0 & 0 & \cdots & 0  
\CodeAfter
\begin{tikzpicture}[] \draw[decorate,decoration={calligraphic brace, amplitude=6pt,raise=5pt},thick] (matrix_2-1-1.north west) -- (matrix_2-1-4.north east) node[midway,yshift=+1.5em]{m-1};
\draw[decorate, decoration={calligraphic brace, mirror, amplitude=6pt,raise=5pt}, thick] (matrix_2-4-1.north west) -- (matrix_2-7-1.south west) node[midway,xshift=-2em]{m-1};
\end{tikzpicture}
\end{pNiceMatrix} \begin{pmatrix}
0 & 1 & 0 & \cdots & 0 \\
0 & 0 & 1 & \cdots & 0 \\
\vdots & \vdots & \vdots & \cdots & \vdots \\
0 & 0 & 0 & \cdots & 1 \\
0 & 0 & 0 & \cdots & 0 
\end{pmatrix} = \begin{pNiceMatrix}[name=matrix_3, ]
0 & 0 & \cdots & 0 & 1 & 0 & \cdots & 0 \\
0 & 0 & \cdots & 0 & 0 & 1 & \cdots & 0 \\
\vdots & \vdots &  & \vdots & \vdots & \vdots &  & \vdots \\
0 & 0 & \cdots & 0 & 0 & 0 & \cdots & 1 \\
0 & 0 & \cdots & 0 & 0 & 0 & \cdots & 0 \\
\vdots & \vdots &  & \vdots & \vdots & \vdots &  & \vdots \\
0 & 0 & \cdots & 0 & 0 & 0 & \cdots & 0  
\CodeAfter
\begin{tikzpicture}[decoration={calligraphic brace, amplitude=6pt,raise=5pt}] \draw[decorate,thick] (matrix_3-1-1.north west) -- (matrix_3-1-4.north east) node[midway,yshift=+1.5em]{m};
\draw[decorate, thick] (matrix_3-4-8.north east) -- (matrix_3-7-8.south east) node[midway,xshift=+1.5em]{m};
\end{tikzpicture}
\end{pNiceMatrix} 
$$
因此假设成立. 当$m = n$时,$\mbf{A}^n  =\mbf{0}$,即当$m \geq n$时,有$\mbf{A}^m = \mbf{0}$. 
\end{example}

\subsection{矩阵交换性}

\begin{example}
\rm 求与矩阵$\mbf{A}$交换的所有矩阵,设
$$
\mbf{A} = \begin{pmatrix}
3 & 1 & 0 \\
0 & 3 & 1 \\
0 & 0 & 3
\end{pmatrix}
$$

\hints\ 首先拆一下$\mbf{A}$,即
$$
\mbf{A} = \begin{pmatrix}
3 & 0 & 0 \\
0 & 3 & 0 \\
0 & 0 & 3
\end{pmatrix} + \begin{pmatrix}
0 & 1 & 0 \\
0 & 0 & 1 \\
0 & 0 & 0
\end{pmatrix} = 3\mbf{I} + \mbf{B}
$$
其中$I$与任意矩阵都是可交换,于是
$$
\mbf{A}\mbf{X} = \mbf{X}\mbf{A} \Leftrightarrow \mbf{B}\mbf{X} = \mbf{X}\mbf{B}
$$
因此有等式
$$
\begin{pmatrix}
0 & 1 & 0 \\
0 & 0 & 1 \\
0 & 0 & 0
\end{pmatrix} \begin{pmatrix}
x_{11} & x_{12} & x_{13} \\
x_{21} & x_{22} & x_{23} \\
x_{31} & x_{32} & x_{33}
\end{pmatrix}  = \begin{pmatrix}
x_{11} & x_{12} & x_{13} \\
x_{21} & x_{22} & x_{23} \\
x_{31} & x_{32} & x_{33}
\end{pmatrix} \begin{pmatrix}
0 & 1 & 0 \\
0 & 0 & 1 \\
0 & 0 & 0
\end{pmatrix} 
$$
可以得到
$$
\begin{pmatrix}
x_{21} & x_{22} & x_{23} \\
x_{31} & x_{32} & x_{33} \\
0 & 0 & 0
\end{pmatrix} = 
\begin{pmatrix}
0 & x_{11} & x_{12} \\
0 & x_{21} & x_{22} \\
0 & x_{31} & x_{32}
\end{pmatrix}
$$
整理一下
$$
\left\{
\begin{array}{ll}
x_{21} = x_{31} = x_{32} = 0 \\
x_{22} = x_{11} = x_{33} \\
x_{23} = x_{12} \\
\end{array} \right.
$$
因此
$$
\mbf{X} = \begin{vmatrix}
x_{11} & x_{12} & x_{13} \\
0 & x_{11} & x_{12} \\
0 & 0 & x_{11}
\end{vmatrix}
$$
其中$x_{11},x_{12},x_{13} \in \mathbb{R}$. 
\end{example}

\begin{example}
\rm 设$\mbf{D}$为主对角线两两不相等的对角矩阵,那么与$\mbf{D}$可交换的矩阵一定是对角矩阵. 
\end{example}

\begin{example}
\rm 设$\mbf{A}$与所有$n$阶矩阵都可交换,则$\mbf{A}$必为数量矩阵. 
\end{example}

\subsection{矩阵表示}

\begin{example}
\rm 任一$n$阶矩阵都可以表示成一个对称矩阵和斜对称矩阵之和,并且表法唯一.

\hints\ 设$\mbf{A} = \mbf{A}_1 + \mbf{A}_2$,其中$\mbf{A}_1$为对称矩阵,$\mbf{A}_2$为斜对称矩阵. 等式两边取转置,得到$\mbf{A}^T = \mbf{A}_1 - \mbf{A}_2$. 联立两个方程得到$\mbf{A}_1 = \frac{\mbf{A} + \mbf{A}^T}{2}$和$\mbf{A}_2 = \frac{\mbf{A} - \mbf{A}^T}{2}$. 
\end{example}

\subsection{特殊矩阵}

\begin{example}
\rm 令
$$
\mbf{C} = \begin{pmatrix}
0 & 1 & 0 & 0 & \cdots & 0 & 0 \\
0 & 0 & 1 & 0 & \cdots & 0 & 0 \\
\vdots & \vdots & \vdots &  \ddots &  & \vdots & \vdots \\
0 & 0 & 0 & 0 & \cdots & 0 & 1 \\
1 & 0 & 0 & 0 & \cdots & 0 & 0 \\
\end{pmatrix}
$$
称$\mbf{C}$为$n$阶循环移位矩阵. 用$\mbf{C}$左乘一个矩阵,就相当于把这个矩阵每一行向上移一行,第一行移到最后一行; 用$\mbf{C}$右乘一个矩阵,就相对于把这个矩阵的每一列想右移动一列,最后一列移到第一列. 
\end{example}

\subsection{矩阵方程}

\begin{example}\label{matrix-equation-1}
\rm 设$\mbf{A},\mbf{B}$分别为$s \times m,s \times n$矩阵,证明: 矩阵方程$\mbf{A}\mbf{X} = \mbf{B}$有解的充要条件为
$$
\rank{\mbf{A}} = \rank{(\mbf{A},\mbf{B})},
$$

\hints\ 设$\mbf{A}$的列向量为$\mbf{\alpha}_1,\mbf{\alpha}_2,\cdots,\mbf{\alpha}_n$,$\mbf{B}$的列向量为$\mbf{\beta}_1,\mbf{\beta}_2,\cdots,\mbf{\beta}_n$. 那么
$$
(\mbf{\alpha}_1,\mbf{\alpha}_2,\cdots,\mbf{\alpha}_n)\begin{pmatrix}
x_{11} & x_{12} & \cdots & x_{1n} \\
x_{21} & x_{22} & \cdots & x_{2n} \\
\vdots & \vdots &  & \vdots \\
x_{m1} & x_{m2} & \cdots & x_{mn} \\
\end{pmatrix} = (\mbf{\beta}_1,\mbf{\beta}_2,\cdots,\mbf{\beta}_n)
$$
因此可以转化为求$n$个方程非齐次线性方程
$$
\mbf{A} \mbf{X}_i = \mbf{\beta}_i.  
$$
其中$\mbf{X}_i = (x_{1i},x_{2i},\cdots,x_{mi})^T$. 上述线性方程组成立等价于$\mbf{\beta}_{i}$是可以被$\mbf{\alpha}_1,\mbf{\alpha}_2,\cdots,\mbf{\alpha}_n$($i=1,2,\cdots,n$),那么也就是
$$
\rank{(\mbf{A},\mbf{B})} = \rank{(\mbf{\alpha}_1,\mbf{\alpha}_2,\cdots,\mbf{\alpha}_n,\mbf{\beta}_1,\mbf{\beta}_2,\cdots,\mbf{\beta}_n)} = \rank{(\mbf{\alpha}_1,\mbf{\alpha}_2,\cdots,\mbf{\alpha}_n)} = \rank{\mbf{A}}.
$$
\end{example}

\begin{example}
\rm 设矩阵矩阵$\mbf{A},\mbf{B}$分别为
$$
\mbf{A} = \begin{pmatrix}
1 & -1 & -1 \\
2 & a & 1 \\
-1 & 1 & a 
\end{pmatrix},
\mbf{B} = \begin{pmatrix}
2 & 2 \\
1 & a \\
-a-1 & -2 
\end{pmatrix}
$$
当$a$为何值时方程有解,方程$\mbf{AX} = \mbf{B}$无界,有唯一解,有多穷多个解. 

\hints\ 显然我们要考察矩阵$(\mbf{A} | \mbf{B})$,对其做初等行变换我们可以得到
$$
(\mbf{A} | \mbf{B}) = \left(\begin{array}{ccc|cc}
1 & -1 & -1 & 2 & 2 \\
0 & a+2 & 3 & -3 & a-4 \\
0 & 0 & a-1 & 1-a & 0
\end{array}\right)
$$
这个时候我们就要根据\ref{matrix-equation-1}来考察一下$\mbf{A}$和$(\mbf{A},\mbf{B})$秩的情况. 当$a \neq 1, a \neq -2$时,$\rank{\mbf{A}} = 3$,显然此时肯定有解且是唯一解. 当$a = 1$时,$\rank{\mbf{A}} = \rank{(\mbf{A},\mbf{B})} = 2$,此时有无穷解. 当$a = -2$时,$\rank{A} = 2$,而$\rank{(\mbf{A},\mbf{B})}=3$,那么方程是无解的. 
\end{example}

\begin{example}
\rm 求下述矩阵$\mbf{A}$
$$
\mbf{A}\begin{pmatrix}
1 & 1 & 1 \\
0 & 2 & 2 \\
0 & 0 & 3
\end{pmatrix} = 
\begin{pmatrix}
2 & 1 & 3 \\
1 & 0 & 2 \\
-3 & 4 & 1
\end{pmatrix}
$$
\hints\  上述矩阵我们记为$\mbf{AB} = \mbf{C}$这里比较直接的手法就是先求出$\mbf{A}$左边这个矩阵的逆,再右乘上等式右边的矩阵. 我们可以简化一下,等式两边先转置一下
$$
\mbf{B}^T\mbf{A}^T = \mbf{C}^T
$$
这样我们就可以对$(\mbf{B}^{T},\mbf{C}^T)$做初等行变换,这样最终会得到$\mbf{A}^T$,之后我们再转置一下就得到了$\mbf{A}$,这样相对直接采用前面的方法要少一个矩阵乘法,是值得的. 
\end{example}


\subsection{$AX=B$当$A$不可逆时求$X$}

\begin{example}
\rm 求矩阵$X$
$$
\begin{pmatrix}
3 & -1 & 2 \\
4 & -3 & 3 \\
1 & 3 & 0
\end{pmatrix}\mbf{X} = \begin{pmatrix}
3 & 9 & 7 \\
1 & 11 & 7 \\
7 & 5 & 7
\end{pmatrix}
$$
\hints\ 注意这里$\mbf{X}$左边这个矩阵不可逆,这里等式两边同乘初等矩阵做初等行变换得到
$$
\begin{pNiceMatrix}
3 & -1 & 2 & 3 & 9 & 7\\
4 & -3 & 3 & 1 & 11 & 7\\
1 & 3 & 0 & 7 & 5 & 7
\CodeAfter
\SubMatrix|{1-4}{3-6}.
\end{pNiceMatrix} \to 
\begin{pNiceMatrix}
1 & 0 & \frac{3}{5} & \frac85 & \frac{16}{5} & \frac{14}{5}\\
0 & 1 & -\frac{1}{5} & \frac{9}{5} & \frac{3}{5} & \frac{7}{5}\\
0 & 0 & 0 & 0 & 0 & 0
\CodeAfter
\SubMatrix|{1-4}{3-6}.
\end{pNiceMatrix}
$$
这里我们设$\mbf{X} =(\mbf{X}_1, \mbf{X}_2, \mbf{X}_3)$,于是
$$
(\mbf{A}\mbf{X}_1, \mbf{A}\mbf{X}_2, \mbf{A}\mbf{X}_3) = (\mbf{\beta}_1, \mbf{\beta}_2, \mbf{\beta}_3),
$$
其中
$$
\mbf{A} = \begin{pmatrix}
1 & 0 & \frac{3}{5} \\
0 & 1 & -\frac{1}{5} \\
0 & 0 & 0 
\end{pmatrix}, (\mbf{\beta}_1, \mbf{\beta}_2, \mbf{\beta}_3) = \begin{pmatrix}
\frac85 & \frac{16}{5} & \frac{14}{5}\\ 
\frac{9}{5} & \frac{3}{5} & \frac{7}{5}\\
0 & 0 & 0
\end{pmatrix}
$$
因此$\mbf{A}\mbf{X}_1 = \mbf{\beta}_1,\mbf{A}\mbf{X}_2 = \mbf{\beta}_2,\mbf{A}\mbf{X}_3 = \mbf{\beta}_3$分别对应的一般解为
$$
\left\{
\begin{array}{ll}
x_1 = -\frac{3}{5}x_3 + \frac{8}{5}\\
x_2 = \frac{1}{5}x_3 + \frac{9}{5} 
\end{array} \right. , \left\{
\begin{array}{ll}
x_1 = -\frac{3}{5}x_3 + \frac{16}{5}\\
x_2 = \frac{1}{5}x_3 + \frac{3}{5} 
\end{array} \right. , \left\{
\begin{array}{ll}
x_1 = -\frac{3}{5}x_3 + \frac{14}{5}\\
x_2 = \frac{1}{5}x_3 + \frac{7}{5} 
\end{array} \right. 
$$
从而
$$
\mbf{X} = (\mbf{X}_1, \mbf{X}_2, \mbf{X}_3) = \begin{pmatrix}
-\frac{3}{5}k_1 + \frac{8}{5} & -\frac{3}{5}k_2 + \frac{16}{5} & -\frac{3}{5}k_3 + \frac{14}{5} \\
\frac{1}{5}k_1 + \frac{9}{5} & \frac{1}{5}k_2 + \frac{3}{5} & \frac{1}{5}k_3 + \frac{7}{5} \\
k_1 & k_2 & k_3
\end{pmatrix}
$$
\end{example}

\subsection{分块矩阵}

\begin{example}
\rm 设$\mbf{A},\mbf{B},\mbf{C},\mbf{D}$都是$n$阶矩阵,且$\mbf{A}\mbf{C} = \mbf{C}\mbf{A}$. 证明:
$$
\begin{pmatrix}
\mbf{A} & \mbf{B} \\
\mbf{C} & \mbf{D}
\end{pmatrix} = |\mbf{AD} - \mbf{CB}|. 
$$
\hints\ 这里肯定要用分块矩阵的性质来简化等式左边这个矩阵,分两种情况考虑. 当$|\mbf{A}| \neq 0$时,这里可以做一个初等行变换
$$
\begin{pmatrix}
\mbf{A} & \mbf{B} \\
\mbf{C} & \mbf{D}
\end{pmatrix} \to 
\begin{pmatrix}
\mbf{A} & \mbf{B} \\
\mbf{0} & \mbf{D}-\mbf{C}\mbf{A}^{-1}\mbf{B} 
\end{pmatrix}
$$
因此
$$
\begin{vmatrix}
\mbf{A} & \mbf{B} \\
\mbf{C} & \mbf{D}
\end{vmatrix} = \begin{vmatrix}
\mbf{A} & \mbf{B} \\
\mbf{0} & \mbf{D}-\mbf{C}\mbf{A}^{-1}\mbf{B} 
\end{vmatrix} =|\mbf{A}||\mbf{D}-\mbf{C}\mbf{A}^{-1}\mbf{B}|=|\mbf{AD}-\mbf{AC}\mbf{A}^{-1}\mbf{B}| = |\mbf{AD}-\mbf{CB}|,
$$
其中最后一个等号用到题目所给条件$\mbf{A}\mbf{C} = \mbf{C}\mbf{A}$. 

当$|\mbf{A}| = 0$时,这里手法就比较巧妙了. 首先构造一个辅助矩阵函数
$$
\mbf{A}(t) = \mbf{A}-t\mbf{I}.
$$
那么$|\mbf{A}(t)| = |\mbf{A}-t\mbf{I}|$就是关于$t$一个$n$次多项式. 由算术基本定理
$$
|\mbf{A}(t)| = 0,
$$
至多有$n$个根,显然$t = 0$是它的一个根. 那么在$t = 0$的邻域附近一定是可以找到一个$t_0$使得$|\mbf{A}(t_0)| \neq 0$,根据前述结论我们有
$$
\begin{vmatrix}
\mbf{A}(t_0) & \mbf{B} \\
\mbf{C} & \mbf{D}
\end{vmatrix} = |\mbf{A}(t_0)||\mbf{D}-\mbf{C}\mbf{A}(t_0)^{-1}\mbf{B}|=|\mbf{A}(t_0)\mbf{D}-\mbf{A}(t_0)\mbf{C}\mbf{A}(t_0)^{-1}\mbf{B}|,
$$
又因
$$
\mbf{A}(t)\mbf{C} = \mbf{AC} - t\mbf{C} = \mbf{CA}-t\mbf{C} = \mbf{C}\mbf{A}(t). 
$$
所以
$$
\begin{vmatrix}
\mbf{A}(t_0) & \mbf{B} \\
\mbf{C} & \mbf{D}
\end{vmatrix}  = |\mbf{A}(t_0)\mbf{B}- \mbf{C}\mbf{B}|.
$$
这里对取极限$t_0 \to 0$,因$\mbf{A}(t)$是连续函数,所以
$$
\begin{pmatrix}
\mbf{A} & \mbf{B} \\
\mbf{C} & \mbf{D}
\end{pmatrix} = |\mbf{AD} - \mbf{CB}|. 
$$
\end{example}


\begin{example}
\rm 设$\mbf{A}$为2阶矩阵,$\mbf{P}=(\mbf{\alpha},\mbf{A\alpha})$,其中$\mbf{\alpha}$为非零向量,且不是$\mbf{A}$的特征向量. 若$\mbf{A}^2\mbf{\alpha}+\mbf{A\alpha}-6\mbf{\alpha} = \mbf{0}$,证明$\mbf{P}$是可逆矩阵,求$\mbf{P}^{-1}\mbf{A}\mbf{P}$. 

\hints\ 根据题意不存在$\lambda$使得$\mbf{A\alpha} = \lambda\mbf{\alpha}$,那么$\mbf{A\alpha}$和$\mbf{\alpha}$是向量无关的. 这个$\mbf{P}^{-1}$估计是不能表示出来,猜测最后结果有这样的形式$\mbf{AP} = \mbf{P}\mbf{B}$,那么来我们来求$\mbf{AP}$.
$$
\begin{aligned}
\mbf{AP} &= (\mbf{A\alpha} + \mbf{A}^2\mbf{\alpha})  \\
&=(\mbf{A\alpha},6\mbf{\alpha}-\mbf{A\alpha}) \\
&=\mbf{P}\begin{pmatrix}
0 & 1 \\
1 & 0
\end{pmatrix}\begin{pmatrix}
1 & 0 \\
0 & 6
\end{pmatrix}
\begin{pmatrix}
1 & -1 \\
0 & 1
\end{pmatrix} \\
& = \mbf{P}\begin{pmatrix}
0 & 6 \\
1 & -1
\end{pmatrix}
\end{aligned}
$$
因此
$$
\mbf{P}^{-1}\mbf{A}\mbf{P} = \begin{pmatrix}
0 & 6 \\
1 & -1
\end{pmatrix}
$$
\end{example}

\newpage
\section{矩阵的逆}

\subsection{凑逆矩阵}

\begin{example}
\rm 若$A^l = \mbf{0}$,其中$l$是使得$A^m = \mbf{0}$中最小的那个$m$,如果存在这样$l$,则称$\mbf{A}$是一个幂零矩阵,它的幂零指数为$l$. 证明在这种情况下$\mbf{I}-\mbf{A}$是可逆矩阵.

\hints\ 这个是比较有构造性,需要把相应的可逆矩阵构造出来,即
$$
\mbf{I} - \mbf{A}^l = \mbf{0} \Rightarrow (\mbf{I}-\mbf{A})(\mbf{I} + \mbf{A} + \mbf{A}^2 +\cdots + \mbf{A}^{l-1}) = \mbf{I} - \mbf{A}^l = \mbf{I},
$$  
这里可以把$\mbf{I} - \mbf{A}^l = \mbf{0}$看做$1-x^n = 0$,那么$\mbf{I}$这个方程的一个根,把这个根先提出来,就可以得到上面的结果.
\end{example}

\begin{example}
\rm 设$\mbf{A}$是$m \times n$矩阵,$\mbf{B}$是$n \times m$矩阵. 证明: 如果$\mbf{I}-\mbf{A}\mbf{B}$可逆,那么$\mbf{I}-\mbf{B}\mbf{A}$也可逆,并求出$(\mbf{I}-\mbf{B}\mbf{A})^{-1}$. 

\hints\ 如果仅仅判定可逆只需要\ref{sylvester determinant}即可,要求出来逆矩阵,那么你要求出来$\mbf{C}$,使得
$$
(\mbf{I}-\mbf{B}\mbf{A})\mbf{C} = \mbf{I},
$$
这里的技巧是我们需要把这个$\mbf{C}$以某种更有价值的方式表示出来,可以猜$\mbf{C}  =  \mbf{I} + \mbf{X}$也有这样的形式,这样表示也是合理的. 于是我们得到
$$
\begin{array}{rl}
(\mbf{I}-\mbf{B}\mbf{A})(\mbf{I} + \mbf{X}) &= \mbf{I} \\
\mbf{I} + \mbf{X} -\mbf{BA} - \mbf{BAX} &= \mbf{I} \\
\mbf{X} - \mbf{BAX} &= \mbf{BA} 
\end{array}
$$
这里需要再凑一下$\mbf{X}$,使得$\mbf{X} = \mbf{B}\mbf{Y}\mbf{A}$,可以得到
$$
\begin{array}{rl}
\mbf{B}\mbf{Y}\mbf{A} - \mbf{BA}\mbf{B}\mbf{Y}\mbf{A} &= \mbf{BA} \\
\mbf{B}(\mbf{Y}-\mbf{ABY})\mbf{A} &= \mbf{BA} \\
\mbf{B}(\mbf{I}-\mbf{AB})\mbf{Y}\mbf{A} &= \mbf{BA}
\end{array}
$$
根据条件$\mbf{I}-\mbf{AB}$是可逆的,因此可以使得$\mbf{Y} = (\mbf{I}-\mbf{AB})^{-1}$让上式成立,从而$\mbf{X} = \mbf{B}(\mbf{I}-\mbf{AB})^{-1}\mbf{A}$,即
$$
(\mbf{I}-\mbf{B}\mbf{A})^{-1} = \mbf{I} + \mbf{B}(\mbf{I}-\mbf{AB})^{-1}\mbf{A}.
$$

再来研究一下这样猜逆矩阵形式的手法,这里同样我们可以用Cayley–Hamilton theorem,因为$\mbf{I}-\mbf{A}\mbf{B}$可逆,设它的逆为$\mbf{X}$,因此$\mbf{X} = f(\mbf{I}-\mbf{A}\mbf{B})$,其中$f(\mbf{I}-\mbf{A}\mbf{B})$是个关于$\mbf{I}-\mbf{A}\mbf{B}$的矩阵多项式,那么$\mbf{X} = g(\mbf{A}\mbf{B})$. 有趣的地方来了,这里有
$$
\mbf{B}\mbf{X}\mbf{A} = h(\mbf{BA}),
$$
这里会得到一个关于$\mbf{BA}$的多项式,但是这个里面不含$\mbf{I}$这一项,这恰恰因此$\mbf{I} +  \mbf{B}\mbf{X}\mbf{A}$有点形似关于$\mbf{I}-\mbf{BA}$的多项式. 
\end{example}

\subsection{拆开求}

\begin{annotation}
\rm 如果遇到求某种形式的矩阵的逆,你有一个直觉它的逆大概是怎样的,你可以按照你的直觉来推一下,如果恰好是这样的,再根据矩阵逆的唯一性,那么你的推测就是正确的. 
\end{annotation}

\begin{example}
\rm 求下述$n$阶矩阵$\mbf{A}$的逆矩阵($n \geq 2$)
$$
\mbf{A} =
\begin{pmatrix}
0 & 1 & 1 & \cdots & 1 \\
1 & 0 & 1 & \cdots & 1 \\
\vdots & \vdots & \vdots &  & \vdots \\
1 & 1 & 1 & \cdots & 0
\end{pmatrix}
$$

\hints\ 这个矩阵的逆可以直接用行变换化单位矩阵的方法来做,但是有比较trick的做法,观察到$\mbf{A} = \mbf{B}-\mbf{I}$,其中$\mbf{B}$是元素全为$1$的矩阵,这里我们猜测$\mbf{A}^{-1} = a\mbf{B} + b\mbf{I}$,那么它一定需要满足
$$
\begin{array}{ll}
\mbf{I}  &= (B-I)(aB + bI) \\
  &= aB^2+(b-a)B -bI \\
  &= anB + (b-a)B - bI \\
  &= (an+b-a)B - bI
\end{array}
$$
因此$b=-1,a = \frac{1}{n-1}$. 注意到这里有一个比较好的性质$\mbf{B}^2 = n\mbf{B}$.

来讲述一下这里为什么可以这样来待定系数做的方法,首先我们观察到$\mbf{A}$是可逆的,因此这里可以有$\mbf{A}^{-1} = f(\mbf{A})$,其中$f(\mbf{A})$表示$\mbf{A}$的多项式,这一细节去看一下Cayley–Hamilton theorem即可. 这里将$\mbf{A}$分解成$\mbf{B} - \mbf{I}$再带入$f(\mbf{A})$可以得到一个关于$\mbf{B}$的多项式$g(\mbf{B})$. 特别地这里$\mbf{B}$秩等于$1$,那么$\mbf{B}$又可以分解为两个一维向量的乘积$\mbf{\alpha} \mbf{\beta}^T$,因此$\mbf{B}^m = k^{m-1}\mbf{B}$. 所以最终有形式$g(\mbf{B}) = a\mbf{B} + b\mbf{I}$. 
\end{example}

\begin{example}
\rm 求下述$n$阶矩阵的逆
$$
\mbf{A} = 
\begin{pmatrix}
1 & b & b^2 & \cdots & b^{n-1} \\
0 & 1 & b & \cdots & b^{n-2}  \\
\vdots & \vdots & \vdots & & \vdots \\
0 & 0 & 0 & \cdots & b \\
0 & 0 & 0 & \cdots & 1 
\end{pmatrix}
$$

由\ref{power-of-matrix:m1},我们将$\mbf{A}$拆开
$$
\mbf{A} = \mbf{I} + b\mbf{B}+b^2\mbf{B}^2 + \cdots + b^{n-1}\mbf{B}^{n-1},
$$
其中
$$
\mbf{B} = \begin{pmatrix}
0 & 1 & 0 & \cdots & 0 \\
0 & 0 & 1 & \cdots & 0 \\
\vdots & \vdots & \vdots & \cdots & \vdots \\
0 & 0 & 0 & \cdots & 1 \\
0 & 0 & 0 & \cdots & 0 
\end{pmatrix}
$$
这样将上述等式两边左乘上$(\mbf{I}-b\mbf{B})$得到
$$
\begin{array}{ll}
(\mbf{I}-b\mbf{B})\mbf{A} &= (\mbf{I}-b\mbf{B})(\mbf{I} + b\mbf{B}+b^2\mbf{B}^2 + \cdots + b^{n-1}\mbf{B}^{n-1}) \\
&=\mbf{I}-b^n\mbf{B}^{n} \\
&=I
\end{array}
$$
因此$\mbf{A}^{-1} = \mbf{I}-b\mbf{B}$. 
\end{example}

\subsection{解方程}
\begin{example}
\rm 求下述$n$阶矩阵$\mbf{A}$的逆矩阵$(n \geq 2)$,
$$
\mbf{A} = \begin{pmatrix}
1 & 2 & 3 & \cdots & n \\
n & 1 & 2 & \cdots & n-1 \\
\vdots & \vdots & \vdots & & \vdots \\
2 & 3 & n & \cdots & 1
\end{pmatrix}
$$

\hints\ 由
$$
\mbf{A}\mbf{B} = \mbf{I} = (\mbf{\varepsilon}_1,\mbf{\varepsilon}_2,\cdots,\mbf{\varepsilon}_n),
$$
同样我们也可以把$\mbf{B}$拆开$(\mbf{X}_1,\mbf{X}_2,\cdots,\mbf{X}_n)$. 因此我们可以分别求解$n$个方程组
$$
\mbf{A}\mbf{X}_i = \mbf{\varepsilon}_i, i=1,2,\cdots,n.
$$
考虑任意一个$X_i$的对应的解为$x_1,x_2,\cdots,x_n$. 按照我们解方程的手法将第一个方程减去第二个方程得到
$$
(1-n)x_1 + (x_2+x_3+\cdots+x_n) = b_1-b_2 ~~~~~~(1).
$$
将所有行都加到第一行有
$$
\frac{(1+n)n}{2}(x_1 + x_2 + \cdots + x_n) = b_1 + b_2 + \cdots + b_n~~~~~~~(2).
$$
讲$(2)$式带入$(1)$,即可得到
$$
x_1 = \frac{1}{n}\left[\frac{2\sum\limits_{i=1}^n x_i}{n(n-1)}-b_1 + b_2 \right].
$$
类似地,当$i \leq n-1$时,用$i+1$行减去$i$行得到
$$
x_i = \frac{1}{n}\left[\frac{2\sum\limits_{i=1}^n x_i}{n(n-1)}-b_i + b_{i+1} \right].
$$
用第$1$行减去第$n$行,得到
$$
x_n =  \frac{1}{n}\left[\frac{2\sum\limits_{i=1}^n x_i}{n(n-1)}-b_n + b_{1} \right]
$$
我们设$s = \frac{2\sum\limits_{i=1}^n x_i}{n(n-1)}$,将$\mbf{\varepsilon}_1,\mbf{\varepsilon}_2,\cdots,\mbf{\varepsilon}_n$分别带入上述通项,得到
$$
\mbf{A}^{-1} = \frac{1}{n} \begin{pmatrix}
s-1 & s+1 & s & \cdots & s \\
s & s-1 & s+1 & \cdots & s \\
\vdots & \vdots & \vdots &  &\vdots \\
s & s & s & \cdots & s+1\\
s+1 & s & s & \cdots & s-1
\end{pmatrix}
$$
\end{example}

\section{正交矩阵}

\subsection{正交矩阵的性质}

\begin{example}
\rm 证明: 如果$n$阶正交矩阵$\mbf{A}$是上三角矩阵,那么$\mbf{A}$是对角矩阵,且$\mbf{A}$的主对角元$1$或者$-1$.

\hints\ 设$\mbf{A}$的列向量为$\mbf{\alpha}_1,\mbf{\alpha}_2,\cdots,\mbf{\alpha}_n$,那么
$$
(\mbf{\alpha}_1,\mbf{\alpha}_2,\cdots,\mbf{\alpha}_n)^T(\mbf{\alpha}_1,\mbf{\alpha}_2,\cdots,\mbf{\alpha}_n) = \mbf{I}
$$
于是则有
$$
\left\{
\begin{array}{ll}
\inp{\mbf{\alpha}_i}{\mbf{\alpha}_i} = 1 & i=1,2,\cdots,n \\
\inp{\mbf{\alpha}_i}{\mbf{\alpha}_j} = 0 & i \neq j
\end{array} \right.
$$ 
由因$\mbf{A}$是上三角矩阵,那么
$$
\begin{array}{ll}
\inp{\mbf{\alpha}_1}{\mbf{\alpha}_1} = a_{11}^2 = 1 \Rightarrow a_{11} = \pm 1 \\
\inp{\mbf{\alpha}_1}{\mbf{\alpha}_2} = a_{11}a_{12} = 0 \Rightarrow a_{12} = 0  \\
\inp{\mbf{\alpha}_2}{\mbf{\alpha}_2} = a_{12}^2 +a_{22}^2 = 1 \Rightarrow a_{22}^2 = \pm 1 \\
\cdots \\
\end{array}
$$
因此$\mbf{A}$是一个主对角线为$\pm 1$的对角矩阵. 
\end{example}

\begin{example}
\rm 设$\mbf{A}$是正交矩阵,证明
\begin{enumerate}
	\item 如果$|\mbf{A}| = 1$,那么$\mbf{A}$的每一个元素等于它自己的代数余子式.
	\item 如果$|\mbf{A}| = -1$,那么$\mbf{A}$的每一个元素等于它自己的代数余子式乘以$-1$.
\end{enumerate}
上述结论反过来也可以推出$\mbf{A}$是正交矩阵. 

\hints\ 设$\mbf{A}$是正交矩阵,因此$\mbf{A}^T = \mbf{A}^{-1} = \frac{\mbf{A}^*}{|\mbf{A}|}$,于是
$$
\mbf{A}(i,j) = \mbf{A}^T(j,i) = \frac{1}{|\mbf{A}|}\mbf{A}^*(j,i) = \frac{1}{|\mbf{A}|}A_{ij}. 
$$
\end{example}

\begin{example}
\rm 设$\mbf{A}$是$n$阶矩阵,$n \geq 3$且$\mbf{A} \neq 0$. 证明
\begin{enumerate}
	\item 如果$\mbf{A}$的每一个元素等于它自己的代数余子式,那么$\mbf{A}$是正交矩阵. 
	\item 如果$\mbf{A}$的每一个元素等于它自己的代数余子式乘上$-1$,那么$\mbf{A}$是正交矩阵. 
\end{enumerate}

\hints\  这里只证明第一个命题. 首先由条件可知
$$
|\mbf{A}| = \sum\limits_{r=1}^n a_{ir}A{ir} > 0,
$$
且
$$
\mbf{A}^T(i,j) = \mbf{A}(j,i) = \mbf{A}^*(i,j).
$$
再由$|\mbf{A}^*| = |\mbf{A}|^{n-1}$,于是
$$
|\mbf{A}^T| = |\mbf{A}| = |\mbf{A}|^{n-1},
$$
当$n \geq 3$时,$n-1 \geq 2$,因此$|\mbf{A}|=1$.  
\end{example}


\begin{example}
\rm 证明: 任意$n$阶矩阵$\mbf{A}$满足下列三个性质中的任意两个性质,那么必有第三个性质: 正交矩阵,对称矩阵,对合矩阵($\mbf{A}^2 = \mbf{I}$). 
\end{example}

\begin{example}
\rm \redt{正交变换不改变向量长度} 设$\mbf{A}$是$n$阶正交矩阵,证明: 对任一列向量$\mbf{\alpha}$,有
$$
\norm{\mbf{A}\mbf{\alpha}} = \norm{\mbf{\alpha}}.
$$

\hints\ 
$$
\norm{\mbf{A}\mbf{\alpha}}^2 = \inp{\mbf{A}\mbf{\alpha}}{\mbf{A}\mbf{\alpha}} = (\mbf{A}\mbf{\alpha})^T(\mbf{A}\mbf{\alpha}) = \mbf{\alpha}^T\mbf{\alpha} = \norm{\mbf{\alpha}}^2
$$
\end{example}

\subsection{施密斯正交化的应用}
\begin{example}
\rm 证明: 可逆$n$阶矩阵$\mbf{A}$可以被唯一分解为正交矩阵$T$和主对角线都为正数的上三角矩阵$\mbf{B}$的乘积,即$\mbf{A} = \mbf{T}\mbf{B}$. \redt{该结论可以推广至任意列满秩或者行满秩$m \times n$矩阵上}. 

\hints\ 设矩阵$\mbf{A}$的列向量为$\mbf{\alpha}_1,\mbf{\alpha}_2,\cdots,\mbf{\alpha}_n$,那么它们是线性无关的. 由施密斯正交化公式可知
$$
\begin{array}{ll}
\mbf{\beta}_1 = \mbf{\alpha}_1 \\
\mbf{\beta}_2 = \mbf{\alpha}_2 - \frac{\inp{\mbf{\alpha}_2}{\mbf{\beta}_1}}{\inp{\mbf{\beta}_1}{\mbf{\beta}_1}}\mbf{\beta}_1\\
\cdots \\
\mbf{\beta}_n = \mbf{\alpha}_n - \sum\limits_{i=1}^{n-1}\frac{\inp{\mbf{\alpha}_n}{\mbf{\beta}_i}}{\inp{\mbf{\beta}_i}{\mbf{\beta}_i}}\mbf{\beta}_i
\end{array}
$$
设$b_{ij} = \frac{\inp{\mbf{\alpha}_j}{\mbf{\beta}_i}}{\inp{\mbf{\beta}_i}{\mbf{\beta}_i}}$,于是
$$
\mbf{A} = (\mbf{\beta}_1,\mbf{\beta}_2,\cdots,\mbf{\beta}_n)
\begin{pmatrix}
1 & b_{12} & b_{13} & \cdots & b_{1n} \\
0 & 1 & b_{23} & \cdots & b_{2n} \\
\vdots & \vdots & \vdots &  & \vdots\\
0 & 0 & 0 & \cdots & b_{n-1,n} \\
0 & 0 & 0 & \cdots & 1 
\end{pmatrix}
$$
其中
$$
\left(\mbf{\beta}_1,\mbf{\beta}_2,\cdots,\mbf{\beta}_n \right) = \left(\frac{\mbf{\beta}_1}{|\mbf{\beta}_1|},\frac{\mbf{\beta}_2}{|\mbf{\beta}_2|},\cdots,\frac{\mbf{\beta}_n}{|\mbf{\beta}_n|}\right)\begin{pmatrix}
|\mbf{\beta}_1| & 0 & 0 & \cdots & 0 \\
0 & |\mbf{\beta}_2| & 0 & \cdots & 0 \\
\vdots & \vdots & \vdots &  & \vdots \\ 
0 & 0 & 0 & \cdots & |\mbf{\beta}_n|
\end{pmatrix}
$$
那么命题得证. 
\end{example}

\subsection{正交向量的性质}

\begin{example}
\rm 设$\mbf{A}$是$m \times n$矩阵,设齐次线性方程组$\mbf{AX}=\mbf{0}$的解空间为$W$. 证明$\mbf{A}$的行向量的转置都和$W$中的任一向量都正交.

\hints\ 设$\mbf{A}$的行向量为$\mbf{\gamma}_1,\mbf{\gamma}_2,\cdots,\mbf{\gamma}_n$,$\mbf{\eta} \in W$,于是
$$
\mbf{A}\mbf{\eta} = \begin{pmatrix}
\mbf{\gamma}_1 \\
\mbf{\gamma}_2 \\
\vdots \\
\mbf{\gamma}_n
\end{pmatrix} \mbf{\eta} = 
\begin{pmatrix}
\mbf{\gamma}_1\mbf{\eta} \\
\mbf{\gamma}_2\mbf{\eta} \\
\vdots \\
\mbf{\gamma}_n\mbf{\eta}
\end{pmatrix} = \begin{pmatrix}
0 \\
0 \\
\vdots \\
0
\end{pmatrix}
$$
因此
$$
\mbf{\gamma}_i\mbf{\eta} = \inp{\mbf{\gamma}_i^T}{\mbf{\eta}} = 0,
$$
\end{example}

\begin{example}
\rm 如果向量$\mbf{\alpha}$和一个正交基的每一个向量都正交,那么$\mbf{\alpha} = \mbf{0}$. 
\end{example}

\subsection{置换矩阵}

\begin{example}
\rm 定义$n$阶矩阵
$$
\mbf{P} = (\mbf{\varepsilon}_{i_1},\mbf{\varepsilon}_{i_2},\cdots,\mbf{\varepsilon}_{i_n}),
$$
其中$\mbf{\varepsilon}_{i_1},\mbf{\varepsilon}_{i_2},\cdots,\mbf{\varepsilon}_{i_n}$是单位向量$\mbf{\varepsilon}_{1},\mbf{\varepsilon}_{2},\cdots,\mbf{\varepsilon}_{n}$的一个全排列. 我们称矩阵$\mbf{P}$为置换矩阵. 证明: 置换矩阵是正交矩阵.

\hints\
$$
\begin{pmatrix}
\mbf{\varepsilon}_{i_1}^T \\
\mbf{\varepsilon}_{i_2}^T \\
\vdots \\ 
\mbf{\varepsilon}_{i_n}^T 
\end{pmatrix}(\mbf{\varepsilon}_{i_1},\mbf{\varepsilon}_{i_2},\cdots,\mbf{\varepsilon}_{i_n}) = \mbf{I}.
$$
\end{example}

\begin{example}\label{transpose-matrix-transform}
\rm 设$i_1,i_2,\cdots,i_n$是$1,2,\cdots,n$的一个全排列,设$n$阶矩阵$\mbf{A}=(a_{ij})$,令
$$
\mbf{B} = \begin{pmatrix}
a_{i_1i_1} & a_{i_1i_2} & \cdots & a_{i_1i_n}\\
a_{i_2i_1} & a_{i_2i_2} & \cdots & a_{i_2i_n}\\
\vdots & \vdots &  & \vdots\\
a_{i_ni_1} & a_{i_ni_2} & \cdots & a_{i_ni_n}\\
\end{pmatrix}
$$
证明: $\mbf{A} \sim \mbf{B}$.

\hints\ 设$\mbf{P} = (\mbf{\varepsilon}_{i_1},\mbf{\varepsilon}_{i_2},\cdots,\mbf{\varepsilon}_{i_n})$
$$
\mbf{P}^T \mbf{A}\mbf{P} = \mbf{B}. 
$$
从这里可以知道\bluet{左乘转置矩阵,就是交换对应的行; 右乘转置矩阵,就是交换对应的列}. 
\end{example}

\begin{example}
\rm 证明$\text{diag}\{\lambda_1,\lambda_2,\cdots,\lambda_n\} \sim \text{diag}\{\lambda_{i_1},\lambda_{i_2},\cdots,\lambda_{i_n}\}$,其中$i_1, i_2, \cdots , i_n$是$1,2,\cdots,n$的一个排列.

\hints\ \ref{transpose-matrix-transform}.
\end{example}

\begin{example}
\rm 设
$$
\mbf{A} = \begin{pmatrix}
0 & 1 & 0 & \cdots & 0 \\
0 & 0 & 1 & \cdots & 0 \\
\vdots & \vdots & \vdots & \cdots & \vdots \\
0 & 0 & 0 & \cdots & 1 \\
0 & 0 & 0 & \cdots & 0 
\end{pmatrix}
$$
证明: $\mbf{A} \sim \mbf{A}^T$. 

\hints\ 设转置矩阵$\mbf{P} = (\mbf{\varepsilon}_n, \mbf{\varepsilon}_{n-1},\cdots,\mbf{\varepsilon}_1)$,则
$$
\mbf{P}^T\mbf{A}\mbf{P} = \mbf{A}^T.
$$
\end{example}


\newpage
\section{矩阵等价}

\subsection{判定等价}

\begin{annotation}
\rm \redt{遵循矩阵等价当且仅当它们秩相同}.
\end{annotation}

\begin{annotation}
\rm 常用推论,任一矩阵可以表示为
$$
\mbf{A} = \mbf{P} \mbf{I}_r \mbf{Q}
$$
其中$r$表示矩阵$\mbf{A}$的秩,$\mbf{P},\mbf{Q}$都为可逆矩阵. 
\end{annotation}

\subsection{矩阵等价的性质}

\begin{example}
\rm 设$m \times n$矩阵$\mbf{A}$的秩为$r(r>0)$,则$\mbf{A}$可以表示成$r$个秩为$1$的矩阵之和. 

\hints\ 根据$\mbf{A}$等价于它的标准型,有
$$
\mbf{A} = \mbf{P}\begin{pmatrix}
\mbf{I}_r & \mbf{0} \\
\mbf{0} & \mbf{0}
\end{pmatrix}\mbf{Q} = \mbf{P}\mbf{E}_{11}\mbf{Q} + \mbf{P}\mbf{E}_{22}\mbf{Q} + \cdots + \mbf{P}\mbf{E}_{rr}\mbf{Q}
$$
\end{example}

\begin{example}
\rm 设$m \times n$矩阵$\mbf{A}$,则$\mbf{A}$的秩为$r$当且仅当存在$m \times r$列满秩矩阵$\mbf{B}$与$r \times n$行满秩矩阵$\mbf{C}$,使得$\mbf{A}=\mbf{B}\mbf{C}$. 

\hints\ \emph{必要性}\ 利用$\mbf{A}$等价于它的标准相似性
$$
\mbf{A} = \mbf{P}\begin{pmatrix}
\mbf{I}_r & \mbf{0} \\
\mbf{0} & \mbf{0}
\end{pmatrix}\mbf{Q} = (\mbf{P}_1, \mbf{P}_2)\begin{pmatrix}
\mbf{I}_r & \mbf{0} \\
\mbf{0} & \mbf{0}
\end{pmatrix}\begin{pmatrix}
\mbf{Q}_1 \\
\mbf{Q}_2 
\end{pmatrix} = (\mbf{P}_1, \mbf{0})\begin{pmatrix}
\mbf{Q}_1 \\
\mbf{Q}_2 
\end{pmatrix} = \mbf{P}_1\mbf{Q}_1.
$$
其中$\mbf{P}_1$是$m \times r$列满秩矩阵,$\mbf{Q}_1$是$r \times n$行满秩矩阵. 

\emph{充分性}\ 由矩阵乘法中秩的上界知道
$$
\rank{\mbf{BC}} \leq r.
$$ 
再由矩阵乘法中秩的下界知道
$$
\rank{\mbf{BC}} \geq \rank{\mbf{B}} + \rank{\mbf{C}} -r =r.
$$
因此$\rank{\mbf{BC}} = \rank{\mbf{A}} = r$. 
\end{example}

\newpage
\section{矩阵相似}

\subsection{相似性质}

\begin{example}
\rm 证明:若两个$n$阶矩阵$\mbf{A},\mbf{B}$满足$\mbf{AB}-\mbf{BA} = \mbf{A}$,那么$\mbf{A}$不可逆. 

\hints\ 假设$\mbf{A}$可逆,
$$
\mbf{B}-\mbf{A}^{-1}\mbf{BA} = \mbf{I} \Rightarrow \mbf{B}+\mbf{I} = \mbf{A}^{-1}\mbf{BA},
$$
这意味着$\mbf{B}$和$\mbf{B}+\mbf{I}$相似的,显然这是不可能的,因为如果$\lambda$是$\mbf{B}$的一个特征值,那么总存在$\lambda+1$是$\mbf{B}+\mbf{I}$的特征值,这意味总是存在不同的特征值. 
\end{example}

\begin{example}
\rm \redt{矩阵的迹在相似下不变} 给定$n$阶矩阵$\mbf{A}$,将$\mbf{A}$的对角线之和记为$\text{trace}(\mbf{A})$. 证明: 若$\mbf{A} \sim \mbf{B}$,那么$\text{trace}(\mbf{A}) = \text{trace}(\mbf{B})$.

\hints\ $\mbf{A}$和$\mbf{B}$相似,则它们有相同的特征值,而$\text{trace}(\mbf{A})$也等于特征值之和,因此$\text{trace}(\mbf{A}) = \text{trace}(\mbf{B})$. 
\end{example}

\subsection{待定系数矩阵相似}

\begin{example}
\rm 设矩阵$\mbf{A},\mbf{B}$分别为
$$
\mbf{A} = 
\begin{pmatrix}
0 & 2 & -3 \\
-1 & 3 & -3  \\
1 & -2 & a 
\end{pmatrix},
\mbf{B} = 
\begin{pmatrix}
1 & -2 & 0 \\
0 & b & 0  \\
0 & 3 & 1 
\end{pmatrix}
$$
若$\mbf{A}$和$\mbf{B}$相似,求$a$和$b$. 

\hints\ 实际上我们并没有比较好的充分条件来判定两个矩阵相似,但是这里我们可以只用两个比较简单的必要条件解出$a,b$
$$
\left\{
\begin{array}{ll}
\text{tr}(A) = \text{tr}(B) \\
|\mbf{A}| = |\mbf{B}|
\end{array}\right.
$$
不要去尝试求特征值,这样做只会让这个题目变得更难. 
\end{example}

\subsection{特殊的相似矩阵}

\begin{example}
\rm 设$\mbf{A}$为$n$阶矩阵,定义
$$
f(\mbf{A}) = a_0\mbf{I} + a_1\mbf{A} + \cdots + a_m\mbf{A}^m 
$$
称矩阵$f(\mbf{A})$是$\mbf{A}$的多项式. 证明: 若$\mbf{A} \sim \mbf{B}$,那么$f(\mbf{A}) \sim f(\mbf{B})$. 

\hints\ 
$$
\begin{array}{ll}
f(\mbf{B}) &= a_0\mbf{I} + a_1\mbf{B} + \cdots + a_m\mbf{B}^m \\
&= a_0\mbf{P}^{-1}\mbf{P} + a_1\mbf{P}^{-1}\mbf{A}\mbf{P} + \cdots + a_m\mbf{P}^{-1}\mbf{A}^m\mbf{P} \\
&=\mbf{P}^{-1}(a_0\mbf{I} + a_1\mbf{A} + \cdots + a_m\mbf{A}^m) \mbf{P}.
\end{array}
$$
\end{example}

\subsection{相似判定}

\begin{proposition}
\rm 常用判定矩阵相似的方法,遇题依次向下使用下述方法.
\begin{enumerate}
	\item 必要条件:相似必行列值相等;
	\item 必要条件:特征值相等;
	\item 充分条件: 对于都可对角化的矩阵,判定其特征值是否相同;
	\item 否命题的充分条件: 一个可对角化,一个不可对角化,则它们不相似;
	\item 对于都不可对角化的矩阵,同一个特征值的特征子空间的维数相同(必要条件);	
	\item 对于都不可对角化的矩阵,设$\mbf{P}\mbf{B} = \mbf{A}$,则对应的特征向量满足: 若$\mbf{B}$对应$\lambda$的特征向量$\alpha$,则$\mbf{A}$对应$\lambda$的特征向量为$P\alpha$. 这里需要求出可逆矩阵$P$
\end{enumerate}
\end{proposition}


\begin{example}
\rm 设$n$阶矩阵$\mbf{A}$和$\mbf{B}$分别为
$$
\mbf{A} = \begin{pmatrix}
1 & 1 & \cdots & 1 \\
1 & 1 & \cdots & 1 \\
\vdots & \vdots &  & \vdots \\
1 & 1 & \cdots & 1 \\
\end{pmatrix},
\mbf{B} = \begin{pmatrix}
0 & 0 & \cdots & 1 \\
0 & 0 & \cdots & 2 \\
\vdots & \vdots &  & \vdots \\
0 & 0 & \cdots & n \\
\end{pmatrix} 
$$
证明$\mbf{A}$和$\mbf{B}$相似. 

\hints\ 经典的秩为$1$的矩阵,想一想矩阵相似我们的充分条件只有可对角下的特征值相等. 首先肯定就是判定它们是否可对角化
$$
\begin{aligned}
|\lambda\mbf{I}-\mbf{A}| &= \begin{pmatrix}
\lambda-1 & -1 & \cdots & -1 \\
-1 & \lambda-1 & \cdots & -1 \\
\vdots & \vdots &  & \vdots \\
-1 & -1 & \cdots & \lambda-1 \\
\end{pmatrix} = (\lambda-n)\lambda^{n-1} \\
|\lambda\mbf{I}-\mbf{B}| &= \begin{pmatrix}
\lambda & 0 & \cdots & -1 \\
0 & \lambda & \cdots & -2 \\
\vdots & \vdots &  & \vdots \\
0 & 0 & \cdots & \lambda-n \\
\end{pmatrix} = (\lambda-n)\lambda^{n-1}
\end{aligned}
$$
它们的特征值相同. 当$\lambda=n$时,容易验证它们的特征值子空间都是$1$维的; 当$\lambda=0$时,也容易验证它们的特征值子空间都是$n-1$维的. 因此它们可以对角化且特征值相同,那么它们相似. 
\end{example}

\subsection{对角化判定}

\begin{proposition}
\rm 常用判定对角化的方法,遇题依次向下使用下述方法
\begin{enumerate}
	\item 实对称矩阵一定相似于对角矩阵;	
	\item 有$n$个不同的特征值,那么一定相似于对角矩阵;
	\item $n$重特征值对应特征子空间是否为$n$维;
\end{enumerate}
\end{proposition}

\subsection{特殊矩阵的对角化}

\begin{example}
\rm 设$\mbf{A}=(a_{ij})$是$n$阶上三角矩阵. 证明
\begin{enumerate}
	\item 如果$a_{11},a_{22},\cdots,a_{nn}$两两不相等,那么$\mbf{A}$可对角化;
	\item 如果$a_{11} = a_{22} = \cdots = a_{nn}$,并且至少有一个$a_{kl} \neq 0(k < l)$,那么$\mbf{A}$不能对角化. 	
\end{enumerate}

\hints\ (1) $|\lambda\mbf{I}-\mbf{A}| = (\lambda-a_{11})(\lambda-a_{22})\cdots (\lambda-a_{nn})$.

(2) 意味着只有一个特征值,只有对应的特征向量空间维数为$n$维才行. 特征向量可以盖住整个$n$维向量空间的,只有数量矩阵$\lambda\mbf{I}$. 
\end{example}


\subsection{对角化的应用}

\begin{example}
\rm 设$n$阶实对称矩阵$\mbf{A}$的全部特征值按大小顺序排成$\lambda_1 \geq  \lambda_2 \geq \cdots \geq \lambda_n$. 证明对应任意的非零列向量$\mbf{\alpha}$,都有
$$
\lambda_n \leq \frac{\mbf{\alpha}^T\mbf{A}\mbf{\alpha}}{\norm{\mbf{a}}^2} \leq \lambda_1
$$

\hints\ $\mbf{B}^T\mbf{A}\mbf{B} = \text{diag}\{\lambda_1,\lambda_2,\cdots,\lambda_n\}$. 
\end{example}

\begin{example}
\rm 设$n$阶实对称矩阵$\mbf{A}$的全部特征值按大小顺序排成$\lambda_1 \geq  \lambda_2 \geq \cdots \geq \lambda_2$. 证明
$$
\lambda_n \leq a_{ii} \leq \lambda_1, i =1,2,\cdots,n.
$$

\hints\ $a_{ii} = \mbf{\varepsilon}_i^T \mbf{A}\mbf{\varepsilon}_i$
\end{example}

\newpage
\section{矩阵特征值和特征向量}

\subsection{特征值和特征向量的一些性质}

\begin{example}\label{a+b-is-not-eigenvector}
\rm 证明: 如果$\mbf{\alpha}$和$\mbf{\beta}$分别是$n$阶矩阵$\mbf{A}$的属于不同特征值的特征向量,那么$\mbf{\alpha}+\mbf{\beta}$不是$\mbf{A}$的特征向量. 

\hints\ 设$\mbf{\alpha}$和$\mbf{\beta}$分别属于特征值$\lambda_1$和$\lambda_2$的特征值向量,且$\lambda_1 \neq \lambda_2$. 假设$\mbf{\alpha}+\mbf{\beta}$是属于特征值$\lambda_3$的特征值向量,那么
$$
\begin{array}{ll}
\mbf{A}(\mbf{\alpha}+\mbf{\beta}) &= \lambda_3(\mbf{\alpha}+\mbf{\beta}) \\
\lambda_1\mbf{\alpha}+\lambda_2\mbf{\beta} &= \lambda_3(\mbf{\alpha}+\mbf{\beta}) \\  
(\lambda_1 - \lambda_2)\mbf{\alpha} + (\lambda_2 - \lambda_3)\mbf{\beta} &= \mbf{0}
\end{array}
$$
因为$\mbf{\alpha}$和$\mbf{\beta}$线性无关,所以有
$$
\left\{
\begin{array}{ll}
\lambda_1 - \lambda_2 = 0\\
\lambda_2 - \lambda_3 = 0
\end{array}\right.
$$
那么这里解得$\lambda_1 = \lambda_2 = \lambda_3$,这是和前提条件矛盾的. 综上$\mbf{\alpha}+\mbf{\beta}$不是$\mbf{A}$的特征向量.
\end{example}

\begin{example}
\rm 设$\mbf{A}$是$n$阶矩阵. 证明: 如果$\mathbb{R}^n$中任意的非零向量列向量都是$\mbf{A}$的特征向量,那么$\mbf{A}=\lambda\mbf{I}$,其中$\lambda$是常数.

\hints\ 那么对任意的两个特征向量$\mbf{\alpha}$和$\mbf{\beta}$,它们之和还是特征向量,因此由\ref{a+b-is-not-eigenvector},可以知道$\mbf{A}$只有一个特征值$\lambda$,同时它对应了$n$个线性无关的向量,因此$\mbf{A}$可以对角化,即$\mbf{P}^{-1}\mbf{A}\mbf{P} = \text{diag}\{\lambda,\lambda,\cdots,\lambda\} = \lambda\mbf{I}$,因此$\mbf{A} = \lambda\mbf{I}$.
\end{example}

\subsection{特殊矩阵的特征值}

\begin{example}
\rm 设$\mbf{A}$是一个$n$阶正交矩阵,证明
\begin{enumerate}
	\item 如果$\mbf{A}$有特征值,那么它的特征值是$1$或者$-1$;
	\item 如果$|\mbf{A}| = -1$,那么$-1$是$\mbf{A}$的特征值;
	\item 如果$|\mbf{A}| = 1$,且$n$是奇数,那么$1$是$\mbf{A}$的特征值;	
\end{enumerate}

\hints\ (1) 设$\lambda$是$\mbf{A}$的一个特征值,那么
$$
\mbf{A}\mbf{\alpha} = \lambda\mbf{\alpha} \Rightarrow \mbf{\alpha}^T\mbf{A}^T = \lambda\mbf{\alpha}^T,
$$
两式相乘得到
$$
\mbf{\alpha}^T\mbf{\alpha} = \lambda^2\mbf{\alpha}^T\mbf{\alpha} \Rightarrow \lambda^2 =  1,
$$
因此$\lambda = \pm 1$. 

(2) 若$|\mbf{A}| = -1$,那么$\mbf{A}$的特征值的乘积也为$-1$,那么必有一个元素为$-1$,不可能全为$1$. 

(3) 若$|\mbf{A}| = 1$,那么$\mbf{A}$的对特征值的乘积也为$1$,而且当$n$为奇数时,特征值闭不可能全为$-1$,因此存在$\lambda = 1$. 
\end{example}

\begin{example}
\rm 设$n$阶矩阵$\mbf{A}$满足$\mbf{A}^2 = \mbf{A}$. 证明$\mbf{A}$的特征值为$1$或者$0$.

\hints\ 设$\lambda$是$\mbf{A}$的一个特征值,于是
$$
\mbf{A}^2\mbf{\alpha} = \lambda\mbf{A\alpha} = \lambda^2\mbf{\alpha}, 
$$
而$\mbf{A}^2 = \mbf{A}$,因此$\lambda^2$也是$\mbf{A}$的特征值. 同时也有$\lambda\mbf{\alpha} = \lambda^2\mbf{\alpha}$,那么$\lambda = 0$或者$\lambda = 1$. 
\end{example}


\begin{example}
\rm 证明: 如果$\mbf{A}$是$m \times n$矩阵,那么$\mbf{A}^T\mbf{A}$的特征值都是非负实数.

\hints\ $\mbf{A}^T\mbf{A}$是对称矩阵. 
\end{example}

\begin{example}
\rm 设$n$阶矩阵$\mbf{A}$的特征值为$\lambda_1,\lambda_2,\cdots,\lambda_3$,设关于$\mbf{A}$的矩阵多项式为$f(\mbf{A}) = a_0\mbf{I} + a_1\mbf{A}  + \cdots + a_m\mbf{A}^m$,那么$f(\mbf{A})$的特征值为$f(\lambda_1),f(\lambda_2),\cdots,\cdots,f(\lambda_m)$.

\hints 
$$
f(\mbf{A})\mbf{\alpha} = (a_0 + a_1\lambda + \cdots + a_m\lambda^m)\mbf{\alpha}. 
$$
\end{example}

\subsection{乘积矩阵的特征值}

\begin{example}\label{eigenvalue-of-AB-and-BA}
\rm 设$\mbf{A},\mbf{B}$分别为$m \times n, n \times m$矩阵. 证明:
\begin{enumerate}
	\item \redt{$\mbf{AB}$与$\mbf{BA}$有相同的非零特征向量,并且重数相同};
	\item 如果$\mbf{\alpha}$是$\mbf{AB}$的属于非零特征值$\lambda$的一个特征向量,那么$\mbf{B\alpha}$是$\mbf{BA}$的属于$\lambda$的一个特征向量. 
\end{enumerate}

\hints\ (1) 这里容易联想到\ref{sylvester determinant},设$\lambda$是$\mbf{AB}$的一个特征值向量,那么
$$
|\lambda\mbf{I} - \mbf{AB}| = \lambda^m|\mbf{I}-\frac{1}{\lambda}\mbf{AB}| = \lambda^m |\mbf{I} - \frac{1}{\lambda}\mbf{BA}| = \lambda^{m-n}|\lambda\mbf{I} - \mbf{BA}| = 0. 
$$
在复数域下展开
$$
|\lambda\mbf{I} - \mbf{AB}| = (\lambda- \lambda_0)^l_0(\lambda- \lambda_1)^l_1\cdots(\lambda- \lambda_{m-1})^l_{m-1}, 
$$
其中$l_0 + l_1 + \cdots + l_{m-1} = m$. 于是
$$
(\lambda- \lambda_0)^{l_0}(\lambda- \lambda_1)^{l_1}\cdots(\lambda- \lambda_{m-1})^{l_{m-1}} = \lambda^{m-n}|\lambda\mbf{I} - \mbf{BA}|,
$$
若$l_i$重根$\lambda_i \neq 0$,那么它必定也是$|\lambda\mbf{I}-\mbf{BA}|$的$l_i$重根. 其实两个相同的多项式必定有相同的根,这里仅仅提出来了$\lambda^{m-n}$对应着$\lambda = 0$的根,不会影响非零根的. 

(2) 若$\mbf{AB\alpha} = \lambda\mbf{\alpha}$,等式两边乘上$\mbf{B}$,即有$\mbf{BAB\alpha} = \lambda\mbf{B\alpha}$. 这样还要验证一下$\mbf{B}\mbf{\alpha} \neq  \mbf{0}$,若$\mbf{B}\mbf{\alpha} \neq  \mbf{0} = 0$,那么则有$\mbf{AB\alpha} = \lambda\mbf{\alpha} = \mbf{0}$,这就得到了$\lambda = 0$,这和前提条件是矛盾的. 
\end{example}


\begin{example}
\rm 设任一非零列向量$\mbf{v}=(v_1,v_2,\cdots,v_n)^T$,求$\mbf{v}\mbf{v}^T$的特征值.

\hints\ 首先由
$$
\mbf{v}\mbf{v}^T\mbf{v} = \left(\sum\limits_{i = 0}^{n}v_i^2 \right)\mbf{v},
$$
所以$\sum\limits_{i = 0}^{n}v_i^2$是其中一个特征值. 设$\mbf{w}$和$\mbf{v}$正交,则有
$$
\mbf{v}\mbf{v}^T\mbf{w} = \mbf{v} \cdot 0 = 0 \cdot \mbf{w}, 
$$
因此$0$也是其中一个特征值. 特别地,$\mbf{w}$和$\mbf{v}$正交,即求解线性方程$\mbf{v}\mbf{X} = \mbf{0}$,即
$$
v_1 x_1 + v_2 x_2 + \cdots + v_n x_n = 0, 
$$
显然它的解空间是$n-1$维,所以有$n-1$特征值$0$,一个特征值$\sum\limits_{i = 0}^{n}v_i^2$. 

换个思考方式,由\ref{eigenvalue-of-AB-and-BA}可知,$\mbf{v}\mbf{v}^T$和$\mbf{v}^T\mbf{v}$的非零特征值相同,而$\mbf{v}^T\mbf{v}$是一个一维矩阵,那么它的非零特征值$\lambda$满足
$$
\begin{array}{ll}
\mbf{v}^T\mbf{v}\mbf{\alpha} &= \lambda \mbf{\alpha} \\
\inp{\mbf{v}}{\mbf{v}} \mbf{\alpha} & = \lambda \mbf{\alpha}
\end{array} 
$$
因此非零特征值只有一个$\lambda = \inp{\mbf{v}}{\mbf{v}}$,且重数为$1$. 特别地$|\mbf{v}\mbf{v}^T| = \mbf{0}$,因此$0$也是该矩阵的一个特征值,接下来类似. 
\end{example}

\newpage
\section{二次型}

\subsection{规范形的使用}

\begin{example}
\rm 设二次型$\mbf{X}^T\mbf{A}\mbf{X}$. 证明: 如果存在列向量$\mbf{\alpha}_1,\mbf{\alpha}_2$,使得$\mbf{\alpha}_1\mbf{A}\mbf{\alpha}_1 >0, \mbf{\alpha}_2\mbf{A}\mbf{\alpha}_2 < 0$,那么存在列向量$\mbf{\alpha}_3$,使得$\mbf{\alpha}_3^T\mbf{A}\mbf{\alpha}_3 = 0$.

\hints\ 首先化规范形,设$\mbf{X}=C\mbf{Y}$,使得
$$
\mbf{X}^T\mbf{A}\mbf{X} = y_1^2 + \cdots + y_p^2-y_{p+1}^2 - \cdots - y_r^2,
$$
若$\mbf{\alpha}_1\mbf{A}\mbf{\alpha}_1 >0$,那么$p > 0$; 若$\mbf{\alpha}_2\mbf{A}\mbf{\alpha}_2 < 0$,那么$r-p > 0$. 换句话说$r,p$均大于$0$,因此可以使得
$$
\mbf{\beta} =(\underbrace{1,0,\cdots,0}_{p},1,0,\cdots,0)^T,
$$
令$\mbf{\alpha}_3 = C\mbf{\beta}$,则有
$$
\mbf{\alpha}_3^T\mbf{A}\mbf{\alpha}_3 = 0. 
$$
\end{example}

\begin{example}
\rm 设$\mbf{A}$为一个$n$阶实对称矩阵,证明: 如果$|\mbf{A}| < 0$,那么存在列向量$\mbf{\alpha}$,使得$\mbf{\alpha}^T\mbf{A}\mbf{\alpha} < 0$.

\hints\ 若$|\mbf{A}| < 0$,那么存在$\mbf{A}$存在奇数个负的特征值. 
\end{example}

\subsection{正定性的判定}

\begin{example}
\rm 如果$\mbf{A}$是$n$阶正定矩阵,那么$\mbf{A}^{-1}$也是正定矩阵.

\hints\ $\mbf{A}=C^T\mbf{I}C$. 
\end{example}

\begin{example}
\rm \redt{必要条件} $n$元二次型为正定的必要条件是,它的$n$个平方项系数都是正. 

\hints\ $\mbf{A}=C^T\mbf{I}C$. 从而
$$
A(i,i) = \sum_{r}^n C^T(i,r)C(r,i) = \sum_{r}^n [C(r,i)]^2 > 0 
$$
\end{example}

\begin{example}
\rm $n$阶矩实对称矩阵$\mbf{A}$是正定的充分必要条件是: 存在$n$阶可逆矩阵$C$使得
$$
\mbf{A}=C^TC
$$
\end{example}

\begin{example}
\rm 设$\mbf{A}$是$n$阶对称矩阵,它的$n$个特征值的绝对值的最大值记为$S_r(A)$. 证明: 当$t > S_r(A)$时,$t\mbf{I}+\mbf{A}$是正定矩阵. 

\hints\ 若$\mbf{A}$的特征值为$\lambda_1,\lambda_2,\cdots, \lambda_n$,那么$t\mbf{I}+\mbf{A}$的特征值为$t+\lambda_1,t+\lambda_2,\cdots, t+\lambda_n$. 当$t > S_r(A)$时,这些特征值显然都是大于$0$的. 
\end{example}

\begin{example}
\rm $n$阶矩实对称矩阵$\mbf{A}$是正定的充分必要条件是: 存在$n$阶可逆矩阵实对称矩阵$C$使得
$$
\mbf{A}=C^2
$$

\hints\ 若对称矩阵$\mbf{A}$是正定的,那么存在正交矩阵$T$
$$
T^{-1}\mbf{A}T = \text{diag}\{\lambda_1,\lambda_2,\cdots,\lambda_n\} \Rightarrow \mbf{A} =  (T\text{diag}\{\sqrt{\lambda_1},\sqrt{\lambda_2},\cdots,\sqrt{\lambda_n}\}T^{-1})(T \text{diag}\{\sqrt{\lambda_1},\sqrt{\lambda_2},\cdots,\sqrt{\lambda_n}\}T^{-1}) = C^2,
$$
必要性得证.

充分性是显然的.  
\end{example}

\begin{example}
\rm $n$阶矩阵$\mbf{A}$是正定的充分必要条件为$\mbf{A}$的所有主子式都大于$0$. 

\hints\ 充分性是显然的,所有主子式包括了顺序主子式. 

证必要性需要用一下乘积矩阵的子式公式. 设$\mbf{A} = C^TC$,那么$\mbf{A}$任一主子式为
$$
\begin{aligned}
\mbf{A}\begin{pmatrix}
i_1 & i_2 & \cdots & i_m \\
i_1 & i_2 & \cdots & i_m
\end{pmatrix} &= C^TC\begin{pmatrix}
i_1 & i_2 & \cdots & i_m \\
i_1 & i_2 & \cdots & i_m
\end{pmatrix} \\
&= \sum\limits_{1 \leq v_1,v_2,\cdots,v_m} C^T\begin{pmatrix}
i_1 & i_2 & \cdots & i_m \\
v_1 & v_2 & \cdots & v_m
\end{pmatrix}C\begin{pmatrix}
v_1 & v_2 & \cdots & v_m \\
i_1 & i_2 & \cdots & i_m
\end{pmatrix} \\
&=\sum\limits_{1 \leq v_1,v_2,\cdots,v_m}\left[C\begin{pmatrix}
i_1 & i_2 & \cdots & i_m \\
v_1 & v_2 & \cdots & v_m
\end{pmatrix}\right]^2
\end{aligned} 
$$
因为$C$是可逆的,因此存在任意阶非零的子式,所有对应$\mbf{A}$主子式也是非零的. 
\end{example}

\begin{example}
\rm 如果$\mbf{A}$与$\mbf{B}$都是正定矩阵,那么$\mbf{A+B}$也是正定矩阵. 

\hints\ $\mbf{\alpha}^T (\mbf{A}+\mbf{B})\mbf{\alpha} = \mbf{\alpha}^T\mbf{A}\mbf{\alpha} + \mbf{\alpha}^T\mbf{B}\mbf{\alpha} > 0$. 
\end{example}

\begin{example}
\rm 如果$\mbf{A}$与$\mbf{B}$都是正定矩阵,那么$\mbf{AB}$也是正定矩阵的充分必要条件是$\mbf{AB} = \mbf{BA}$. 

\hints\ $\mbf{A},\mbf{B}$对称加上$\mbf{A}\mbf{B}$交换,可以得到$\mbf{A}\mbf{B}$也是对称矩阵. 设$\mbf{A}=C_1^TC_1,\mbf{B}=C_2^TC_2$,那么
$$
\begin{array}{ll}
\mbf{AB} &= C_1^TC_1C_2^TC_2 \\
C_2\mbf{AB}C_2^{-1} &= C_2C_1^TC_1C_2^T  \\
& = (C_1C_2^T)^TC_1C_2^T
\end{array}
$$
$C_2\mbf{AB}C_2^{-1}$合同于单位矩阵,那么它是正定矩阵,同时它还相似于$\mbf{AB}$,即有相同的特征值,最终$\mbf{AB}$也是正定的. 
\end{example}

\begin{example}
\rm 设
$$
\mbf{M} = \begin{pmatrix}
\mbf{A} & \mbf{B} \\
\mbf{B}^T & \mbf{C}
\end{pmatrix}
$$
是正定矩阵,其中$\mbf{A}$是$r$阶矩阵$(r < n)$. 证明$\mbf{A},\mbf{C},\mbf{C}-\mbf{B}^T\mbf{A}^{-1}\mbf{B}$都是正定矩阵. 

\hints\ $\mbf{A}$的顺序主子式都是$\mbf{M}$的顺序主子式,$\mbf{C}$的主子式都是$\mbf{M}$的主子式,因此$\mbf{A}$和$\mbf{C}$都是正定的. 再做分块矩阵的初等变换
$$
 \begin{pmatrix}
\mbf{A} & \mbf{B} \\
\mbf{B}^T & \mbf{C}
\end{pmatrix} \to \begin{pmatrix}
\mbf{A} & \mbf{0} \\
\mbf{0} & \mbf{C}-\mbf{B}^T\mbf{A}^{-1}\mbf{B} 
\end{pmatrix}
$$
右边这个矩阵是对称矩阵,且相似于$\mbf{M}$,那么它是正定的. 再根据已经证明的结论$\mbf{C}-\mbf{B}^T\mbf{A}^{-1}\mbf{B}$也是正定的. 
\end{example}

\begin{example}
\rm 若$\mbf{A}$是$n$阶可逆矩阵,那么$\mbf{A}\mbf{A}^T$正定. 

\hints\ 显然$\mbf{A}\mbf{A}^T$是一个对称矩阵,设$\mbf{B} = \mbf{A}^T$,那么
$$
\begin{aligned}
\mbf{\alpha}^T\mbf{A}\mbf{A}^T\mbf{\alpha} &= \mbf{\alpha}^T\mbf{B}^T\mbf{B}\mbf{\alpha} \\
&= (\mbf{B}\mbf{\alpha})^T \mbf{B}\mbf{\alpha},
\end{aligned} 
$$
设$\mbf{y} = \mbf{B}\mbf{\alpha}$,当$\mbf{\alpha} \neq \mbf{0}$时,因$\mbf{A}$是可逆矩阵,那么$\mbf{y} \neq \mbf{0}$,所以
$$
\mbf{y}\mbf{y}^T > 0. 
$$
\end{example}

\subsection{正定矩阵的性质}

\begin{example}
\rm 设$\mbf{A}$是$n$阶正定矩阵,$\mbf{B}$是$n$阶对称矩阵,则存在一个$n$阶可逆矩阵$C$使得$C^T\mbf{A}C$和$C^T\mbf{B}C$都是对角矩阵. 

\hints\ 由于$\mbf{A}$是正定矩阵,那么$\mbf{A}$合同于$\mbf{I}$,即存在可逆矩阵$C_1$使得
$$
C_1^T\mbf{A}C_1 = \mbf{I},
$$
再构造一个对称矩阵$C_1^T\mbf{B}C_1$,那么存在正交矩阵$C_2$使得
$$
C_2^T(C_1^T\mbf{B}C_1)C_2 = \text{diag}\{\lambda_1,\lambda_2,\cdots,\lambda_n\}.
$$
令$C = C_1C_2$,可以验证它是满足命题要求的可逆矩阵. 
\end{example}





\end{document}