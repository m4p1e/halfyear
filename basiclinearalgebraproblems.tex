\documentclass{article}

\usepackage{ctex}
\usepackage{tikz}
\usetikzlibrary{calc}
\usetikzlibrary{cd}
\usetikzlibrary{decorations.pathreplacing}

\usepackage{amsthm}
\usepackage{amsmath}
\usepackage{amssymb}
\usepackage{mathdots} %iddots

\usepackage{pgfplots}
\pgfplotsset{compat=newest}

\usepackage{hyperref} %url
\hypersetup{
    colorlinks=true,
    linkcolor=blue,
    filecolor=magenta,      
    urlcolor=cyan,
    pdftitle={Overleaf Example},
    pdfpagemode=FullScreen,
    }

\usepackage{enumitem}

\usepackage[textwidth=18cm]{geometry} % 设置页宽=18

\usepackage{blindtext}
\usepackage{bm}
\parindent=0pt
\setlength{\parindent}{2em} 
\usepackage{indentfirst}


\usepackage{xcolor}
\usepackage{titlesec}
\titleformat{\section}[block]{\color{blue}\Large\bfseries\filcenter}{}{1em}{}
\titleformat{\subsection}[hang]{\color{red}\Large\bfseries}{}{0em}{}
%\setcounter{secnumdepth}{1} %section 序号

\newtheorem{theorem}{Theorem}[section]
\newtheorem{lemma}[theorem]{Lemma}
\newtheorem{corollary}[theorem]{Corollary}
\newtheorem{proposition}[theorem]{Proposition}
\newtheorem{example}[theorem]{Example}
\newtheorem{definition}[theorem]{Definition}
\newtheorem{remark}[theorem]{Remark}
\newtheorem{exercise}{Exercise}[section]
\newtheorem{annotation}[theorem]{Annotation}

\newcommand*{\xfunc}[4]{{#2}\colon{#3}{#1}{#4}}
\newcommand*{\func}[3]{\xfunc{\to}{#1}{#2}{#3}}

\newcommand\Set[2]{\{\,#1\mid#2\,\}} %集合
\newcommand\SET[2]{\Set{#1}{\text{#2}}} %

\newcommand{\norm}[1]{\left\lVert#1\right\rVert} % 范数
\newcommand{\vect}[1]{\mathbf{#1}} % vector
\newcommand{\hints}{{\color{blue} \text{hints}}}
\newcommand{\mbf}[1]{\bm{#1}} 
\newcommand{\rank}[1]{\text{rank}\left(#1\right)} % rank
\newcommand\inp[2]{\langle #1, #2 \rangle} %inner product

\newcommand{\redt}[1]{\textcolor{red}{#1}}
\newcommand{\bluet}[1]{\textcolor{blue}{#1}}


\begin{document}
\title{考研高数习题集}
\author{枫聆}
\maketitle
\tableofcontents

\section{行列式}

\subsection{定义}

\begin{annotation}
\rm 这类题特征
\begin{enumerate}
	\item 按照行列式的完全展开式来计算某种特殊的矩阵
	\item 给定某个具体的行列式值的基础上,通过行列式的性质来计算行列式. 
\end{enumerate}
\end{annotation}

\begin{example}
\rm 证明: 如果在$n$阶行列式中,第$i_1,i_2,\cdots,i_k$行分别与第$j_1,j_2,\cdots,j_l$列交叉位置的元素都是$0$,并且$k+l > n$,那么这个行列式的值等于$0$.
\begin{proof}
按照行列式的完全展开式,每一项都必须要包含第$i_1,i_2,\cdots,i_k$行中位于不用列的元素,则有$k$个元素. 由已知的条件,第$i_1,i_2,\cdots,i_k$行只与$j_1,j_2,\cdots,j_l$之外的$n-l$元素可能不为零,但是$k > n-l$,说明每一项必取到0,因此行列式为$0$. 
\end{proof}
\end{example}

\begin{example}
\rm 证明
$$
\begin{vmatrix}
a_1 + c_1 &  b_1 + a_1 & c_1+b_1 \\
a_2 + c_2 &  b_2 + a_2 & c_2+b_2 \\
a_3 + c_3 &  b_3 + a_3 & c_3+b_3 \\
\end{vmatrix} = 2 \begin{vmatrix}
a_1 & b_1 & c_1 \\
a_2 & b_2 & c_2 \\
a_3 & b_3 & c_3 	
\end{vmatrix}
$$
\end{example}

\begin{example}
\rm \redt{行列式最大值} 求元素为$1$和$0$的三阶行列式可取的最大值.

\hints\ 从完全展开式我们应使得带正号的项尽可能的都是1,而带负号项尽可能是0
\begin{enumerate}
	\item 若3个正项都是1,那么此时行列式等于0;
	\item 若2个正项是1,此时任取一个2个项是正的行列式其行列式均等于2,且其他负项也都是0,例如
	$$
	\begin{vmatrix}
	0 & 1 & 1\\
	1 & 0 & 1\\
	1 & 1 & 0\\
	\end{vmatrix}
	$$
\end{enumerate}
综上最大值肯定就是$2$. 
\end{example}

\begin{example}
\rm 设$n \geq 2$,证明: 如果$n$阶矩阵$\mbf{A}$的元素为$1$或者$-1$,则$|A|$闭为偶数. 

\hints\ 这个证明按行展开可能更简单
$$
|\mbf{A}| = a_{11}A_{11} + a_{12}A_{12} +a_{13}A_{13},
$$
其中三个代数余子式都是二阶行列式的正值或者负值,那么我们来看一下元素为$1$或者$-1$的二阶行列式
$$
\begin{vmatrix}
b_{11} & b_{12} \\
b_{21} & b_{22} \\
\end{vmatrix}  = b_{11}b_{22} - b_{12}b{21}
$$
这里一共有4种可能的取值,由$b_{11}b_{22} = \pm 1, b_{12}b{21} = 、\pm 1$决定,经过计算该二阶行列式取值可能为$0,2$,因此$|A|$是由3个偶数相加得到的,那么$|A|$也一定是偶数. 
\end{example}

\begin{example}
\rm 求元素为$1$或者$-1$的三阶行列式的最大值. 

\hints\ 从完全展开式出发,三阶行列式有6项,其中每一项只可能为$-1$和$1$. 再有前面证明,我们知道这样的三阶行列式的值只能是偶数,那么最大的偶数就是6项全为1加起来为6,即
$$
\begin{array}{ll}
a_{11}a_{22}a_{33} = 1, a_{12}a_{23}a_{31} = 1, a_{13}a_{21}a_{32} = 1 \\
-a_{13}a_{22}a_{31} = 1, -a_{12}a_{21}a_{33} = 1, -a_{11}a_{23}a_{32} = 1 
\end{array}
$$
由此得出
$$
\begin{array}{ll}
a_{11}a_{22}a_{33}a_{12}a_{23}a_{31}a_{13}a_{21}a_{32} = 1 \\
a_{13}a_{22}a_{31}a_{12}a_{21}a_{33}a_{11}a_{23}a_{32} = -1 
\end{array}
$$
这是矛盾的. 因此我们再考虑行列式最大值为$4$的可能,可以找到
$$
\begin{vmatrix}
1 & -1 & -1\\
-1 & 1 & -1 \\
1 & 1 & 1 
\end{vmatrix} = 1 + 1 + 1 + 1 - 1 + 1 = 4
$$
那么这样的行列式的最大值为$4$. 
\end{example}

\begin{example}
\rm 设$n \leq 3$,证明: 元素为$1$或者$-1$的$n$阶行列式的绝对值不超过$(n-1)!(n-1)$.

\hints\ 借助前面的例子做归纳. 
\end{example}

\begin{example}
\rm 设$n \geq 2$,证明: 元素为$1$或$-1$的$n$阶行列式的值能被$2^{n-1}$整除. 

\hints\ 设$|\mbf{A}|$是这样的行列式,首先将第一列上的$(-1)$所在行都提一个$-1$出来
$$
|\mbf{A}| = (-1)^m \begin{vmatrix}
1 & b_{12} & \cdots & b_{1n} \\
1 & b_{22} & \cdots & b_{2n} \\
\vdots & \vdots & \cdots & \vdots \\
1 & b_{n2} & \cdots & b_{nn} \\
\end{vmatrix} = (-1)^m \begin{vmatrix} 
1 & b_{12} & \cdots & b_{1n} \\
0 & c_{22} & \cdots & c_{2n} \\
\vdots & \vdots & \cdots & \vdots \\
0 & c_{n2} & \cdots & c_{nn} \\
\end{vmatrix}
$$
其中$c_{ij}$值可能为$2,-2,0$,因此从第$2$列到第$n$列都是可以提一个因子$2$出来,最终就可以凑成$2^{n-1}$. 
\end{example}



\subsection{化行阶梯形}

\begin{annotation}
\rm 不是特殊矩阵的第一选择.
\end{annotation}

\subsection{按一行展开}

\begin{annotation}
\rm 若是可以将某一行或者某一列消去,只留下一个非零元素,按行和按列展开是不错的选择. 
\end{annotation}

\begin{example}
\rm 计算
$$
|\mbf{A}| = \begin{vmatrix}
1 & 2 & 3 & \cdots & n-1 & n \\
1 & -1 & 0 & \cdots & 0 & 0 \\
0 & 2 & -2 & \cdots & 0 & 0 \\
\vdots & \vdots & \vdots &  & \vdots & \vdots \\
0 & 0 & 0 & \cdots & n-1 & 1-n \\
\end{vmatrix}
$$
\hints 可以考虑把所有列都加到第一列,再按第一列展开
$$
|\mbf{A}| = \begin{vmatrix}
\frac{(1+n)n}{2} & 2 & 3 & \cdots & n-1 & n \\
0 & -1 & 0 & \cdots & 0 & 0 \\
0 & 2 & -2 & \cdots & 0 & 0 \\
\vdots & \vdots & \vdots &  & \vdots & \vdots \\
0 & 0 & 0 & \cdots & n-1 & n-1 \\
\end{vmatrix} = \frac{(1+n)n}{2} \begin{vmatrix}
 -1 & 0 & \cdots & 0 & 0 \\
2 & -2 & \cdots & 0 & 0 \\
 \vdots & \vdots &  & \vdots & \vdots \\
  0 & 0 & \cdots & n-1 & n-1 \\
\end{vmatrix} 
$$
同样上述矩阵也是所有列加到第一列,最终有$|\mbf{A}| = (-1)^{n-1}\frac{(n+1)!}{2}$.
\end{example}

\subsection{按多行展开}

\begin{annotation}
\rm 好像没有直接使用拉普拉斯定理的习惯,比较特殊的分块矩阵可以考虑. 
\end{annotation}

\subsection{特殊矩阵}



\begin{annotation}
\rm 常见的特殊矩阵\url{https://www.bilibili.com/read/cv266516}
\begin{enumerate}
	\item 范德蒙德行列式
	\item 爪型行列式	
\end{enumerate}
\end{annotation}

\subsection{数学归纳法}

\begin{annotation}
\rm 通常证明手法也是按行或者列展开.
\end{annotation}

\begin{example}
\rm 计算$n$阶行列式
$$
\mbf{D}_n = \begin{vmatrix}
x & 0 & 0  & \cdots & 0 & 0 & a_0 \\
-1 & x & 0  & \cdots & 0 & 0 & a_1 \\
0 & -1 & x  & \cdots & 0 & 0 & a_2 \\
\vdots & \vdots & \vdots  & \cdots & \vdots & \vdots & \vdots \\
0 & 0 & 0  & \cdots & -1 & x & a_{n-2} \\
0 & 0 & 0  & \cdots & 0 & -1 & x+a_{n-1} \\
\end{vmatrix}
$$
\begin{proof}
当$n=2$时,有
$$
\mbf{D}_2 = \begin{vmatrix}
x & a_0 \\
-1 & x+a_1 
\end{vmatrix} = x^2 + a_1x+ a_0
$$
假设对于上述形式的$n-1$阶行列式,有
$$
\begin{vmatrix}
x & 0 & 0  & \cdots & 0 & 0 & a_0 \\
-1 & x & 0  & \cdots & 0 & 0 & a_1 \\
0 & -1 & x  & \cdots & 0 & 0 & a_2 \\
\vdots & \vdots & \vdots  & \cdots & \vdots & \vdots & \vdots \\
0 & 0 & 0  & \cdots & 0 & -1 & x+a_{n-2} \\
\end{vmatrix} = x^{n-1} + a_{n-2}x^{n-2} + \cdots + a_1x + a_0
$$
那么$n$阶行列式,把它按第一行展开,有
$$
\begin{array}{ll}
\mbf{D}_n &= x\begin{vmatrix}
 x & 0  & \cdots & 0 & 0 & a_1 \\
 -1 & x  & \cdots & 0 & 0 & a_2 \\
 \vdots & \vdots  & \cdots & \vdots & \vdots & \vdots \\
 0 & 0  & \cdots & -1 & x & a_{n-2} \\
 0 & 0  & \cdots & 0 & -1 & x+a_{n-1} \\
\end{vmatrix}+ (-1)^{1+n}a_0 \begin{vmatrix}
-1 & x & 0  & \cdots & 0 & 0  \\
0 & -1 & x  & \cdots & 0 & 0 \\
\vdots & \vdots & \vdots  & \cdots & \vdots & \vdots  \\
0 & 0 & 0  & \cdots & -1 & x  \\
0 & 0 & 0  & \cdots & 0 & -1  \\
\end{vmatrix}  \\
&=x(x^{n-1} + a_{n-1}x^{n-2} + \cdots + a_2x + a_1) + (-1)^{1+n}a_0(-1)^{n-1} \\
&=  x^n + a_{n-1}x^{n-1} + a_{n-2}x^{n-2} + \cdots + a_1x + a_0
\end{array}
$$
\end{proof}
\end{example}

\subsection{递推式}


\begin{example}
\rm 计算$n$阶行列式
$$
\mbf{D}_n = \begin{vmatrix}
2 & -1 & 0 & 0 & \cdots & 0 & 0 & 0\\
-1 & 2 & -1 & 0 & \cdots & 0 & 0 & 0\\
0 & -1 & 2 & -1 & \cdots & 0 & 0 & 0\\
\vdots & \vdots & \vdots & \vdots & & \vdots & \vdots & \vdots\\
0 & 0 & 0 & 0 & \cdots & -1 & 2 & -1\\
0 & 0 & 0 & 0 & \cdots & 0 & -1 & 2\\
\end{vmatrix}
$$
\hints\ 显然$\mbf{D}_1 = 2$. 将$\mbf{D}_n$按第一列展开,则有
$$
\mbf{D}_n = 2\mbf{D}_{n-1}+
\begin{vmatrix}
 -1 & 0 & 0 & \cdots & 0 & 0 & 0\\
 -1 & 2 & -1 & \cdots & 0 & 0 & 0\\
 \vdots & \vdots & \vdots & & \vdots & \vdots & \vdots\\
 0 & 0 & 0 & \cdots & -1 & 2 & -1\\
 0 & 0 & 0 & \cdots & 0 & -1 & 2\\
\end{vmatrix} = 2\mbf{D}_{n-1} - \mbf{D}_{n-2}
$$
这也意味着$\mbf{D}_n - \mbf{D}_{n-1} = \mbf{D}_{n-1}-\mbf{D}_{n-2}$,可以马上推出$\mbf{D}_n - \mbf{D}_{n-1} = \mbf{D}_2-\mbf{D}_1 = 1$,即该行列式是一个等差数列$D_n = 2+(n-1) = n+1$. 
\end{example}


\begin{example}
\rm 计算$n$阶行列式
$$
\mbf{D}_n = \begin{vmatrix}
a+b & ab & 0 & 0 & \cdots & 0 & 0 & 0\\
1 & a+b & ab & 0 & \cdots & 0 & 0 & 0\\
0 & 1 & a+b & ab & \cdots & 0 & 0 & 0\\
\vdots & \vdots & \vdots & \vdots & & \vdots & \vdots & \vdots\\
0 & 0 & 0 & 0 & \cdots & 1 & a+b & ab\\
0 & 0 & 0 & 0 & \cdots & 0 & 1 & a+b\\
\end{vmatrix}
$$
其中$a \neq b$.

\hints\ 还是按第一列展开
$$
\mbf{D}_n = (a+b)\mbf{D}_{n-1}-\begin{vmatrix}
 ab & 0 & 0 & \cdots & 0 & 0 & 0\\
 1 & a+b & ab & \cdots & 0 & 0 & 0\\
 \vdots & \vdots & \vdots & & \vdots & \vdots & \vdots\\
 0 & 0 & 0 & \cdots & 1 & a+b & ab\\
 0 & 0 & 0 & \cdots & 0 & 1 & a+b\\ 
\end{vmatrix} = (a+b)\mbf{D}_{n-1} -ab\mbf{D}_{n-2}. 
$$
注意其上最后一个等式成立的条件是$a \neq 0$和$b \neq 0$,那么推出
$$
\begin{array}{ll}
\mbf{D}_n - a\mbf{D}_{n-1} = b(\mbf{D}_{n-1}-a\mbf{D}_{n-2}) \Rightarrow \mbf{D}_n -a\mbf{D}_{n-1} = (\mbf{D}_2 - a\mbf{D}_1)b^{n-2}  \\
\mbf{D}_n - b\mbf{D}_{n-1} = a(\mbf{D}_{n-1}-b\mbf{D}_{n-2}) \Rightarrow \mbf{D}_n -b\mbf{D}_{n-1} = (\mbf{D}_2 - b\mbf{D}_1)a^{n-2}
\end{array}
$$
而$\mbf{D}_1 = a+b$,$\mbf{D}_2 = a^2 + ab + b^2$. 因此
$$
\begin{array}{ll}
\mbf{D}_n -a\mbf{D}_{n-1}  = b^n \\
\mbf{D}_n -b\mbf{D}_{n-1}  = a^n
\end{array}
$$
所以$D_n  = \frac{b^{n+1}-a^{n+1}}{b-a}$. 

当$a = 0$时,$\mbf{D}_n = b^n$; 当$b=0$时,$\mbf{D}_n=a^n$. 
\end{example}

\subsection{Laplace展开}

\begin{example}
\rm 计算下述$2n$阶行列式
$$
\mbf{D}_{2n} = \begin{vmatrix}
a  & & & &  & b \\
   & \ddots & & & \iddots \\
  & & a & b &  &  \\
  & & b & a &  & \\
  &  \iddots & & &  \ddots \\
  b& & & &  &a\\
\end{vmatrix}
$$
\hints\ 尝试从第一行和最后一行展开,于是得到
$$
\begin{array}{ll}
\mbf{D}_{2n} &= \begin{vmatrix}
a & b \\
b & a 
\end{vmatrix}\cdot (-1)^{(1+2n)+(1+2n)}\cdot \mbf{D}_{2n-2} \\
&=(a^2-b^2)\mbf{D}_{2n-2}
\end{array} 
$$
而$\mbf{D}_2=a^2 - b^2$,因此$\mbf{D}_{2n} = (a^2-b^2)^{n}$ 
\end{example}


\subsection{用矩阵运算拆开计算}

\begin{example}
\rm 设
$$
s_k = x_1^k +x_2^k + \cdots + x_n^k, ~ k = 0,1,2,\cdots, 
$$
设$\mbf{A} = (a_{ij})_{n \times n}$,其中
$$
a_{ij} = s_{i+j-2}, \, i,j = 1,2,\cdots,n.
$$
证明: $|\mbf{A}| = \prod\limits_{1 \leq j < i \leq n}(x_i - x_j)^2$. 

\hints\ 这个左边是范德蒙行列式的通项的平方. 其中
$$
a_{ij} = x_1^{i-1}x_1^{j-1} + x_2^{i-1}x_2^{j-1} + \cdots + x_n^{i-1}x_n^{j-1} = \sum\limits_{r=1}^n x_r^{i-1}x_r^{j-1}. 
$$
因此
$$
|\mbf{A}| = \begin{vmatrix}
1 & 1  & \cdots & 1 & 1 \\
x_1 & x_2 & \cdots & x_{n-1} & x_n \\
\vdots & \vdots &  & \vdots & \vdots \\
x_1^{n-2} & x_2^{n-2} & \cdots & x_{n-1}^{n-2} & x_n^{n-2} \\
x_1^{n-1} & x_2^{n-1} & \cdots & x_{n-1}^{n-1} & x_n^{n-1} \\
\end{vmatrix}
\begin{vmatrix}
1 & x_1  & \cdots & x_1^{n-2} & x_1^{n-1} \\
1 & x_2 & \cdots & x_2^{n-2} & x_2^{n-1} \\
\vdots & \vdots &  & \vdots & \vdots \\
1 & x_{n-1} & \cdots & x_{n-1}^{n-2} & x_{n-1}^{n-1} \\
1 & x_{n} & \cdots & x_{n}^{n-1} & x_n^{n-1} \\
\end{vmatrix}  
$$
显然这是两个范德蒙行列式的乘积
\end{example}

\begin{example}
\rm 设
$$
u_j = \sum\limits_{i=1}^{n}c_i{a_i}^{j},\, 1 \leq j < 2n. 
$$
令
$$
\mbf{A} = \begin{pmatrix}
u_1 & u_2 & \cdots & u_n \\
u_2 & u_3 & \cdots & u_{n+1} \\
\vdots & \vdots &  & \vdots \\
u_n & u_{n+1} & \cdots & u_{2n-1}\\
\end{pmatrix}
$$
证明: 对任意$\mbf{\beta} \in \mathbb{R}^n$,线性方程组$\mbf{A}\mbf{X} = \mbf{\beta}$有唯一解的充分必要条件是$a_1,a_2,\cdots,a_n$两两不等,且$a_i$和$c_i$全不为$0$($i=1,2,\cdots,n$). 

\hints\ 就是我们要考察行列式$|\mbf{A}|$. 观察$\mbf{A}$也是可以拆开的
$$
\begin{array}{ll}
\mbf{A} &= \begin{pmatrix}
c_1 & c_2 & \cdots & c_{n-1} & c_n \\
c_1a_1 & c_2a_1 & \cdots & c_{n-1}a_{n-1} & c_na_n \\
\vdots & \vdots &  & \vdots & \vdots \\
c_1a_1^{n-2} & c_2a_2^{n-2} & \cdots & c_{n-1}a_{n-1}^{n-2} & c_na_n^{n-2} \\
c_1a_1^{n-1} & c_2a_2^{n-1} & \cdots & c_{n-1}a_{n-1}^{n-1} & c_na_n^{n-1}
\end{pmatrix} 
\begin{pmatrix}
a_1 & a_1^2 & \cdots & a_1^{n-1} & a_1^n \\
a_2 & a_2^2 & \cdots & a_2^{n-1} & a_2^n \\
\vdots & \vdots &  & \vdots & \vdots \\
a_{n-1} & a_{n-1}^2 & \cdots & a_{n-1}^{n-1} & a_{n-1}^{n} \\
a_{n} & a_n^2 & \cdots & a_n^{n-1} & a_n^{n}  
\end{pmatrix} \\
&= c_1\cdots c_n a_1\cdots a_n \begin{pmatrix}
1 & 1 & \cdots & 1 & 1 \\
a_1 & a_2 & \cdots & a_{n-1} & a_n \\
\vdots & \vdots &  & \vdots & \vdots \\
a_1^{n-2} & a_2^{n-2} & \cdots & a_{n-1}^{n-2} & a_n^{n-2} \\
a_1^{n-1} & a_2^{n-1} & \cdots & a_{n-1}^{n-1} & a_n^{n-1}
\end{pmatrix} 
\begin{pmatrix}
1 & a_1 & \cdots & a_1^{n-2} & a_1^{n-1} \\
1 & a_2 & \cdots & a_2^{n-2} & a_2^{n-1} \\
\vdots & \vdots &  & \vdots & \vdots \\
1 & a_{n-1} & \cdots & a_{n-1}^{n-2} & a_{n-1}^{n-1} \\
1 & a_n & \cdots & a_n^{n-2} & a_n^{n-1}  
\end{pmatrix} \\
&= c_1\cdots c_2 a_1 \cdots a_n \sum\limits_{1 \leq j < i \leq n}(a_i -a_j)^2  
\end{array}
$$
\end{example}

\begin{example}
\rm 计算$n$阶行列式
$$
|\mbf{A}| = 
\begin{vmatrix}
a_1 - b_1 & a_1 - b_2 & \cdots & a_1 - b_n \\
a_2 - b_1 & a_2 - b_2 & \cdots & a_2 - b_n \\
\vdots & \vdots && \vdots \\
a_n - b_1 & a_n - b_2 & \cdots & a_n - b_n
\end{vmatrix}
$$

\hints\ 矩阵$\mbf{A}$是可以拆开的
$$
\mbf{A} = \begin{pmatrix}
a_1 & -1 \\
a_2 & -1 \\
\vdots & \vdots \\
a_{n-1} & -1 \\
a_n  & -1 
\end{pmatrix} \begin{pmatrix}
1 & 1 & \cdots & 1 & 1\\ 
b_1 & b_2 & \cdots & b_{n-1} & b_n
\end{pmatrix}
$$
当$n > 2$时,$\rank{\mbf{A}} \leq 2$,即$|\mbf{A}| = 0$. 

当$n = 2$时,$|\mbf{A}| = (a_2 - a_1)(b_2 - b_1)$.

当$n = 1$时,$|\mbf{A}| = a_1 - b_1$. 
\end{example}

\subsection{矩阵行列式Lemma}

\begin{lemma}
\rm 给定两个列向量$\mbf{u},\mbf{v}$,则有
$$
|\mbf{I} + \mbf{u}\mbf{v}^{T}| = (1 + \mbf{v}^{T}\mbf{u}). 
$$
\end{lemma}

\begin{lemma}
\rm 设$\mbf{A}$可逆,给定两个列向量$\mbf{u},\mbf{v}$,则有
$$
|\mbf{A} + \mbf{u}\mbf{v}^{T}| = (1+\mbf{v}^{T}\mbf{A}^{-1}\mbf{u})|\mbf{A}|.
$$
\end{lemma}



\newpage
\section{向量空间}

\subsection{向量运算下的线性性质判定}

\begin{example}
\rm 证明: 如果向量组$\mbf{\alpha}_1,\mbf{\alpha}_2,\mbf{\alpha}_3$线性无关,那么向量组$3\mbf{\alpha}_1-\mbf{\alpha}_2,5\mbf{\alpha}_2 + 2\mbf{\alpha}_3, 4\mbf{\alpha}_3-7\mbf{\alpha}_1$
\hints\ 直接用线性无关的定义,即
$$
k_1(3\mbf{\alpha}_1-\mbf{\alpha}_2) + k_2(5\mbf{\alpha}_2 + 2\mbf{\alpha}_3)+ k_3(4\mbf{\alpha}_3-7\mbf{\alpha}_1) = \mbf{0}
$$
此时需要证明$k_1 = k_2 = k_3$,展开上式
$$
(3k_1-7k_3)\mbf{\alpha}_1 + (5k_2 - k_1)\mbf{\alpha}_2 + (2k_2 +4k_3)\mbf{\alpha}_3 = 0. 
$$
根据已知条件,于是得到下述方程组
$$
\left \{
\begin{array}{ll}
3k_1-7k_3  = 0 \\
5k_2 - k_1 = 0 \\
2k_2 +4k_3 = 0
\end{array} \right.
$$
由此构造系数矩阵,最后系数矩阵的行列式不为零,即原方程只有零解,命题得证. 
\end{example}

\subsection{向量组的极大线性无关组}

\begin{annotation}
\rm 若给定$n$个列向量$\mbf{\alpha}_1,\cdots,\mbf{\alpha}_n$,操作步骤为
\begin{enumerate}
	\item 写出对应矩阵的形式$\mbf{A}$,做初等行变换化行阶梯型矩阵$\mbf{D}$.
	\item 观察行列式不为零最大子式所在的列,它们构成一个极大线性无关组. 
\end{enumerate}
\end{annotation}


\newpage
\section{秩}

\subsection{特殊矩阵的秩}

\begin{example}
\rm 设$n$阶矩阵$\mbf{A}$,满足
$$
|a_{ii}| >  \sum\limits_{j = 1 \atop j \neq i}^{n} |a_{ij}|, \, i = 1,2,\cdots,n.
$$
证明$\mbf{A}$的秩等于$n$. 这样的矩阵称为\redt{主对角占优矩阵}. 

\hints\  设$\mbf{A}$的列向量为$\mbf{\alpha}_1,\mbf{\alpha}_2,\cdots,\mbf{\alpha}_n$,那么只需证明$\mbf{A}$的列向量都是线性无关的. 假设$\mbf{A}$的列向量是线性相关的,则存在
$$
k_1\mbf{\alpha}_1 + k_2\mbf{\alpha}_2 + \cdots + k_n\mbf{\alpha}_n = \mbf{0},
$$
其中$k_1,k_2,\cdots,k_n$不全为$0$. 取
$$
|k_l| = \max\{|k_1|,|k_2|,\cdots,|k_n|\}. 
$$
然后我们列向量的第$l$个分量有
$$
k_1a_{l1} + k_1a_{l2} + \cdots + k_na_{ln} = 0,
$$
等式两边除以$k_l$,得到
$$
a_{ll} = - \frac{k_1}{k_l}a_{l1} - \frac{k_2}{k_l}a_{l2} - \cdots - - \frac{k_n}{k_l}a_{ln} =  -\sum\limits_{j = 1 \atop j \neq l}^{n} \frac{k_j}{k_l}|a_{lj}|.
$$
因此有
$$
|a_{ll}| \leq \sum\limits_{j = 1 \atop j \neq l}^{n} |a_{lj}|,
$$
与前提条件矛盾. 从而$\mbf{A}$等于$n$. 
\end{example}

\subsection{矩阵运算中的秩}

\begin{example}
\rm 证明: $\rank{\mbf{A}+\mbf{B}} \leq \rank{\mbf{A}} + \rank{\mbf{B}}$.

\hints\ 设$\mbf{A} = (\mbf{\alpha}_1,\mbf{\alpha}_2,\cdots,\mbf{\alpha}_n),\mbf{B}=(\mbf{\beta}_1,\mbf{\beta}_2,\cdots,\mbf{\beta}_n)$,那么$\mbf{A}+\mbf{B} = (\mbf{\alpha}_1+\mbf{\beta}_1,\mbf{\alpha}_2+\mbf{\beta}_2,\cdots,\mbf{\alpha}_n+\mbf{\beta}_n)$. 设$\mbf{A}$列向量的一个极大无关组$\mbf{\alpha}_{i_1},\mbf{\alpha}_{i_2},\cdots,\mbf{\alpha}_{i_n}$,$\mbf{B}$列向量的一个极大无关组$\mbf{\beta}_{j_1},\mbf{\beta}_{j_2},\cdots,\mbf{\beta}_{j_n}$. 显然$mbf{\alpha}_1+\mbf{\beta}_1,\mbf{\alpha}_2+\mbf{\beta}_2,\cdots,\mbf{\alpha}_n+\mbf{\beta}_n$可以由$\mbf{\alpha}_{i_1},\mbf{\alpha}_{i_2},\cdots,\mbf{\alpha}_{i_n},\mbf{\beta}_{j_1},\mbf{\beta}_{j_2},\cdots,\mbf{\beta}_{j_n}$线性表出. 
\end{example}


\begin{example}
\rm 设$\mbf{A}$为$m \times n$矩阵,则
$$
\rank{\mbf{A}\mbf{A}^T} = \rank{\mbf{A}^T\mbf{A}} = \rank{\mbf{A}}. 
$$
\hints\ 在note上记录了一种化行阶梯形的证明方法,这里从判定$n$元齐次线性方程组$(\mbf{A}^T\mbf{A})\mbf{X}=\mbf{0}$与$\mbf{A}\mbf{X}= \mbf{0}$同解出发,同解意味着解空间维数相同,从而对应的系数矩阵的秩相同. 

若$\mbf{\eta}$是$\mbf{A}\mbf{X}= \mbf{0}$的一个解,那么显然$\mbf{\eta}$也是$(\mbf{A}^T\mbf{A})\mbf{X}=\mbf{0}$的解. 

若$\mbf{\delta}$是$(\mbf{A}^T\mbf{A})\mbf{X}=\mbf{0}$的解,于是有$(\mbf{A}^T\mbf{A})\mbf{\delta}=\mbf{0}$,将这个等式两边乘上$\mbf{\delta}^T$,得到
$$
\mbf{\delta}^T(\mbf{A}^T\mbf{A})\mbf{\delta}=\mbf{0} \Rightarrow (\mbf{\delta}^T\mbf{A}^T)(\mbf{A}\mbf{\delta})=\mbf{0} \Rightarrow (\mbf{A}\mbf{\delta})^T(\mbf{A}\mbf{\delta})=\mbf{0}
$$
那么设$(\mbf{A}\mbf{\delta})^T = (c_1,c_2,\cdots,c_n)$,那么则有
$$
c_1^2 + c_2^2 + \cdots + c_n^2 = 0,
$$
因此$c_1 = c_2 = \cdots = c_n = 0$,所以$\mbf{\delta}$也是$\mbf{A}\mbf{X}= \mbf{0}$的解. 

最后
$$
\rank{\mbf{A}\mbf{A}^T} = \rank{(\mbf{A}^T)^T\mbf{A}^T} = \rank{\mbf{A}^T} = \rank{\mbf{A}}.   
$$
\end{example}

\subsection{秩为1的矩阵相关问题}

\begin{example}
\rm MSE上相关问题收集\href{https://math.stackexchange.com/questions/904926/determinant-of-a-rank-1-update-of-a-scalar-matrix-or-characteristic-polynomia}{determinant-of-a-rank-1-update-of-a-scalar-matrix-or-characteristic-polynomia}
\begin{enumerate}
	\item \href{https://math.stackexchange.com/q/153457/18880}{元素全是$1$的$n$阶矩阵的特征多项式通项}
	\item \href{https://math.stackexchange.com/q/55165/18880}{秩为$1$的$n$矩阵的特征值$uv^T$}
	\item \href{https://math.stackexchange.com/q/577937/18880}{计算$A+I$的行列式,其中$A$为秩为1的$n$阶矩阵}
	\item \href{https://math.stackexchange.com/q/84206/18880}{计算对角线元素都是$0$,其他元素都是$1$的$n$阶矩阵的行列式}
	\item \href{https://math.stackexchange.com/q/86644/18880}{计算对角线元素都是$a$,其他元素都是$b$的$n$阶矩阵的行列式}
	\item \href{https://math.stackexchange.com/questions/629892/determinant-of-a-special-n-times-n-matrix}{计算对角线元素都是$2$,其他元素都是$1$的$n$阶矩阵的行列式}
\end{enumerate}
\end{example}

\newpage
\section{方程组的解}

\subsection{方程组性质}

\begin{example}
\rm 证明: 给定$s$个$n$线性方程组成的线性方程组,如果该方程组的增广矩阵的第$i$个行向量$\mbf{\alpha}_i$可以由其余行向量线性表出,即
$$
\mbf{\alpha}_i = k_1\mbf{\alpha}_1 + k_{i-1}\mbf{\alpha}_{i-1} + k_{i+1}\mbf{\alpha}_{k+1} + \cdots + k_s\mbf{\alpha}_s 
$$
那么将第$i$个方程去掉之后得到的方程组与原方程组通解. 
\end{example}

\begin{example}
\rm 设一个$m \times n$矩阵$H$的列向量组为$\mbf{\alpha}_1,\cdots,\mbf{\alpha}_n$. 证明: $H$的任意$s$列都线性无关当且仅当,齐次线性方程组
$$
x_1 \mbf{\alpha}_1 + \cdots + x_n \mbf{\alpha}_n = \mbf{0}
$$
的任一非零解的非零分量的数目大于$s$. 

\hints\ \emph{必要性}\ 若$H$的任意$s$列都线性无关. 假设上述线性方程存在一个非零解为
$$
\eta = (0,\cdots,c_{i_1},\cdots,c_{i_j},\cdots)^{T}. 
$$
其中$c_{i_1},\cdots,c_{i_j}$均不为零,且$ l \leq s$. 则
$$
c_{i_1}\mbf{\alpha}_{i_1}+\cdots+c_{i_j}\mbf{\alpha}_{i_j} = \mbf{0},
$$
这意味着存在$l$列线性相关,与前提数矛盾,因此任一非零解的非零分量的数目大于$s$

\emph{充分性}\ 若$H$的任一非零向量的非零分量的数目大于$s$. 假设有$l\leq s$个列向量线性相关
$$
k_{i_1}\mbf{\alpha}_{i_1}+\cdots+k_{i_j}\mbf{\alpha}_{i_j} = \mbf{0},
$$
那么存在一个非零解,即
$$
\eta = (0,\cdots,k_{i_1},\cdots,k_{i_j},\cdots)^{T}. 
$$
就是将其他分量扩充为0即可,这样和前提是矛盾的. 因此$H$的任意$s$列都线性无关. 
\end{example}

\subsection{给定方程组解的情况}


\subsection{带参数的方程组解的情况}


\subsection{线性方程充要条件}

\begin{annotation}
\rm \redt{系数矩阵的秩和增广矩阵的秩相同}!
\end{annotation}

\begin{example}
\rm 证明: 线性方程组
$$
\left\{
\begin{array}{l}
a_{11}x_1 + a_{12}x_2 + \cdots + a_{1n}x_n = b_1, \\
a_{21}x_1 + a_{22}x_2 + \cdots + a_{2n}x_n = b_2, \\
\cdots \\
a_{m1}x_1 + a_{m2}x_2 + \cdots + a_{mn}x_n = b_m, \\
\end{array} \right.
$$
有解的当且仅当下述线性方程
$$
\left\{
\begin{array}{l}
a_{11}x_1 + a_{21}x_2 + \cdots + a_{s1}x_m = 0, \\
a_{12}x_1 + a_{22}x_2 + \cdots + a_{s2}x_m = 0, \\
\cdots \\
a_{1m}x_1 + a_{2m}x_2 + \cdots +a_{sm}x_m = 0, \\
b_{1}x_1 + b_{2}x_2 + \cdots+ b_{m}x_m = 1, \\
\end{array} \right.
$$

\hints\ 设第一个方程组的系数矩阵为$\mbf{A}$,增广矩阵为$\tilde{\mbf{A}}$. 那么第二个方程组的系数矩阵及增广矩阵分别为
$$
\begin{array}{ll}
\mbf{B} = \tilde{\mbf{A}}^T \\
\tilde{\mbf{B}}= \begin{pmatrix}
\mbf{A}^T & \mbf{0} \\\
\mbf{\beta} & 1
\end{pmatrix}
\end{array}
$$
其中$\mbf{\beta} = (b_1,\cdots,b_m)$. 这里存在一些等式
$$
\begin{array}{ll}
\rank{\mbf{B}} = \rank{\tilde{\mbf{A}}} \\
\rank{\tilde{\mbf{B}}} = \rank{\mbf{A}} + 1.
\end{array}
$$

当$\rank{\mbf{A}} = \rank{\tilde{\mbf{A}}}$时,有$\rank{\mbf{B}} < \rank{\tilde{\mbf{B}}}$. 

当$\rank{\mbf{B}} < \rank{\tilde{\mbf{B}}}$时,有$\rank{\tilde{\mbf{A}}} + 1 = \rank{\mbf{A}} + 1$. \bluet{这里能得到这个结果是因为增广矩阵的秩那么等于系数矩阵的秩,那么等于系数矩阵的秩加1}. 
\end{example}

\subsection{齐次线性方程组的性质}

\begin{example}
\rm 设$n$个的方程的$n$元齐次线性方程组的系数矩阵$\mbf{A}$的行列式等于$0$,且$\mbf{A}$的$(k,l)$元的代数余子式$A_{kl} \neq 0$. 证明$\mbf{\eta}=(A_{k1},A_{k2},\cdots,A_{kn})^T$是原齐次线性方程组的一个基础解系. 

\hints\ 证明的要求暗示了解空间是一维的. 这是因为由$A_{kl} \neq 0$,意味着$\mbf{A}$存在一个$n-1$阶的子式行列式不为0,而$|\mbf{A}| = 0$,因此$\rank{\mbf{A}} = n-1$. 

将该$\mbf{\eta}$带入原方程,当$i = k$时
$$
a_{i1}A_{k1} + a_{i1}A_{k1} +\cdots +a_{in}A_{kn} = |\mbf{A}| = 0; 
$$
当$i \neq k$时,显然有
$$
a_{i1}A_{k1} + a_{i1}A_{k1} +\cdots +a_{in}A_{kn} = 0.
$$
因此$\mbf{\eta}$的确是原方程组的一个解,其中第$k$个分量$A_{kl} \neq 0$,结合解空间是一维的,从而$\mbf{\eta}$是一个基础解系. 
\end{example}

\subsection{非齐次线性方程组的性质}

\begin{example}
\rm 给定$n$元非齐次线性方程组
$$
x_1\mbf{\alpha}_1 + x_2\mbf{\alpha}_2 + \cdots + x_n\mbf{\alpha}_n = \mbf{\beta},
$$
它的一个特解为$\mbf{\gamma}_0$和基础解系为$\mbf{\eta}_1,\mbf{\eta}_2,\cdots,\mbf{\eta}_t$. 令
$$
\mbf{\gamma}_1 = \mbf{\gamma}_0 + \mbf{\eta}_1, \mbf{\gamma}_2 = \mbf{\gamma}_0 + \mbf{\eta}_2,\cdots,\mbf{\gamma}_n = \mbf{\gamma}_0 + \mbf{\eta}_t.
$$
那么该非齐次线性方程的解集可以表示为
$$
U = \Set{k_0\mbf{\gamma}_0+k_1\mbf{\gamma}_1+\cdots+k_t\mbf{\gamma}_t}{k_0 + k_1 + \cdots + k_t = 1, k_i \in \mathbb{R},i =0,2,\cdots,t}
$$
\bluet{这个例子表明非齐次线性方程组也可以用有限个线性无关的解向量来表示解空间,但是系数上是有限制的}.
\end{example}

\newpage
\section{矩阵运算}

\subsection{矩阵乘法性质}

\begin{example}
\rm 设$\mbf{A}$和$\mbf{B}$均为$n$阶矩阵. 若$\mbf{A}^2 = \mbf{B}^2$,不一定有$\mbf{A} = \mbf{B}$或者$\mbf{A} = -\mbf{B}$.

\hints\ 首先$\mbf{A}$和$\mbf{B}$不一定交换,即
$$
\mbf{A}^2 - \mbf{B}^2 \neq (\mbf{A} + \mbf{B})(\mbf{A} - \mbf{B}).
$$
即使有$\mbf{A}$和$\mbf{B}$,即
$$
(\mbf{A} + \mbf{B})(\mbf{A} - \mbf{B}) = \mbf{0},
$$
也不一定有$\mbf{A} + \mbf{B} = \mbf{0}$或者$\mbf{A} - \mbf{B} = 0$. 例如$\mbf{A} = \mbf{I}_2$,$\mbf{B} = \begin{pmatrix}
1 & 0 \\
0 & -1
\end{pmatrix}.$
\end{example}

\subsection{幂运算}


\begin{example}
\rm 设$\mbf{A} = \begin{pmatrix}
2 & 3 \\
0 & 2
\end{pmatrix}$,求$\mbf{A}^m$. 

\hints\ 将$\mbf{A}$拆开
$$
\mbf{A} = \begin{pmatrix}
2 & 0 \\
0 & 2 
\end{pmatrix} + \begin{pmatrix}
0 & 3 \\
0 & 0
\end{pmatrix} = 2\mbf{I} + 3\mbf{B}.
$$
其中$\mbf{B}= \begin{pmatrix}
0 & 1 \\
0 & 0
\end{pmatrix}$. 我们的目标很明确了是尝试用二项式展开,因为这里
$$
\mbf{B}^2 = \mbf{0}. 
$$
还有一个关键是$\mbf{B}$和$\mbf{I}$是可交换的,因此才有
$$
(2\mbf{I}3\mbf{B})(2\mbf{I}3\mbf{B})\cdots(2\mbf{I}3\mbf{B}) = 2^m3^m \mbf{I}^m \mbf{B}^m,
$$ 
不然这里是用不了二项式展开的. 于是
$$
\mbf{A}^m = (2\mbf{I} + 3\mbf{B})^m = (2\mbf{I})^m + C_m^1 (2\mbf{I})^{m-1}3\mbf{B} = 2^m \mbf{I} + 3m 2^{m-1}\mbf{B} = \begin{pmatrix}
2^m & 3m2^{m-1} \\
0 & 2^m 
\end{pmatrix}
$$
\end{example}


\begin{example}
\rm 设$\mbf{A} = \begin{pmatrix}
a & c \\
0 & b
\end{pmatrix}$,求$\mbf{A}^m$. 


\hints\ 这里是不能用二项式展开的,因为不确定$\begin{pmatrix}
a & 0 \\
0 & b\\
\end{pmatrix}$和$\begin{pmatrix}
0 & c \\
0 & 0
\end{pmatrix}$是否交换. 可以先写出前几项,猜结果用数学归纳法来证. 
$$
\begin{array}{ll}
\mbf{A}^2 = \begin{pmatrix}
a & c \\
0 & b\\
\end{pmatrix}\begin{pmatrix}
a & c \\
0 & b\\
\end{pmatrix} = \begin{pmatrix}
a^2 & c(a+b) \\
0 & b^2 
\end{pmatrix} \\
 \mbf{A}^3 = \begin{pmatrix}
a^2 & c(a+b) \\
0 & b^2 
\end{pmatrix}\begin{pmatrix}
a & c \\
0 & b\\
\end{pmatrix} = \begin{pmatrix}
a^3 & c(a^2 + ab + b^2) \\
0 & b^3  
\end{pmatrix} \\
 \mbf{A}^4 = \begin{pmatrix}
a^3 & c(a^2 + ab + b^2) \\
0 & b^3  
\end{pmatrix}\begin{pmatrix}
a & c \\
0 & b\\
\end{pmatrix} = \begin{pmatrix}
a^4 & c(a^3 + a^2b + ab^2 + b^3) \\
0 & b^4  
\end{pmatrix}
\end{array} 
$$
因此我们猜测其通项为
$$
\mbf{A}^m = \begin{pmatrix}
a^m & c(a^{m-1} + a^{m-2}b + \cdots + ab^{m-2}+b^{m-1}) \\
0 & b^m
\end{pmatrix}
$$
当$m = 2$时,显然是满足. 假设$m = k-1$时满足,考虑$m = k$时. 根据幂运算,有
$$
\begin{array}{ll}
\mbf{A}^k = \mbf{A}^{k-1} \mbf{A} = \mbf{A}^m &= \begin{pmatrix}
a^{k-1} & c(a^{k-2} + a^{k-3}b + \cdots + ab^{k-3}+b^{k-2}) \\
0 & b^{k-1}
\end{pmatrix}\begin{pmatrix}
a & c \\
0 & b\\
\end{pmatrix} \\
&= \begin{pmatrix}
a^{k} & c(a^{k-1} + a^{k-2}b + \cdots + ab^{k-2}+b^{k-1}) \\
0 & b^{k}
\end{pmatrix}
\end{array}
$$
\end{example}

\subsection{矩阵交换性}

\begin{example}
\rm 求与矩阵$\mbf{A}$交换的所有矩阵,设
$$
\mbf{A} = \begin{pmatrix}
3 & 1 & 0 \\
0 & 3 & 1 \\
0 & 0 & 3
\end{pmatrix}
$$

\hints\ 首先拆一下$\mbf{A}$,即
$$
\mbf{A} = \begin{pmatrix}
3 & 0 & 0 \\
0 & 3 & 0 \\
0 & 0 & 3
\end{pmatrix} + \begin{pmatrix}
0 & 1 & 0 \\
0 & 0 & 1 \\
0 & 0 & 0
\end{pmatrix} = 3\mbf{I} + \mbf{B}
$$
其中$I$与任意矩阵都是可交换,于是
$$
\mbf{A}\mbf{X} = \mbf{X}\mbf{A} \Leftrightarrow \mbf{B}\mbf{X} = \mbf{X}\mbf{B}
$$
因此有等式
$$
\begin{pmatrix}
0 & 1 & 0 \\
0 & 0 & 1 \\
0 & 0 & 0
\end{pmatrix} \begin{pmatrix}
x_{11} & x_{12} & x_{13} \\
x_{21} & x_{22} & x_{23} \\
x_{31} & x_{32} & x_{33}
\end{pmatrix}  = \begin{pmatrix}
x_{11} & x_{12} & x_{13} \\
x_{21} & x_{22} & x_{23} \\
x_{31} & x_{32} & x_{33}
\end{pmatrix} \begin{pmatrix}
0 & 1 & 0 \\
0 & 0 & 1 \\
0 & 0 & 0
\end{pmatrix} 
$$
可以得到
$$
\begin{pmatrix}
x_{21} & x_{22} & x_{23} \\
x_{31} & x_{32} & x_{33} \\
0 & 0 & 0
\end{pmatrix} = 
\begin{pmatrix}
0 & x_{11} & x_{12} \\
0 & x_{21} & x_{22} \\
0 & x_{31} & x_{32}
\end{pmatrix}
$$
整理一下
$$
\left\{
\begin{array}{ll}
x_{21} = x_{31} = x_{32} = 0 \\
x_{22} = x_{11} = x_{33} \\
x_{23} = x_{12} \\
\end{array} \right.
$$
因此
$$
\mbf{X} = \begin{vmatrix}
x_{11} & x_{12} & x_{13} \\
0 & x_{11} & x_{12} \\
0 & 0 & x_{11}
\end{vmatrix}
$$
其中$x_{11},x_{12},x_{13} \in \mathbb{R}$. 
\end{example}

\begin{example}
\rm 设$\mbf{D}$为主对角线两两不相等的对角矩阵,那么与$\mbf{D}$可交换的矩阵一定是对角矩阵. 
\end{example}

\begin{example}
\rm 设$\mbf{A}$与所有$n$阶矩阵都可交换,则$\mbf{A}$必为数量矩阵. 
\end{example}

\subsection{矩阵表示}

\begin{example}
\rm 任一$n$阶矩阵都可以表示成一个对称矩阵和斜对称矩阵之和,并且表法唯一.

\hints\ 设$\mbf{A} = \mbf{A}_1 + \mbf{A}_2$,其中$\mbf{A}_1$为对称矩阵,$\mbf{A}_2$为斜对称矩阵. 等式两边取转置,得到$\mbf{A}^T = \mbf{A}_1 - \mbf{A}_2$. 联立两个方程得到$\mbf{A}_1 = \frac{\mbf{A} + \mbf{A}^T}{2}$和$\mbf{A}_2 = \frac{\mbf{A} - \mbf{A}^T}{2}$. 
\end{example}

\subsection{特殊矩阵}

\begin{example}
\rm 令
$$
\mbf{C} = \begin{pmatrix}
0 & 1 & 0 & 0 & \cdots & 0 & 0 \\
0 & 0 & 1 & 0 & \cdots & 0 & 0 \\
\vdots & \vdots & \vdots &  \ddots &  & \vdots & \vdots \\
0 & 0 & 0 & 0 & \cdots & 0 & 1 \\
1 & 0 & 0 & 0 & \cdots & 0 & 0 \\
\end{pmatrix}
$$
称$\mbf{C}$为$n$阶循环移位矩阵. 用$\mbf{C}$左乘一个矩阵,就相当于把这个矩阵每一行向上移一行,第一行移到最后一行; 用$\mbf{C}$右乘一个矩阵,就相对于把这个矩阵的每一列想右移动一列,最后一列移到第一列. 
\end{example}

\newpage
\section{矩阵的逆}

\subsection{复杂矩阵的逆}

\begin{example}
\rm 若$A^l = \mbf{0}$,其中$l$是使得$A^m = \mbf{0}$中最小的那个$m$,如果存在这样$l$,则称$\mbf{A}$是一个幂零矩阵,它的幂零指数为$l$. 证明在这种情况下$\mbf{I}-\mbf{A}$是可逆矩阵.

\hints\ 这个是比较有构造性,需要把相应的可逆矩阵构造出来,即
$$
\mbf{I} - \mbf{A}^l = \mbf{0} \Rightarrow (\mbf{I}-\mbf{A})(\mbf{I} + \mbf{A} + \mbf{A}^2 +\cdots + \mbf{A}^{l-1}) = \mbf{I} - \mbf{A}^l = \mbf{I},
$$  
这里可以把$\mbf{I} - \mbf{A}^l = \mbf{0}$看做$1-x^n = 0$,那么$\mbf{I}$这个方程的一个根,把这个根先提出来,就可以得到上面的结果.
\end{example}

\subsection{直觉猜}

\begin{annotation}
\rm 如果遇到求某种形式的矩阵的逆,你有一个直觉它的逆大概是怎样的,你可以按照你的直觉来推一下,如果恰好是这样的,再根据矩阵逆的唯一性,那么你的推测就是正确的. 
\end{annotation}

\begin{example}
\rm 求下述$n$阶矩阵$\mbf{A}$的逆矩阵($n \geq 2$)
$$
\mbf{A} =
\begin{pmatrix}
0 & 1 & 1 & \cdots & 1 \\
1 & 0 & 1 & \cdots & 1 \\
\vdots & \vdots & \vdots &  & \vdots \\
1 & 1 & 1 & \cdots & 0
\end{pmatrix}
$$

\hints\ 这个矩阵的逆可以直接用行变换化单位矩阵的方法来做,但是有比较trick的做法,观察到$\mbf{A} = \mbf{B}-\mbf{I}$,其中$\mbf{B}$是元素全为$1$的矩阵,这里我们猜测$\mbf{A}^{-1} = a\mbf{B} + b\mbf{I}$,那么它一定需要满足
$$
\begin{array}{ll}
\mbf{I}  &= (B-I)(aB + bI) \\
  &= aB^2+(b-a)B -bI \\
  &= anB + (b-a)B - bI \\
  &= (an+b-a)B - bI
\end{array}
$$
因此$b=-1,a = \frac{1}{n-1}$. 注意到这里有一个比较好的性质$\mbf{B}^2 = n\mbf{B}$.

来讲述一下这里为什么可以这样来待定系数做的方法,首先我们观察到$\mbf{A}$是可逆的,因此这里可以有$\mbf{A}^{-1} = f(\mbf{A})$,其中$f(\mbf{A})$表示$\mbf{A}$的多项式,这一细节去看一下Cayley–Hamilton theorem即可. 这里将$\mbf{A}$分解成$\mbf{B} - \mbf{I}$再带入$f(\mbf{A})$可以得到一个关于$\mbf{B}$的多项式$g(\mbf{B})$. 特别地这里$\mbf{B}$秩等于$1$,那么$\mbf{B}$又可以分解为两个一维向量的乘积$\mbf{\alpha} \mbf{\beta}^T$,因此$\mbf{B}^m = k^{m-1}\mbf{B}$. 所以最终有形式$g(B) = a\mbf{B} + b\mbf{I}$. 
\end{example}

\newpage
\section{矩阵等价}

\subsection{判定等价}

\begin{annotation}
\rm \redt{遵循矩阵等价当且仅当它们秩相同}.
\end{annotation}

\subsection{矩阵等价的性质}

\begin{example}
\rm 设$m \times n$矩阵$\mbf{A}$的秩为$r(r>0)$,则$\mbf{A}$可以表示成$r$个秩为$1$的矩阵之和. 

\hints\ 根据$\mbf{A}$相似于它的标准型,有
$$
\mbf{A} = \mbf{P}\begin{pmatrix}
\mbf{I}_r & \mbf{0} \\
\mbf{0} & \mbf{0}
\end{pmatrix}\mbf{Q} = \mbf{P}\mbf{E}_{11}\mbf{Q} + \mbf{P}\mbf{E}_{22}\mbf{Q} + \cdots + \mbf{P}\mbf{E}_{rr}\mbf{Q}
$$
\end{example}

\begin{example}
\rm 设$m \times n$矩阵$\mbf{A}$,则$\mbf{A}$的秩为$r$当且仅当存在$m \times r$列满秩矩阵$\mbf{B}$与$r \times n$行满秩矩阵$\mbf{C}$,使得$\mbf{A}=\mbf{B}\mbf{C}$. 

\hints\ \emph{必要性}\ 利用$\mbf{A}$等价于它的标准相似性
$$
\mbf{A} = \mbf{P}\begin{pmatrix}
\mbf{I}_r & \mbf{0} \\
\mbf{0} & \mbf{0}
\end{pmatrix}\mbf{Q} = (\mbf{P}_1, \mbf{P}_2)\begin{pmatrix}
\mbf{I}_r & \mbf{0} \\
\mbf{0} & \mbf{0}
\end{pmatrix}\begin{pmatrix}
\mbf{Q}_1 \\
\mbf{Q}_2 
\end{pmatrix} = (\mbf{P}_1, \mbf{0})\begin{pmatrix}
\mbf{Q}_1 \\
\mbf{Q}_2 
\end{pmatrix} = \mbf{P}_1\mbf{Q}_1.
$$
其中$\mbf{P}_1$是$m \times r$列满秩矩阵,$\mbf{Q}_1$是$r \times n$行满秩矩阵. 

\emph{充分性}\ 由矩阵乘法中秩的上界知道
$$
\rank{\mbf{BC}} \leq r.
$$ 
再由矩阵乘法中秩的下界知道
$$
\rank{\mbf{BC}} \geq \rank{\mbf{B}} + \rank{\mbf{C}} -r =r.
$$
因此$\rank{\mbf{BC}} = \rank{\mbf{A}} = r$. 
\end{example}

\newpage
\section{矩阵相似}

\subsection{相似判定}

\begin{proposition}
\rm 常用判定矩阵相似的方法,遇题依次向下使用下述方法.
\begin{enumerate}
	\item 必要条件:相似必行列值相等;
	\item 必要条件:特征值相等;
	\item 充分条件: 对于都可对角化的矩阵,判定其特征值是否相同;
	\item 否命题的充分条件: 一个可对角化,一个不可对角化,则它们不相似;
	\item 对于都不可对角的矩阵,同一个特征值的特征子空间的维数相同;	
	\item 对于都不可对角的矩阵,则对应的特征向量满足: 若$\mbf{B}$对应$\lambda$的特征向量$\lambda$,则$\mbf{A}$对应$\lambda$的特征向量为$P\alpha$. 这里需要求出可逆矩阵$P$
\end{enumerate}
\end{proposition}


\subsection{对角化判定}

\begin{proposition}
\rm 常用判定对角化的方法,遇题依次向下使用下述方法
\begin{enumerate}
	\item 实对称矩阵一定相似于对角矩阵;	
	\item 有$n$个不同的特征值,那么一定相似于对角矩阵;
	\item $n$重特征值对应特征子空间是否为$n$维;
\end{enumerate}
\end{proposition}

\section{二次型}

\subsection{正定性的判定}



\end{document}